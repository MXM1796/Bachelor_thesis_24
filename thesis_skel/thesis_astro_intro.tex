% !TEX root = mythesis.tex

%==============================================================================
\chapter{Introduction}
\label{sec:intro}
%==============================================================================

The introduction usually gives a few pages of introduction to the
whole subject, maybe even starting with the Greeks.

For more information on \LaTeX{} and the packages that are available
see for example the books of Kopka~\citep{kopka04} and Goossens et
al~\citep{goossens04}.

A lot of useful information on particle physics can be found in the
\enquote{Particle Data Book}~\citep{pdg2010}.

I have resisted the temptation to put a lot of definitions into the
file \texttt{thesis\_defs.sty}, as everyone has their own taste as
to what scheme they want to use for names.
However, a few examples are included to help you get started:
\begin{itemize}
\setlength{\itemsep}{0pt}\setlength{\parskip}{0pt}
\item cross-sections are measured in \unit{\pb} and integrated
  luminosity in \unit{\invpb};
\item the \KoS is an interesting particle;
\item the missing transverse momentum, \pTmiss, is often called
  missing transverse energy, even though it is calculated using a vector sum.
\end{itemize}
Note that the examples of units assume that you are using the
\texttt{siunitx} package.

It also is probably a good idea to include a few well formatted
references in the thesis skeleton. More detailed suggestions on what
citation types to use can be found in the \enquote{Thesis Guide}~\cite{thesis-guide}:

\begin{itemize}
\item articles in refereed journals~\citep{pdg2010,Aad:2010ey};
\item a book~\citep{Halzen:1984mc};
\item a PhD thesis~\citep{tlodd:2012} and a Diplom thesis~\citep{mergelmeyer:2011};
\item a collection of articles~\citep{lhc:vol1};
\item a conference note~\citep{ATLAS-CONF-2011-008};
\item a preprint~\citep{atlas:perf:2009} (you can also use
  \texttt{@online} or \texttt{@booklet}for such things;
\item something that is only available online~\citep{thesis-guide}.
\end{itemize}
Note that astronomy publications use citation commands defined by \Package{natbib}.
You should choose which one you want to use:
\begin{itemize}
\item \textbackslash citep: a citation~\citep{pdg2010,Aad:2010ey};
\item \textbackslash citet: a citation~\citet{pdg2010,Aad:2010ey};
\item \textbackslash citealt: a citation~\citealt{pdg2010,Aad:2010ey}.
\end{itemize}

At the end of the introduction it is normal to say briefly what comes
in the following chapters.

The line at the beginning of this file is used by TeXstudio etc.\ to
specify which is the master \LaTeX\ file, so that you can compile your thesis
directly from this file.
If your thesis is called something other than \texttt{mythesis}, adjust it as appropriate.

For demonstration purposes,
we include a figure and a table that a referenced using the \texttt{cleveref} package.
\Cref{fig:nothing} does not show much,
while \cref{tab:little} is not much better.

\begin{figure}[htbp]
  \centering
  \fbox{\textcolor{red}{This is not really a figure!}}
  \caption{A caption for a figure that is not really there.
    Just for fun we can refer to the contents of \cref{tab:little}.}
  \label{fig:nothing}
\end{figure}

\begin{table}[htbp]
  \caption{A table with not much in it.
    Just for fun we can refer to the contents of \cref{fig:nothing}.}
  \label{tab:little}
  \centering
  \begin{tabular}{ll}
    \toprule
    Number & Letter \\
    \midrule
    1 & A \\
    2 & B \\
    \bottomrule  
  \end{tabular}
\end{table}

% Print the bibliography starting on the same page at the end of the chapter.
% \printbibliography[heading=subbibliography]
