\documentclass[Bachelor, ngerman, UKenglish]{scrbook}
%------------------------------------------------------------------------------
% This file contains a skeleton thesis for
% a Physics or Astronomy Institute in the University of Bonn.

% Specify the thesis type as an option: PhD, Master, Diplom, Bachelor.
% Specify the thesis stage as an option: Draft (default), Submit, Final, PILibrary.

% Specify the language(s) in the class and then use babel.
% If you need more than one language, give the default language last,
% e.g. ngerman, UKenglish for a thesis in British (UK) English where you want
% to be able to set the language to German for some part of it.

%------------------------------------------------------------------------------
% Pass TeX Live version to the package.
% Use command pdflatex --version to find out which version you are running.
% twoside=true is suitable for printing, while twoside=false is probably better for PDF version.
\usepackage[twoside=true]{ubonn-thesis}

%------------------------------------------------------------------------------
% Adjustments to standard biblatex style.
% Change option to backref=false when your thesis is ready to turn off back-referencing.
% Pass the option showurl=false to shorten your bibliography by not including url fields.
\usepackage[backref=true]{ubonn-biblatex}

%------------------------------------------------------------------------------
% Glossary package
% \usepackage[acronym,toc,nosuper]{glossaries}
% TikZ packages and libraries
% \usepackage{tikz}
% \usepackage{tikz-3dplot}
% \usepackage{pgfplots}
% \usetikzlibrary{positioning,shapes,arrows}
% \usetikzlibrary{decorations.pathmorphing}
% \usetikzlibrary{decorations.markings}
\usepackage{thesis_defs}
\usepackage{color}
\usepackage{amsfonts}
\usepackage{import}
\usepackage{float}
\usepackage{placeins}
\usepackage{environ}

%\RenewEnviron{figure}{}% Gobble figure environment
%\RenewEnviron{table}{}% Gobble table environment


%------------------------------------------------------------------------------
% Instead of colouring  links, cites, table of contents etc.
% put them in a coloured box for the screen version.
% This is probably a good idea when you print your thesis.
% \hypersetup{colorlinks=false,
%   linkbordercolor=blue,citebordercolor=magenta,urlbordercolor=darkgreen
% }

%------------------------------------------------------------------------------
% When writing your thesis it is often helpful to have the date and
% time in the output file. Comment this out for the final version.
\ifoot[\today{} \thistime]{\today{} \thistime}

% In order to check if your labels are referenced try the refcheck package
% \usepackage{refcheck}

%------------------------------------------------------------------------------
% biblatex is included by ubonn-thesis. Look there for the settings used.
% See the options for settings that can be changed easily.
% For further changes copy the \RequirePackage[...]{biblatex} here
% and include ubonn-thesis with the option biblatex=false.

% Specify the bibliography files here and not at the end!
% Use standard_refs-bibtex if you use bibtex or bibtex8
% and standard_refs-biber  if you use biber
\addbibresource{bib/thesis_refs.bib}
\addbibresource{bib/standard_refs-biber.bib}

%------------------------------------------------------------------------------
% The following definitions are used to produce the title pages
% needed at various stages
\newcommand{\thesistitle}{Characterization of a Superconducting nanowire single photon detector}
\newcommand*{\thesisauthor}{Max Reicherd}
\newcommand*{\thesistown}{Gotha}
\renewcommand*{\InstituteName}{\IAP}
\renewcommand*{\inInstitute}{\inIAP}
\renewcommand*{\InstituteAddress}{\IAPaddress}
% Adjust \thesisreferee...text depending on male/female referee
\newcommand*{\thesisrefereeonetext}{1.\ Reviewer}
\newcommand*{\thesisrefereeone}{Prof.\ Dr.\ Sebastian Hofferberth}
\newcommand*{\thesisrefereetwotext}{2.\ Reviewer}
\newcommand*{\thesisrefereetwo}{Prof.\ Dr.\ Daqing Wang}
% Date when thesis was submitted (Master/Diplom)
% Year or Month, Year when thesis was submitted (PhD)
\newcommand*{\thesissubmit}{31.07.2024}
% \newcommand*{\thesissubmit}{Month 2024}
% Date of thesis examination (PhD)
\newcommand*{\thesispromotion}{31.07.2024}
% Month and year of the final printed version of the thesis
\newcommand*{\thesismonth}{Juli}
\newcommand*{\thesisyear}{2024}
\newcommand*{\thesisnumber}{BONN-IR-2024-XXX}
% Dedication
% \newcommand*{\thesisdedication}{}

%------------------------------------------------------------------------------
% The abstract is only needed for the printed version and should be in
% English regardless of the language of the thesis
\newcommand{\thesisabstract}{%
  \begin{otherlanguage}{UKenglish}
    This is your thesis abstract. It may be in a language that is
    different from the rest of your thesis.
  \end{otherlanguage}
}

%------------------------------------------------------------------------------
% \includeonly can be used to select which chapters you want to process
% A simple \include command just inserts a \clearpage before and after the file
% Note that \includeonly can be quite picky! Do not forget to put a
% comma after the filename, otherwise it will simply be ignored!
 \includeonly{
   Latex_Contentfiles/thesis_acknowledge,
   Latex_Contentfiles/Introduction_to_SNSPD,
   Latex_Contentfiles/Working_principle_of_SNSPD,
   Latex_Contentfiles/Building_set_up_for_Detector_Characterization,
   Latex_Contentfiles/Characterization_of_SNSPD,
   Latex_Contentfiles/Conclusion_and_Outlook,
   Latex_Contentfiles/thesis_appendix
 }

%------------------------------------------------------------------------------
% Give a list of directories where figures can be found. Do not leave
% any spaces in the list and end the directory name with a /
\graphicspath{%
  {figs/}%
  {figs/cover/}%
  {/Users/maxim.re/Studium/Physik B.Sc./Semester_8_SS24/Proseminar/Figs Single Photon Detection/}%
  {\string~/sciebo/Bilder_BA_MaxR/}%
}

%------------------------------------------------------------------------------
% Make a glossary and a list of acronyms
% \makeglossaries

% Glossary entries
% \input{thesis_glossary}

% Draft version - add the word DRAFT on the cover pages
% Also add the date and time of compilation and turn on line numbers (if option is set).
%\ifthenelse{\equal{\ThesisVersion}{Draft}}{%
%  \usepackage{background}
%  \backgroundsetup{contents=DRAFT, color=blue!30}
%  \ifoot[\today{} \thistime]{\today{} \thistime}
%  \ifthenelse{\boolean{ThesisLineno}}{\linenumbers}{}
% }

%------------------------------------------------------------------------------
\begin{document}

% Make cover and title pages
\makethesistitle

%\pagestyle{scrplain}

%------------------------------------------------------------------------------
% You can add your acknowledgements here - don't forget to also add
% them to \includeonly above
%------------------------------------------------------------------------------
\chapter*{Acknowledgements}
\label{sec:ack}
%------------------------------------------------------------------------------

I would like to thank ...

You should probably use \texttt{\textbackslash chapter*} for
acknowledgements at the beginning of a thesis and
\texttt{\textbackslash chapter} for the end.

%%% Local Variables: 
%%% mode: latex
%%% TeX-master: "../mythesis"
%%% End: 


\tableofcontents

\mainmatter
\pagestyle{scrheadings}

%------------------------------------------------------------------------------
% Add your chapters here - don't forget to also add them to \includeonly above
% !TEX root = mythesis.tex

%==============================================================================
\chapter{Introduction}
\label{sec:SNSPD_Introduction}
%==============================================================================

Single photon detection is an essential part in nonlinear quantum optics with Rydberg atoms. 
Nonlinear quantum optics with Rydberg atoms is used for the study of fundamental light matter interactions in 
the HQO (Hybrid Quantum Optics) experiment at the University of Bonn and might result in innovations for promising
applications like optical quantum computing \cite{firstenberg-2016, gao-2011}.

In the HQO experiment properties of Rydberg atoms will be used to interact with an electromechanical acoustic oscillator \cite{}.
For this study of those interactions, it is required to place the Rydberg atoms with the electromechanical acoustic oscillator
in a 4K environment inside a cryostat (science chamber).
Inside the science chamber Rydberg atoms will be trapped over atom chip.
The goal is to excite Rydberg atoms and cool down the electromechanical acoustic oscillator to its ground state through the interaction with
the Rydberg atoms.



The challenge of detecting single photons is to translate the low energies (input variable - e.g.
for the optical/infrared range: $\text{E} \approx 0.5 - 3.3 \si{\eV}$) into measurable electrical signals (output variable).
Four central requirements have been established in the literature to quantitatively describe the quality of
EPD \cite{hadfield-2009, shalm_single-photon_2013}.


% !TEX root = mythesis.tex
\graphicspath{{/Users/maxim.re/Studium/Physik B.Sc./Semester_8_SS24/Proseminar/Figs Single Photon Detection/}}

%==============================================================================
\chapter{Working principle of a SNSPD}
\label{sec:SNSPD_working_principle}
%==============================================================================
This chapter will introduce the working principle of superconducting nanowire single photon detectors (SNSPD).
This chapter is structured in three parts.
The first part describes the essential detector elements and their functions.
The second part explains the detection process and provides its principles in terms of their phenomenological aspects.
For further details regarding the physics and mathematics behind superconducting nanowires, the reader is directed
to the work of Gol’tsman et al. \cite{goltsman-2001} and Hadfield et al. \cite{natarajan-2012}.
The third part will explain the reason why the detected photons need to be aligned to the slow axis of the optical fibre.

In general a SNSPD consists of four parts, as shown in figure \ref{fig:SNSPD_rough_structure}.
The most important part is detection area and consists of a superconducting nanowire ($\approx 100nm$ wide) on a sapphire base.
In general, superconductors have the property of abruptly losing all internal resistance once the temperature
$T_{\text{krit}}$ falls \textit{below} a critical temperature.
If the critical temperature is \textit{exceeded} (e.g. due to the environment or excessive current),
the resistance suddenly increases again.
Theoretically, this behaviour is described using Cooper pairs in the BCS theory \cite{bardeen-1957}.
A more detailed description of superconductors is omitted in this chapter.
To collect the entire output of the optical fibre, a pattern of a
thin superconducting film (such as niobium nitride) is shaped into a meandering nanowire by nanofabrication \cite{single-quantum-2022}.
Another element is the sapphire layer, which is used to dissipate the heat when the wire heats up.
Further, a gold contact supplies a bias current through the superconducting nanowire, and an optical fibre is coupled to the detection area.
To operate the system, this setup is cooled below the critical temperature of the superconductor to
2-3K and a DC current (Bias current $I_{B}$) is applied to the superconducting nanowire.

\begin{figure}[!hbt]
    \centering
    \includegraphics[width=0.6\linewidth]{figs/HQO_20240721_SNSPD_elementary_setup}
    \caption{Scematic illustration of the fundamental parts of a superconducting nanowire single photon detector based on \cite{steudle-2012}.
    Detector area is made of a superconducting nanowire and is contacted with gold from two sides.
    An optical fibre is coupled to the detector area.
    The fibre coupling realization of the characterized detector from Single Quantum is shown in figure \ref{fig:SNSPD_fiber_coupling}.}
    \label{fig:SNSPD_rough_structure}
\end{figure}

\FloatBarrier

In this second part, the detection process, shown in figure \ref{fig: SNSPD_process} is outlined.
Starting point (i) is the detector area in the superconducting state and an applied Bias current which is below the critical current.
Then, if photons hit the superconducting nanowire (ii), they break up individual Cooper pairs.
This leads to a local reduction of the critical current below the bias current and in turn to a localized area, where
the superconductivity is interrupted.
This local area forms the so-called \("\)hotspot\("\) (iii).
This hotspot forms a resistance area because the critical temperature is exceeded by the energy of the photon.
In response, the current flows around this hotspot (iv), whereby the local current density in the side areas next to the
hotspot again exceeds the critical current, due to a higher current density.
If the critical current is exceeded, the superconductivity also breaks down in these areas.
The excess also causes a resistance in the side channels of the nanowire (v).
This rapid increase in resistance can ultimately be measured in form of a voltage pulse, which can be seen in figure \ref{fig:SNSPD_single_voltage_pulse}.
The local non-superconducting area is then cooled down by the cryogenic environment and returns to the superconducting state
(iv—i).

\begin{figure}[!hbt]
    \centering
    \includegraphics[width=0.8\linewidth]{figs/HQO_20240725_detection_process_circle}
    \caption{Schematic detection process cycle of a superconducting nanowire single photon detector based on \cite{singlequantum_snsd_nodate}.
    The dotted innercircle shows the parts of the dead time $\tau_{\text{dead}}$ and the reset time $\tau_{\text{reset}}$
    The outer circle shows the parts of the detection process.
    In the upper left side the whole detector area is shown, the grey bars show the current flow in the zoomed superconducting nanowire.}
    \label{fig: SNSPD_process}
\end{figure}

\begin{figure}
    \centering
    \includegraphics[width=0.8\linewidth]{figs/HQO_20240725_analog_signal_31.2uA_tv_600mV}
    \caption{Single analog voltage pulse signal, recorded with a Lecroi oscilloscope with a time resolution of 500MHz.
    Voltage is depicted on the Y-axis and time in ns on the X-axis. A single photon steems from a faint laser source (780nm) attenuated to a photon rate of 1.772KHz.}
    \label{fig:SNSPD_single_voltage_pulse}
\end{figure}

In the detection process the photon polarization plays a crucial role.
Due to a polarization dependent absorption efficiency, it is important to consider the technical detail of the geometry of the meander.
In the used SNSPD, the meander design allows for a higher absorption efficiency if the E-field of the photons
is polarized parallel to the wire direction rather than orthogonally polarized \cite{single-quantum-2022}.
As depicted in figure \ref{fig:SNSPD_fiber_coupling} the slow axis of the coupled fibre in the characterized
detector is aligned parallel to the nanowire to maximize the absorption efficiency.
Other geometries are being investigated and might enable a high absorption efficiency independent of polarization \cite{zheng-2016}.

\begin{figure}
    \centering
    \includegraphics[width=\linewidth]{SNSPD_Meander_fiber_coupling}
    \caption{Fibre coupling set up to the superconducting nanowire detector area of the characterized detector from the Single Quantum \cite{singlequantum_snsd}.
    Fibre coupling is made in-house and is ajusted to maximize efficiency. It cannot be adjusted manually.}
    \label{fig:SNSPD_fiber_coupling}
\end{figure}

%Nur Technische Details die für die Charakterisierung relevant sind (max 2 Seiten)
%Physikalische Motivation (Warum frage sehr wichtig)
% !TEX root = mythesis.tex

%==============================================================================
\chapter{Faint laser source for detector characterization}
\label{sec:SNSPD_setup}
%==============================================================================

Counting photons one has to consider the characteristics of the emitter source as well.
Here I focus only on the characteristics of a laser source affecting the characterization I made.
For this first, I briefly sum up characteristics of a laser light source and its conditions it gives us for our detector
characterization.
Further, I make considerations regarding the inherent blindspots of laser light due to the randomness nature of photons
in coherent light.
Afterward, I introduce the setup I build for characterizing our SNSPD.\\

\section{Characteristics of faint laser sources}

Using a laser source enables us, considering the emitting light as monochromatic beam with angular frequency $\omega$
and constant Intensity I. The Photon flux of a laser is defined as the photon average number passing through
a cross-section in unit time:

\begin{align}
    \Phi = \frac{I A}{\hbar \omega} = \frac{P}{\hbar \omega} \text{photons $s^{-1}$}
\end{align}

where \textit{I} is the current of photon, \textit{A} the cross-section, \textit{P} the laser power and $\omega$ the angular frequency which
depends on the wavelength.

The average number of registered counts $N(T)$ for a given detection time T by a detector is given by:

\begin{align}
    \text{N(T)} = \text{T} \Phi \eta= \frac{P T \eta}{\hbar \omega} \text{photons}
\end{align}

and hence the registered counts $\mathbb{R}$ per unit time by:

\begin{align}
    \mathcal{R} = \frac{\text{N(T)}}{\text{T}} = \eta \Phi= \frac{P \eta}{\hbar \omega} \text{photons  $s^{-1}$}
\end{align}

where \textit{$\eta$} is the efficiency of the detector system.

This detection count rate is restricted by the largest amount on the dead time of the detector.

\begin{align}
    \mathcal{R_{\max}} \propto \frac{1}{\tau_{d}}
\end{align} \\

Because the used laser has a minimum threshold power which corresponds to a photonrate above the maximum one,
one has to attenuate the laser power in order to detect all events.

The photon statistic of coherent light, in our case (in reasonable approximation) of our laser light,
is given by poisson statistics.
This characteristic stems from discrete nature of photons and hence non-equidistant spacing between photons.
Measuring single photons, one has to ensure that we have a neglectable amount of photons in the segment
of the deadtime because else our light characteristics inherently forbid us measuring each of the incoming photons.\\

This is calculated by looking at the propability of measuring one photon per length segment, given by the deadtime.
First, we consider one Length segment given by the deadtime $\tau_{\text{d}}$ and the measurement time $\tau_{\text{m}}$:
\begin{align}
    L_{\text{d}} = \frac{c}{\tau_{\text{d}}}\\
    L_{\text{m}} = \frac{c}{\tau_{\text{m}}}
\end{align}

Through this, we can calculate for a given measurement time the average photon rate per length segment $L_{\text{m}}$
and further the probability of finding one photon in the particular line segment $L_{\text{d}}$:

\begin{align}
    \bar{n} = \Phi \frac{L_{\text{m}}}{c}\\
    p = \frac{\bar{n}}{\text{N}}
\end{align}

Where $N = \frac{c}{L_{\text{d}}}$ are the subsegments of the measured length segment $L_{\text{m}}$.

This enables us to calculate the probability p of finding n Photons per deadtime segment $L_{\text{d}}$ and
including this in our measurements.

\section{Experimental setup}

Though not measured by ourselves first, it is known from \textcolor{red}{Zitat} deadtime of the detector is around 20-25ns.
This gives us a theoretical maximum detection rate of $\mathcal{R_{\max}} \propto \frac{1}{\tau_{d}} = 25-50\si{M\Hz}$.
In order to realize the laser attenuation the following setup was build:

The first coupling of the laser light was done in order to operate with the beam on a lower stage, because the laserbeam 
was due to its construction on an uplifted stage. 
Afterward the beam passes a pbs to filter the horizontal polarized E field out. 
Further a galile telescope was build out of one focal and one diffusing lens for minimizing the beam width so it fits fully 
on the surface of the AOM crystal. The first order of the AOM was set for flexible voltage modulation of the laser.
A cover was used to filter out the first from the zeroth order of the AOM. Then a flip mount was placed, where 
Neutral density (ND) filers could be placed in and flexible placed in and out of the laser beam. 
The ND filters have the function to attenuate the laser light. 
At the end, before the laser light was again coupled in two waveplates where used to stabilize the light polarization regarding the
slow axis of the fibre. Afterwards the light was coupled back into a fibre, so it can be send to the detector.
It was important that the light was coupled in to a APC/PC to FC/PC optical fibre because the detector only had a FC/PC optical fibre
input, in order to maintain higher efficiency coupling in the light.\\

Besides, this optical setup had to be protected from environmental light. For this, the room where the setup was running was shielded 
with alu foil which has a reflection coefficient of almost 90$\%$ at the operating wavelenght of 780nm. 
Morverover a black box was build. It has the function to avoid further environmental light coupling into the fibre. Additionally, 
the optical fibre running from the optical setup to the detector was shielded with alu foil as well to avoid absoption from the optical 
fibre. 



%Relying on: https://nano-optics.physik.uni-siegen.de/education/teaching/lab_courses/sps-exp-manual.pdf

- Basics of photon distribution of laser, attenuation, poisson statistics \\
- Set up for laser attenuation \\
- Single Photon Detection paper \\




% !TEX root = mythesis.tex


%==============================================================================
\chapter{Characterization of a SNSPD by Single Quantum}
\label{sec:SNSPD_Characterization}
%==============================================================================

In literature, four central characteristics have emerged to quantify the quality of single photon detectors and
make their performance comparable \cite{natarajan-2012, hadfield-2009}.
These characteristics are the system detection efficiency $\eta_{\text{sde}}$, the dark count rate (DCR), the recovery time ($\tau_{\text{recovery}} = \tau_{\text{rec}}$)
and the timing jitter.
In this thesis, I focus on the detector efficiency $\eta_{\text{sde}}$, the dark count rate (DCR) and the recovery time ($\tau_{\text{rec}}$).
In general there are more than these introduced figures of merits, like after-pulsing but these will not be considered in this thesis.

\section{Dark Count Rate}\label{sec:dark-count-rate}
The DCR is the rate of measured detection events not intentionally sent from the source (here the faint laser source).
It is measured in counts per second and can be caused by statistical fluctuations in the measurement electronics.
A low DCR is important for a high signal-to-noise ratio and means easy interpretable results which are not distorted by noise \cite{wikipedia-contributors-2024}.

In the context of SNSPDs, the DCR is dependent on the bias current applied to the nanowire.
This is due to the fact that if the bias current approaches the critical current, less current $\Delta I= I_c - I_B$ is needed to exceed the critical current .
Therefore, electronic fluctuation, close to the critical current will cause a breakdown of the superconducting state and hence a dark count more often.
Furthermore, it is important to perform DCR measurements first in the characterization process because it determines the
limit, where general measurements are not distorted by high DCR noise.

\subsection*{Measurement and results}

In order to evaluate the DCR, it is necessary to perform measurements in two different setups.
First, a setup in which no optical fibre is connected to the detector and the port is covered.
In such a setup, it can be assumed that no photons from the surrounding environment are striking the detector.
This allows for the measurement of the DCR only triggered by the electronics noise depending on the detector's bias current and trigger voltage.

The required measurement setup consists out of the detector with the protection cap on the output port of the detector.
This configuration represents the most shielded environment from external light sources and serves as the reference
value for optimal DCR values achievable in single photon measurements.
The measurement was conducted by sweeping the bias current from 0 to 35$\si{\micro \A}$in 0.1$\si{\micro \A}$increments with an integration time of 200ms at each step

The second setup involves connecting the detector to the faint laser source of section \ref{sec:characteristics_faint laser sources}.
The laser source was turned off, so no photons from the source were sent to the detector.
This is done in order determine the DCR for consecutive measurements and improve the light shielding of the setup and the
optical fibre.
This allows one to find the optimal shielding configuration for the highest signal-to-noise ratio.
Once more, the bias current was swept from 0 to 35 µA in 0.1 µA increments with an integration time of the count rates for 200 ms.
In fig \subref{fig: DCR_black_box_to_exp} the results are shown for an optimized and a non optimized case.

At the initial, non optimized configuration an optical fibre was connected to the detector's output port and the
fibre output coupling of the experiment.
The laser source was turned off, so only electronic fluctuations and ambient light hitting the detector can cause detection events.
To reduce dark counts due to ambient light a black box was constructed that covers the laser setup.
Further, the fiber was wrapped in aluminum foil to prevent ambient light to couple to the core through the cladding.

As expected, the DCR rise in each case in \subref{fig: DCR_black_box_to_exp} with decreasing difference $ \Delta I= I_c - I_B$
due to raising probability that weak electronic noises trigger a signal.
The orange curve in \subref{fig: DCR_black_box_to_exp} demonstrates that in the absence of protection, a significant number of photons from the environment
are able to enter the detector through various potential pathways like the fibre cladding or the coupling connection to the laser setup.\\

In contrast, the measurement results with full shielding, depicted in figure \ref{fig: DCR_black_box_to_exp} as well,
show, that the DCR of the coupled and protected setup is the same as to the DCR with a cap on.
The peaks in the green curve at $\approx 3$ and $\approx 14 \si{\micro \A}$are artifacts resulting from some leakages in the protection.
Nevertheless, these leakages are not substantial when viewed in the context of the total photon count rate,
particularly when considering the anticipated photon rates from the faint weak laser source operating
in the high kHz and MHz frequency ranges.

Lastly, one can conclude from the investigation of the DCR of channel 1, that all further measurements have to be done
at a bias current of $ I_B < \approx 31.2\si{\micro \A}$
As mentioned above the final figure of merit for the DCR is depending on the bias current working point.
In this work, five 60s measurements were done for three different bias currents (24, 28 and 31.2$\si{\micro \A}$).

The averaged results for channel one of the detector yield a DCR of:

\begin{align}
    &DCR_{24\si{\micro \A}} = (5 \pm 0.0001) \si{\Hz} \\
    &DCR_{28\si{\micro \A}} = (4 \pm 0.0001) \si{\Hz} \\
    &DCR_{31.2\si{\micro \A}} = (3 \pm 0.0001) \si{\Hz}
\end{align}

\begin{figure}
    \centering
    \includegraphics[width=\linewidth]{figs/HQO_20240708_DCR_cap_on_Channel_1_Ba_thesis}
    \caption{Channel 1 DCR measurements for different bias currents at a trigger voltage of 200MHz.
    The blue curve shows the DCR with a cap on the output port of the detector.
    The orange curve shows the DCR with a fibre connected to the detector and the fibre output coupling of the experiment.
    The green curve shows the DCR with a fibre connected to the detector, the fibre output coupling of the experiment and alumni foil
    wrapped around the optical fibre.}
    \label{fig: DCR_black_box_to_exp}
\end{figure}

\FloatBarrier

\section{Recovery time}\label{sec:recovery-time}
The concept of the recovery time is visually depicted in fig \ref{fig:Recovery_time}.
When a photon hits the detector and is absorbed, the efficiency of the detector drops to zero and no further photons can
be measured for a certain period of time.
This elapsed time is called the dead time $\tau_{\text{dead}} = \tau_{\text{d}}$.
The efficiency then rises again to the original device efficiency $\eta_{G}$.
This period is called the reset time $\tau_{\text{reset}} = \tau_{\text{r}}$.
The vertical dashed line forms the starting point where the efficiency rises again to the original device efficiency $\eta_{G}$.
Finally, the sum $\tau_{\text{rec}} = \tau_{\text{r}} + \tau_{\text{d}}$ of both times forms the recovery time $\tau_{\text{rec}}$.

The recovery time is important because it determines the rate the detector can detect photons.
The lower the recovery time, as higher the counting rate.

\begin{figure}[H]
 \centering
 \includegraphics[width=0.8\textwidth]{figs/HQO_20240712_recoverytime_visualized_engl}
 \caption{Schematic efficiency curve for the detection of a photon\cite{shalm_single-photon_2013}. On the Y axis is the
 efficiency $\eta$, where $\eta_{G}$ is the device efficiency. On the X axis is the time course of the efficiency
 The trajectory of the initial device efficiency, represented by the variable , does not align with the illustration}.
 \label{fig:Recovery_time}
\end{figure}

\FloatBarrier

\subsection*{Measurement and results}
In this work, the recovery time of the detector is determined through an autocorrelation method based on a
continuous wave laser source (a faint laser source), a technique that has been previously employed by other research groups.
\cite{autebert-2020,miki-2017}.
The measurement was conducted with the setup shown in fig \ref{fig: recovery_time_setup}.
The raw analog signals from the detector were directly transmitted to a time tagger unit (Time Tagger 20)
by Swabian instruments.
with self-adjustable trigger voltages, a device deadtime of 6ns and a maximal counting rate of 9MHz.

\begin{figure}[H]
 \centering
 \includegraphics[width=0.8\textwidth]{figs/HQO_20240712_recoverytime_setup}
 \caption{Experimental setup for measuring the recovery time. Optical setup of "faint laser source in black box" is depicted in
 \ref{fig: faint_laser_source_full_set_up}}.
 \label{fig: recovery_time_setup}
\end{figure}

This unit enabled the tagging of incoming signals with a time tag, as implied by its name.
Subsequently, the tags were used to process the time distances between all signals.
The histogram of these distances provide an autocorrelation in time (here not normalized).
The autocorrelation was measured for one channel for four different bias currents (26$\si{\micro \A}$, 28$\si{\micro \A}$, 31$\si{\micro \A}$ and 31.2$\si{\micro \A}$) and
trigger voltages (300$\si{\milli \V}$, 400$\si{\milli \V}$, 500$\si{\milli \V}$ and 600$\si{\milli \V}$).

These measurements were done to determine the recovery time and analyze it dependencies.
The results for a fixed bias current of 31.2$\si{\micro \A}$ are shown in figure \ref{fig: recovery_time_measurement_31_2uA}
and for 26uA in \ref{fig: recovery_time_measurement_25uA}.
The other measurement results are presented in the appendix \ref{sec:Recovery time measurements}.

\begin{figure}[H]
  \begin{subfigure}[t]{.5\textwidth}
    \includegraphics[width=\linewidth]{figs/HQO_20240723_recovery_time_Channel_1_Bias_31_2uA_trigg_300-1000mV_thesis}
    \caption{}
    \label{fig: recovery_time_measurement_31_2uA}
  \end{subfigure}
  \hfill
  \begin{subfigure}[t]{.5\textwidth}
    \includegraphics[width=\linewidth]{figs/HQO_20240723_recovery_time_Channel_1_Bias_25uA_trigg_300-1000mV_thesis}
    \caption{}
    \label{fig: recovery_time_measurement_25uA}
  \end{subfigure}
  \caption{Autocorrelation of distances between two photon detection events for \subref*{fig: recovery_time_measurement_31_2uA} $I_{B} = 31.2uA$ and \subref{fig: recovery_time_measurement_25uA} $I_{B} = 25\si{\micro \A}$.
  The X-axis represents the time distance between two signals in 1ns steps and the Y-axis the counts per bin.}
  \label{fig: recovery_time_measurement_31_2_and_25uA}
\end{figure}

The results of the autocorrelation show three major features.
First, for low trigger voltages, the dead time is longer and decrease for increasing trigger voltages.
This is true for both, the lower and higher bias current.
The reason for this behavior can be explained best by looking at an exemplary raw analog signal (see fig \ref{fig: analog_signal_31_2_uA_double_peak})
of two consequent pulses.
In figure \ref{fig: analog_signal_31_2_uA_double_peak} one can see the peaks of two consecutive detection signals,
where the second pulse starts ($\approx 20ns$) before the falling edge of the first pulse ends.
Physically, that means, that before the first signal spike has fully decayed a second photon, already hit the detector,
got detected and produced a second spike.

If the trigger is below 500$\si{\milli \V}$ the time tagger will count this signal as one count, since the second pulse came when
the remaining voltage of the wire was above 500$\si{\milli \V}$.
If the trigger is above 500$\si{\milli \V}$ both pulses will be counted.
This allows successive events with smaller time delay between them and therefore reduces the perceived recovery time.


\begin{figure}
 \centering
 \includegraphics[width=0.5\textwidth]{~/sciebo/Bilder_BA_MaxR/HQO_20240710_Ana_signal_30uA_tv_600mV_double_single}
 \caption{Analog signal screenshot of an oscilloscope from channel 1 for a bias current of 31.2$\si{\micro \A}$and a trigger of 600$\si{\milli \V}$.
 X-axis: time in 50ns steps (straight vertical yellow lines), Y-axis voltage in 500$\si{\milli \V}$ steps (straight horizontal yellow lines)}
 \label{fig: analog_signal_31_2_uA_double_peak}
\end{figure}

Secondly,for the lower bias current (26$\si{\micro \A}$), the rising count curves for each trigger voltage converge earlier
in comparison to the bias current of 31.2 $\si{\micro \A}$.
At the bias current of 31.2 $\si{\micro \A}$, the four different count curves remain distinct until they reach their peak.
This can be attributed to the differing pulse heights, dependent on the bias current.
According to Ohm's law, for the same resistivity, a lower bias current corresponds to lower voltage pulses and vice versa.
Due to the lower pulse, the regime, where pulses can be resolved by a trigger voltage of 600$\si{\milli \V}$ but not 500$\si{\milli \V}$
becomes smaller.
The different pulse heights can also be verified by the recorded analog signals shown in figure:
\ref{fig: analog_signals_comparison}.

The third interesting feature is the count peak at 24-27ns for 31.2$\si{\micro \A}$ and a trigger level of 300 $\si{\milli \V}$
\ref{fig: recovery_time_measurement_31_2uA}.

This small peak is less visible at lower bias currents or higher trigger levels.
Closer examination of this trend are shown in the appendix \ref{sec:app}.

This behavior can be understood by taking into account that the bias current needs a finite amount of time to reach
its target value once the superconductivity is restored and will also overshoot a bit after reaching the target value.
A sketch of the expected behavior is shown in figure \ref{fig: Oscillating_bias_current}.

\begin{figure}
 \centering
 \includegraphics[width=0.8\textwidth]{~/sciebo/Bilder_BA_MaxR/HQO_20240710_oscilating_bias_current}
 \caption{A sketch of the assumed bias current behaviour is presented herewith. Once the bias current has been reached, the current undergoes a brief oscillation before stabilising at the bias current level.}
 \label{fig: Oscillating_bias_current}
\end{figure}

Following the detection of a current, the course of the current does not proceed directly and precisely to the bias current.
Instead, it oscillates for a brief period and then rapidly reaches equilibrium.

If the target bias current is close to the critical current the overshooting might cause a breaking of the
superconductivity leading to a time correlated increase in the dark current rate (seen in figure \ref{fig: DCR_black_box_to_exp}).
This feature is less severe for higher tigger voltages which suggests that the self triggered pulse is typically
of smaller height.

It can be concluded that the optimal recovery time is achieved when the detector is operated at a bias current maintained
at a level that is close but not excessive to critical current of 31.2$\si{\micro \A}$ and a trigger voltage of 600v.

This enables the generation of a higher pulse, which in turn results in a steeper reset time, a shorter dead time,
and consequently, a shorter recovery time.

\textcolor{red}{DISCLAIMER: ALL THE RESULTS AND METHODS ARE NOT FINALIZED AND WILL BE ADAPTED, IF METHOD IS AGREED}

Finally, a reasonable trigger point for a bias current of 31.2$\si{\micro \A}$ is 600$\si{\milli \V}$, which yields the lowest recovery time.
The calculation of the $\tau_{\text{rec}}$ is done, by calculating the average of counts per bin from 30$\si{\micro \A}$ till
the end of the measurement period.
Here, I choose 30$\si{\micro \A}$ as the starting point, because from this point a constant curve is visible (saturation point).
Furthermore, I fitted a function to estimate the point were $50\%$ and $90\%$ of the full efficiency is reached.
Moreover, the raise of the function for the dividing line between dead and reset time is calculated by the point where
the gradient of the fitting function is not zero any more.
The calculated points are visualized in fig: \ref{fig: recovery_time} and the final recovery time is
$\tau^{90\%}_{rec} = (21.21 \pm 0.34) ns$, where $\tau_{\text{d}} = (14.13 \pm 0.14) ns$ is the dead time and $\tau_{\text{r}} = (7.08 \pm 0.36) ns$  the reset time.
Moreover, the time the detector is back at efficiency of $\eta_{\text{sde}} = 50 \%$  is $\tau^{50\%}_{rec} = (18.21 \pm 0.24) ns$.

\begin{figure}
 \centering
 \includegraphics[width=0.5\textwidth]{figs/HQO_20240710_Deadtime_Channel_1_Bias_31_2uA_trig_600mV_thesis}
 \caption{Histogram of distances between signals for $I_{B} = 31.2uA$ and 600$\si{\milli \V}$. H- and v-lines indicate the dead-;
 reset- and recovery time}
 \label{fig: recovery_time}
\end{figure}

\FloatBarrier

\section{Efficiency}\label{sec:efficiency}
There are three types of efficiencies that describe independent loss processes in single photon detection.
An efficiency can be equated with the probability that a quantum mechanically process under consideration will occur.\\
These three efficiencies are the coupling efficiency ($\eta_{\text{K}}$),
the absorption efficiency ($\eta_{\text{A}}$) and the registration efficiency ($\eta_{\text{R}}$).
The graph \ref{fig: single_efficiency_terms} shows schematically where the different loss in the detection
process appears.
When a photon is sent to a detector via an optical fibre, not all photons can be coupled into the
fibre.
The probability of coupling is called the \textit{coupling efficiency}.
When photons hit the detector, there is always a probability that the photon will not be absorbed by the detector.
This is due to material and symmetry properties.
This is described by the \textit{absorption efficiency}.
Finally, there is always a probability that the photon will not be registered by the measuring electronics.
This is expressed with the \textit{Registration efficiency}.

\begin{figure}
    \centering
    \includegraphics[width=0.8\textwidth]{figs/HQO_20240712_systemd_detection_efficiency_visualized_engl}
    \caption{Sketch of the components in the detector setup where photonlosses appear and consequently a propability
     ($\eta_{\text{K}}$, $\eta_{\text{A}}$ or $\eta_{\text{R}}$) has to be considered.}
    \label{fig: single_efficiency_terms}
\end{figure}

In literature, these terms are summarized in two general efficiency terms: the device detection efficiency
($\eta_{\text{dde}} = \eta_{\text{A}} \cdot \eta_{\text{R}}$) and the system efficiency
($\eta_{\text{sde}} = \eta_{\text{A}} \cdot \eta_{\text{R}} \cdot \eta_{\text{K}}$) \cite{natarajan-2012}.
The device's efficiency $\eta_{\text{dde}}$ corresponds to the efficiency of the device itself, when photons are sent
to the detector in a free environment without any fibre coupling.
In this way only the efficiency of the elements detecting photons are considered.
The system detection efficiency $\eta_{\text{sde}}$ takes the coupling losses to the optical fibre into account.
This is the case if the detector is connected to a fibre, as the device properties or the experiment does not allow
photon detection in a free environment.\\

\subsection*{Measurement and results}

\textcolor{red}{DISCLAIMER: ALL THE RESULTS AND METHODS ARE NOT FINALIZED AND WILL BE ADAPTED, IF METHOD IS AGREED}

In the given setup, only the system detection efficiency c is measured, because the detector is
already prebuild with a fixed coupling to a fibre.
This internal fibre is connected to a fibre to fibre port (FC/PC to FC/PC).
Through this one, one can connect the detector with an external fibre and send photons from the experiment
to the detector.

The system detection efficiency $eta_{\text{sde}}$ is measured in different ways, each pointing out a different dependency.
Each measurement was done in the setup explained in part \ref{sec:SNSPD_setup}.
In addition, the order of the measurements is a relevant factor, as the conclusions drawn for one measurement
influence the preceding measurements.

In conclusion, it is first necessary to align the polarization of the laser light with the slow axis of the fibre
connected to the output port of the detector.
Along the manual the coupled light needs to polarized along the slow axis of the fibre \cite{manual_single_quantum_snspd}.
This is explained by the technical fact that only the slow axis of the fibre is coupled to the output port of
the detector.
The reason for this preselection of polarization is the related maximum $\eta_{A}$, explained
in part \ref{sec:SNSPD_working_principle}.\\

By adjusting the laser beam linear with a quarter-wave plate first and rotating the half-wave plate
in 10 degree steps afterwards, the polarization axis was rotated.
With this, it was possible to find the angle configuration were the maximum of light was coupled to the slow axis of the fibre.
This is important since measuring subsequent efficiency measurements aligned to a different axis would
put a systematic downshift on the true efficiency of the detector.\\
In the figures \ref{fig: angle_dependend_countrate_sde} the count rate and the resulting system detection efficiencies are depicted.
For preceding measurements the polarization was aligned to an angle, where a maximum of $eta_{\text{sde}} = 81.652 \pm 10.872$ was reached.

\begin{figure}
  \begin{subfigure}[t]{.5\textwidth}
    \includegraphics[width=\linewidth]{figs/HQO_2024011_countrate_angle_thesis}
    \caption{}
    \label{fig: angle_dependend_countrate}
  \end{subfigure}
  \hfill
  \begin{subfigure}[t]{.5\textwidth}
    \includegraphics[width=\linewidth]{figs/HQO_2024011_sde_angle_thesis}
    \caption{}
    \label{fig: angle_dependend_countrate_sde}
  \end{subfigure}
  \caption{\subref*{fig: angle_dependend_countrate} Angle dependent countrate \subref*{fig: angle_dependend_countrate_sde}
  Angle dependent system detection efficiency $eta_{\text{sde}}$}
\end{figure}

In a second measurement the bias current und trigger voltage dependency was investigated.
For this the bias current was swept from 0 to 35$\si{\micro \A}$in 0.1$\si{\micro \A}$steps and events within 200ms integration
time were counted.

\begin{figure}
    \centering
    \includegraphics[width=0.5\textwidth]{figs/HQO_20240711_sde_bias_current_tv_300_1000_thesis}
    \caption{System detection efficiency for different bias currents and trigger voltages}
    \label{fig: sde_bias_current_tv_300_1000}
\end{figure}

In fig \ref{fig: sde_bias_current_tv_300_1000} one can see that at lower trigger voltage of 300$\si{\milli \V}$ the count rate oscillates
a bit, which again corresponds likely to the increased dark count rates as explained in \ref{fig: Oscillating_bias_current}.
Furthermore, one can see that for lower trigger voltages the detected counts for lower bias currents are higher.
This is due to the consideration of lower voltage pulses in lower bias current regimes if the trigger voltage is low.
Hence, with a lower trigger voltage one can count already signals, however, with a very low system detection efficiency
$\eta_{\text{sde}}$.
Another behaviour is the saturation, which is reached by each trigger voltage configuration
at around $I_{Bias} \approx 20uA$ .
Here the maximum count rate is reached and the efficiency course continues constant without any gradient.
At the end at a bias current of  $I_{Bias} \approx 32uA$ the efficiency drops.
This is because the critical current is reached.
After this, along the \cite{manual_single_quantum_snspd}, the detector stops sending countable analog signals,
which corresponds to an efficiency of zero.

Finally, measurements for different count rates were done, to determine the bandwidth where the photons are detected with
the highest efficiency.
The count rates were varied by using different ND filtersa and was done by putting together different ND filter
combinations in order to get different OD values and therefore different count rates.
To avoid the oscilation of count rates near the critical current as seen by the blue course in fig \ref{fig: sde_bias_current_tv_300_1000}
a trigger voltage of 750$\si{\milli \V}$ is used.

\begin{figure}
  \begin{subfigure}[t]{.5\textwidth}
    \centering
    \includegraphics[width=\linewidth]{figs/HQO_20240711_sde_bias_current_750_thesis}
    \caption{}
    \label{fig: sde_bias_current_750_thesis}
    \end{subfigure}
  \hfill
  \begin{subfigure}[t]{.5\textwidth}
    \centering
    \includegraphics[width=\linewidth]{figs/HQO_20240711_sde_photon_rate_750_thesis}
    \caption{}
    \label{fig: sde_count_rates_750_thesis}
  \end{subfigure}
  \caption{\subref*{fig: sde_bias_current_750_thesis} Course of system detection efficiency for different bias current and count rates \subref*{fig: sde_count_rates_750_thesis} Course of system detection efficiency for different count rates}
\end{figure}

In fig \ref{fig: sde_bias_current_750_thesis} one can see that the system detection efficiency is decreasing with increasing
count rates.
Moreover, only for the measurement of 1.391MHz one gets a range of bias currents where the efficiency is constant.
All the other measurements with higher count rates show a continues increase in $\eta_{\text{sde}}$ up to the point the
detector shut down.
Furthermore, the shut-down happens earlier, when the count rate is higher.
This dynamics follows from the fact that the detector is not able to recover in time, when the count rate is too high.
Hence, the critical temperature is reached earlier, when the count rate is raising and the detections is shut down.

In fig \ref{fig: sde_count_rates_750_thesis} the maximum system detection efficiency for each count rate is shown.
One see clearly a downward trend for raising count rates.
Moreover, the measurement shows the saturating and therefore a maximal system detection efficiency of $(87.308 \pm 9.159) \%$
for channel 1. This $\eta_{\text{sde}}$ is reached at a count rate of $1.391 \pm 0.32$MHz.


\FloatBarrier

\section{Discussion}

%#TODO - Freitag
% durch lesen und verbessern
% Diskussion schreiben




%Influence of timing jitter on efficiency and recovery time.
%- No Afterpulsing
%- Temperature 2.9 instead of 2.5Í
%- Photon counters formel für NQO Excelitas SPCS - Note #7

% !TEX root = mythesis.tex

%==============================================================================
\chapter{Summary and Outlook}
\label{sec:Summary_and_Outlook}
%==============================================================================

In this thesis, both the quantitative verification of the company's specification data for one channel of the SNSPD
and the analysis of the dependencies on the bias current, trigger voltage, polarization and count rate of the SNSPD
were successfully completed.
The results of the system detection efficiency, dark count rate, recovery time,  dead time and maximum count rate are summarized in table \ref{tab:final_results_SNSPD_channel1}
and compared with the specification available \cite{tech_sheet_single_quantum}.

The basis for the characterization was a faint laser source setup.
This served as a source for measuring single photons and made measurements of SNSPD
characteristics possible.
For realizing a faint laser source set up a laser with a wavelength of $\lambda = 780\si{\nano \m}$ was used and
attenuated by ar ND filters.
For the precise measurement of the OD values of the filters, a calibration of the filters was done in two ways to consider statistical and systematic errors.
Moreover, the faint laser source set up was operated in a self-build optical enclosure (black box) to avoid
environmental light coupling into the fibre.\\

After the setup was built, first Dark count measurements were performed.
Different settings were investigated \ref{fig: DCR_black_box_to_exp} to find the optimal settings for the lowest Dark count rate.
The lowest dark count rate was achieved by operating the faint laser source in the black box and coating an aluminium
foil around the optical fibre connecting it to the SNSPD.
A dark count rate of $DCR_{31.2\si{\micro \A}} = (1.40 \pm 1.36) \si{\Hz}$ was achieved and yields the same low values as for
the case, where the detector has no connection to the experiment.
Afterwards the recovery, dead and reset time was determined with a time tagger unit (Time Tagger 20) by Swabian instruments.
With a common used autocorrelation evaluation \cite{autebert-2020,miki-2017} of the time distances between detected photons,
the three times were determined.
The results yield for the bias current of $I_{\text{B}} = 31.2\si{\micro \A}$ and
a trigger voltage of 600$\si{\milli \V}$ the following values:

\begin{itemize}
    \item Recovery time: $\tau_{\text{rec}} = (17.156 \pm 0.0445) \si{\nano \s}$
    \item Dead time: $t_{\text{dead}} =  (13.950 \pm 0.050) \si{\nano \s}$
    \item Reset time: $t_{\text{reset}} = (3.206 \pm 0.447) \si{\nano \s}$
\end{itemize}

The results are in the range of the values of the companies specification data \cite{tech_sheet_single_quantum} and would make the
detection of single photons with count rate up to  $(71.68 \pm 0.25)\si{\mega \Hz} $ possible.
Furthermore, in the evaluation of the recovery time three interesting features were observed.
First, for low trigger voltages, the dead time is longer and shortens for increasing trigger voltage up to a trigger voltage of
800$\si{\milli \V}$ (seen in figure \ref{fig: recovery_time_measurement_31_2uA}).
For higher voltages the dead time stays constant.
Second, for lower bias current (26$\si{\micro \A}$), the rising curves for each trigger voltage converge earlier in comparison
to the bias current of 31.2$\si{\micro \A}$.
Moreover, the curve back to full efficiency is steeper for the higher bias current 31.2$\si{\micro \A}$.
The third interesting feature is the additional "counts per bin" peak at (24-27 $\si{\nano \s}$) for all trigger voltages and
bias currents (seen in figures \ref{fig: recovery_time_measurement_25uA}, \ref{fig: recovery_time_measurement_31_2uA} and \ref{fig:Recovery_time}).
This peak might be explained by a brief overshooting of the bias current and therefore a resulting rise in dark counts
(as seen in figure \ref{fig: DCR_black_box_to_exp}) due to short time excess of the critical current.

Finally, the system detection efficiency was measured.
First the polarization dependency of the setup was analyzed and the optimal settings were set in order to achieve the highest efficiency (shown in figure \ref{fig: angle_dependend_countrate}).
After this, the trigger voltage and bias current dependency were evaluated.
It was found that the efficiency is stable and independent of the trigger voltage, for an input count rate up to
2.446$\si{\mega \Hz}$ (shown in figure \ref{fig:countrate_bias_current_tv_300_600_900_8_92} and \ref{fig:countrate_bias_current_tv_300_600_900_9_72}).
At the end, the relation to the count rate was investigated and a expected downward trend for higher count rates was observed and
a constant system detection efficiency in the range between $(89.833 \pm 9.447) \%$ and $(92.292 \pm 9.550) \%$ for count rates up to
9.575 $\si{\mega \Hz}$ was found \ref{fig:sde_count_rate_log_thesis}. \\

Next steps would be the investigations and characterizations of the other seven available channels.
Further, a standard procedure can be developed for accelerating the characterization process.
Moreover, when the characterization of all channels is finished, the SNSPD can be used to measure in a Hanbury Brown and
Twiss set up.
This set up can measure the second order correlation function $g^{(2)}(\tau)$ of the faint laser source and confirm
the characteristic constant curve for coherent light sources \cite{glauber-1963}.
Finally, the SNSPD can be used in the HQO experiment for measurements of strong nonlinearities in Rydberg physics.



% Uncomment the following command to get references per chapter.
% Put it inside the file or change \include to \input if you do not want the references
% on a separate page
% \printbibliography[heading=subbibliography]

%------------------------------------------------------------------------------
\appendix
% \part*{Appendix}
% Add your appendices here - don't forget to also add them to \includeonly above
%------------------------------------------------------------------------------
\chapter{Appendix}
\label{sec:app}
%------------------------------------------------------------------------------

\section{ND filter calibration}\label{sec:ND_filter_calibration}

To calculate the OD value of the ND Filters, we first calculate the Transmission value:

$T=\frac{P_{detected}}{P_{input}}$. 

For the Error calculation we have to consider two error sources.
Once the measurement uncertainty of the photodiode and the measurement uncertainty from reading the value from the Powermeter.
Since the uncertainty from the photodiode is always 3$\%$ of the measured value, it does not contribute to the error,
since the transmission is a division and this does not affect the overall relation.
However, reading of the value from the powermeter provides always a different error because the last displayed order of
magnitude of the displayed result on the powermeter fluctuates.

Its also important to mention that after each adjustment for the fiber coupling, it was not possible to get to the same
initial value.
Therefore, we always have different power values we compare, though the filtering process was done always with the same
amount of light power.
So we have to consider an error from the fiber coupling and optimization.
For this our $P_{input}$ value for the Transmission is in reality not the measured value we coupled in but
the averaged value with an standard deviation as error.
Also, this only applies for the first measurement method because we adjust there our fiber coupling.
In the second measurement we have a stable $P_{input}$

So for the transmission error we got:

\begin{align}
    \Delta T= \sqrt{\left(\frac{\Delta P_{detected}}{P_{input}}\right)^2 + \left(\frac{\Delta P_{input} P_{detected}}{ P_{input}^2}\right)^2}
\end{align}.

Where.

\begin{align}
    \Delta P_{Input} = \sqrt{(P^{Pow.meter read off}_{detected})^2 + (P^{STD, coupling}_{detected})^2}
\end{align}

And with the relation: $\text{OD}= \log(\frac{1}{T})$ we can calculate the error for the OD:
$\Delta OD = \frac{\Delta T}{\ln(10) \cdot T}$ 

Regarding the errors it is important to mention that the error of the photodiode is not relevant if we look at the error
of the transmission rate because its the same for the measurement before and after the ND Filters.

In general the measurement error of the photodiode is about 3$\%$ of the measured value \href{https://www.thorlabs.com/newgrouppage9.cfm?objectgroup_id=3328 Thorlabs - Powermeter specs}{Powermeter specs}.
This error I considered only in the calculation for the output power error:

Table with measurement values and results for Transmission value and OD Value

\begin{table}[!ht]
    \centering
    \resizebox{\columnwidth}{!}{%
    \label{tab:ND_filter_calibration}
    \begin{tabular}{|l|l|l|l|l|l|l|l|}
    \hline
        \textbf{ID} & \textbf{Expectetd OD Value} & \textbf{Measured T Value} & \textbf{T\_Value Error Systematic} & \textbf{T Value Error stat} & \textbf{T Value Error} & \textbf{Mean OD Value} & \textbf{OD Value Error} \\ \hline
        TP03337667 & 1,00 & 1,43E-01 & 67,697E-4 & 3,223E-03 & 7,50E-03 & 0,844 & 0,023 \\ \hline
        TP03337667 (2) & 1,00 & 1,47E-01 & 50,474E-04 & 3,073E-03 & 5,91E-03 & 0,832 & 0,017 \\ \hline
        TP03366490 & 3,00 & 8,26E-03 & 32,418E-05 & 1,882E-04 & 3,75E-04 & 2,083 & 0,020 \\ \hline
        TP03366490 (2) & 3,00 & 7,93E-03 & 52,291E-05 & 1,744E-04 & 5,51E-04 & 2,101 & 0,030 \\ \hline
        TP03275234 & 4,00 & 1,00E-03 & 5,318E-05 & 4,381E-05 & 6,89E-05 & 2,999 & 0,030 \\ \hline
        TP03312353 & 4,00 & 1,04E-03 & 7,142E-05 & 4,312E-05 & 8,34E-05 & 2,983 & 0,035 \\ \hline
        TP03271009 & 4,00 & 1,02E-03 & 4,842E-05 & 4,432E-05 & 6,56E-05 & 2,991 & 0,028 \\ \hline
        TP03275234 (2) & 4,00 & 1,03E-03 & 6,566E-05 & 4,303E-05 & 7,85E-05 & 2,986 & 0,033 \\ \hline
        TP03324728 & 5,00 & 1,80E-04 & 9,945E-06 & 3,843E-05 & 3,96E-05 & 3,744 & 0,095 \\ \hline
        TP03287742 & 5,00 & 1,78E-04 & 4,196E-06 & 3,731E-05 & 3,75E-05 & 3,749 & 0,092 \\ \hline
        TP03348187 (2) & 5,00 & 1,82E-04 & 4,506E-06 & 3,921E-05 & 3,95E-05 & 3,739 & 0,094 \\ \hline
        TP03348187 & 5,00 & 1,82E-04 & 5,792E-06 & 3,862E-05 & 3,91E-05 & 3,739 & 0,093 \\ \hline
    \end{tabular}%
    }
\end{table}


Calculating the error for stacked ND filters. 
The Error from

\begin{align}
    P_{input} = P^{initial}_{Input} \cdot 10^{-OD}
\end{align}
constructed out of the uncertainty of the Photodiode  $\Delta P^{initial}_{Input} = 0.03 \cdot P^{initial}_{Input} $ and the error of the ND Filters
\begin{align}
    \Leftarrow \Delta P_{input} = \sqrt{(\Delta P^{initial}_{Input} \cdot 10^{-OD})^2 + (P^{initial}_{Input} \cdot \log(10) \cdot \Delta OD)^2}
\end{align}.

The Error  $\Delta P_{detected} $ is reformed to a rate only in the order of a few hundred Herz and
is neglectable compared to the  $\Delta P_{input}$.


\section{Angle dependent countrate and $\eta_{\text{sde}}$}\label{sec:countrate_sde_results}

\begin{table}[!ht]
    \centering
    \begin{tabular}{|l|l|l|l|l|l|l|}
    \hline
        Input laser Power (W) & OD Filter (multiple Added) & OD Error & Expected Output (W) & Outputerror (Measurment uncertainty)(+-) (W) & Output error (OD and Measurement uncertainty combined) (W) & Output error (OD and Measurement uncertainty combined) (MHz) \\ \hline
        5.111E-04 & 9.574 & 0.135 & 1.363E-13 & 4.090E-15 & 1.835E-14 & 0.072 \\ \hline
        5.111E-04 & 9.574 & 0.135 & 1.363E-13 & 4.090E-15 & 1.835E-14 & 0.072 \\ \hline
        5.111E-04 & 9.574 & 0.135 & 1.363E-13 & 4.090E-15 & 1.835E-14 & 0.072 \\ \hline
        5.111E-04 & 9.574 & 0.135 & 1.363E-13 & 4.090E-15 & 1.835E-14 & 0.072 \\ \hline
        5.111E-04 & 9.574 & 0.135 & 1.363E-13 & 4.090E-15 & 1.835E-14 & 0.072 \\ \hline
        5.111E-04 & 9.574 & 0.135 & 1.363E-13 & 4.090E-15 & 1.835E-14 & 0.072 \\ \hline
        5.111E-04 & 9.574 & 0.135 & 1.363E-13 & 4.090E-15 & 1.835E-14 & 0.072 \\ \hline
        5.111E-04 & 9.574 & 0.135 & 1.363E-13 & 4.090E-15 & 1.835E-14 & 0.072 \\ \hline
        5.111E-04 & 9.574 & 0.135 & 1.363E-13 & 4.090E-15 & 1.835E-14 & 0.072 \\ \hline
        5.111E-04 & 9.574 & 0.135 & 1.363E-13 & 4.090E-15 & 1.835E-14 & 0.072 \\ \hline
        5.111E-04 & 9.574 & 0.135 & 1.363E-13 & 4.090E-15 & 1.835E-14 & 0.072 \\ \hline
        5.111E-04 & 9.574 & 0.135 & 1.363E-13 & 4.090E-15 & 1.835E-14 & 0.072 \\ \hline
        5.111E-04 & 9.574 & 0.135 & 1.363E-13 & 4.090E-15 & 1.835E-14 & 0.072 \\ \hline
        5.111E-04 & 9.574 & 0.135 & 1.363E-13 & 4.090E-15 & 1.835E-14 & 0.072 \\ \hline
        5.111E-04 & 9.574 & 0.135 & 1.363E-13 & 4.090E-15 & 1.835E-14 & 0.072 \\ \hline
        5.111E-04 & 9.574 & 0.135 & 1.363E-13 & 4.090E-15 & 1.835E-14 & 0.072 \\ \hline
        5.111E-04 & 9.574 & 0.135 & 1.363E-13 & 4.090E-15 & 1.835E-14 & 0.072 \\ \hline
        5.111E-04 & 9.574 & 0.135 & 1.363E-13 & 4.090E-15 & 1.835E-14 & 0.072 \\ \hline
        5.111E-04 & 9.574 & 0.135 & 1.363E-13 & 4.090E-15 & 1.835E-14 & 0.072 \\ \hline
        5.111E-04 & 9.574 & 0.135 & 1.363E-13 & 4.090E-15 & 1.835E-14 & 0.072 \\ \hline
        5.111E-04 & 9.574 & 0.135 & 1.363E-13 & 4.090E-15 & 1.835E-14 & 0.072 \\ \hline
    \end{tabular}
\end{table}

\begin{table}[!ht]
    \centering
    \begin{tabular}{|l|l|l|l|l|l|}
    \hline
        Input laser Power (W) & Output error (OD and Measurement uncertainty combined) (MHz) & Expected Photon count (MHz) & Measured Photon Count in MHZ & SDE  & SDE Error  \\ \hline
        5.111E-04 & 0.072 & 0.537 & 0.437 & 81.365 & 10.952 \\ \hline
        5.111E-04 & 0.072 & 0.537 & 0.418 & 77.827 & 10.476 \\ \hline
        5.111E-04 & 0.072 & 0.537 & 0.375 & 69.821 & 9.398 \\ \hline
        5.111E-04 & 0.072 & 0.537 & 0.330 & 61.443 & 8.271 \\ \hline
        5.111E-04 & 0.072 & 0.537 & 0.310 & 57.719 & 7.769 \\ \hline
        5.111E-04 & 0.072 & 0.537 & 0.314 & 58.463 & 7.870 \\ \hline
        5.111E-04 & 0.072 & 0.537 & 0.358 & 66.656 & 8.972 \\ \hline
        5.111E-04 & 0.072 & 0.537 & 0.402 & 74.848 & 10.075 \\ \hline
        5.111E-04 & 0.072 & 0.537 & 0.438 & 81.551 & 10.977 \\ \hline
        5.111E-04 & 0.072 & 0.537 & 0.440 & 81.923 & 11.028 \\ \hline
        5.111E-04 & 0.072 & 0.537 & 0.412 & 76.710 & 10.326 \\ \hline
        5.111E-04 & 0.072 & 0.537 & 0.367 & 68.332 & 9.198 \\ \hline
        5.111E-04 & 0.072 & 0.537 & 0.323 & 60.139 & 8.095 \\ \hline
        5.111E-04 & 0.072 & 0.537 & 0.308 & 57.346 & 7.719 \\ \hline
        5.111E-04 & 0.072 & 0.537 & 0.320 & 59.581 & 8.020 \\ \hline
        5.111E-04 & 0.072 & 0.537 & 0.351 & 65.352 & 8.797 \\ \hline
        5.111E-04 & 0.072 & 0.537 & 0.399 & 74.290 & 10.000 \\ \hline
        5.111E-04 & 0.072 & 0.537 & 0.430 & 80.061 & 10.777 \\ \hline
        5.111E-04 & 0.072 & 0.537 & 0.435 & 80.992 & 10.902 \\ \hline
        5.111E-04 & 0.072 & 0.537 & 0.404 & 75.221 & 10.125 \\ \hline
        5.111E-04 & 0.072 & 0.537 & 0.353 & 65.725 & 8.847 \\ \hline
    \end{tabular}
\end{table}


\section{Recovery time measurements}\label{sec:Recovery time measurements}

\subsection{Recovery time measurements - Oscillating bias current} \label{subsec:recovery-time-measurements---oscillating_bias_current}
\subsection{Polarization alignment for system detection efficiency measurements}\label{subsec:polarization_alignment_for_system_detection_efficiency_measurments}

%{| class="wikitable"
%! Input Laser Power (W)  !! Wavelenght (nm)
%|-
%| 5,39E-04 || 780
%|}
%
%{| class="wikitable"
%! nr. !! ID !! Laser Power before (W) !! Laser Power before Mean (W) !! Laser Power before Std (W) !! Laser Power before Error (W) !! Measured Laser Power after (W) !! Measured T Value !! T Value Error !! Expectetd OD Value !! Measured OD Value !! OD Value Error !! Deviation in %
%|-
%| 1 || TP03337667 || 5,1170E-04 || 5,2670E-04 || 1,03621E-05 || 1,05534E-05 || 7,8840E-05 || 1,50E-01 || 3,18E-03 || 1,00 || 0,8248 || 0,0092 || 17,52
%|-
%| 2 || TP03337667 (2) || 5,3000E-04 || 5,2670E-04 || 1,03621E-05 || 1,05534E-05 || 8,0240E-05 || 1,52E-01 || 3,01E-03 || 1,00 || 0,8172 || 0,0086 || 18,28
%|-
%| 3 || TP03366490 || 5,1300E-04 || 5,2670E-04 || 1,03621E-05 || 1,05534E-05 || 4,5210E-06 || 8,58E-03 || 1,85E-04 || 3,00 || 2,0663 || 0,0094 || 93,37
%|-
%| 4 || TP03366490 (2) || 5,3000E-04 || 5,2670E-04 || 1,03621E-05 || 1,05534E-05 || 4,4500E-06 || 8,45E-03 || 1,71E-04 || 3,00 || 2,0732 || 0,0088 || 92,68
%|-
%| 5 || TP03275234 || 5,2600E-04 || 5,2670E-04 || 1,03621E-05 || 1,05534E-05 || 5,5500E-07 || 1,05E-03 || 4,35E-05 || 4,00 || 2,9773 || 0,0179 || 102,27
%|-
%| 6 || TP03312353 || 5,3900E-04 || 5,2670E-04 || 1,03621E-05 || 1,05534E-05 || 5,8500E-07 || 1,11E-03 || 4,28E-05 || 4,00 || 2,9544 || 0,0167 || 104,56
%|-
%| 7 || TP03271009 || 5,2300E-04 || 5,2670E-04 || 1,03621E-05 || 1,05534E-05 || 5,6220E-07 || 1,07E-03 || 4,40E-05 || 4,00 || 2,9717 || 0,0179 || 102,83
%|-
%| 8 || TP03275234 (2) || 5,3900E-04 || 5,2670E-04 || 1,03621E-05 || 1,05534E-05 || 5,7800E-07 || 1,10E-03 || 4,26E-05 || 4,00 || 2,9596 || 0,0169 || 104,04
%|-
%| 9 || TP03324728 || 5,2400E-04 || 5,2670E-04 || 1,03621E-05 || 1,05534E-05 || 1,0030E-07 || 1,90E-04 || 3,84E-05 || 5,00 || 3,7203 || 0,0875 || 127,97
%|-
%| 10 || TP03287742 || 5,3900E-04 || 5,2670E-04 || 1,03621E-05 || 1,05534E-05 || 9,6000E-08 || 1,82E-04 || 3,73E-05 || 5,00 || 3,7393 || 0,0888 || 126,07
%|-
%| 11 || TP03348187 (2) || 5,1300E-04 || 5,2670E-04 || 1,03621E-05 || 1,05534E-05 || 9,8370E-08 || 1,87E-04 || 3,92E-05 || 5,00 || 3,7287 || 0,0911 || 127,13
%|-
%| 12 || TP03348187 || 5,2040E-04 || 5,2670E-04 || 1,03621E-05 || 1,05534E-05 || 9,9150E-08 || 1,88E-04 || 3,86E-05 || 5,00 || 3,7253 || 0,0891 || 127,47
%|-
%| 13 || TP03312353 + TP03275234 (2) || 5,3900E-04 || 5,2670E-04 || 1,03621E-05 || 1,05534E-05 || 1,1170E-09 || 2,12E-06 || 3,71E-05 || 8,00 || 5,6735 || 0,0238 || 232,65
%|}


%{| class="wikitable"
%! nr. !! ID !! Laser Power before (W) !! Measured Laser Power after (W) !! Measured T Value !! T Value Error !! Expectetd OD Value !! Measured OD Value !! OD Value Error !! Deviation in %
%|-
%| 1 || TP03337667 || 5,13E-04 || 6,98E-05 || 0,1361 || 5,32E-04 || 1,00 || 0,8660 || 0,0017 || 13,40
%|-
%| 2 || TP03337667 (2) || 5,13E-04 || 7,30E-05 || 0,1422 || 5,56E-04 || 1,00 || 0,8469 || 0,0017 || 15,31
%|-
%| 3 || TP03366490 || 5,13E-04 || 4,07E-06 || 0,0079 || 3,12E-05 || 3,00 || 2,1004 || 0,0017 || 89,96
%|-
%| 4 || TP03366490 (2) || 5,13E-04 || 3,80E-06 || 0,0074 || 2,91E-05 || 3,00 || 2,1306 || 0,0017 || 86,94
%|-
%| 5 || TP03275234 || 5,13E-04 || 4,86E-07 || 0,0009 || 5,37E-06 || 4,00 || 3,0235 || 0,0025 || 97,65
%|-
%| 6 || TP03312353 || 5,13E-04 || 4,96E-07 || 0,0010 || 5,43E-06 || 4,00 || 3,0142 || 0,0024 || 98,58
%|-
%| 7 || TP03271009 || 5,13E-04 || 4,98E-07 || 0,0010 || 5,43E-06 || 4,00 || 3,0130 || 0,0024 || 98,70
%|-
%| 8 || TP03275234 (2) || 5,13E-04 || 4,96E-07 || 0,0010 || 5,42E-06 || 4,00 || 3,0150 || 0,0024 || 98,50
%|-
%| 9 || TP03324728 || 5,13E-04 || 8,75E-08 || 0,0002 || 7,71E-07 || 5,00 || 3,7682 || 0,0020 || 123,18
%|-
%| 10 || TP03287742 || 5,13E-04 || 8,92E-08 || 0,0002 || 7,82E-07 || 5,00 || 3,7598 || 0,0020 || 124,02
%|-
%| 11 || TP03348187 (2) || 5,13E-04 || 9,12E-08 || 0,0002 || 7,95E-07 || 5,00 || 3,7502 || 0,0019 || 124,98
%|-
%| 12 || TP03348187 || 5,13E-04 || 9,06E-08 || 0,0002 || 7,92E-07 || 5,00 || 3,7529 || 0,0019 || 124,71
%|}

\textbf{Method 1 and 2 combined:} In both tables above one can see that the second method has slightly higher values.
This indicates a systematic error we dont know.
Till now, we only considered statistical errors.
Now if we want to calculate the true value we have to consider the systematic error as well.
Here we determine the systematic error as half of the difference between the values from method one and
method two $\Delta T_{syst} = \frac{T_{meth_2} - T_{meth_1}}{2}$  .
So the final transmission error of the combined measurements is:
$\Delta T = \sqrt{\Delta T_{stat}^2 + \Delta T_{syst}^2}$  

%{| class="wikitable"
%! ID !! Expectetd OD Value !! Measured T Value !! T_Value Error Systematic !! T Value Error stat !! T Value Error !! Mean OD Value !! OD Value Error
%|-
%| TP03337667 || 1,00 || 1,43E-01 || 0,006769666 || 3,22E-03 || 7,50E-03 || 0,8449 || 0,0228
%|-
%| TP03337667 (2) || 1,00 || 1,47E-01 || 0,005047419 || 3,07E-03 || 5,91E-03 || 0,8318 || 0,0174
%|-
%| TP03366490 || 3,00 || 8,26E-03 || 0,000324182 || 1,88E-04 || 3,75E-04 || 2,0830 || 0,0197
%|-
%| TP03366490 (2) || 3,00 || 7,93E-03 || 0,000522915 || 1,74E-04 || 5,51E-04 || 2,1010 || 0,0302
%|-
%| TP03275234 || 4,00 || 1,00E-03 || 5,31863E-05 || 4,38E-05 || 6,89E-05 || 2,9998 || 0,0299
%|-
%| TP03312353 || 4,00 || 1,04E-03 || 7,14296E-05 || 4,31E-05 || 8,34E-05 || 2,9833 || 0,0349
%|-
%| TP03271009 || 4,00 || 1,02E-03 || 4,84206E-05 || 4,43E-05 || 6,56E-05 || 2,9918 || 0,0280
%|-
%| TP03275234 (2) || 4,00 || 1,03E-03 || 6,56618E-05 || 4,30E-05 || 7,85E-05 || 2,9864 || 0,0330
%|-
%| TP03324728 || 5,00 || 1,80E-04 || 9,94546E-06 || 3,84E-05 || 3,96E-05 || 3,7436 || 0,0954
%|-
%| TP03287742 || 5,00 || 1,78E-04 || 4,19645E-06 || 3,73E-05 || 3,75E-05 || 3,7494 || 0,0915
%|-
%| TP03348187 (2) || 5,00 || 1,82E-04 || 4,50636E-06 || 3,92E-05 || 3,95E-05 || 3,7393 || 0,0940
%|-
%| TP03348187 || 5,00 || 1,82E-04 || 5,79273E-06 || 3,86E-05 || 3,91E-05 || 3,7388 || 0,0930
%|}

%This values kind of closely match with the values from thorlabs. 
%[[Image:HQO_20240625_Photodiode_literature_Thorlabs.png|400px]] [[Image:HQO_20240624_ND_Filter_literature_Thorlabs.png|400px]]  

In the appendix you usually include extra information that should be
documented in your thesis, but not interrupt the flow.


% \printbibliography[heading=subbibliography]

%------------------------------------------------------------------------------
% Use biblatex for the bibliography
% Add bibliography to Table of Contents
% Comment out this command if your references are printed for each chapter.
\printbibliography[heading=bibintoc]
%------------------------------------------------------------------------------
% Include the following lines and comment out \printbibliography if
% you use BiBTeX for the bibliography.
% If you use biblatex package the files should be specified in the preamble.
% \KOMAoptions{toc=bibliography}
% {\raggedright
%   \bibliographystyle{../refs/atlasBibStyleWithTitle.bst}
%   % \bibliographystyle{unsrt}
%   \bibliography{./thesis_refs,../refs/standard_refs-bibtex}
% }

%------------------------------------------------------------------------------
% Declare lists of figures and tables and acknowledgements as backmatter
% Chapter/section numbers are turned off
\backmatter


%------------------------------------------------------------------------------
% Print the glossary and list of acronyms
% \printglossaries

%------------------------------------------------------------------------------
% You could instead add your acknowledgements here - don't forget to
% also add them to \includeonly above
% %------------------------------------------------------------------------------
\chapter*{Acknowledgements}
\label{sec:ack}
%------------------------------------------------------------------------------

I would like to thank ...

You should probably use \texttt{\textbackslash chapter*} for
acknowledgements at the beginning of a thesis and
\texttt{\textbackslash chapter} for the end.

%%% Local Variables: 
%%% mode: latex
%%% TeX-master: "../mythesis"
%%% End: 


\end{document}
