% !TEX root = mythesis.tex

%==============================================================================
\chapter{Faint laser source for detector characterization}
\label{sec:SNSPD_setup}
%==============================================================================

In order to characterize a detector, it is necessary to consider not only the characteristics of the detector itself,
but also those of the emitter source.
This section will focus on the characteristics of the laser setup, which serves as a light source for the detector
characterization.
In order to provide a brief overview of the characteristics of a coherent laser light source and the conditions it gives
for detector characterization, I will first present a summary of the relevant information.
Furthermore, I will introduce the setup that I have constructed for characterizing the SNSPD.


\section{Characteristics of faint laser sources}\label{sec:characteristics_faint laser sources}

%Question: How monochromatic is the laser light?

Using a laser source enables us considering the emitting light as monochromatic beam with angular frequency $\omega$
and constant Intensity I. The Photon flux of a laser is defined as the photon average number passing through
a cross-section in one unit time:

\begin{align}
    \Phi = \frac{\text{I} \text{A}}{\hbar \omega} = \frac{\text{P}}{\hbar \omega} \text{photons $s^{-1}$}
\end{align}

where \textit{I} is the current of photon, \textit{A} the cross-section, \textit{P} the laser power and $\omega = \frac{2 \pi c}{\lambda}$ the angular frequency which
depends on the wavelength.
The average number of registered counts N(T) for a given detection time T by a detector is given by:

\begin{align}
    \text{N(T)} = \text{T} \Phi \eta= \frac{\text{P T} \eta}{\hbar \omega} \text{photons}
\end{align}

and hence the registered counts $\mathcal{R}$ per unit time is described by:

\begin{align}
    \mathcal{R} = \frac{\text{N(T)}}{\text{T}} = \eta \Phi= \frac{P \eta}{\hbar \omega} \text{photons  $s^{-1}$}
\end{align}

where \textit{$\eta$} is the efficiency of the detector system which is described in more detail in section \ref{sec:efficiency}.

The detector has a maximal detection count rate, and it is restricted by the recovery time ($\tau_{\text{recovery}} = \tau_{\text{rec}}$)\ref{sec:recovery-time} of the detector
$\mathcal{R_{\max}} \propto \frac{1}{\tau_{\text{rec}}}$.

Sending too many photons at once to the detector causes latching and prevents counting \cite{single-quantum-2022}.
Therefore, the photon rate send to the detector has to be below the maximal detection count rate of the detector.
As consequence for this experiment, the laser power has to be attenuated to a level where the photon rate is below the
maximal detection count rate of the detector.

The photon statistic of coherent light, in our case (in reasonable approximation) of our laser light,
is given by poisson statistics.

This property originates from the discrete nature of photons, which are randomly distributed with non-equidistant
spacing between each other.

Considering the fact, that the detector is not able to detect photons within the deadtime and with significant lower efficiency
in the reset time, one has to look at the probability of photons in those blind spots to include this in later efficiency evaluation.

The amount of photons is calculated by looking at the probability measuring one photon per length segment, given by
the deadtime and the reset time.
First, we consider one Length segment given by the deadtime $\tau_{\text{d}}$:

\begin{align}
    L_{\text{d}} = c \cdot \tau_{\text{d}} \\
\end{align}

With this, we can calculate for a given measurement time the average photon rate $\bar{n}$ per length segment $L_{\text{d}}$

\begin{align}
    \bar{n} = \Phi \frac{L_{\text{d}}}{c} = \Phi \cdot \tau_{\text{d}}\\
\end{align}


This enables us to calculate the probability, given by the poisson distribution,

\begin{align}
    \text{P}_{\text{dead}}(n) = \frac{(\bar{n})^n}{n!} e^{-\bar{n}}
\end{align}

of finding n Photons within the length segment, defined by the deadtime $\tau_{\text{d}}$.

In accordance with the specified probability, the quantity of photons that are inherently unmeasurable due to the
constraints of the detector can be calculated.


\section{Experimental setup}\label{sec:experimental_setup}

From factory details it is known that the recovery time of the detector is located between 17 and 23ns, depending on the
detector settings.
This details can be confirmed by our measurements as well \ref{sec:recovery-time}.
This gives a theoretical maximum detection rate with constant maximal possible efficiency of $\mathcal{R_{\max}} \propto \frac{1}{\tau_{d}} \approx 43-58\si{M\Hz}$.
However, this rate would only hold, if the emitter would be a true single photon source with a consistent emitting rate, where
photons are equidistantly spaced.
Since we use a coherent laser source the counting rate is significantly lower than the theoretical maximum detection rate.
From the factory we got specification of about 2-4 MHz as detection count rate with maximal detection efficiency \textcolor{red}{Source}.

% Ist die theoretisch mögliche count rate nur durch den Fakt, dass wir keine EPD haben limitiert?
% Das nur dewegen die Count rate so viel niedriger ist?

The initial situation is that the minimum laser power of the laser source is higher than the maximal detection count rate of the detector.
The reason for this is that the laser source does only lase with a constant power with a certain minimum power.
Therefore, the laser power can not be regulated down by adjusting the input power of the laser, but has to be attenuated
by neutral density filters afterwards.

In order to realize the laser attenuation the setup shown in fig \ref{fig: faint_laser_source_full_set_up} was build.
The first coupling of the laser light at the upper stage was done in order to operate with the beam on the lower stage, because the laserbeam
was due to its construction on an uplifted stage. 
Afterwards, on the lower stage the beam passes a half-wave plate $\lambda_2$ and polarization beam splitter (pbs) to filter the horizontal
polarized E-field out.
Further, a galilei telescope was build ($l_1$ and $l_2$) out of one focal and one diffusing lens for reducing the beam width, so it fits fully
on the surface of the crystal of the Acousto-optic modulator (AOM).
The first order of the AOM was used for flexible voltage modulation of the laser.
A cover was used to filter out the first from the zeroth order of the AOM.
Then a flip mount was placed, where Neutral density (ND) filers could be placed in and flexible placed in and out of the laser beam.
The ND filters have the function to attenuate the laser light. 
At the end, before the laser light was again coupled in, again a half-wave plate $\lambda_2$ and quarter-wave plate $\lambda_4$ were used to
stabilize and control the light polarization coupling in the fibre.
As mentioned in \ref{sec:SNSPD_working_principle} the light was polarized according to the slow axis of the fibre.
Afterwards the light was coupled back into a fibre, so it was directed to the detector.
Technically, it was important coupling to a APC/PC to FC/PC optical fibre because the detector only had
an FC/PC optical fibre input port, in order to maintain higher efficiency coupling in the light \cite{single-quantum-2022}.\\

Besides, this optical setup had to be protected from environmental light.
For this, the room where the setup was running was shielded
with alu foil which has a reflection coefficient of almost 90$\%$ at the operating wavelength of 780nm \textcolor{red}{Source}.
Moreover, a black box was build.
It has the function to avoid further environmental light coupling into the fibre.
Additionally, the optical fibre running from the optical setup to the detector was shielded with alu foil as well to
avoid absorption from the optical fibre. \\

\begin{figure}
    \centering
    \includegraphics[width=0.8\textwidth]{figs/HQO_20240712_faint_laser_source_full_set_up_thesis}
    \caption{Set up for attenuation of a 780nm laser source from the Russian company \href{http://vitawave.ru/eng/}{Vitawave}}
    \label{fig: faint_laser_source_full_set_up}
\end{figure}


Based on those theoretical considerations and our experimental setup, one can approach the central goal to determine the
amount of photons and with this, determine the efficiency of the detector.

Central, in order to do efficiency measurements, on the one hand one must send only a few photons (low power) and one must
exactly know how many photons are send.
For this we need to measure the power of the laser light, we are sending to the detector.
However, the desired light is so weak, we cannot measure it with power-meters or any other measurement devices available to us.
So our photon rate we are expecting to send to the detector depends on the ND filters and their optical density (OD) value, since it is the sole factor
we can adjust our laser power (here transmission)($\text{T} = 10^{-\text{OD}}$), hence our amount of photons we send to the detector.\\

The challenge relying on ND filters is that the fabric values of their OD are not precise, hence the filters need to be calibrated by oneself and 
as precise as possible, to get accurate and stable OD values for the ND filters.

To get accurate results, the OD value of each filter is measured with two methods to reduce systematic errors.
The first method measures the OD values, where the ND filters are set in the flip mount as shown in graph \ref{fig: faint_laser_source_full_set_up}.
The second method measures the OD values after the fibre coupling directly in front of the powermeter
outside the blackbox to avoid straying light from the elements in the setup.

The attenuation of ND filters is quantized via the OD value and is connected to the transmission value $ \text{T} = 10^{(\text{OD}}$ of the outgoing light.
Due to the logarithmic definition of the OD value, the OD values of the ND filters are added up when they are stacked on each other.

The measurement of the OD values was done by sending light on the filter and measuring the power with and without the filter.
Moreover, only single ND filter measurements were in the measurable range, since the powermeter was not able 
to measure the power of the attenuated laser light by several stacked ND filters. 

The OD value is then calculated by the logarithmic value of the inverse transmission value $\log_{10}(\frac{1}{\text{T}})$ .
The transmission value is given by the proportion of the power with the filter compared without the filter $T = \frac{P_{\text{out}}}{P_{\text{in}}}$.
Afterwards, the final OD values of each method were combined and the corresponding systematic and statistical errors are
considered in order to get accurate OD values for the ND filters.
Results of the measurements and error calculations can be found in section \ref{sec:ND_filter_calibration}.

Based one these OD values the photon rate was determined.

%uncertainties package (python)





%Relying on: https://nano-optics.physik.uni-siegen.de/education/teaching/lab_courses/sps-exp-manual.pdf

%- Basics of photon distribution of laser, attenuation, poisson statistics \\
%- Set up for laser attenuation \\
%- Single Photon Detection paper \\
%
%Second, one single ND filter cannot attenuate the laser power enough and therefore one has to stack several ND
%filters on each other, since their  values are adding up.

\FloatBarrier
%==============================================================================

