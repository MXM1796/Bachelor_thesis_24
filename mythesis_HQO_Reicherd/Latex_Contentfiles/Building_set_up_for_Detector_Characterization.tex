% !TEX root = mythesis.tex

%==============================================================================
\chapter{Faint laser source for detector characterization}
\label{sec:SNSPD_setup}
%==============================================================================

Counting photons one has to consider the characteristics of the emitter source as well.
Here I focus only on the characteristics of a laser source affecting the characterization I made.
For this first, I briefly sum up characteristics of a laser light source and its conditions it gives us for our detector
characterization.
Further, I make considerations regarding the inherent blindspots of laser light due to the randomness nature of photons
in coherent light.
Afterward, I introduce the setup I build for characterizing our SNSPD.\\

\section*{Characteristics of faint laser sources}

Using a laser source enables us, considering the emitting light as monochromatic beam with angular frequency $\omega$
and constant Intensity I. The Photon flux of a laser is defined as the photon average number passing through
a cross-section in unit time:

\begin{align}
    \Phi = \frac{I A}{\hbar \omega} = \frac{P}{\hbar \omega} \text{photons $s^{-1}$}
\end{align}

where \textit{I} is the current of photon, \textit{A} the cross-section, \textit{P} the laser power and $\omega$ the angular frequency which
depends on the wavelength.

The average number of registered counts $N(T)$ for a given detection time T by a detector is given by:

\begin{align}
    \text{N(T)} = \text{T} \Phi \eta= \frac{P T \eta}{\hbar \omega} \text{photons}
\end{align}

and hence the registered counts $\mathbb{R}$ per unit time by:

\begin{align}
    \mathcal{R} = \frac{\text{N(T)}}{\text{T}} = \eta \Phi= \frac{P \eta}{\hbar \omega} \text{photons  $s^{-1}$}
\end{align}

where \textit{$\eta$} is the efficiency of the detector system.

This detection count rate is restricted by the largest amount on the dead time of the detector.

\begin{align}
    \mathcal{R_{\max}} \propto \frac{1}{\tau_{d}}
\end{align} \\

Because the used laser has a minimum threshold power which corresponds to a photonrate above the maximum one,
one has to attenuate the laser power in order to detect all events.

The photon statistic of coherent light, in our case (in reasonable approximation) of our laser light,
is given by poisson statistics.
This characteristic stems from discrete nature of photons and hence non-equidistant spacing between photons.
Measuring single photons, one has to ensure that we have a neglectable amount of photons in the segment
of the deadtime because else our light characteristics inherently forbid us measuring each of the incoming photons.\\

This is calculated by looking at the propability of measuring one photon per length segment, given by the deadtime.
First, we consider one Length segment given by the deadtime $\tau_{\text{d}}$ and the measurement time $\tau_{\text{m}}$:
\begin{align}
    L_{\text{d}} = \frac{c}{\tau_{\text{d}}}\\
    L_{\text{m}} = \frac{c}{\tau_{\text{m}}}
\end{align}

Through this, we can calculate for a given measurement time the average photon rate per length segment $L_{\text{m}}$
and further the probability of finding one photon in the particular line segment $L_{\text{d}}$:

\begin{align}
    \bar{n} = \Phi \frac{L_{\text{m}}}{c}\\
    p = \frac{\bar{n}}{\text{N}}
\end{align}

Where $N = \frac{c}{L_{\text{d}}}$ are the subsegments of the measured length segment $L_{\text{m}}$.

This enables us to calculate the probability p of finding n Photons per deadtime segment $L_{\text{d}}$ and
including this in our measurements.

\section*{Experimental setup}

Though not measured by ourselves first, it is known from \textcolor{red}{Zitat} deadtime of the detector is around 20-25ns.
This gives us a theoretical maximum detection rate of $\mathcal{R_{\max}} \propto \frac{1}{\tau_{d}} = 25-50\si{M\Hz}$.
In order to realize this laser attenuation the following set was build:




%Relying on: https://nano-optics.physik.uni-siegen.de/education/teaching/lab_courses/sps-exp-manual.pdf

- Basics of photon distribution of laser, attenuation, poisson statistics \\
- Set up for laser attenuation \\
- Single Photon Detection paper \\



