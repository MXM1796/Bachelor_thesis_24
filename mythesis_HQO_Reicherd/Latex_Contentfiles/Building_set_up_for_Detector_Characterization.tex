% !TEX root = mythesis.tex

%==============================================================================
\chapter{Faint laser source for detector characterization}
\label{sec:SNSPD_setup}
%==============================================================================

In order to characterize a detector, it is necessary to consider not only the characteristics of the detector itself,
but also those of the emitter source.
This section will focus on the characteristics of the laser setup, which serves as a coherent light source for the detector
characterization.
A brief overview of the characteristics of a coherent laser light source and the conditions it gives
for detector characterization will be provided.
Furthermore, I will introduce the setup of the faint laser source that I have constructed for characterizing the SNSPD.

\section{Characteristics of faint laser sources}\label{sec:characteristics_faint laser sources}

%Question: How monochromatic is the laser light?

The emitting light from a laser is considered as monochromatic beam with frequency f
and constant Intensity I. The average rate of a laser is given by:

\begin{align}
    \Phi = \frac{\text{P}}{h \text{f}}
\end{align}

where \textit{I} is the intensity, \textit{P} the laser power and $\text{f} = \frac{c}{\lambda}$ the frequency which
depends on the wavelength.
The average number of registered counts N(T) for a given detection time T by a detector is given by:

\begin{align}
    \text{N(T)} = \text{T} \Phi \eta= \frac{\text{P T} \eta}{h \text{f}}
\end{align}

where \textit{$\eta$} is the efficiency of the detector system which is described in more detail in section \ref{sec:efficiency}.

Since, we are interested in detecting single photons, the photon rate has to be low, which corresponds to
$\text{N(T)} \cdot \eta << 1$ \cite{hadfield-2009}.

Based on this for an ideal detector with a recovery time of zero the registered counts $\mathcal{R}$ per unit time would be:

\begin{align}
    \mathcal{R} = \frac{\text{N(T)}}{\text{T}} = \eta \Phi= \frac{P \eta}{h \text{f}}
\end{align}

However, for a real detector the true count rate is restricted by the dead time $\tau_{\text{d}}$ and
the DCR of the detector.
Therefore, the true count rate $\mathcal{R}'$ is given by:

\begin{align}
    \mathcal{R}' = \left(\frac{\mathcal{R}}{1 - \tau_{\text{d}} \cdot \mathcal{R}}
    - \frac{\text{DCR}}{1 - \tau_{\text{d}} \cdot \text{DCR}}\right)
\end{align}

where $\tau_{\text{d}}$ is the dead time of the detector and DCR is the dark count rate of the detector.

Furthermore, the maximal count rate of the detector is restricted by the dead time $\tau_{\text{d}}$ and
sending too many photons at once to the detector causes latching and prevents counting \cite{single-quantum-2022}.

\section{Experimental setup}\label{sec:experimental_setup}
As mentioned above, we are interested in detecting single photons, corresponding to $\text{N(T)} \cdot \eta << 1$ \cite{hadfield-2009}.
Therefore, the laser power has to be low enough to detect single photons.
The minimum laser power of the laser source is higher than the maximal detection count rate of the detector.
The reason for this is that the laser source only starts to lase given a minimum input power,
Therefore, the laser power can not be regulated down by adjusting the input power of the laser, but has to be attenuated
by neutral density filters.\\
%From factory details it is known that the recovery time of the detector is located between 17 and 23ns, depending on the
%detector settings.
%This details can be confirmed by our measurements as well \ref{sec:recovery-time}.
%This gives a theoretical maximum detection rate with constant maximal possible efficiency of $\mathcal{R_{\max}} \propto \frac{1}{\tau_{d}} \approx 43-58\si{M\Hz}$.
%However, this rate would only hold, if the emitter would be a true single photon source with a consistent emitting rate, where
%photons are equidistantly spaced.
%Since we use a coherent laser source the counting rate is significantly lower than the theoretical maximum detection rate.
%From the factory we got specification of about 2-4 MHz as detection count rate with maximal detection efficiency \textcolor{red}{Source}.

% Ist die theoretisch mögliche count rate nur durch den Fakt, dass wir keine EPD haben limitiert?
% Das nur dewegen die Count rate so viel niedriger ist?

In order to realize the laser attenuation the setup shown in figure \ref{fig: faint_laser_source_full_set_up} and
\ref{fig: faint_laser_source_full_set_up} was build.

The starting point is the upper stage (left) and the optical axis runs parallel to the table.
Laser light is emitted from a 780nm laser source from the Russian company \href{http://vitawave.ru/eng/}{Vitawave} and
is coupled into a single mode polarization maintaining fibre.
The quarter-wave plate $\lambda_4$ and half-wave plate $\lambda_2$ were used to set the polarization of the light vertical
to the optical axis.
The laser beam was coupled at the upper stage, so it could be used at the lower stage.

Afterwards, on the lower stage the beam passes a half-wave plate $\lambda_2$ and polarization beam splitter (pbs) to filter the horizontal
polarized E-field out.
Further, a Galilean telescope was built with one focal ($l_1$) and one diffusing lens ($l_2$)a to reduce the beam width.
This reduction was done to not exceed the surface of the crystal of the Acousto-optic modulator (AOM).
After the AOM, a cover was used to select the first order of the AOM.
The first order of the AOM was used for flexible intensity modulation of the laser.
Then a flip mount was placed where anti reflection coatet (AR) neutral density (ND) filters could be set in and out of the laser beam.
The ND filters have the function to attenuate the laser light.

Before the laser light was again coupled in the optical fibre, again a half-wave plate $\lambda_2$ and a quarter-wave
plate $\lambda_4$ were used to control the light polarization coupling in the fibre.
As mentioned in section \ref{sec:SNSPD_working_principle} the light was polarized according to the slow axis of the fibre.
Afterwards the light is coupled into the fibre, guiding it to the detector.

Regarding the final coupling it was important, to couple to a APC/PC (fibre channel / angled physical contact) to FC/PC (fibre channel / physical contact) optical fibre.
The coupling from the experiment to a APC/PC optical fibre was done because it yields lower reflection losses.
On the other side, the detector had only an FC/PC optical fibre input port, in order to maintain higher efficiency coupling \cite{single-quantum-2022}.\\

Besides, this optical setup had to be protected from environmental light.
For this, the windows in the room, where the setup was installed was shielded
with aluminium foil which has a reflection coefficient of almost 90$\%$ at the operating wavelength of 780nm \cite{Alu_reflectionsgrad}.
Moreover, an optical enclosure (black box) was  made from black plexiglass and wood (seen in figure \ref{}),
where the optical setup was placed in.
It has the function to avoid environmental light coupling into the fibre.
Additionally, the optical fibre going from the optical setup to the detector was covered with aluminium foil to avoid
absorption of stray light through the cladding.\\

\begin{figure}
    \centering
    \includegraphics[width=0.8\textwidth]{figs/HQO_20240712_faint_laser_source_full_set_up_thesis}
    \caption{Set up for attenuation of a 780nm laser source from the Russian company \href{http://vitawave.ru/eng/}{Vitawave}}
    \label{fig: faint_laser_source_full_set_up}
\end{figure}


Based on those theoretical considerations and the experimental setup, one can approach the central goal to determine the
amount of photons and with this, determine the efficiency of the detector.

In order to do efficiency measurements, one must send only a few photons (low power) and one must
exactly know how many photons are send.
For this the power of the laser light sending to the detector need to be measured.

However, the desired light is so weak, one cannot measure it with power-meters or any other measurement devices available to us.
Therefore, the expected photon rate send to the detector depends on the ND filters and their optical density (OD) value.
Since, no time was left to characterize the AOM, the ND filter were the sole factor to adjust the photon rate, given a stable laser power.
%(here transmission)($\text{T} = 10^{-\text{OD}}$).

The challenge relying on ND filters is that the fabric values of their OD are not precise.
Thus, the filters need to be calibrated, to get accurate OD values for the ND filters.
The OD value of each filter is measured with two methods to respect systematic errors.
In the first method the ND filters are set in the flip mount as shown in graph \ref{fig: faint_laser_source_full_set_up}.
In the second method, measurements with and without filters were done directly in front of the powermeter
outside the black box to avoid straying light inside the black box.

The attenuation of ND filters is quantized by the OD value and is connected to the transmission value $ \text{T}_{\text{ND}} = 10^{\text{OD}}$ of the outgoing light.
Due to the logarithmic definition of the OD value ($\text{OD} = \log_{10}(\frac{1}{\text{T}_{\text{ND}}})$), the OD values of the
ND filters are added up when they are stacked on each other.

The measurement of the OD values was done by measuring the power with and without the filter.
Moreover, only measurements for one single ND filter were done.
The reason is the measurable range of the power metre.
The power meter was unable to accurately measure the power of the attenuated laser light through more than one ND filter.

The transmission then corresponds to:
\begin{align}
    \text{T} = \frac{\text{P}_{\text{ND}}}{\text{P}} = \frac{\mathcal{R}'}{\mathcal{R}}
\end{align}

where, $\text{P}_{\text{ND}}$ is the power of the laser light and the corresponding corrected count rate $\mathcal{R}'$ measured by the
single photon detector with the ND filters.
$\text{P}$ is the power of the laser light and $\mathcal{R}$ corresponding count rate
without the ND filter, measured with the power metre.

The final OD values of each method were combined and the corresponding systematic and statistical errors are
considered in order to get accurate OD values for the ND filters.
Results of the measurements and error calculations can be found in section \ref{sec:ND_filter_calibration_appendix}.
Based one these OD values the photon rate was determined.

%uncertainties package (python)





%Relying on: https://nano-optics.physik.uni-siegen.de/education/teaching/lab_courses/sps-exp-manual.pdf

%- Basics of photon distribution of laser, attenuation, poisson statistics \\
%- Set up for laser attenuation \\
%- Single Photon Detection paper \\
%
%Second, one single ND filter cannot attenuate the laser power enough and therefore one has to stack several ND
%filters on each other, since their  values are adding up.

\FloatBarrier
%==============================================================================

