% !TEX root = mythesis.tex

%==============================================================================
\chapter{Introduction}
\label{sec:SNSPD_Introduction}
%==============================================================================

%Single photon detection is an essential part in nonlinear quantum optics with Rydberg atoms. 
%It is used for the study of fundamental light matter interactions in
%the HQO (Hybrid Quantum Optics) experiment at the University of Bonn and might result in innovations for promising
%applications like optical quantum computing \cite{firstenberg-2016, gao-2011}.
%Figure \ref{fig: HQO_full_experiment} shows the setup of the HQO experiment.
%In the HQO experiment Rydberg atoms will be used to interact with an electromechanical acoustic oscillator.
%For the study of those interactions, it is required to place the Rydberg atoms with the electromechanical acoustic oscillator
%in a 4K environment inside a cryostat (here: science chamber) \cite{oconnell-2010}.
%Inside the science chamber Rydberg atoms will be magnetically trapped over a superconducting atom chip, where the electromechanical
%acoustic oscillator is placed on.
%The goal is to excite Rydberg atoms and cool down the electromechanical acoustic oscillator to its ground state through the interaction with
%the Rydberg atoms.
%
%\begin{figure}[hbt!]
% \centering
% \includegraphics[width=\textwidth]{figs/HQO_full_experiment}
% \caption{Experimental setup of the HQO experiment. Rydberg atoms will be cooled and trapped in
% a Magneto optical trap (MOT).
% The atoms will then be transferred to the science chamber and trapped and exited over an atom chip where the
% electromechanical acoustic oscillator is placed on. Courtesy of Cedric Wind.}
% \label{fig: HQO_full_experiment}
%\end{figure}
%
%In genereal, strong nonlinearities of Rydberg atoms appear in a Rydberg blockage at an interaction level of a few photons.
%The Rydberg blockade describes the creation process of a Rydberg polariton (with a significantly large energy shift) by a photon and a
%resulting shift of Rydberg states of nearby atoms \cite{lukin-2001}.
%Then an excitation from a further photon of another Rydberg atom in this blockade is not possible due to the induced shift of the Rydberg states.
%This excitation suppression appears in a radius of 10$\si{\micro \m}$ around the exited Rydberg atom \cite{urban-2009}.
%This will be used to construct an effectively one dimensional chain of Rydberg atoms above the electromechanical acoustic oscillator.
%With the magnetic trap, the distance between the Rydberg atom and the oscillator on the atom chip can be adjusted and hence, the coupling
%strength between them can be controlled.
%
%The experimentation of Rydberg atoms and corresponding quantum nonlinearities, like dissipative interactions (single
%photons are transmitted better than photon pairs) and dispersive interactions (single photons and photon pairs are equally transmitted)
%are investigated by measurements of single photons \cite{firstenberg-2016}.
%However, the measurement of single photons is challenging.
%
%The challenge of detecting single photons is to translate the low energies (i.e. for the optical/infrared
%range: $\text{E} \approx 0.5 - 3.3 \si{\eV}$) into measurable electrical signals.
%Up till today, the most precise way, to detect single photons is with a superconducting nanowire single photon detector (SNSPD).
%
%In this thesis, a SNSPD for single photon detection by the company Single Quantum is characterized.
%This characterization is done in the context of the HQO experiment and has two main goals.
%The first goal is to check if the specifications, given by the company \cite{tech_sheet_single_quantum} are correct and can be confirmed by our own measurements.
%The second goal is to investigate the different dependencies of the detector characteristics.
%
%The thesis is structured as follows:
%The second chapter, entitled \ref{sec:SNSPD_working_principle}, introduces the operational principles of the SNSPD.
%The chapter begins with an introduction to the phenomenological background of how single photon detection with superconducting nanowires works.
%Afterwards, the reason for sending photons with the polarization aligned to the slow axis of the optical fibre is explained.
%
%In the third chapter \ref{sec:SNSPD_setup}, a short%overview of relevant physical relations for the properties evaluation is given.
%Further, the experimental setup of a faint laser source is described.
%In chapter \ref{sec:SNSPD_Characterization}, an investigation of the Dark count rate, the recovery time and the system detection efficiency of the SNSPD is done.
%For this a variety of measurements were done and evaluated in terms of various dependencies.
%The chapter ends with a summary and discussion of the results.

Single photon detectors enable the study of optical interaction on the single photon level which makes them a powerful
tool for many quantum experiments \cite{charaev-2023, unknown-author-2023}.
Many different technologies can be used to detect single photons, ranging from photomultiplier tubes to
superconducting detectors such as the Transition Edge Sensor or the Superconducting Nanowire Single Photon Detector (SNSPD)
\cite{eisaman-2011}.
Recent developments in SNSPDs have made them the first choice detector for the detection of single photons
in the visible to infrared range at high repetition rates of ~10MHz with extremely high quantum efficiency of over 90$\%$,
low timing jitter and extremely low dark count rates \cite{you-2020}.

For that reason a state-of-the-art SNSPD from the company Single Quantum \footnote{\href{https://www.singlequantum.com/technology/snspd/}{Single Quantum homepage}}
was acquired for the new Hybrid Quantum Optics (HQO) experiment in the Nonlinear Quantum Optics group of Sebastian Hofferberth.

The goal of the HQO experiment (set up shown in \ref{fig:HQO_full_experiment}) is to couple an ultra-cold ensemble of rubidium atoms excited to the Rydberg state to
an electromechanical oscillator in a cryogenic environment \cite{stevenson-2016, gao-2011}.

\begin{figure}[hbt!]
 \centering
 \includegraphics[width=\textwidth]{figs/HQO_full_experiment}
 \caption{Experimental setup of the HQO experiment. Rydberg atoms will be cooled and trapped in
 a Magneto optical trap (MOT).
 The atoms will then be transferred to the science chamber and trapped and exited over an atom chip where the
 electromechanical acoustic oscillator is placed on. Courtesy of Cedric Wind.}
 \label{fig:HQO_full_experiment}
\end{figure}

Such electromechanical oscillators are a promising candidate for quantum memory when coupled to superconducting
transmon qubits due to their long coherence times at microwave frequencies \cite{chu-2017}
and have also been used for fundamental studies of large quantum systems \cite{bild-2023}.
The proposed hybrid system is of particular interest in this context because it would allow the optical control of the
mechanical motion of the oscillator mediated by the strong microwave transitions of the Rydberg atoms.
The first experimental goal is to use the interaction with the Rydberg atoms to cool the mechanical oscillator to its
mechanical ground state.
For the Rydberg excitation, a two-photon process consisting of a 780 nm probe and a 480 nm control will be used.
A typical tool for the detection of Rydberg atoms is field ionisation \cite{low-2012} but further information about the
Rydberg excitation and in particular the interaction with the electromechanical oscillator can be obtained by
measuring the probe photons transmitted through the atomic ensemble at the single photon level \cite{peyronel-2012, firstenberg-2013}.

In this thesis, the SNSPD has been characterised for later use in the HQO experiment with two main objectives.
First, to provide a precise characterisation of the detector in order to have a reliable knowledge of the system
parameters for a proper evaluation of the measurement results in the experiment.
Second, to find the optimal operating parameters for the SNSPD, because unlike avalanche photo diodes, the bias current
of a SNSPD has to be adjusted and optimized for different photon rates.

The thesis is structured as follows:
Chapter \ref{sec:SNSPD_working_principle}, introduces the working principle of a SNSPD.
The third chapter \ref{sec:SNSPD_setup} gives a short overview of relevant physical relations for the evaluation of the upcoming characteristica in chapter \ref{sec:SNSPD_Characterization}.
Furthermore, in this chapter the experimental setup of a faint laser source is described.
In chapter \ref{sec:SNSPD_Characterization}, an investigation of the dark count rate, the recovery time and the system detection efficiency of the SNSPD is done.
For this a variety of measurements were done and evaluated in terms of various dependencies.
The chapter ends with a summary of the results.

%Chapter 2 of this thesis introduces the working principle of SNSPDs.
%Chapter 3 introduces and explains the relevant characteristics of single photon detectors.
%Furthermore, the experimental setup of a faint laser source used for the detector characterisation performed for this thesis is described.
%Chapter 4 presents the measurement of the dark count rate, the recovery time and the system detection efficiency of the SNSPD.
%The dependencies of these system characteristics on the operating parameters are discussed and presented.