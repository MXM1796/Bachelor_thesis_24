% !TEX root = mythesis.tex

%==============================================================================
\chapter{Introduction}
\label{sec:SNSPD_Introduction}
%==============================================================================

Single photon detection is an essential part in nonlinear quantum optics with Rydberg atoms. 
Nonlinear quantum optics with Rydberg atoms is used for the study of fundamental light matter interactions in 
the HQO (Hybrid Quantum Optics) experiment at the University of Bonn and might result in innovations for promising
applications like optical quantum computing \cite{firstenberg-2016, gao-2011}.

In the HQO experiment properties of Rydberg atoms will be used to interact with an electromechanical acoustic oscillator \cite{}.
For this study of those interactions, it is required to place the Rydberg atoms with the electromechanical acoustic oscillator
in a 4K environment inside a cryostat (science chamber).
Inside the science chamber Rydberg atoms will be trapped over atom chip.
The goal is to excite Rydberg atoms and cool down the electromechanical acoustic oscillator to its ground state through the interaction with
the Rydberg atoms.



The challenge of detecting single photons is to translate the low energies (input variable - e.g.
for the optical/infrared range: $\text{E} \approx 0.5 - 3.3 \si{\eV}$) into measurable electrical signals (output variable).
Four central requirements have been established in the literature to quantitatively describe the quality of
EPD \cite{hadfield-2009, shalm_single-photon_2013}.

