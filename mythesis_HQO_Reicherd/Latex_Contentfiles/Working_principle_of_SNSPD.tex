% !TEX root = mythesis.tex
\graphicspath{{/Users/maxim.re/Studium/Physik B.Sc./Semester_8_SS24/Proseminar/Figs Single Photon Detection/}}

%==============================================================================
\chapter{Working principle of SNSPDs}
\label{sec:SNSPD_working_principle}
%==============================================================================

In its essence, the SNSPD consists of four parts, seen in fig. \ref{fig:SNSPD_rough_structure}.
A saphire underground, a detection area made out of a superconducting nanowire in a serpentine winding and placed ont the
saphire underground, a gold contact to supply a bias current through the superconducting nanowire and a fiber coupled
to the detection area.\\
The sapphire layer is used to efficiently dissipate the heat when the wire heats up. \\
Moreover, the set up is first cooled below the critical temperature of the superconductor ($2-3 \text{K}$),
and a bias current is applied to the superconductor that is lower than the critical current.

\begin{figure}[hhh]
\includegraphics[width=12cm]{SNSPD_grober_aufbau}
\caption{(a) Schematic structure of a superconducting nanowire single photon detector \cite{steudle-2012}}
\label{fig:SNSPD_rough_structure}
\end{figure}

The detection process can be understood along the detection circle in fig. \ref{fig: SNSPD_process}.

\begin{figure}[hhh]
\includegraphics[width=12cm]{SNSPD_Ablauf}
\caption{Schematic detection cycle of a superconducting nanowire single photon detector \cite{singlequantum_snsd_nodate} }
\label{fig: SNSPD_process}
\end{figure}

Single photons hitting a superconducting nanowire (ii) and break up individual Cooper pairs.
This leads to a local reduction of the critical current below the bias current and in turn to a localised area where
the superconductivity is interrupted, this local area forms the so-called \("\)hotspot\("\) (iii).
This hotspot forms a resistance area because the bias current exceeds the critical current.
In response, the current flows around this hotspot (iv), whereby the local current density in the side areas next to the
hotspot again exceeds the critical current, due to a higher current density.
This excess also causes a resistance in the side channels due to the critical temperature being exceeded (v).
Ultimately, this increase in resistance can be measured in the form of a voltage pulse.
The non-superconducting area is then cooled down by the cryogenic environment and returns to the superconducting state
(iv—i).\\

One important technical detail is the fiber coupling of this detector because the efficiency
and timing jitter depend on it.
Depending on in which polarization the light hits the meander, the efficiency changes.
When light hits the wire orthogonally polarized to the wire direction, the photon is less efficiently absorbed, than
polarized parallel to the wire. \\ As seen in fig. \ref{fig:SNSPD_fiber_coupling} the coupled fibre in our characterized
detector is parallel polarized to the nanowire.\\
Moreover, to cover the whole light, shined out of the fibre, the geometry of the detection zone is constructed as a
round plate and has roughly the same diameter as the fibre output (FC/PC). A smaller area would risk not absorbing each
photon and a larger area would increase the time leading the signal to the computer unit.

\begin{figure}[hhh]
\includegraphics[width=12cm]{SNSPD_Meander_fiber_coupling}
\caption{Scematic set up of the fiber coupling of a superconducting nanowire single photon detector \textcolor{red}{Zitieren!}}
\label{fig:SNSPD_fiber_coupling}
\end{figure}
%Nur Technische Details die für die Charakterisierung relevant sind (max 2 Seiten)
%Physikalische Motivation (Warum frage sehr wichtig)