% !TEX root = mythesis.tex
\graphicspath{{/Users/maxim.re/Studium/Physik B.Sc./Semester_8_SS24/Proseminar/Figs Single Photon Detection/}}

%==============================================================================
\chapter{Working principle of a SNSPD}
\label{sec:SNSPD_working_principle}
%==============================================================================
This chapter will introduce the working principle of superconducting nanowire single photon detectors (SNSPD).
This chapter has three parts.
The first part describes the essential detector elements and their functions.
The second part explains the detection process and provides its principles in terms of their phenomenological aspects.
For further details regarding the physics and mathematics behind SNSPDs, the reader is directed
to the work of Gol’tsman et al. \cite{goltsman-2001} and Hadfield et al. \cite{natarajan-2012}.
The third part will explain the reason why the detected photons need to be aligned to the slow axis of the optical fibre.\\

In general a SNSPD consists of four parts, as shown in figure \ref{fig:SNSPD_rough_structure}.
The most important part is the detection area and consists of a superconducting nanowire ($\approx 100nm$ wide) on a sapphire base.
In general, superconductors have the property of abruptly losing all internal resistance once the temperature
falls \textit{below} a critical temperature $T_{\text{krit}}$.
If the critical temperature is \textit{exceeded} (e.g. due to the environment or excessive current),
the resistance suddenly increases again.
Theoretically, this behaviour is described by Cooper pairs in the BCS theory \cite{bardeen-1957}.
A more detailed description of superconductors is omitted in this chapter.

To collect the entire output of the optical fibre, a pattern of a
thin superconducting film (such as niobium nitride (NbN)) is shaped into a meandering nanowire by nanofabrication \cite{single-quantum-2022}.
Another element is the sapphire layer, which is used to dissipate the heat when the wire heats up.
Further, a gold contact supplies a bias current through the superconducting nanowire, and an optical fibre is coupled to the detection area.
To operate the system, this setup is cooled below the critical temperature of the superconductor to
2-3K and a DC current (Bias current $I_{B}$) is applied to the superconducting nanowire.

\begin{figure}[!hbt]
    \centering
    \includegraphics[width=0.6\linewidth]{figs/HQO_20240721_SNSPD_elementary_setup}
    \caption{Scematic illustration of the fundamental parts of a superconducting nanowire single photon detector based on \cite{steudle-2012}.
    Detector area is made of a superconducting nanowire and is contacted with gold from two sides.
    An optical fibre is coupled to the detector area.
    The fibre coupling realization of the characterized detector from Single Quantum is shown in figure \ref{fig:SNSPD_fiber_coupling}.}
    \label{fig:SNSPD_rough_structure}
\end{figure}

\FloatBarrier

In this second part, the detection process, shown in figure \ref{fig: SNSPD_process} is outlined.
Starting point (i) is the detector area in the superconducting state and an applied Bias current which is below the critical current.
Then, if photons hit the superconducting nanowire (ii), they break up individual Cooper pairs.
This leads to a local reduction of the critical current below the bias current and in turn to a localized area, where
the superconductivity is interrupted.
This local area forms the so-called \("\)hotspot\("\) (iii).
This hotspot forms a resistance area because the critical temperature is exceeded by the energy of the photon.
In response, the current flows around this hotspot (iv), whereby the local current density in the side areas next to the
hotspot again exceeds the critical current, due to a higher current density.
If the critical current is exceeded, the superconductivity also breaks down in these areas.
The excess also causes a resistance in the side channels of the nanowire (v).
This whole rapid increase in resistance can ultimately be measured in form of a voltage pulse, which can be seen in figure \ref{fig:SNSPD_single_voltage_pulse}.
The local non-superconducting area is then cooled down by the cryogenic environment and returns to the superconducting state
(iv—i).

\begin{figure}[!hbt]
    \centering
    \includegraphics[width=0.8\linewidth]{figs/HQO_20240725_detection_process_circle}
    \caption{Schematic detection process cycle of a superconducting nanowire single photon detector based on \cite{singlequantum_snsd_nodate}.
    The dotted innercircle shows the parts of the dead time $\tau_{\text{dead}}$ and the reset time $\tau_{\text{reset}}$.
    The outer circle shows the parts of the detection process.
    In the upper left side the whole detector area is shown, the grey bars show the current flow in the zoomed superconducting nanowire.}
    \label{fig: SNSPD_process}
\end{figure}

\begin{figure}
    \centering
    \includegraphics[width=0.8\linewidth]{figs/HQO_20240725_analog_signal_31.2uA_tv_600mV}
    \caption{Single analog voltage pulse signal, recorded with a Lecroi oscilloscope with a time resolution of 500MHz.
    Voltage is depicted on the Y-axis and time in ns on the X-axis. A single photon steems from a faint laser source (780nm) attenuated to a photon rate of 1.772KHz.}
    \label{fig:SNSPD_single_voltage_pulse}
\end{figure}

In the detection process the photon polarization plays a crucial role.
Due to a polarization dependent absorption efficiency, it is important to consider the technical detail of the geometry of the meander.
In the used SNSPD, the meander design allows for a higher absorption efficiency if the E-field of the photons
is polarized parallel to the wire direction rather than orthogonally polarized \cite{single-quantum-2022}.
As depicted in figure \ref{fig:SNSPD_fiber_coupling} the slow axis of the coupled fibre in the characterized
detector is aligned parallel to the nanowire to maximize the absorption efficiency.
Other geometries are being investigated and might enable a high absorption efficiency independent of polarization \cite{zheng-2016}.

\begin{figure}
    \centering
    \includegraphics[width=\linewidth]{SNSPD_Meander_fiber_coupling}
    \caption{Fibre coupling set up to the superconducting nanowire detector area of the characterized detector from the company Single Quantum \cite{singlequantum_snsd}.
    Fibre coupling is made in-house and is ajusted to maximize efficiency. It cannot be adjusted manually.}
    \label{fig:SNSPD_fiber_coupling}
\end{figure}

%Nur Technische Details die für die Charakterisierung relevant sind (max 2 Seiten)
%Physikalische Motivation (Warum frage sehr wichtig)