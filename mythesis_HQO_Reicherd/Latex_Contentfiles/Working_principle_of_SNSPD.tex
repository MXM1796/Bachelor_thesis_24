% !TEX root = mythesis.tex
\graphicspath{{/Users/maxim.re/Studium/Physik B.Sc./Semester_8_SS24/Proseminar/Figs Single Photon Detection/}}

%==============================================================================
\chapter{Working principle of SNSPDs}
\label{sec:SNSPD_working_principle}
%==============================================================================
This chapter will introduce the working principle of superconducting nanowire single photon detectors (SNSPDs).
%1-2 sentences to history for introduction
%Wikipedia provides good stuff for introduction
The following section will provide an introduction to the principles in terms of their phenomenological aspects.
For further details regarding the physics and mathematics behind superconducting nanowires, the reader is directed
to the work of Gol’tsman et al \cite{goltsman-2001} and Hadfield et al \cite{natarajan-2012}.

The SNSPD has four parts, as shown in figure \ref{fig:SNSPD_rough_structure}.
The central detection area is made of a superconducting nanowire ($\approx 100nm$ wide) on a sapphire base.
To collect the whole output of the optical fibre, a pattern out of a
thin superconducting film (like niobium nitride) is shaped into a meandering nanowire through nanofabrication \cite{single-quantum-2022}.
The sapphire layer is used to dissipate the heat when the wire heats up.
Further, a gold contact supplies a bias current through the superconducting nanowire and an optical fibre is coupled to the detection area.

To operate the system, the setup is cooled below the critical temperature of the superconductor to
2-3K and a DC current is applied to the nanowire.

\begin{figure}[hhh]
    \centering
    \includegraphics[width=12cm]{SNSPD_grober_aufbau}
    \caption{(a) Schematic structure of a superconducting nanowire single photon detector \cite{steudle-2012}}
    \label{fig:SNSPD_rough_structure}
\end{figure}

The detection process is shown in the figure \ref{fig: SNSPD_process}.

\begin{figure}[hhh]
    \centering
    \includegraphics[width=12cm]{SNSPD_Ablauf}
    \caption{Schematic detection cycle of a superconducting nanowire single photon detector \cite{singlequantum_snsd_nodate}}
    \label{fig: SNSPD_process}
\end{figure}

Photons hit the superconducting nanowire (ii) and break up individual Cooper pairs.
This leads to a local reduction of the critical current below the bias current and in turn to a localised area where
the superconductivity is interrupted, this local area forms the so-called \("\)hotspot\("\) (iii).
This hotspot forms a resistance area because the critical temperature is exceeded by the energy of the photon.
In response, the current flows around this hotspot (iv), whereby the local current density in the side areas next to the
hotspot again exceeds the critical current, due to a higher current density.
If the critical current is exceeded, the superconductivity also breaks down in these areas.
This excess also causes a resistance in the side channels of the nanowire (v).
Ultimately, this rapid increase in resistance can be measured in form of a voltage pulse, which can be seen in figure \ref{fig: SNSPD_single_voltage_pulse}.
The local non-superconducting area is then cooled down by the cryogenic environment and returns to the superconducting state
(iv—i).\\


%\begin{align}
%    \centering
%    \includegraphics[width=12cm]{SNSPD_Meander_fiber_coupling}
%    \caption{Exemplary Analog voltage pulse of a SNSPD from an oscilloscope from channel 1 for a bias current of 31.2$\si{\micro \A}$and a trigger of 600$\si{\milli \V}$.
% X-axis: time in 50ns steps (straight vertical yellow lines), Y-axis voltage in 500$\si{\milli \V}$ steps (straight horizontal yellow lines)}
%    \label{fig:SNSPD_single_voltage_pulse}
%\end{align}}

Due to a polarization dependent absorption efficiency it is important to consider the technical detail of the geometry of the meander.
In the used SNSPD, the meander design allows for a higher absorption efficiency if the e-field of the photons
is parallel polarized to the wire direction than orthogonally polarized ones \cite{single-quantum-2022}.
As depicted in figure \ref{fig:SNSPD_fiber_coupling} the coupled fibre slow axis in the characterized
detector is parallel aligned to the nanowire to maximize the absorption efficiency.
Other geometries are investigated and might enable a high absorption efficiency independent of the polarization \cite{zheng-2016}.



\begin{figure}[hhh]
    \centering
    \includegraphics[width=12cm]{SNSPD_Meander_fiber_coupling}
    \caption{Scematic set up of the fiber coupling of a superconducting nanowire single photon detector \textcolor{red}{Zitieren!}}
    \label{fig:SNSPD_fiber_coupling}
\end{figure}


%Nur Technische Details die für die Charakterisierung relevant sind (max 2 Seiten)
%Physikalische Motivation (Warum frage sehr wichtig)