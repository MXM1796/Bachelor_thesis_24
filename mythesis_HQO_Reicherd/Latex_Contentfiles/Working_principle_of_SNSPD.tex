% !TEX root = mythesis.tex
\graphicspath{{/Users/maxim.re/Studium/Physik B.Sc./Semester_8_SS24/Proseminar/Figs Single Photon Detection/}}

%==============================================================================
\chapter{Working principle of SNSPDs}
\label{sec:SNSPD_working_principle}
%==============================================================================
This chapter will introduce the working principle of superconducting nanowire single photon detectors (SNSPDs).
1-2 sentences to history for introduction
Wikipedia provides good stuff for introduction
Here only phenomenological explanations is given. More details can be found in ..


The SNSPD has four parts, as shown in figure \ref{fig:SNSPD_rough_structure}.
The central detection area is made of a superconducting nanowire on a sapphire base.
The superconducting nanowire is made of a thin film of a superconducting material, like niobium nitride (NbN).
The sapphire layer is used to dissipate the heat when the wire heats up.
Further, a gold contact supplies a bias current through the superconducting nanowire and an optical fibre is coupled to the detection area.

To operate the system, the setup is cooled below the critical temperature of the superconductor to
2-3K and a DC current is applied to the nanowire.

\begin{figure}[hhh]
    \centering
    \includegraphics[width=12cm]{SNSPD_grober_aufbau}
    \caption{(a) Schematic structure of a superconducting nanowire single photon detector \cite{steudle-2012}}
    \label{fig:SNSPD_rough_structure}
\end{figure}

The detection process is shown in the figure \ref{fig: SNSPD_process}.

\begin{figure}[hhh]
    \centering
    \includegraphics[width=12cm]{SNSPD_Ablauf}
    \caption{Schematic detection cycle of a superconducting nanowire single photon detector \cite{singlequantum_snsd_nodate}}
    \label{fig: SNSPD_process}
\end{figure}

Photons hit the superconducting nanowire (ii) and break up individual Cooper pairs.
This leads to a local reduction of the critical current below the bias current and in turn to a localised area where
the superconductivity is interrupted, this local area forms the so-called \("\)hotspot\("\) (iii).
This hotspot forms a resistance area because the critical temperature is exceeded by the energy of the photon.
In response, the current flows around this hotspot (iv), whereby the local current density in the side areas next to the
hotspot again exceeds the critical current, due to a higher current density.
If the critical current is exceeded, the superconductivity also breaks down in these areas.
This excess also causes a resistance in the side channels of the nanowire (v).
Ultimately, this rapid increase in resistance can be measured in form of a voltage pulse.
The local non-superconducting area is then cooled down by the cryogenic environment and returns to the superconducting state
(iv—i).\\

One important technical detail is the fiber coupling of this detector because the efficiency
and timing jitter depend on it.
Depending on in which polarization the light hits the meander, the efficiency changes.
When light hits the wire orthogonally polarized to the wire direction, the photon is less efficiently absorbed, than
polarized parallel to the wire. \\ As seen in fig. \ref{fig:SNSPD_fiber_coupling} the coupled fibre in our characterized
detector is parallel polarized to the nanowire.\\
Moreover, to cover the whole light, shined out of the fibre, the geometry of the detection zone is constructed as a
round plate and has roughly the same diameter as the fibre output (FC/PC). A smaller area would risk not absorbing each
photon and a larger area would increase the time leading the signal to the computer unit.

\begin{figure}[hhh]
    \centering
    \includegraphics[width=12cm]{SNSPD_Meander_fiber_coupling}
    \caption{Scematic set up of the fiber coupling of a superconducting nanowire single photon detector \textcolor{red}{Zitieren!}}
    \label{fig:SNSPD_fiber_coupling}
\end{figure}


%Nur Technische Details die für die Charakterisierung relevant sind (max 2 Seiten)
%Physikalische Motivation (Warum frage sehr wichtig)