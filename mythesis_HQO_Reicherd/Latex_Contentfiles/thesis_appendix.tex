%------------------------------------------------------------------------------
\chapter{Appendix}
\label{sec:app}
%------------------------------------------------------------------------------

\section{Set up and elements}\label{sec:Set_up_and_Elements_appendix}

\begin{figure}
    \centering
    \includegraphics[width=\textwidth]{figs/HQO_20240730_set_in_black_box}
    \caption{Set up for attenuation of a 780nm laser source from the Russian company Vitawave}
    \label{fig: faint_laser_source_full_set_up_real_life}
\end{figure}

\begin{figure}
    \centering
    \includegraphics[width=0.8\textwidth]{figs/HQO_20240730_fibre_connecton_to_detector}
    \caption{Fibre connection to the detector from the black box. The optical fibre is going through a hole in the black box and is connected to the
    outcoupler of the faint laser source.}
    \label{fig: fibre_connection_real_life}
\end{figure}

\begin{figure}
    \centering
    \includegraphics[width=0.8\textwidth]{figs/HQO_20240703_self_build_black_box}
    \caption{Picture of the self build optical enclosure (black box) from the outside.}
    \label{fig: black_box_outside}
\end{figure}

\begin{figure}
    \centering
    \includegraphics[width=0.8\textwidth]{figs/HQO_20240703_self_build_black_box_inside}
    \caption{Picture of the self build optical enclosure (black box) from the inside}
    \label{fig:black_box_inside}
\end{figure}

\FloatBarrier

\section{Neutral density filter calibration - Error calculation}\label{sec:ND_filter_calibration_appendix}

To calculate the OD value of the ND Filters, we have to consider the statistical error from the
transmission value and the systematic error, we get from measuring the OD with two methods, as explained in \ref{sec:experimental_setup}.

We first consider the statistical error from the transmission value.
The transmission is given by:

\begin{align}
    \text{T}=\frac{P_{\text{ND}}}{P}
\end{align}

Where $P$ is the power detected by the power metre without the ND filter and $P_{\text{ND}}$
is the power we measured with applied ND filter.

For the Error calculation the following considerations are done: 

First, for both $P$ and $P_{\text{ND}}$ the measurement uncertainty of the photodiode from the photodiode is always 3$\%$
of the measured value \cite{unknown-author-no-date}.
Since this uncertainty does not affect the overall relation between the two measurements it does not
contribute to the error calculation.
However, the measurement uncertainty from reading the value from the Powermeter contributes to both,
$P$ and $P_{\text{ND}}$, since reading the value from the powermeter provides always a different error because
the last displayed order of magnitude of the displayed result on the powermeter fluctuates.

Further, regarding  $\Delta P$ it is also important to mention that after each ND filter change,
the coupling was readjusted to the maximal vale.
However, after the ND filter was removed it was not possible to get to the same initial value.
Therefore, we always have different power values we compare, though the filtering process was done always with the same
fixed laser power setting.
Based on this, $P$ is the mean value of all initial values was taken and for the error the standard
deviation was calculated.
Due to this we have also to consider an $\textit{additional error source} $ $\Delta P^{\text{STD}}$ from the fiber
coupling and optimization in $\Delta P$ .

\textit{Important}: This only applies for the first measurement method because we adjust there our fiber coupling.
In the second measurement we have a stable $P$ because there, the fiber coupling was not adjusted, once the maximum was
reached.

So, $\Delta P$ for the first measurement method is calculated as follows:

\begin{align}
    \Delta P = \sqrt{(\Delta P_{\text{Read off}})^2 + (\Delta P_{\text{STD}})^2}
\end{align}

For $P_{\text{ND}}$ applies:

\begin{align}
    P_{\text{ND}} = P \cdot 10^{-OD}
\end{align}

The corresponding error is  $\Delta P = 0.03 \cdot P$ (uncertainty of the photodiode) and hence the total error for the power after the ND filter is calculated as follows:
\begin{align}
    \Leftarrow \Delta P_{\text{ND}} = \sqrt{(\Delta P \cdot 10^{-\text{OD}})^2 + (P \cdot \log(10) \cdot \Delta \text{OD})^2}
\end{align}.

An additional error from reading off the count rate (hence, the corresponding Power) from the computer screen was also considered,
since measurement with the power metre was not able.
However, because the fluctuating rate of the count rate was in the order of a few Hz, it was neglected for the final error calculation.

As result, we get the following error for the transmission rate:

\begin{align}
    \Delta T_{\text{stat}}= \sqrt{\left(\frac{\Delta P_{\text{ND}}}{P}\right)^2 + \left(\frac{\Delta P P_{\text{ND}}}{ P^2}\right)^2}
\end{align}.

Further for calculating the total transmission value and its error, we have to include the measurements and errors for the second method.
Here, we have stable power $P$ because the fiber coupling was not adjusted, once the maximum was reached.
The transmission error for the second method is also calculated the same way as above:

\begin{align}
    \Delta T= \sqrt{\left(\frac{\Delta P_{\text{ND}}}{P}\right)^2 + \left(\frac{\Delta P P_{\text{ND}}}{ P^2}\right)^2}
\end{align}.

but both values $\Delta P_{\text{ND}}$ and $\Delta P$ are estimated by the read off error of the powermeter.

%The values for this method are shown in table \ref{tab:}:
% Table has to be included!

With booth transmission values, the total transmission value and its total error (systematic and statistical) can be calculated as follows:

\begin{align}
    T_{\text{total}} = \frac{T_{\text{stat}} + T_{\text{systematic}}}{2}\\
    \Delta T_{\text{total}} = \sqrt{\Delta T_{\text{stat}}^2 + \Delta T_{\text{systematic}}^2}
\end{align}

Since, we do not know from which method the results are closer to the real OD value, we take the mean value of both values.

Continuing with the calculation of the OD value, we have according to \cite{Thorlabs-OD} the relation:
 $\text{OD}= \log(\frac{1}{T})$.
From this, we can calculate the error for the OD: $\Delta OD = \frac{\Delta T}{\ln(10) \cdot T}$

%# the final errors for the OD value are presented in table:
% Table has to be included!


Below are the tables with the calculated values and results for the first and second method, as well for the combined results
which lastly provide the final OD values used for the calculations:

%%! Author = maxim.re
%! Date = 27.07.24


\begin{table}[!ht]
    \centering
    \begin{tabular}{|l|l|l|l|l|l|l|l|}
    \hline
        \textbf{ID}& \textbf{$T_{\text{measured}}$} & \textbf{$\Delta T_{\text{stat}}$} \\ \hline
        TP0333766 & 1,43E-01 & 3,223E-03 \\ \hline
        TP03337667 (2)  & 1,47E-01 & 3,073E-03\\ \hline
        TP03366490  & 8,26E-03 & 1,882E-04\\ \hline
        TP03366490 (2)  & 7,93E-03 & 1,744E-04 \\ \hline
        TP03275234  & 1,00E-03 & 4,381E-05 \\ \hline
        TP03312353  & 1,04E-03 & 4,312E-05 \\ \hline
        TP03271009  & 1,02E-03 & 4,432E-05 \\ \hline
        TP03275234 (2)  & 1,03E-03 & 4,303E-05 \\ \hline
        TP03324728  & 1,80E-04 & 3,843E-05\\ \hline
        TP03287742  & 1,78E-04 & 3,731E-05\\ \hline
        TP03348187 (2)  & 1,82E-04 & 3,921E-05\\ \hline
        TP03348187  & 1,82E-04 & 3,862E-05\\ \hline
    \end{tabular}
    \caption{Results for the transmission rate and its statistcal error.
    The ID is the identification number of the OD filter.}
    \label{tab: ND_filter_calibration}
\end{table}
%! Author = maxim.re
%! Date = 30.07.24

\begin{table}[!ht]
    \centering
    \begin{tabular}{|l|l|l|l|l|l|l|l|l|l|l|l|}
    \hline
ID & $P_{\text{before}} \times 10^{-4}$  [W]  & $\bar(P_{\text{before}}) \times 10^{-4}   [W] & Std $P_{\text{before}} \times 10^{-5}$ [W] & $ \Delta P_{\text{before}} \times 10^{-5}$ [W] & $P^{\text{measured}}_{\text{after}}$ [W] & $T_{\text{Measured}}$ & $ \Delta T_{\text{Measured}}$ & $\text{OD}_{\text{Expected}}$ & $\text{OD}_{\text{measured}}$ & $\Delta \text{OD}_{\text{measured}}$ & Deviation (\%) \\ \hline
    TP03337667 & $5.117  $ & $5.257  $ & $1.011  $ & $1.031  $ & $7.884 \times 10^{-5}$ & $1.500 \times 10^{-1}$ & $3.104 \times 10^{-3}$ & 1.00 & 0.8240 & 0.0090 & 17.60 \\ \hline
    TP03337667 (2) & $5.300  $ & $5.257  $ & $1.011  $ & $1.031  $ & $8.024 \times 10^{-5}$ & $1.526 \times 10^{-1}$ & $2.944 \times 10^{-3}$ & 1.00 & 0.8163 & 0.0084 & 18.37 \\ \hline
    TP03366490 & $5.130  $ & $5.257  $ & $1.011  $ & $1.031  $ & $4.521 \times 10^{-6}$ & $8.600 \times 10^{-3}$ & $1.813  $ & 3.00 & 2.0655 & 0.0092 & 93.45 \\ \hline
    TP03366490 (2) & $5.300  $ & $5.257  $ & $1.011  $ & $1.031  $ & $4.450 \times 10^{-6}$ & $8.465 \times 10^{-3}$ & $1.676  $ & 3.00 & 2.0724 & 0.0086 & 92.76 \\ \hline
    TP03275234 & $5.260  $ & $5.257  $ & $1.011  $ & $1.031  $ & $5.550 \times 10^{-7}$ & $1.056 \times 10^{-3}$ & $4.328 \times 10^{-5}$ & 4.00 & 2.9764 & 0.0178 & 102.36 \\ \hline
    TP03312353 & $5.390  $ & $5.257  $ & $1.011  $ & $1.031  $ & $5.850 \times 10^{-7}$ & $1.113 \times 10^{-3}$ & $4.252 \times 10^{-5}$ & 4.00 & 2.9536 & 0.0166 & 104.64 \\ \hline
    TP03271009 & $5.230  $ & $5.257  $ & $1.011  $ & $1.031  $ & $5.622 \times 10^{-7}$ & $1.069 \times 10^{-3}$ & $4.372 \times 10^{-5}$ & 4.00 & 2.9708 & 0.0178 & 102.92 \\ \hline
    TP03275234 (2) & $5.390  $ & $5.257  $ & $1.011  $ & $1.031  $ & $5.780 \times 10^{-7}$ & $1.100 \times 10^{-3}$ & $4.240 \times 10^{-5}$ & 4.00 & 2.9588 & 0.0167 & 104.12 \\ \hline
    TP03324728 & $5.240  $ & $5.257  $ & $1.011  $ & $1.031  $ & $1.003 \times 10^{-7}$ & $1.908  $ & $3.835 \times 10^{-5}$ & 5.00 & 3.7194 & 0.0873 & 128.06 \\ \hline
    TP03287742 & $5.390  $ & $5.257  $ & $1.011  $ & $1.031  $ & $9.600 \times 10^{-8}$ & $1.826  $ & $3.726 \times 10^{-5}$ & 5.00 & 3.7384 & 0.0886 & 126.16 \\ \hline
    TP03348187 (2) & $5.130  $ & $5.257  $ & $1.011  $ & $1.031  $ & $9.837 \times 10^{-8}$ & $1.871  $ & $3.918 \times 10^{-5}$ & 5.00 & 3.7279 & 0.0909 & 127.21 \\ \hline
    TP03348187 & $5.204  $ & $5.257  $ & $1.011  $ & $1.031  $ & $9.915 \times 10^{-8}$ & $1.886  $ & $3.862 \times 10^{-5}$ & 5.00 & 3.7244 & 0.0889 & 127.56 \\ \hline
    \end{tabular}
    \caption{Measurement results for various samples.}
    \label{tab:measurement_results}
\end{table}
%! Author = maxim.re
%! Date = 30.07.24

\begin{table}[!hbt]
    \centering
    \resizebox{\columnwidth}{!}{%
    \begin{tabular}{|l|l|l|l|l|l|l|l|l|}
    \hline
        ID & $P_{\text{before}} \times 10^{-4}$ [W] & $P^{\text{measured}}_{\text{after}}$ [W] & $T_{\text{Measured}}$ & $\Delta T_{\text{Measured}}$ & $\text{OD}_{\text{Expected}}$ & $\text{OD}_{\text{measured}}$ & $\Delta \text{OD}_{\text{measured}}$ & Deviation (\%) \\ \hline
        TP03337667 & $5.13 \times 10^{-4}$ & $6.98 \times 10^{-5}$ & 0.1361 & $5.32 \times 10^{-4}$ & 1.00 & 0.8660 & 0.0017 & 13.40 \\ \hline
        TP03337667 (2) & $5.13 \times 10^{-4}$ & $7.30 \times 10^{-5}$ & 0.1422 & $5.56 \times 10^{-4}$ & 1.00 & 0.8469 & 0.0017 & 15.31 \\ \hline
        TP03366490 & $5.13 \times 10^{-4}$ & $4.07 \times 10^{-6}$ & 0.0079 & $3.12 \times 10^{-5}$ & 3.00 & 2.1004 & 0.0017 & 89.96 \\ \hline
        TP03366490 (2) & $5.13 \times 10^{-4}$ & $3.80 \times 10^{-6}$ & 0.0074 & $2.91 \times 10^{-5}$ & 3.00 & 2.1306 & 0.0017 & 86.94 \\ \hline
        TP03275234 & $5.13 \times 10^{-4}$ & $4.86 \times 10^{-7}$ & 0.0009 & $5.37 \times 10^{-6}$ & 4.00 & 3.0235 & 0.0025 & 97.65 \\ \hline
        TP03312353 & $5.13 \times 10^{-4}$ & $4.96 \times 10^{-7}$ & 0.0010 & $5.43 \times 10^{-6}$ & 4.00 & 3.0142 & 0.0024 & 98.58 \\ \hline
        TP03271009 & $5.13 \times 10^{-4}$ & $4.98 \times 10^{-7}$ & 0.0010 & $5.43 \times 10^{-6}$ & 4.00 & 3.0130 & 0.0024 & 98.70 \\ \hline
        TP03275234 (2) & $5.13 \times 10^{-4}$ & $4.96 \times 10^{-7}$ & 0.0010 & $5.42 \times 10^{-6}$ & 4.00 & 3.0150 & 0.0024 & 98.50 \\ \hline
        TP03324728 & $5.13 \times 10^{-4}$ & $8.75 \times 10^{-8}$ & 0.0002 & $7.71 \times 10^{-7}$ & 5.00 & 3.7682 & 0.0020 & 123.18 \\ \hline
        TP03287742 & $5.13 \times 10^{-4}$ & $8.92 \times 10^{-8}$ & 0.0002 & $7.82 \times 10^{-7}$ & 5.00 & 3.7598 & 0.0020 & 124.02 \\ \hline
        TP03348187 (2) & $5.13 \times 10^{-4}$ & $9.12 \times 10^{-8}$ & 0.0002 & $7.95 \times 10^{-7}$ & 5.00 & 3.7502 & 0.0019 & 124.98 \\ \hline
        TP03348187 & $5.13 \times 10^{-4}$ & $9.06 \times 10^{-8}$ & 0.0002 & $7.92 \times 10^{-7}$ & 5.00 & 3.7529 & 0.0019 & 124.71 \\ \hline
    \end{tabular}%
    }
    \caption{Measurement results for measurements of the OD values of the ND filters with the method 2. The ID is the identification number of the OD filter.}
    \label{tab:measurement_results_method_2}
\end{table}
%! Author = maxim.re
%! Date = 30.07.24
\sisetup{round-precision=3}
\begin{table}[!hbt]
    \centering
    \resizebox{\columnwidth}{!}{%
    \begin{tabular}{|l|l|l|l|l|l|l|l|}
    \hline
        ID & $\text{OD}_{\text{Expected}}$ & $T_{\text{Measured}}$ & $\Delta T_{\text{Systematic}}$ & $\Delta T_{\text{Stat}}$ & $\Delta T$ & $\text{OD}_{\text{mean}}$ & $\Delta \text{OD}$ \\ \hline
        TP03337667 & 1.00 & $1.43 \times 10^{-1}$ & $6.916 \times 10^{-3}$ & $3.149 \times 10^{-3}$ & $7.599 \times 10^{-3}$ & 0.8445 & 0.0231 \\ \hline
        TP03337667 (2) & 1.00 & $1.47 \times 10^{-1}$ & $5.196 \times 10^{-3}$ & $2.997 \times 10^{-3}$ & $5.998 \times 10^{-3}$ & 0.8314 & 0.0177 \\ \hline
        TP03366490 & 3.00 & $8.27 \times 10^{-3}$ & $3.326 \times 10^{-4}$ & $1.840 \times 10^{-4}$ & $3.800 \times 10^{-4}$ & 2.0826 & 0.0199 \\ \hline
        TP03366490 (2) & 3.00 & $7.93 \times 10^{-3}$ & $5.312 \times 10^{-4}$ & $1.701 \times 10^{-4}$ & $5.577 \times 10^{-4}$ & 2.1005 & 0.0305 \\ \hline
        TP03275234 & 4.00 & $1.00 \times 10^{-3}$ & $5.421 \times 10^{-5}$ & $4.361 \times 10^{-5}$ & $6.958 \times 10^{-5}$ & 2.9993 & 0.0302 \\ \hline
        TP03312353 & 4.00 & $1.04 \times 10^{-3}$ & $7.251 \times 10^{-5}$ & $4.286 \times 10^{-5}$ & $8.423 \times 10^{-5}$ & 2.9828 & 0.0352 \\ \hline
        TP03271009 & 4.00 & $1.02 \times 10^{-3}$ & $4.946 \times 10^{-5}$ & $4.405 \times 10^{-5}$ & $6.624 \times 10^{-5}$ & 2.9914 & 0.0282 \\ \hline
        TP03275234 (2) & 4.00 & $1.03 \times 10^{-3}$ & $6.673 \times 10^{-5}$ & $4.274 \times 10^{-5}$ & $7.925 \times 10^{-5}$ & 2.9860 & 0.0333 \\ \hline
        TP03324728 & 5.00 & $1.81 \times 10^{-4}$ & $1.013 \times 10^{-5}$ & $3.836 \times 10^{-5}$ & $3.968 \times 10^{-5}$ & 3.7431 & 0.0954 \\ \hline
        TP03287742 & 5.00 & $1.78 \times 10^{-4}$ & $4.374 \times 10^{-6}$ & $3.727 \times 10^{-5}$ & $3.753 \times 10^{-5}$ & 3.7490 & 0.0914 \\ \hline
        TP03348187 (2) & 5.00 & $1.82 \times 10^{-4}$ & $4.688 \times 10^{-6}$ & $3.918 \times 10^{-5}$ & $3.946 \times 10^{-5}$ & 3.7389 & 0.0939 \\ \hline
        TP03348187 & 5.00 & $1.83 \times 10^{-4}$ & $5.976 \times 10^{-6}$ & $3.862 \times 10^{-5}$ & $3.908 \times 10^{-5}$ & 3.7384 & 0.0929 \\ \hline
    \end{tabular}%
    }
    \caption{Combined results for measurements of the OD values of the ND filters with method 1 and 2. The ID is the identification number of the OD filter.}
    \label{tab:OD_calibration_methods_combined}
\end{table}

\begin{figure}
    \centering
    \includegraphics[width=0.8\textwidth]{figs/HQO_20240730_OD_calibrtion_method_2}
    \caption{Set up for the ND filter calibration (method 2) - Measurement of ND filter after the coupling}
    \label{fig: OD_calibration_method_2}
\end{figure}

\FloatBarrier
%
This final values are in good agreement with the values from the manufacturer Thorlabs as shown on their website \cite{Thorlabs-OD}.
The self calculated errors even provide a narrower error range than the errors provided by Thorlabs.
Unfortunately, it can not be proved how close the calculated values are to the real OD values from Thorlabs, because no
exact values are given.
Only a rough curve (wavelength vs OD) is provided.
Here, exact determination of an expectation value can not be done.

\newpage
\section{Angle dependent count rate measurement data and results}\label{sec:angle_dependent_countrate_results_appendix}

\begin{table}[!ht]
    \centering
    \begin{tabular}{|l|l|l|l|l|l|l|}
    \hline
        Input laser Power (W) & OD Filter (multiple Added) & OD Error & Expected Output (W) & Outputerror (Measurment uncertainty)(+-) (W) & Output error (OD and Measurement uncertainty combined) (W) & Output error (OD and Measurement uncertainty combined) (MHz) \\ \hline
        5.111E-04 & 9.574 & 0.135 & 1.363E-13 & 4.090E-15 & 1.835E-14 & 0.072 \\ \hline
        5.111E-04 & 9.574 & 0.135 & 1.363E-13 & 4.090E-15 & 1.835E-14 & 0.072 \\ \hline
        5.111E-04 & 9.574 & 0.135 & 1.363E-13 & 4.090E-15 & 1.835E-14 & 0.072 \\ \hline
        5.111E-04 & 9.574 & 0.135 & 1.363E-13 & 4.090E-15 & 1.835E-14 & 0.072 \\ \hline
        5.111E-04 & 9.574 & 0.135 & 1.363E-13 & 4.090E-15 & 1.835E-14 & 0.072 \\ \hline
        5.111E-04 & 9.574 & 0.135 & 1.363E-13 & 4.090E-15 & 1.835E-14 & 0.072 \\ \hline
        5.111E-04 & 9.574 & 0.135 & 1.363E-13 & 4.090E-15 & 1.835E-14 & 0.072 \\ \hline
        5.111E-04 & 9.574 & 0.135 & 1.363E-13 & 4.090E-15 & 1.835E-14 & 0.072 \\ \hline
        5.111E-04 & 9.574 & 0.135 & 1.363E-13 & 4.090E-15 & 1.835E-14 & 0.072 \\ \hline
        5.111E-04 & 9.574 & 0.135 & 1.363E-13 & 4.090E-15 & 1.835E-14 & 0.072 \\ \hline
        5.111E-04 & 9.574 & 0.135 & 1.363E-13 & 4.090E-15 & 1.835E-14 & 0.072 \\ \hline
        5.111E-04 & 9.574 & 0.135 & 1.363E-13 & 4.090E-15 & 1.835E-14 & 0.072 \\ \hline
        5.111E-04 & 9.574 & 0.135 & 1.363E-13 & 4.090E-15 & 1.835E-14 & 0.072 \\ \hline
        5.111E-04 & 9.574 & 0.135 & 1.363E-13 & 4.090E-15 & 1.835E-14 & 0.072 \\ \hline
        5.111E-04 & 9.574 & 0.135 & 1.363E-13 & 4.090E-15 & 1.835E-14 & 0.072 \\ \hline
        5.111E-04 & 9.574 & 0.135 & 1.363E-13 & 4.090E-15 & 1.835E-14 & 0.072 \\ \hline
        5.111E-04 & 9.574 & 0.135 & 1.363E-13 & 4.090E-15 & 1.835E-14 & 0.072 \\ \hline
        5.111E-04 & 9.574 & 0.135 & 1.363E-13 & 4.090E-15 & 1.835E-14 & 0.072 \\ \hline
        5.111E-04 & 9.574 & 0.135 & 1.363E-13 & 4.090E-15 & 1.835E-14 & 0.072 \\ \hline
        5.111E-04 & 9.574 & 0.135 & 1.363E-13 & 4.090E-15 & 1.835E-14 & 0.072 \\ \hline
        5.111E-04 & 9.574 & 0.135 & 1.363E-13 & 4.090E-15 & 1.835E-14 & 0.072 \\ \hline
    \end{tabular}
\end{table}

\FloatBarrier

\section{Count rate dependent efficiency $\eta_{\text{sde}}$}\label{sec:countrate_sde_results_appendix}

%! Author = maxim.re
%! Date = 29.07.24

\begin{table}[!hbt]
    \centering
    \begin{tabular}{|l|l|l|l|l|l|l|l|l|}
    \hline
         $\eta_{\text{sde}} $ [$\%$] &  $ \Delta \eta_{\text{sde}} $ [$\%$] & $ \#_{incident} $ [$\si{\mega \Hz}$]
    & $ \Delta \#_{incident} $ [$\si{\mega \Hz}$] \\ \hline
        87.242 & 4.955  & 2.446            & 0.139\\ \hline
        90.408 & 9.529  & 0.531  & 5.596  $10^{-2}$\\ \hline
        91.834 & 9.550  & 0.388  & 4.0314 $10^{-2}$\\ \hline
        89.760 & 9.448  & 0.067  & 7.086 $10^{-3}$\\ \hline
        90.793 & 9.657  & 1.230 $10^{-3}$   & 1.299 $10^{-3}$\\ \hline
        90.332 & 10.416 & 1.802 $10^{-3}$  & 0.201 $10^{-3}$\\ \hline
        88.392 & 9.149  & 9.575            & 0.991\\ \hline
        55.312 & 5.580  & 67.363   & 6.797\\ \hline
    \end{tabular}
    \caption{ Efficiency values $\eta_{\text{sde}}$ for different count rates.
    Measurement were done with 518.1$ \si{\micro \W}$ input laser power.}
    \label{tab:sde_count_rate_table}
\end{table}




\FloatBarrier

\section{Bias sweeping for different count rates}\label{sec:bias_sweeping_countrate_appendix}

\begin{figure}[!hbt]
    \centering
    \begin{tabular}{cc}
        \subcaptionbox{Input photon rate: 67.363MHz \label{fig:bias_sweep_7_48_appendix}}{\includegraphics[width=0.45\textwidth]{figs/HQO_20240729_OD 7_48_countrate_bias_sweep_tv_300_900_thesis}} &
        \subcaptionbox{Input photon rate: 9.575MHz \label{fig:bias_sweep_8_32_appendix}}{\includegraphics[width=0.45\textwidth]{figs/HQO_20240729_OD 8_32_countrate_bias_sweep_tv_300_900_thesis}} \\
        \subcaptionbox{Input photon rate: 2.446MHz \label{fig:bias_sweep_8_92_appendix}}{\includegraphics[width=0.45\textwidth]{figs/HQO_20240729_OD 8_92_countrate_bias_sweep_tv_300_900_thesis}} &
        \subcaptionbox{Input photon rate: 0.531MHz \label{fig:bias_sweep_9_58_appendix}}{\includegraphics[width=0.45\textwidth]{figs/HQO_20240729_OD 9_58_countrate_bias_sweep_tv_300_900_thesis}} \\
        \subcaptionbox{Input photon rate: 0.388MHz \label{fig:bias_sweep_9_72_appendix}}{\includegraphics[width=0.45\textwidth]{figs/HQO_20240729_OD 9_72_countrate_bias_sweep_tv_300_900_thesis}} &
        \subcaptionbox{Input photon rate: 0.067MHz \label{fig:bias_sweep_10_48_appendix}}{\includegraphics[width=0.45\textwidth]{figs/HQO_20240729_OD 10_48_countrate_bias_sweep_tv_300_900_thesis}} \\
%        \subcaptionbox{Caption 9\label{fig:9}}{\includegraphics[width=0.3\textwidth]{figs/HQO_20240729_OD 7_48_countrate_bias_sweep_tv_300_900_thesis}} \\
    \end{tabular}
    \caption{Bias sweep measurement (Bias current vs. Count rate) - Error bars are not shown for better visibility. }
    \label{fig:grid_bias_sweep_appendix_1}
\end{figure}

\begin{figure}[!hbt]
    \centering
    \begin{tabular}{cc}
        \subcaptionbox{Input photon rate: 0.012MHz \label{fig:bias_sweep_11_22_appendix}}{\includegraphics[width=0.45\textwidth]{figs/HQO_20240729_OD 11_22_countrate_bias_sweep_tv_300_900_thesis}} &
        \subcaptionbox{Input photon rate: 0.0017MHz \label{fig:bias_sweep_12_06_appendix}}{\includegraphics[width=0.45\textwidth]{figs/HQO_20240729_OD 12_06_countrate_bias_sweep_tv_300_900_thesis}} \\
    \end{tabular}
    \caption{Bias sweep measurement (Bias current vs. Count rate) - Error bars are not shown for better visibility. }
    \label{fig:grid_bias_sweep_appendix_2}
\end{figure}


\FloatBarrier

\section{Recovery time measurements}\label{sec:Recovery time measurements_appendix}

\begin{figure}[!hbt]
    \centering
    \begin{tabular}{cc}
        \subcaptionbox{Bias current: 25$\si{\micro \A}$\label{fig:recovery_25_appendix}}{\includegraphics[width=0.5\textwidth]{figs/HQO_20240723_recovery_time_Channel_1_Bias_25uA_trigg_300-900mV_thesis}} &
        \subcaptionbox{Bias current: 27$\si{\micro \A}$\label{fig:recovery_27_appendix}}{\includegraphics[width=0.5\textwidth]{figs/HQO_20240723_recovery_time_Channel_1_Bias_27uA_trigg_300-1000mV_thesis}} \\
    \end{tabular}
    \caption{Results of autocorrelation evaluation (Input photon rate: 0.531MHz ) - bins are in ns and represent the different time intervals between the incoming pulses.
        - Error bars are not shown for better visibility.}
    \label{fig:grid_recovery_time_appendix}
\end{figure}

\begin{figure}[!hbt]
    \centering
    \begin{tabular}{cc}
        \subcaptionbox{Bias current: 29$\si{\micro \A}$\label{fig:recovery_29_appendix}}{\includegraphics[width=0.5\textwidth]{figs/HQO_20240723_recovery_time_Channel_1_Bias_29uA_trigg_300-1000mV_thesis}} &
        \subcaptionbox{Bias current: 31.2$\si{\micro \A}$\label{fig:recovery_31_2_appendix}}{\includegraphics[width=0.5\textwidth]{figs/HQO_20240723_recovery_time_Channel_1_Bias_31_2uA_trigg_300-1000mV_thesis}}
    \end{tabular}
    \caption{Results of autocorrelation evaluation (Input photon rate: 0.531MHz ) - bins are in ns and represent the different time intervals between the incoming pulses.
        - Error bars are not shown for better visibility.}
    \label{fig:grid_recovery_time_appendix_2}
\end{figure}
\FloatBarrier

\section{Recordings of analog pulses}\label{sec:analog_pulse_recordings_appendix}
\begin{figure}[!hbt]
    \centering
    \begin{tabular}{cc}
        \subcaptionbox{Bias current: 25$\si{\micro \A}$\label{fig:analog_pulse_appendix_25uA}}{\includegraphics[width=0.45\textwidth]{figs/HQO_20240730_analog_signal_25uA_tv_600mV}} &
        \subcaptionbox{Bias current: 31.2$\si{\micro \A}$ \label{fig:analog_pulse_appendix_31_2uA}}{\includegraphics[width=0.45\textwidth]{figs/HQO_20240730_analog_signal_31_2uA_tv_600mV}} \\
    \end{tabular}
    \caption{Recordings of analog pulses for different bias currents. Recordings are done with oscilloscope with time resolution of 500MHz.}
    \label{fig:grid_analog_pulse_appendix_1}
\end{figure}

\begin{figure}[!hbt]
    \centering
    \begin{tabular}{cc}
        \subcaptionbox{Bias current: 31.2$\si{\micro \A}$\label{fig:analog_pulse_appendix_31_2uA_OD_9_57}}{\includegraphics[width=0.45\textwidth]{figs/HQO_20240730_analog_signal_31_2uA_tv_600mV_OD_9_57}} &
        \subcaptionbox{Bias current: 33$\si{\micro \A}$\label{fig:analog_pulse_appendix_33uA}}{\includegraphics[width=0.45\textwidth]{figs/HQO_20240730_analog_signal_33uA_tv_600mV}} \\
         \subcaptionbox{Bias current: 35$\si{\micro \A}$ with latching\label{fig:analog_pulse_appendix_35uA}}{\includegraphics[width=0.45\textwidth]{figs/HQO_20240730_analog_signal_35uA_tv_600mV}} &
        \subcaptionbox{Bias current: 35$\si{\micro \A}$ \label{fig:analog_pulse_appendix_35uA_OD_9_57_single}}{\includegraphics[width=0.45\textwidth]{figs/HQO_20240730_analog_signal_35uA_tv_600mV_single}} \\
    \end{tabular}
    \caption{Recordings of analog pulses for different bias currents. Recordings are done with oscilloscope with time resolution of 500MHz.}
    \label{fig:grid_analog_pulse_appendix_2}
\end{figure}


