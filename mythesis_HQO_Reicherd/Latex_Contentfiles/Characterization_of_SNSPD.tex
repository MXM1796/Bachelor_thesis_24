% !TEX root = mythesis.tex


%==============================================================================
\chapter{Characterization of a SNSPD by Single Quantum}
\label{sec:SNSPD_Characterization}
%==============================================================================

In literature, four central characteristics have emerged to quantify the quality of single photon detectors and
make their performance comparable.
These characteristics are the system detection efficiency $\eta_{sde}$, the dark count rate (DCR), the recovery time ($\tau_{recovery} = \tau_{rec}$)
and the timing jitter.
In this thesis, I focus on the detector efficiency $\eta_{sde}$, the dark count rate (DCR) and the recovery time ($\tau_{rec}$).
With its importance will be dealt in the single sections.
In general there are more than these introduced figures of merits, like after-pulsing but will not be considered in this thesis.

\section{Dark Count Rate}\label{sec:dark-count-rate}
The dark count rate (DCR) is the rate of measured detection events not intentionally sent from the source (here the faint laser source).
It is measured in counts per second and can be caused by statistical fluctuations in the measurement electronics
or by scattered or ambient light from the environment.
A low DCR is important and allows differentiation between intended detected counts from the source and detected counts
by environment and electronic noise.
Furthermore, an increase in the count resolution is achieved by low DCR, enabling the detection of high signal-to-noise ratios at
lower input frequencies.\\

In the context of SNSPDs, the DCR is dependent on the bias current applied to the nanowire.
This is due to the fact that if the bias current approaches the critical current, the electronic fluctuates more frequently and
exceeds more frequent the critical current which results in a triggered pulse.
Furthermore, it is important to perform DCR measurements first in the characterization process because it determines the
limit, where general measurements are not distorted by high DCR noise.
Consequently, modifications to the DCR configuration serve as the baseline for all subsequent measurements
with the detector and the associated setup.

\subsection*{Measurement and results}

In order to evaluate the DCR, it is necessary to perform measurements in two different setups.
First, a setup in which the detector is disconnected from the source and covered entirely in darkness.
In such a setup, it can be assumed that no photons from the surrounding environment are striking the detector.
This allows for the measurement of the DCR only triggered by the electronics noise depending on the detector's operating mode.

Depending on the operation mode means, the probability a signal is triggered by the detector raises if the bias current increases,
hence the difference $\Delta I= I_c - I_B$ to the critical current decreases.
This is due to the lower current gap its needed exceeding the critical current.

The measurement setup for this consists out of the detector with the protection cap on the output port of the detector.
This configuration represents as described the most shielded environment from external light sources and serves as the reference
value for optimal DCR values.
The measurement was conducted by sweeping the bias current from 0 to 35 µA in 0.1 µA increments with an
integration time of the count rates for 200 ms.
The result is displayed in fig \subref{fig: DCR_black_box_to_exp}.


The second setup involves connecting the detector to the faint laser source of section \ref{sec:characteristics_faint laser sources}.
This is done in order to check how isolated the experimental set up.
Moreover, one can use the comparison to the first set up to adjust the setup in order to minimize noise from the
environment.
This allows the optimal experimental configuration for the highest signal-to-noise ratio.
For this it is assumed that the first setup has the lowest DCR.
Again the bias current was sweeped from 0 to 35 µA in 0.1 µA increments with an integration time of the count rates for 200 ms.
In fig \subref{fig: DCR_black_box_to_exp} the results are shown for optmized and non optimized case.

At the initial, non optimized configuration an optical fibre was linked to the detector's output channel and the
experiment's output connector.
The laser source was turned off, so no signal noise except the described noises above trigger detection events.
Afterwards, the optimized realization, the same connection was maintained, but the setup was fully protected by
a self-constructed black box as showed in fig \ref{fig: self_build black box}.
Furthermore, the optical fibre was enclosed within an aluminium foil layer to prevent photons from coupling through the fibre's cladding.

As expected, the DCR raises in each case in \subref{fig: DCR_black_box_to_exp} and \subref{fig: DCR_Cap_on} with decreasing difference $ \Delta I= I_c - I_B$
due to raising probability that weak electronic noises trigger a signal.
The orange course in \subref{fig: DCR_black_box_to_exp} demonstrates that in the absence of protection, a significant number of photons from the environment
are able to enter to the detector through various potential pathways like the fibre cladding and the coupling connection to the experiment.\\
In contrast, the fully protected setup, shown by the green course, exhibits an analogous course to the setup with the cap
on fig \ref{fig: DCR_Cap_on}.
The results show, that the DCR of the coupled and protected setup is comparable to the DCR with a cap on.
The peaks in the green course at $\approx 3$ and $\approx 14$uA are artifacts resulting from some leakages in the protection.
Nevertheless, these leakages are not substantial when viewed in the context of the total photon count,
particularly when considering the anticipated photon rates from the faint weak laser source operating
in the high kHz and MHz frequency ranges.

Lastly, one can conclude from the investigation of the DCR of channel 1, that all further measurements have to be done
at a bias current of $ I_B < \approx 32$uA.
To get a final figure of merit for the DCR of channel 1 , all counts up to the high raise are averaged and a
measurement error of 1 Hz was estimated.
The final DCR of this channel yields $(0.5 \pm 1) Hz$.
Thus, measurements up to the predetermined limit are accompanied by a DCR of $(0.5 \pm 1) Hz$.

\begin{figure}
    \begin{subfigure}[t]{.5\textwidth}
        \centering
        \includegraphics[width=\linewidth]{figs/HQO_20240708_DCR_cap_on_Channel_1_Ba_thesis}
        \caption{}
        \label{fig: DCR_Cap_on}
    \end{subfigure}
    \begin{subfigure}[t]{.5\textwidth}
        \centering
        \includegraphics[width=\linewidth]{figs/HQO_20240708_DCR_Channel_1_Ba_thesis}
        \caption{}
        \label{fig: DCR_black_box_to_exp}
    \end{subfigure}
    \caption{\subref*{fig: DCR_Cap_on} Dark count rate (DCR) where the output channel of the detector which was protected by an aluminium cap.
    \subref*{fig: DCR_black_box_to_exp} Dark count rate (DCR) of both measurement realizations, compared to the DCR with the cap on.
  }
\end{figure}


\FloatBarrier

\section{Recovery time}\label{sec:recovery-time}
The concept of the recovery time is visually depicted in fig \ref{fig:Recovery_time}.
When a photon hits the detector and is absorbed, the efficiency of the detector drops to zero and no further photons can
be measured for a certain period of time.
This elapsed time is called the dead time $\tau_{d}$.
The efficiency then rises again to its original device efficiency $\eta_{G}$.
This period is called the reset time $\tau_{r}$.
The characteristic curve between this two times forms the start of the increase to full efficiency.
Moreover, the course to the original device efficiency $\eta_{G}$ is not as depicted in fig \ref{fig:Recovery_time} and serve
only for visualization purposes.
Finally, the sum $\tau_{rec} = \tau_{r} + \tau_{d}$ of both times forms the recovery time $\tau_{rec}$.

The recovery time is important because it determines the rate the detector can detect photons.
As lower the recovery time, as higher the counting rate.

\begin{figure}[H]
 \centering
 \includegraphics[width=0.8\textwidth]{figs/HQO_20240712_recoverytime_visualized_engl}
 \caption{Schematic efficiency curve for the detection of a photon\cite{shalm_single-photon_2013}. On the Y axis is the
 efficiency $\eta$, where $\eta_{G}$ is the device efficiency. On the X axis is the time course of the efficiency}.
 \label{fig:Recovery_time}
\end{figure}

\FloatBarrier

\subsection*{Measurement and results}
In this work the recovery time measurement of this detector is based on an autocorrelation method of a continuous
wave laser source (faint laser source) and was performed already by other groups \cite{autebert-2020,miki-2017}.
The measurement was conducted with the setup shown in fig \ref{fig: recovery_time_setup}.
The raw analog signals from the detector were directly transmitted to a time tagger unit (Time Tagger 20)
by Swabian instruments (\href{https://www.swabianinstruments.com/time-tagger/}{link})
with self-adjustable trigger voltages, a device deadtime of 6ns and a maximal counting rate of 9MHz.

\begin{figure}[H]
 \centering
 \includegraphics[width=0.8\textwidth]{figs/HQO_20240712_recoverytime_setup}
 \caption{Experimental setup for measuring the recovery time. Optical setup of "faint laser source in black box" is depicted in
 \ref{fig: faint_laser_source_full_set_up}}.
 \label{fig: recovery_time_setup}
\end{figure}

This unit enabled the tagging of incoming signals with a time tag, as implied by its name.
Subsequently, the tags were used to process the time distances between all signals.
The histogram of these distances provide an autocorrelation in time (here not normalized).
The autocorrelation was measured for one channel for four different bias currents (26, 28, 31 and 31.2uA) and
trigger voltages (300, 400, 500 and 600mV).

These measurements were done to determine the recovery time and analyze it dependencies.
Results of the autocorrelation from two out of four bias current dependencies are presented
in fig \ref{fig: recovery_time_measurement_31_2_and_26uA}.
The remaining other two bias current dependencies are shown in the appendix \ref{sec:app}.

\begin{figure}[H]
  \begin{subfigure}[t]{.5\textwidth}
    \includegraphics[width=\linewidth]{figs/HQO_20240708_Deadtime_Channel_1_Bias_31_2uA_trigg_300-600mV_thesis}
    \caption{}
    \label{fig: recovery_time_measurement_31_2uA}
  \end{subfigure}
  \hfill
  \begin{subfigure}[t]{.5\textwidth}
    \includegraphics[width=\linewidth]{figs/HQO_20240708_Deadtime_Channel_1_Bias_26uA_trigg_300-600mV_thesis}
    \caption{}
    \label{fig: recovery_time_measurement_26uA}
  \end{subfigure}
  \caption{\subref*{fig: recovery_time_measurement_31_2uA} Autocorrelation of distances between signals for $I_{B} = 31.2uA$ and \subref{fig: recovery_time_measurement_26uA} $I_{B} = 26uA$}
  \label{fig: recovery_time_measurement_31_2_and_26uA}
\end{figure}


The results of the autocorrelation show three major trends.
First, for low trigger voltages, the dead time is longer and decrease for increasing trigger voltages.
This can be seen for the lower and higher bias current.
The reason for this behavior can be explained best by looking at an exemplary raw analog signal (see fig \ref{fig: analog_signal_31_2_uA_double_peak})
of two consequent pulses.
One can see two pulses, where the second signal comes close ($\approx 20ns$) after the first one.
So it is a pulse on the falling edge of the first one.
Physically, that means, while the current in the superconducting wire wasn't fully at the default bias current of
31.2uA, a second photon, during $\tau_{r}$ already hit the detector and got absorbed.
If the trigger is below 500mV the time tagger will count this signal as one count, since the second pulse came when
the remaining voltage of the wire was above 500mV.
However, if the trigger is over 500mV both consecutive pulses will be counted by the time tagging unit as single counts.
This allows more individual pulses to be counted separately with small gaps as close as the deadtime between them.
However, only with a lower efficiency as depicted in \ref{fig: recovery_time}, since in the time regime
$\tau_{r}$ the nanowire has not fully recovered yet.
Therefore, we see this earlier raise in fig \ref{fig: recovery_time_measurement_31_2_and_26uA}.

\begin{figure}
 \centering
 \includegraphics[width=0.5\textwidth]{~/sciebo/Bilder_BA_MaxR/HQO_20240710_Ana_signal_30uA_tv_600mV_double_single}
 \caption{Analog signal screenshot of an oscilloscope from channel 1 for a bias current of 31.2uA and a trigger of 600mV.
 X-axis: time in 50ns steps (straight vertical yellow lines), Y-axis voltage in 500mV steps (straight horizontal yellow lines)}
 \label{fig: analog_signal_31_2_uA_double_peak}
\end{figure}

Secondly, for the lower bias current (26uA), the slope tends to decrease towards the end of the saturation point.
Moreover, the courses of the raising counts for each trigger voltage are converging towards each other earlier.
In comparison, with a bias current of 31.2uA the four different courses are separate till they reach their peak.
This can be explained by the fact that, depending on the bias current, the triggered resistance caused by a
photon hitting the superconducting nanowire is different.
This behaviour can also be verfied by the different pulse heights of the signals in fig:
\ref{fig: analog_signals_comparison}.
According to Ohm's law, a low bias current corresponds to lower voltage pulses and vice versa, for the same resistivity.
This is also comprehensible by Ohm's law.
Due to the lower pulse, the regime, where one pulse can be resolved by a trigger voltage of 600mV but not 500mV
becomes smaller.
Hence, one can see that the counts per bin equalize for increasing time distances.

The third trend concerns the small peak, in the range of 24-27ns as can be seen clearly in the course with
a bias current of 31.2uA \ref{fig: recovery_time_measurement_31_2uA}.
This small peak is more suppressed with a lower bias current and higher trigger voltages.
Closer examination of this trend are shown in the appendix \ref{sec:app}.
A plausible solution for this behaviour can be understood with the fig \ref{fig: Oscillating_bias_current}.

\begin{figure}
 \centering
 \includegraphics[width=0.8\textwidth]{~/sciebo/Bilder_BA_MaxR/HQO_20240710_oscilating_bias_current}
 \caption{Scetch of brief oscillating current and subsequent levelling to the bias current}
 \label{fig: Oscillating_bias_current}
\end{figure}

Following the detection of a current, the course of the current does not proceed directly and precisely to the bias current.
Instead, it oscillates for a brief period and then rapidly reaches equilibrium.
In the small oscillating period, the Dark count rate raises exponentially (seen in fig: \ref{fig: DCR_black_box_to_exp})
and adds to the detected counts.
It can be concluded that the optimal recovery time is achieved when the detector is operated in closeness to the bias current.
This enables the generation of a higher pulse, which in turn results in a steeper reset time, a smaller dead time,
and consequently, a shorter recovery time.

\textcolor{red}{DISCLAIMER: ALL THE RESULTS AND METHODS ARE NOT FINALIZED AND WILL BE ADAPTED, IF METHOD IS AGREED}

Finally, a reasonable trigger point for a bias current of 31.2uA is 600mV, which yields to the lowest recovery time.
The calculation of the $\tau_{rec}$ is done, by calculating the average of counts per bin from 30uA till the end.
Here, I chose 30uA as the starting point, because from this point a clear saturation and a constant course is
visible.
Furthermore, I fitted a function to estimate the point were $50\%$ and $90\%$ of the full efficiency is reached.
Moreover, the raise of the function for the dividing line between dead and reset time is calculated by the point where
the gradient of the fitting function is not zero any more.
The calculated points are visualized in fig: \ref{fig: recovery_time} and the final recovery time is
$\tau^{90\%}_{rec} = (21.21 \pm 0.34) ns$, where $\tau_{d} = (14.13 \pm 0.14) ns$ is the dead time and $\tau_{r} = (7.08 \pm 0.36) ns$  the reset time.
Moreover, the time the detector is back at efficiency of $\eta_{sde} = 50 \%$  is $\tau^{50\%}_{rec} = (18.21 \pm 0.24) ns$.

\begin{figure}
 \centering
 \includegraphics[width=0.5\textwidth]{figs/HQO_20240710_Deadtime_Channel_1_Bias_31_2uA_trig_600mV_thesis}
 \caption{Histogram of distances between signals for $I_{B} = 31.2uA$ and 600mV. H- and v-lines indicate the dead-;
 reset- and recovery time}
 \label{fig: recovery_time}
\end{figure}

\FloatBarrier

\section{Efficiency}
There are three types of efficiencies that describe independent loss processes in single photon detection.
An efficiency can be equated with the probability that a quantum mechanically process under consideration will occur.\\
These three efficiencies are the coupling efficiency ($\eta_{\text{K}}$),
the absorption efficiency ($\eta_{\text{A}}$) and the registration efficiency ($\eta_{\text{R}}$).
The graph \ref{fig: single_efficiency_terms} shows schematically where the different loss in the detection
process appears.
When a photon is sent to a detector via an optical fibre, not all photons can be coupled into the
fibre.
The probability of coupling is called the \textit{coupling efficiency}.
When photons hit the detector, there is always a probability that the photon will not be absorbed by the detector.
This is due to material and symmetry properties.
This is described by the \textit{absorption efficiency}.
Finally, there is always a probability that the photon will not be registered by the measuring electronics.
This is expressed with the \textit{Registration efficiency}.

\begin{figure}
    \centering
    \includegraphics[width=0.8\textwidth]{figs/HQO_20240712_systemd_detection_efficiency_visualized_engl}
    \caption{Sketch of the components in the detector setup where photonlosses appear and consequently a propability
     ($\eta_{\text{K}}$, $\eta_{\text{A}}$ or $\eta_{\text{R}}$) has to be considered.}
    \label{fig: single_efficiency_terms}
\end{figure}

In literature, these terms are summarized in two general efficiency terms: the device detection efficiency
($\eta_{\text{dde}} = \eta_{\text{A}} \cdot \eta_{\text{R}}$) and the system efficiency
($\eta_{\text{sde}} = \eta_{\text{A}} \cdot \eta_{\text{R}} \cdot \eta_{\text{K}}$) \cite{natarajan-2012}.
The device's efficiency $\eta_{\text{dde}}$ corresponds to the efficiency of the device itself, when photons are sent
to the detector in a free environment without any fibre coupling.
In this way only the efficiency of the elements detecting photons are considered.
The system detection efficiency $\eta_{\text{sde}}$ takes the coupling losses to the optical fibre into account.
This is the case if the detector is connected to a fibre, as the device properties or the experiment does not allow
photon detection in a free environment.\\

\subsection*{Measurement and results}

\textcolor{red}{DISCLAIMER: ALL THE RESULTS AND METHODS ARE NOT FINALIZED AND WILL BE ADAPTED, IF METHOD IS AGREED}

In the given setup, only the system detection efficiency c is measured, because the detector is
already prebuild with a fixed coupling to a fibre.
This internal fibre is connected to a fibre to fibre port (FC/PC to FC/PC).
Through this one, one can connect the detector with an external fibre and send photons from the experiment
to the detector.

The system detection efficiency $eta_{\text{sde}}$ is measured in different ways, each pointing out a different dependency.
Each measurement was done in the setup explained in part \ref{sec:SNSPD_setup}.
In addition, the order of the measurements is a relevant factor, as the conclusions drawn for one measurement
influence the preceding measurements.

In conclusion, it is first necessary to align the polarization of the laser light with the slow axis of the fibre
connected to the output port of the detector.
Along the manual the coupled light needs to polarized along the slow axis of the fibre \cite{manual_single_quantum_snspd}.
This is explained by the technical fact that only the slow axis of the fibre is coupled to the output port of
the detector.
The reason for this preselection of polarization is the related maximum $\eta_{A}$, explained
in part \ref{sec:SNSPD_working_principle}.\\

By adjusting the laser beam linear with a quarter-wave plate first and rotating the half-wave plate
in 10 degree steps afterwards, the polarization axis was rotated.
With this, it was possible to find the angle configuration were the maximum of light was coupled to the slow axis of the fibre.
This is important since measuring subsequent efficiency measurements aligned to a different axis would
put a systematic downshift on the true efficiency of the detector.\\
In the figures \ref{fig: angle_dependend_countrate_sde} the count rate and the resulting system detection efficiencies are depicted.
For preceding measurements the polarization was aligned to an angle, where a maximum of $eta_{\text{sde}} = 81.652 \pm 10.872$ was reached.

\begin{figure}
  \begin{subfigure}[t]{.5\textwidth}
    \includegraphics[width=\linewidth]{figs/HQO_2024011_countrate_angle_thesis}
    \caption{}
    \label{fig: angle_dependend_countrate}
  \end{subfigure}
  \hfill
  \begin{subfigure}[t]{.5\textwidth}
    \includegraphics[width=\linewidth]{figs/HQO_2024011_sde_angle_thesis}
    \caption{}
    \label{fig: angle_dependend_countrate_sde}
  \end{subfigure}
  \caption{\subref*{fig: angle_dependend_countrate} Angle dependent countrate \subref*{fig: angle_dependend_countrate_sde}
  Angle dependent system detection efficiency $eta_{\text{sde}}$}
\end{figure}

In a second measurement the bias current und trigger voltage dependency was investigated.
For this the bias current was swept from 0 to 35uA in 0.1uA steps and events within 200ms integration
time were counted.

\begin{figure}
    \centering
    \includegraphics[width=0.5\textwidth]{figs/HQO_20240711_sde_bias_current_tv_300_1000_thesis}
    \caption{System detection efficiency for different bias currents and trigger voltages}
    \label{fig: sde_bias_current_tv_300_1000}
\end{figure}

In fig \ref{fig: sde_bias_current_tv_300_1000} one can see that at lower trigger voltage of 300mV the count rate oscillates
a bit, which again corresponds likely to the increased dark count rates as explained in \ref{fig: Oscillating_bias_current}.
Furthermore, one can see that for lower trigger voltages the detected counts for lower bias currents are higher.
This is due to the consideration of lower voltage pulses in lower bias current regimes if the trigger voltage is low.
Hence, with a lower trigger voltage one can count already signals, however, with a very low system detection efficiency
$\eta_{sde}$.
Another behaviour is the saturation, which is reached by each trigger voltage configuration
at around $I_{Bias} \approx 20uA$ .
Here the maximum count rate is reached and the efficiency course continues constant without any gradient.
At the end at a bias current of  $I_{Bias} \approx 32uA$ the efficiency drops.
This is because the critical current is reached.
After this, along the \cite{manual_single_quantum_snspd}, the detector stops sending countable analog signals,
which corresponds to an efficiency of zero.

Finally, measurements for different count rates were done, to determine the bandwidth where the photons are detected with
the highest efficiency.
The count rates were varied by using different ND filtersa and was done by putting together different ND filter
combinations in order to get different OD values and therefore different count rates.
To avoid the oscilation of count rates near the critical current as seen by the blue course in fig \ref{fig: sde_bias_current_tv_300_1000}
a trigger voltage of 750mV is used.

\begin{figure}
  \begin{subfigure}[t]{.5\textwidth}
    \centering
    \includegraphics[width=\linewidth]{figs/HQO_20240711_sde_bias_current_750_thesis}
    \caption{}
    \label{fig: sde_bias_current_750_thesis}
    \end{subfigure}
  \hfill
  \begin{subfigure}[t]{.5\textwidth}
    \centering
    \includegraphics[width=\linewidth]{figs/HQO_20240711_sde_photon_rate_750_thesis}
    \caption{}
    \label{fig: sde_count_rates_750_thesis}
  \end{subfigure}
  \caption{\subref*{fig: sde_bias_current_750_thesis} Course of system detection efficiency for different bias current and count rates \subref*{fig: sde_count_rates_750_thesis} Course of system detection efficiency for different count rates}
\end{figure}

In fig \ref{fig: sde_bias_current_750_thesis} one can see that the system detection efficiency is decreasing with increasing
count rates.
Moreover, only for the measurement of 1.391MHz one gets a range of bias currents where the efficiency is constant.
All the other measurements with higher count rates show a continues increase in $\eta_{\text{sde}}$ up to the point the
detector shut down.
Furthermore, the shut-down happens earlier, when the count rate is higher.
This dynamics follows from the fact that the detector is not able to recover in time, when the count rate is too high.
Hence, the critical temperature is reached earlier, when the count rate is raising and the detections is shut down.

In fig \ref{fig: sde_count_rates_750_thesis} the maximum system detection efficiency for each count rate is shown.
One see clearly a downward trend for raising count rates.
Moreover, the measurement shows the saturating and therefore a maximal system detection efficiency of $(87.308 \pm 9.159) \%$
for channel 1. This $\eta_{\text{sde}}$ is reached at a count rate of $1.391 \pm 0.32$MHz.


\FloatBarrier

\section{Discussion}

%#TODO - Freitag
% durch lesen und verbessern
% Diskussion schreiben




%Influence of timing jitter on efficiency and recovery time.
%- No Afterpulsing
%- Temperature 2.9 instead of 2.5Í
%- Photon counters formel für NQO Excelitas SPCS - Note #7

