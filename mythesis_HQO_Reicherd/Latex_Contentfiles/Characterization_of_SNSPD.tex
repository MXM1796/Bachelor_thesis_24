% !TEX root = mythesis.tex

%==============================================================================
\chapter{Characterization of a SNSPD by Single Quantum}
\label{sec:SNSPD_Characterization}
%==============================================================================

In literature, four central characteristics have emerged to quantify the quality of single photon detectors and
make their performances comparable \cite{natarajan-2012, hadfield-2009}.
These characteristics are the system detection efficiency ($\eta_{\text{sde}}$), the dark count rate (DCR), the recovery time ($\tau_{\text{recovery}} = \tau_{\text{rec}}$)
and the timing jitter ($\Delta t$).
In this chapter, I investigate the detector efficiency ($\eta_{\text{sde}}$), the dark count rate (DCR) and the recovery time ($\tau_{\text{rec}}$).
At the end, the results are summarized.
The timing jitter ($\Delta t$) was not analyzed due to missing equipment like a fast pulse laser and time constraints.

\section{Dark count rate}\label{sec:dark-count-rate}
The DCR is the rate of false positive events which are not intentionally generated from the source (here the faint laser source).
It is measured in counts per second and can be caused by statistical fluctuations in the measurement electronics.
A low DCR is important for a high signal-to-noise ratio and means easy interpretable results which are not distorted by noise \cite{wikipedia-contributors-2024}.

In the context of SNSPDs, the DCR is dependent on the bias current applied to the nanowire.
This is due to the fact that if the bias current approaches the critical current, less current $\Delta I= I_c - I_B$ is needed to exceed the critical current .

Therefore, thermal electronic fluctuations, close to the critical current will cause a breakdown of the superconducting state and hence more dark counts are detected.
It is important to perform DCR measurements first in the characterization process because it determines the
bias current limit, where general measurements are not distorted by high DCR.

\subsection*{Measurement and results}

In order to evaluate the DCR, it is necessary to perform measurements in two different setups.
In the first setup (Setup 1) no optical fibre is connected to the detector and the detector port is covered with a protection cap.
In such a setup, it can be assumed that no photons from the environment strike the detector.
This allows for measurements of the DCR only triggered by thermal electronic fluctuations and depending on the detector's bias current.
This configuration represents the most shielded environment from external light sources and serves as reference
value for the lowest DCR values.
The measurement was conducted by sweeping the bias current from 0 to 35$\si{\micro \A}$, at a trigger voltage of
200$\si{\milli \V}$ in 0.1$\si{\micro \A}$ increments with an integration time of 200ms at each step.

In the second setup (Setup 2) the detector is connected with an optical fibre to the faint laser source (introduced in section \ref{sec:characteristics_faint laser sources}).
The set up stands in free space and the laser source was turned off, so no photons from the source were sent to the detector.
The orange curve in \ref{fig: DCR_black_box_to_exp} demonstrates that in the absence of any protection, a significant number of photons from the environment
are able to enter the detector through various potential pathways like the fibre cladding or the coupling connection to the laser setup.\\

To reduce dark counts due to ambient light a Blackbox was constructed (as discussed in section \ref{sec:SNSPD_setup}) that covers the laser setup.
Further, the optical fiber was wrapped in aluminum foil to prevent ambient light coupling to the core through the cladding (seen in figure \ref{fig: fibre_connection_real_life}).
Once more, the bias current was swept from 0 to 35 µA in 0.1 µA increments with an integration time of the count rates for 200 ms.
In figure \ref{fig: DCR_black_box_to_exp} the results for this optimized case are shown by the green curve (Setup 2 - optimized).

The measurement results for the optimized case show that the DCR of the coupled and protected setup are the same as the
DCR with the protection cap on.
The peaks in the green curve at $\approx 3\si{\micro \A}$ and $\approx 14 \si{\micro \A}$ are artifacts resulting from some leakages in the protection.
Nevertheless, these leakages are not substantial when viewed in the context of the total photon count rate.
Particularly, when considering anticipated photon rates from the faint laser source in the high kHz and MHz range.

One can conclude from the investigation of the DCR of channel 1 that all further measurements should be done
at a bias current below $ I_B < \approx 31.2\si{\micro \A}$.
As mentioned above, the final DCR is depending on the bias current working point.
Therefore, five 60s measurements were done for three different bias currents (24$\si{\micro \A}$, 28$\si{\micro \A}$ and 31.2$\si{\micro \A}$).
The averaged results for channel one of the detector yield a DCR of:

\begin{align}
    &DCR_{24\si{\micro \A}} = (1.40 \pm 1.02) \si{\Hz} \\
    &DCR_{28\si{\micro \A}} = (1.20 \pm 0.98) \si{\Hz} \\
    &DCR_{31.2\si{\micro \A}} = (1.40 \pm 1.36) \si{\Hz}
\end{align}

where the error is the standard deviation of the mean.
These results are in agreement to the companies specifications of: DCR <  5 \cite{tech_sheet_single_quantum}.

\begin{figure}
    \centering
    \includegraphics[width=0.8\linewidth]{figs/HQO_20240708_DCR_cap_on_Channel_1_Ba_thesis}
    \caption{Channel 1 DCR measurements for different bias currents at a trigger voltage of 200MHz.
    The blue curve shows the DCR with a cap on the output port of the detector.
    The orange curve shows the DCR with an optical fibre connected to the detector and the experiment (in free space).
    The green curve shows the DCR with an optical fibre connected to the detector and the experiment (in a Blackbox).
    Additionally, the optical fiber was wrapped in aluminum foil.}
    \label{fig: DCR_black_box_to_exp}
\end{figure}

\FloatBarrier

\section{Recovery time}\label{sec:recovery-time}
The concept of the recovery time is visually depicted in figure \ref{fig:Recovery_time}.
When a photon hits the detector and is absorbed, the efficiency ($\eta_{\text{sde}}$) of the detector drops to zero and \textit{no further photons} can
be measured for a certain period of time.
This elapsed time is called the dead time ($\tau_{\text{dead}} = \tau_{\text{d}}$).
The efficiency then rises again to the original system detection efficiency ($\eta_{\text{sde}}$).
This period is called the reset time ($\tau_{\text{reset}} = \tau_{\text{r}}$).
The vertical dashed line forms the starting point where the efficiency rises again to the system detection efficiency ($\eta_{\text{sde}}$).
Finally, the sum of both times forms the recovery time ($\tau_{\text{rec}} = \tau_{\text{r}} + \tau_{\text{d}}$).

The recovery time is important because it determines the temporal resolution of the detector \cite{hadfield-2009}.
The temporal resolution describes the minimal time interval where the detector can distinguish between two photons.
The lower the recovery time, as higher is the temporal resolution, or maximal count rate of the detector.

\begin{figure}
 \centering
 \includegraphics[width=0.8\textwidth]{figs/HQO_20240712_recoverytime_visualized_engl}
 \caption{Schematic efficiency curve for the detection of a photon\cite{shalm_single-photon_2013}. On the Y axis is the
 efficiency $\eta$, where ($\eta_{\text{sde}}$) is the  system detection efficiency. On the X axis is the time course of the efficiency.
 The efficiency trajectory does not align with the real efficiency trajectory, and serves only for a visualization purpose.}
 \label{fig:Recovery_time}
\end{figure}

\FloatBarrier

\subsection*{Measurement and results}
In this work, the recovery time of the detector is determined through an autocorrelation method based on a
continuous wave laser source (a faint laser source).
This technique has been previously employed by other research groups
\cite{autebert-2020,miki-2017}.
The measurement was conducted with the setup shown in figure \ref{fig: recovery_time_setup}.
The analog signals from the detector were directly guided with a SMA cable (SubMiniature version A) to a time tagger unit (Time Tagger 20)
by Swabian instruments.
The time tagger unit has self-adjustable trigger voltages, a device dead time of 6ns and a maximal counting rate of 9MHz.
This unit enabled the tagging of incoming signals with a time tag.
Subsequently, the tags were used to process the time distances between all signals.

\begin{figure}
 \centering
 \includegraphics[width=0.8\textwidth]{figs/HQO_20240712_recoverytime_setup}
 \caption{Schematic illustration of experimental setup for measuring the recovery time. Optical setup of the "faint laser source in a Blackbox" is depicted in
 \ref{fig: faint_laser_source_full_set_up}.}
 \label{fig: recovery_time_setup}
\end{figure}

Histograms of the time distances between photons provide a distribution of the time distances, which
yields an autocorrelation of the photon distances.

In order to determine the recovery time and analyze it dependencies on the bias current and the trigger voltage,
the autocorrelation was measured for four different bias currents (25$\si{\micro \A}$, 27$\si{\micro \A}$, 29$\si{\micro \A}$ and 31.2$\si{\micro \A}$) and
trigger voltages from 300$\si{\milli \V}$ to 900$\si{\milli \V}$ in 100$\si{\milli \V}$ steps for a measurement period of 30s each.
The results for a fixed bias current of 31.2$\si{\micro \A}$ are shown in figure \ref{fig: recovery_time_measurement_31_2uA}
and for 25$\si{\micro \A}$ in \ref{fig: recovery_time_measurement_25uA}.
The other measurement results are presented in the appendix \ref{sec:Recovery time measurements_appendix}.\\

\begin{figure}
  \begin{subfigure}[t]{.5\textwidth}
    \includegraphics[width=\linewidth]{figs/HQO_20240723_recovery_time_Channel_1_Bias_31_2uA_trigg_300-900mV_thesis}
    \caption{}
    \label{fig: recovery_time_measurement_31_2uA}
  \end{subfigure}
  \hfill
  \begin{subfigure}[t]{.5\textwidth}
    \includegraphics[width=\linewidth]{figs/HQO_20240723_recovery_time_Channel_1_Bias_25uA_trigg_300-900mV_thesis}
    \caption{}
    \label{fig: recovery_time_measurement_25uA}
  \end{subfigure}
  \caption{Autocorrelation of distances between two photon detection events for \subref*{fig: recovery_time_measurement_31_2uA} $I_{B} = 31.2uA$ and \subref*{fig: recovery_time_measurement_25uA} $I_{B} = 25\si{\micro \A}$.
  Measurements were done with an input photon rate of 0.531 MHz.
  The X-axis represents the time distance between two signals in 1ns steps and the Y-axis the counts per "time distance bin".}
\end{figure}

\newpage

The results of the autocorrelation show three major features.
First, for low trigger voltages, the dead time is longer and decrease for increasing trigger voltages.
This is true for both, the lower and higher bias current.
The reason for this behavior can be explained best by looking at an exemplary analog signal (see figure \ref{fig: analog_signal_31_2_uA_double_peak})
of two consequent pulses.
In figure \ref{fig: analog_signal_31_2_uA_double_peak} one can see the peaks of two consecutive detection signals,
where the second pulse starts ($\approx 25ns$) before the falling edge of the first pulse ends.
Physically, that means, that before the first signal spike has fully decayed a second photon, already hit the detector,
got detected and produced a second spike.

Therefore, if the trigger is \textit{below} a certain threshold $\text{Tresh}^{*}$ the time tagger will count this signal as one count, since the second pulse came when
the remaining voltage of the wire was still \textit{above} $\text{Tresh}^{*}$.
If the trigger is \textit{above} $\text{Tresh}^{*}$ both pulses will be counted.
This allows counting of successive events with smaller time delay between them and therefore reduces the perceived recovery time.
Due to time constraints, no further investigation was done to record double pulses with a distance below ($\approx 25ns$).

\begin{figure}
 \centering
 \includegraphics[width=0.8\textwidth]{figs/HQO_20240725_analog_signal_double_pulse_31.2uA_tv_600mV}
 \caption{Analog signal of three consecutive pulses recorded with a Lecroi oscilloscope with trigger at 600$\si{\milli \V}$.
 Detector was set up to a bias current of 31.2$\si{\micro \A}$ and the faint laser source was attenuated down to 0.531 MHz.
 The time resolution of the oscilloscope was 500MHz.}
 \label{fig: analog_signal_31_2_uA_double_peak}
\end{figure}

Second, for the lower bias current (25$\si{\micro \A}$), the rising curves for each trigger voltage converge earlier
in comparison to the bias current of 31.2 $\si{\micro \A}$.
At the bias current of 31.2 $\si{\micro \A}$ the four different curves remain distinct until they reach their peak.
This can be attributed to the differing pulse heights, dependent on the bias current.
According to Ohm's law, for the same resistivity, a lower bias current corresponds to lower voltage pulses and vice versa.
Due to the lower pulse, the regime, where pulses can be resolved by a trigger voltage of 600$\si{\milli \V}$ but not 500$\si{\milli \V}$
becomes smaller.
The different pulse heights can also be verified by the recorded analog signals shown in figure \ref{fig:grid_analog_pulse_appendix_1}.

The third interesting feature is the peak at 24$\si{\nano \s}$-27$\si{\nano \s}$ in all
autocorrelation measurements as can be seen in figure \ref{fig: recovery_time_measurement_31_2uA}, \ref{fig: recovery_time_measurement_25uA} and \ref{fig:grid_recovery_time_appendix}.
This behavior can be understood by taking into account that the bias current needs a finite amount of time to reach
its target value once the superconductivity is restored and will also overshoot a bit after reaching the target value.
A sketch of the idea of the expected behavior is shown in figure \ref{fig: Oscillating_bias_current}.

\begin{figure}
 \centering
 \includegraphics[width=\textwidth]{~/sciebo/Bilder_BA_MaxR/HQO_20240710_oscilating_bias_current}
 \caption{A sketch of the assumed bias current behaviour. Once the bias current has been reached, the current undergoes a brief oscillation before stabilising at the bias current level.}
 \label{fig: Oscillating_bias_current}
\end{figure}

The current during the nanowire recovery does not proceed directly and precisely to the bias current.
Instead, it oscillates for a brief period and then rapidly reaches equilibrium.
An overshooting in the current oscillation might cause a breaking of the
superconductivity leading to a time correlated rapid increase in the dark count rate (seen in figure \ref{fig: DCR_black_box_to_exp}).

Overall, a reasonable working point would be at a bias current of 31.2$\si{\micro \A}$ and a trigger voltage of 600$\si{\milli \V}$, since it yields the best
compromise between a short recovery time and a low additional oscillation signal as discussed above.

Finally, the calculation of the recovery time ($\tau_{\text{rec}}$) is done by measuring 10 times the autocorrelation for
this configuration.
Afterwards, analogue bin values are averaged and the error is calculated by the standard deviation.

To calculate the recovery time, first the bin counts from 28$\si{\nano \s}$ till the end of the measurement period were averaged.
The time 28$\si{\nano \s}$ is chosen as the starting point because from this point a constant curve, hence maximal detection efficiency, is assumable (saturation value).
Based on this saturation value a $05\%$, $50\%$ and $90\%$ bin count threshold is calculated.
The threshold is then used to define the dead time $\tau_{\text{dead}}$ as the time,
where the detector is reached $05\%$ of the saturation value.
Furthermore, the reset time $\tau_{\text{reset}}$ is defined as the time from $05\%$ till $90\%$ of the saturation value.
The calculated points are visualized in figure \ref{fig: recovery_time}.
The recovery time is then calculated by the sum of the dead and reset time $\tau_{\text{rec}} = \tau_{\text{dead}} + \tau_{\text{reset}}$.
Errors are calculated by using the mean root square of the corresponding standard deviation of the dead and reset time.\\

The final results are $\tau^{90\%}_{rec} = (17.156 \pm 0.045) ns$,  where
$\tau_{\text{d}} = (13.950 \pm 0.050) ns$ is the dead time and $\tau_{\text{r}} = (3.206 \pm 0.447) ns$  the reset time.
Moreover, the time the detector is back at efficiency of $\eta_{\text{sde}} = 50 \%$ is $\tau^{50\%} = (14.900 \pm 0.082) ns$.
The dead time also defines a maximal possible count rate of  $\mathcal{R}_{\max} =  \frac{1}{\tau_{\text{d}}}= (71.68 \pm 0.25)\si{\mega \Hz}$.

\begin{figure}
    \centering
    \includegraphics[width=0.7\textwidth]{figs/HQO_20240710_Deadtime_Channel_1_Bias_31_2uA_trig_600mV_thesis}
    \caption{Autocorrelation of averaged distances between two photon detection events for $I_{B} = 31.2\si{\micro \A}$ and 600$\si{\milli \V}$.
    The X-axis represents the time distance between two signals in 1ns steps and the Y-axis the counts per bin.
    The dashed h- and v-lines indicate the thresholds used for the recovery time evaluation.}
    \label{fig: recovery_time}
\end{figure}

\FloatBarrier

\section{Efficiency}\label{sec:efficiency}
Efficiency is a way of measuring how likely a process is to happen.
There are three types of efficiencies that describe independent loss processes in single photon detection,
the coupling efficiency ($\eta_{\text{C}}$), the absorption efficiency ($\eta_{\text{A}}$) and the registration efficiency ($\eta_{\text{R}}$).
The graph \ref{fig: single_efficiency_terms} shows schematically where the different losses in the detection
process appear.
When a photon is sent to a detector via an optical fibre, not all photons can be coupled into the
fibre.
The probability of coupling is the so called \textit{coupling efficiency}.
When photons hit the detector, there is always a probability that the photon will not be absorbed by the detector.
This is due to material and symmetry properties in the design of the superconducting nanowire \cite{hadfield-2009}.
This is described by the \textit{absorption efficiency}.
Finally, there is always a probability that the photon will not be registered by the measuring electronics.
This is expressed with the \textit{registration efficiency}.

\begin{figure}
    \centering
    \includegraphics[width=0.8\textwidth]{figs/HQO_20240712_systemd_detection_efficiency_visualized_engl}
    \caption{Sketch of the components in the detector setup where photonlosses appear and consequently a propability
     ($\eta_{\text{K}}$, $\eta_{\text{A}}$ or $\eta_{\text{R}}$) has to be considered.}
    \label{fig: single_efficiency_terms}
\end{figure}

In literature, these terms are summarized in two general efficiency terms: the device detection efficiency
($\eta_{\text{dde}} = \eta_{\text{A}} \cdot \eta_{\text{R}}$) and the system detection efficiency
($\eta_{\text{sde}} = \eta_{\text{A}} \cdot \eta_{\text{R}} \cdot \eta_{\text{K}}$) \cite{natarajan-2012, hadfield-2009}.
The device detection efficiency $\eta_{\text{dde}}$ corresponds to the efficiency of the device itself and
neglects coupling inefficiencies.
This gives an idealized upper bound to the achievable efficiency.
For perfect optical coupling, the device detection efficiency is equal to the system detection efficiency ($\eta_{\text{dde}} = \eta_{\text{sde}}$).
The system detection efficiency $\eta_{\text{sde}}$ takes the coupling losses to the optical fibre into account.
This is the case if the detector is connected to a fibre, as the device properties or the experiment does not allow
photon detection in a free environment.\\

\FloatBarrier

\subsection*{Measurement and results}

In the given setup, only the system detection efficiency $\eta_{\text{sde}}$ is measured, because the detector is
already prebuild with a fixed coupling to a fibre \cite{single-quantum-2022}.
This internal fibre is connected to a single mode fibre to fibre port for the connection type FC/PC.
The connector type FC/PC is used in order to maintain higher efficiency coupling \cite{single-quantum-2022}.
Through this port, one can connect the detector with an external optical fibre and send photons from the experiment
to the detector.

The measured system detection efficiency $\eta_{\text{sde}}$ depends on the photon polarization (see section \ref{subsec:polarization_dependency}),
the applied bias current, the chosen trigger voltage (see section \ref{subsec:trigger_voltage_and_bias_current_dependency}) and
 and the photon rate send to the detector (see section \ref{subsec:photon_rate_dependency}).
These dependencies will be investigated and the system detection efficiency will be determined.
Each measurement was done with the optimized faint laser source setup, as discussed in section \ref{sec:SNSPD_setup}.

\subsubsection*{Polarization dependency}\label{subsec:polarization_dependency}

The measurements were done in a specific order since the conclusions drawn for certain measurements
influence the preceding measurements.
Therefore, it is first necessary to align the polarization of the laser light with the slow axis of the fibre
connected to the output port of the detector.
According to the manual the coupled light needs to the polarized along the slow axis of the fibre \cite{single-quantum-2022}.
This is explained by the fact that the absorption efficiency ($\eta_{\text{A}}$) is maximized when the light is polarized parallel
to the superconducting nanowire of the detector as explained in section \ref{sec:SNSPD_working_principle}.

A combination of a quarter-wave plate $\lambda_4$ and a half-wave plate $\lambda_2$ is used in front of the coupling to the connection fibre
to alter the polarization of the input laser light.
The quarter-wave plate $\lambda_4$ is used to pre compensate the stress induced birefringence of the fiber input
while the half-wave plate $\lambda_2$ is used to rotate the linear input polarization.
The measurement of the polarization was done with the polarization analyzer SK010PA by Schäfter + Kirchhoff.
Furthermore, the laser power input was set to 511.1$\si{\micro \V}$, corresponding to a photon rate of 2.006 $\si{\peta \Hz}$,
and attenuated by three ND filters with a total OD of ($9.574 \pm 0.105 $), in order to make sure the condition of $\text{N(T)} \cdot \eta \ll 1$
as discussed in section \ref{sec:characteristics_faint laser sources}.
Different polarization angles for measuring the count rate of the detector are then realized by rotating the
half-wave plate $\lambda_2$ while the quarter wave plate is kept in its position.
%Final alignment of the polarization are shown in the Appendix \ref{subsec:polarization_alignment_for_system_detection_efficiency_measurments}.

By rotating the half-wave plate $\lambda_2$ in $\theta_{\text{rel}} = (10 \pm 2)\si{\degree}$ steps, the polarization axis was rotated relative to the slow axis of the fibre.
With this, it was possible to find the angle configuration were the maximum of light was coupled to the slow axis of the fibre.
This is important since measuring subsequent efficiency measurements aligned to a different axis would
put a systematic downshift on the true efficiency of the detector.\\
In the figure \ref{fig: angle_dependend_countrate} the count rates are depicted.
For preceding measurements the polarization was aligned to the relative angle of $ (0 \pm 2) \si{\degree}$, where the maximum count rate was reached.
%\begin{figure}[hhh]
%  \begin{subfigure}[t]{.5\textwidth}
%    \includegraphics[width=\linewidth]{figs/HQO_2024011_countrate_angle_thesis}
%    \caption{}
%    \label{fig: angle_dependend_countrate}
%  \end{subfigure}
%  \hfill
%  \begin{subfigure}[t]{.5\textwidth}
%    \includegraphics[width=\linewidth]{figs/HQO_2024011_sde_angle_thesis}
%    \caption{}
%    \label{fig: angle_dependend_countrate_sde}
%  \end{subfigure}
%  \caption{On both X axis the relative angle to the slow axis is depicted.
%  \subref*{fig: angle_dependend_countrate} shows the angle dependent countrate, \subref*{fig: angle_dependend_countrate_sde}
%  shows the angle dependent  $\eta_{\text{sde}}$.
%      Countrate error in \subref*{fig: angle_dependend_countrate} and corresponding $\Delta \eta_{\text{sde}}$ in \subref*{fig: angle_dependend_countrate_sde}
%  are depicted in table \ref{tab:angle_dependend_countrate_results} and \ref{tab:angle_dependend_sde_results} and are caculated according considerations mentioned in \ref{sec:ND_filter_calibration}}
%\end{figure}

\begin{figure}
    \centering
    \includegraphics[width=0.8\textwidth]{figs/HQO_2024011_countrate_angle_thesis}
    \caption{On the X-axis the relative angle to the slow axis is depicted with an estimated error of $2 \si{\degree}$. The Y-axis shows the angle dependent count rates.
      Countrate errors are listed in table \ref{tab:angle_dependend_countrate_results} and caculated according to the calculation mentioned in \ref{sec:ND_filter_calibration_appendix}.}
    \label{fig: angle_dependend_countrate}
\end{figure}

%\begin{figure}[hhh]
%  \begin{subfigure}[t]{.5\textwidth}
%    \includegraphics[width=\linewidth]{figs/HQO_2024011_countrate_angle_thesis}
%    \caption{}
%    \label{fig: angle_dependend_countrate}
%  \end{subfigure}
%  \hfill
%  \begin{subfigure}[t]{.5\textwidth}
%    \includegraphics[width=\linewidth]{figs/HQO_2024011_sde_angle_thesis}
%    \caption{}
%    \label{fig: angle_dependend_countrate_sde}
%  \end{subfigure}
%  \caption{On both X axis the relative angle to the slow axis is depicted.
%  \subref*{fig: angle_dependend_countrate} shows the angle dependent countrate, \subref*{fig: angle_dependend_countrate_sde}
%  shows the angle dependent  $\eta_{\text{sde}}$.
%      Countrate error in \subref*{fig: angle_dependend_countrate} and corresponding $\Delta \eta_{\text{sde}}$ in \subref*{fig: angle_dependend_countrate_sde}
%  are depicted in table \ref{tab:angle_dependend_countrate_results} and \ref{tab:angle_dependend_sde_results} and are caculated according considerations mentioned in \ref{sec:ND_filter_calibration}}
%\end{figure}


\FloatBarrier


\subsubsection*{Trigger voltage and bias current dependency} \label{subsec:trigger_voltage_and_bias_current_dependency}

In a second measurement the trigger voltage and bias current dependency was investigated for different count rates.
For this measurement a different power as input was used because the power drifted over time and could not be reproduced to the power level
used in the polarization alignment.
The new power level was set to 518.1$\si{\micro \V}$, corresponding to a photon rate of 2.034 $\si{\peta \Hz}$.
Furthermore, the ND filters were used for attenuation to reach different count rates in the MHz regime.

In order to investigate the trigger voltage and bias current dependency the bias current was swept from 0 to 35
$\si{\micro \A}$ in 0.1$\si{\micro \A}$ steps and events within 1s integration time were counted.

\begin{figure}
  \begin{subfigure}[t]{.5\textwidth}
      \includegraphics[width=\linewidth]{figs/HQO_20240726_countrate_bias_sweep_tv_3_6_9_thesis_8_92}
      \caption{}
      \label{fig:countrate_bias_current_tv_300_600_900_8_92}
  \end{subfigure}
  \hfill
  \begin{subfigure}[t]{.5\textwidth}
        \includegraphics[width=\linewidth]{figs/HQO_20240726_countrate_bias_sweep_tv_3_6_9_thesis_9_72}
        \caption{}
        \label{fig:countrate_bias_current_tv_300_600_900_9_72}
  \end{subfigure}
  \caption{Count rates for different bias currents and trigger voltages.
    The count rates are measured for a trigger voltage of 300$\si{\milli \V}$, 600$\si{\milli \V}$ and 900$\si{\milli \V}$.
    \subref*{fig:countrate_bias_current_tv_300_600_900_8_92} shows the measured count rates for an input count rate of 2.45MHz
  and \subref*{fig:countrate_bias_current_tv_300_600_900_9_72} for 0.39MHz.
    The curves for a trigger voltage of 300$\si{\milli \V}$  and 600$\si{\milli \V}$ are laying on each other and are not distinguishable in the lower bias current regions.}
\end{figure}

In figure \ref{fig:countrate_bias_current_tv_300_600_900_8_92} and \ref{fig:countrate_bias_current_tv_300_600_900_9_72} one can see that at a lower trigger voltage of 300$\si{\milli \V}$ the count rate oscillates
a bit.
This likely corresponds to the increased dark count rates, due to the overshoot of the critical current, as explained in figure \ref{fig: Oscillating_bias_current}.

Furthermore, one can see that in the lower bias current regime the count rates are higher for the lower trigger voltages.
This corresponds to the detection of lower voltage pulses.
Another behaviour is the constant count rate, beginning at $I_{Bias} \approx 20 \si{\micro \A}$.
A constant count rate is reached independent of the trigger voltage (except for the light fluctuation at a trigger voltage of $300 \si{\milli \V}$).
At the end of the curve, for the higher count rate \subref{fig:countrate_bias_current_tv_300_600_900_8_92} the counting of the detector drops at a bias current of  $I_{Bias} \approx 32uA$ and
for the lower count rates \subref{fig:countrate_bias_current_tv_300_600_900_9_72} at $I_{Bias} \approx 34uA$ .
One can assume that the critical current is reached earlier when more photons are send to the detector, because the nanowire
is not able to recover that fast enough.
Overall, one can also conclude that the behaviour of the count rates does not change significantly for the trigger voltages of 600$\si{\milli \V}$ and 900$\si{\milli \V}$.

Looking at the analog pulses, an interesting observation is that if the count rate reached a certain high amount (here at 0.547MHz), the produced voltage peak is decreasing.
This behaviour counteracts the behaviour of a rising voltage peak, when the bias current is increased.
This can be seen in recordings of voltage pulses with the same bias current and trigger voltage \textit{but different count rates}
(shown in figure \ref{fig:analog_signals_comparison}).
Moreover, if several photons are detected in a short time period, the voltage amplitude of successive pulses are fluctuating.
This is also observable in the comparison in the three consecutive pulses of the orange curve in figure \ref{fig:analog_signals_comparison}.

%\begin{figure}
%  \begin{subfigure}[t]{.5\textwidth}
%      \includegraphics[width=\linewidth]{figs/HQO_20240726_analog_signal_OD_9.57_31.2uA_tv_600mV}
%      \caption{}
%      \label{fig: analog_signals_OD_9.57_31.2uA_tv_600mV}
%  \end{subfigure}
%  \hfill
%  \begin{subfigure}[t]{.5\textwidth}
%        \includegraphics[width=\linewidth]{figs/HQO_20240726_analog_signal_31.2uA_tv_600mV_OD_12_06}
%        \caption{}
%        \label{fig:analogs_signals_OD_12_06_31.2uA_tv_600mV}
%  \end{subfigure}

%  \caption{Analog voltage pulse signals for the count rates 547.546KHz \subref*{fig: analog_signals_OD_9.57_31.2uA_tv_600mV}
%  and 1.771KHz \subref*{fig:analogs_signals_OD_12_06_31.2uA_tv_600mV}.
% In both recordings the same bias current $I_{\text{b}} = 31.2 \si{\micro \A}$ and trigger voltage $600 \si{\milli \V}$ is used.
%  In \subref*{fig:analogs_signals_OD_12_06_31.2uA_tv_600mV} at $\approx$ 26ns minor peak on the descending curve is visible.
%  This may be attributed to back reflection in the cable.
%  }
%  \label{fig:analog_signals_comparison}
%\end{figure}

\begin{figure}
    \centering
    \includegraphics[width=0.8\textwidth]{figs/HQO_20240726_analog_signal_OD_9_57_and_12_06_31_2uA_tv_600mV}
    \caption{Analog voltage pulse signals for the count rates 0.544MHz (orange curve) and 1.771KHz (blue curve).
    In both recordings the same bias current $I_{\text{b}} = 31.2 \si{\micro \A}$ and trigger voltage $600 \si{\milli \V}$ is used.
    At $\approx$ 26ns a minor peak on both descendings curves is visible.
    This may be attributed to back reflection in the cable.
    Furthermore, signals at higher count rates are followed by more underground noise.
    This can be seen at the underground offset of the orange curve.
      }
    \label{fig:analog_signals_comparison}
\end{figure}

This behaviour can also be seen in figure \ref{fig:bias_sweep_7_48_appendix} along the lower count rates for higher trigger voltages in the appendix \ref{sec:bias_sweeping_countrate_appendix}.

One possible explanation for this phenomenon is that the hotspot in the nanowire has not been entirely dissipated,
resulting in incomplete restoration of the initial superconductivity.
Therefore, successive triggered events have not the same pulse amplitude.

Accordingly, the optimal working point with regard to bias current and trigger voltage is dependent upon the photon rate being transmitted to the detector.
A midpoint trigger voltage of 600 mV is reasonable when measuring with a count rate in the 300 kHz regime.
For higher count rates, a reduction in the trigger voltage could be appropriate.
However, this may result in an increased risk of detecting noise signals, such as cable back reflection seen in \ref{fig:analog_signals_comparison}.

\FloatBarrier

\subsubsection*{Input photon rate dependency}\label{subsec:photon_rate_dependency}

Finally, efficiency measurements for different count rates were analyzed, to determine the bandwidth where events are detected with
the constant maximum efficiency.

The count rates were varied by using different combinations of ND filters.
With this, six different count rates (2.446MHz, 0.531MHz, 0.388MHz, 0.0674MHz, 12.300KHz, 1.800KHz) were generated.

To avoid the oscillation of count rates near the critical current as seen in the blue curve in figure \ref{fig:countrate_bias_current_tv_300_600_900_8_92}
and \ref{fig:countrate_bias_current_tv_300_600_900_9_72} a trigger voltage of 600$\si{\milli \V}$ is used.
Further, the bias current was set to 31.2$\si{\micro \A}$, since it yields the best recovery time as discussed in section \ref{sec:recovery-time}.

However, with this operating bias current it is not possible to measure higher count rates than 2.45MHz (OD $\leq$ 8.92),
due to an earlier breakdown of counting corresponding to the failing recovery of the superconducting nanowire.

For each of the 6 input count rates the detected count rate was measured for 30s with an integration time of 1 second.
This measurement was repeated 5 times.
Afterwards, the system detection efficiency $\eta_{\text{sde}}$ was calculated by the formula
$\eta_{\text{sde}} = \frac{\mathcal{R}_{\text{mean}}}{\mathcal{R}_{\text{incident}}'} \cdot 100$.
Where $\mathcal{R}_{\text{mean}}$ are the averaged count rate of the 5 measurements and $\mathcal{R}_{\text{incident}}'$ is the incident photon rate.
$\mathcal{R}_{\text{incident}}'$ was calculated via the OD of the ND filters and the photon number per second of the initial laser power $\mathcal{R}_{\text{incident}'} = \Phi \cdot 10^{-\text{OD}}$.
The resulting 6 efficiency values are plotted vs the input count rates in figure \ref{fig:sde_count_rates_600mV_thesis}.

\begin{figure}
    \centering
    \includegraphics[width=0.8\textwidth]{figs/HQO_20240727_sde_count_rate_thesis}
    \caption{System detection efficiency appears in a range between $(87.242 \pm 4.955) \%$ and $(91.834 \pm 9.559) \%$.
    Errors are calculated according to \ref{sec:ND_filter_calibration_appendix} and due to the
    high uncertainty in the OD calculation unphysical values of over $\eta_{\text{sde}} = 100\%$ are included in the error range.
    Calculated values are listed in table \ref{tab:sde_count_rate_table}.}
    \label{fig:sde_count_rates_600mV_thesis}
\end{figure}

In figure \ref{fig:sde_count_rates_600mV_thesis} one can see a rather constant trend of $\eta_{\text{sde}}$.
No significant decrease in efficiency can be observed.
In order to investigate the behaviour further, measurements were done for two higher count rates
(67.363MHz and 9.575MHz).

From the count rates of a bias sweep measurement (seen in figure \ref{fig:sde_bias_current_600_thesis}),
the point with the highest count rates was selected.
With these additional values, the  $\eta_{\text{sde}}$ was again plottet against the count rate (log scale) as shown in figure \ref{fig:sde_count_rate_log_thesis}.

\begin{figure}
  \begin{subfigure}[t]{.5\textwidth}
    \centering
    \includegraphics[width=\linewidth]{figs/HQO_20240727_sde_bias_sweep_tv_600_thesis}
    \caption{}
    \label{fig:sde_bias_current_600_thesis}
    \end{subfigure}
  \hfill
  \begin{subfigure}[t]{.5\textwidth}
    \centering
    \includegraphics[width=\linewidth]{figs/HQO_20240727_sde_count_rate_log_thesis}
    \caption{}
    \label{fig:sde_count_rate_log_thesis}
  \end{subfigure}
  \caption{\subref*{fig:sde_bias_current_600_thesis} Curves of system detection efficiency for different bias current.
  The curve for 0.39MHz was left out because the high relative fluctuations made the other cuvers unreadable.
  The raising edges at the end of the curve are due to the rising noise, as already seen in \ref{fig: DCR_black_box_to_exp}.
  In \subref*{fig:sde_count_rate_log_thesis} curve of system detection efficiency for two further higher count rates plotted on a logarithmic X - axis to
  emphazise the efficiency bandwidth.
  For better visualizaton, the errors for $\mathcal{R}_{\text{incident}}$ were scaled down by a factor of 10.
  }
\end{figure}

In figure \ref{fig:sde_bias_current_600_thesis} one can clearly see two different behaviours.
First, for higher count rates the maximum bias current that the detector can handle is reached earlier and no constant count rate is reached.
This dynamic follows from the fact that the detector is not able to recover in time, when the count rate is too high.
Hence, the critical temperature is reached earlier and does recover again.
As result the detector shuts down.
This makes the counting process highly unstable and unreliable, because one wants to measure with a high but also stable count rate.
This is not the case for the count rate of 67.363MHz. No stable count rate is reached, and the highest
count rate is close to the edge of the breakdown, which is not desirable.

Secondly, the system detection efficiency $\eta_{\text{sde}}$ sinks for higher count rates.
This can be explained by the fact that at certain count rate, the average photon per dead time unit is above 1, which
directly corresponds to a decrease in the absorption efficiency $\eta_{\text{A}}$.

Finally, the measurements in figure \ref{fig:sde_count_rate_log_thesis} show a constant efficiency course for count rates lower than 9.575MHz.
Than, the efficiency decreases for higher count rates.
Due to time constraints, no further measurements were done to investigate the efficiency behaviour for other count rates.
All results are listed in table \ref{tab:sde_count_rate_table}.
All system detection efficiency values appear in a range between $\eta_{\text{sde}} = (87.242 \pm 4.955) \%$ for a
count rate of 2.446 MHz at a count rate of 0.3876 MHz.

\begin{figure}
  \begin{subfigure}[t]{.45\textwidth}
    \centering
    \includegraphics[width=\linewidth]{figs/HQO_20240730_sde_fit_thesis}
    \caption{}
    \label{fig:linear_fit_sde}
    \end{subfigure}
  \hfill
  \begin{subfigure}[t]{.45\textwidth}
    \centering
    \includegraphics[width=\linewidth]{figs/HQO_20240730_count_vs_photon_rate_log_thesis}
    \caption{}
    \label{fig:count_vs_photon_rate_log_thesis}
  \end{subfigure}
  \caption{\subref*{fig:linear_fit_sde} Shows the linear Orthogonal Distance Regression (ODR) fit of $\mathcal{R}_{\text{incident}}'$ against $\mathcal{R}_{\text{mean}}$.
    The fit is done with \texttt{SciPy} and considers the uncertainties of x- and y.
  The fit yields a slope of $m =0.893 \pm 0.008$ with $\chi^2 = 705.610$.
    In \subref*{fig:count_vs_photon_rate_log_thesis} the extented count rate (log scale) is plotted against the extended incident photon rate.
  }
\end{figure}

These results are in agreement with the specification of the company, which measured a system detection efficiency of 87 $\pm 3 \%$ for a
incident count rate of 395 KHz \cite{tech_sheet_single_quantum}.
However, the results bring two caviats.
First, the error of $\eta_{\text{sde}}$ is very large.
This lowers the significance of the results.
Second, the error of $\mathcal{R}_{\text{incident}}'$ are correlated with the error of $\eta_{\text{sde}}$.
This dependent errors have the effect of distorting the presentation of the results.


In order to compensate this effect, $\mathcal{R}_{\text{incident}}'$ was plotted against $\mathcal{R}_{\text{mean}}$ and
a linear orthogonal distance regression (ODR) with \texttt{SciPy} was done in order to consider the x- and y-errors.
The regression was done, because it is assumed that $\eta_{\text{sde}}$ stays constant, hence the relation scales
linearly with $\mathcal{R}_{\text{mean}} \leq 2.446 MHz $.
The results of the ODR fit are shown in figure \ref{fig:linear_fit_sde}.
The fit yields a slope of $0.893 \pm 0.008$ with $\chi^2 = 705.610$, correspoding to efficiency values of
$\eta_{\text{sde}} = (89.30 \pm 0.78) \%$ for an incident photon rate below $\mathcal{R}_{\text{incident}}'\leq 2.446$MHz.

In addition to the system detection efficiency, in figure \ref{fig:count_vs_photon_rate_log_thesis}
the count rate is plotted against the incident photon rate for the extended count rates of 67.363MHz and 9.575MHz.
This plot show the same behaviour as figure \ref{fig:sde_count_rate_log_thesis}, but displayed
with $\mathcal{R}_{\text{mean}}$ against $\mathcal{R}_{\text{incident}}'$ resulting in smaller errors in the presentation.
Moreover, one can see in this plot, that $\mathcal{R}_{\text{mean}}$ would converge for high $\mathcal{R}_{\text{incident}}'$
to the calculated maximum count rate of $\mathcal{R}_{\text{max}} = (71.68 \pm 0.25)$MHz.

Finally, one has to consider the correction of $\mathcal{R}_{\text{mean}}$.
The count rate the detector is showing us, are not only limited by the system detection efficiency $\eta_{\text{sde}}$
but also by our faint laser source.
It means, considering poisson distributed photons (for $\text{N(T)} \cdot \eta \ll 1$) there is always the probability
that a certain amount of photons is timely spaced by less than the dead time of the detector.
As described \ref{sec:characteristics_faint laser sources} the true count rate of our laser source is then given by
equation \eqref{eq:true_count_rate}.
Redoing the linear orthogonal distance regression (ODR) with the corrected count rates (seen in figure \ref{fig:corrected_linear_fit_sde}),
gives a slope of $ m = 0.905 \pm 0.004$ with $\chi^2 =  176.014$ corresponding to a system detection efficiency of $\eta_{\text{sde}} = (90.47 \pm 0.36) \%$.
The value $\eta_{\text{sde}} = (90.47 \pm 0.36) \%$ is not in agreement with the company specification of 87 $\pm 3 \%$ \cite{tech_sheet_single_quantum}.
However, one has to consider, that the specifications of the company are based on a laser source with a wavelength of 785nm \cite{tech_sheet_single_quantum}.
Since the used laser has wavelength of 780nm and the efficiency also depends on the wavelength \cite{natarajan-2012}, all values are not exact comparable.
Nevertheless, this value is used as the final result for the system detection efficiency $\eta^{\text{ch1}}_{\text{sde}}$ of the first channel of the SNSPD
since it comes very close to the given specification \cite{tech_sheet_single_quantum}.

\begin{figure}
    \centering
    \includegraphics[width=0.6\textwidth]{figs/HQO_20240730_sde_fit_corrected_thesis}
    \caption{Shows the linear ODR fit of $\mathcal{R}_{\text{incident}}'$ against $\mathcal{R}^{corrected}_{\text{mean}}$.
    The fit is done with \texttt{SciPy} and considers the uncertainties of x- and y.
  The fit yields a slope of $m = 0.900 \pm 0.002$ with $\chi^2 =  0.000187$.}
    \label{fig:corrected_linear_fit_sde}
\end{figure}

\newpage

\FloatBarrier
\newpage

\section{Summary of the results}\label{sec:results_summary}

These final results (listed in table \ref{tab:final_results_SNSPD_channel1}) yield the measured characteristics of channel 1 of the SNSPD.
They are listed in the column \("\)Measurement\("\) next to the given specification in column \("\)Single Quantum\("\) \cite{tech_sheet_single_quantum}.
The column \("\)Agreement\("\) lists, if the measured value is in agreement with the specification.

%! Author = maxim.re
%! Date = 29.07.24
\begin{table}[!hbt]
    \centering
    \begin{tabular}{|l|l|l|l|}
    \hline
    Characteristics & Single Quantum & Measurement & Agreement \\ \hline
    $\eta^{\text{ch1}}_{\text{sde}}$ [$\%$] & $(87 \pm 3)\%$  & $(90.47 \pm 0.36) \%$  & Yes \\ \hline
    DCR [Hz] & < 5 $\si{\Hz}$ & $(1.40 \pm 1.36) \si{\Hz}$ & Yes \\ \hline
    $\tau_{\text{recovery}}$ [ns] & - & $(17.156 \pm 0.045) \si{\nano\second}$ & - \\ \hline
    $\tau_{\text{dead}}$ [ns] & - & $(13.950 \pm 0.050) \si{\nano\second}$ & - \\ \hline
    Maximum count rate [MHz] & - & $(71.68 \pm 0.25) \si{\mega\Hz}$ & - \\ \hline
        Timing jitter [ps] & $ 17.52 \si{\pico\second}$ & - & - \\ \hline
        Figure of merit $H$ & - &  3.628 $\times 10^{10}$& - \\ \hline
    \end{tabular}
    \caption{Final measurement results of the SNSPD Characterization of Channel 1.
    Missing or not calculatable values are marked with a "-".}
    \label{tab:final_results_SNSPD_channel1}
\end{table}

For the recovery time, dead time and maximum count rate no specification was available, however the measurement were reasonably close to other specification data of the company
for detectors in similar wavelength range \cite{singlequantum_eos}.
Additionally, the timing jitter $\Delta t$ measured by the company v\cite{tech_sheet_single_quantum} and a figure of merit $H = \frac{\eta_{\text{sde}}}{DCR \Delta t}$ according to \cite{hadfield-2009} is calculated.
This can be used as compact comparison value for the quality of the detector, where higher $H$ values indicate better detectors.

\FloatBarrier
