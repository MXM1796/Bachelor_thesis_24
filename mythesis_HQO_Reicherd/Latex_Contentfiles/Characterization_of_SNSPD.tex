% !TEX root = mythesis.tex

% Add the external directory to the graphics path
\graphicspath{{/Users/maxim.re/Studium/Physik B.Sc./Semester_8_SS24/Proseminar/Figs Single Photon Detection/}}

%==============================================================================
\chapter{Characterization of our SNSPD}
\label{sec:SNSPD_Characterization}
%==============================================================================
Vom Experiment aus Theorie erklären \\ schauen ob Unterkapitel passt


In literature, four central characteristics have emerged to quantize the quality of single photon detectors and
make their performance comparable.
These characteristics are the detector efficiency, the dead count rate, the recovery time and the timing jitter.
In this thesis, I focus on the first three characteristics.
In general, there are more than these introduced ones,
like an after-pulsing or signal-to-noise ratio; however, these are the most important in the context of this thesis.

\section{Dark Count Rate}
The \textbf{dark count rate} (DCR) is the rate of measured detection events that were not intentionally sent to the source.
It is measured in counts per Second and can be caused by statistical fluctuations in the measurement electronics
or by scattered or ambient light from the environment.
A low DCR is important because it allows differentiating between detected counts from the source and detected counts by
environment and electronic noise. \\
It increases the count resolution and enables at lower frequencies still high signal-to-noise ratios.
Moreover, SNSPDs DCR depends on the bias current applied to the nanowire.
Since the deadtime and the efficiency depends on the bias current as well, one has to find the best adjustment,
where low DCR is assured as well as a low deadtime and high efficiency.


\subsection*{Measurement setup}

Evaluating the DCR needs measurements in two measurement setups.
First, a setup where the detector is unplugged from the source and covered fully in darkness, so one can assume no
photons from the environment are hitting the detector.
This enables measuring the DCR triggered by the electronics and the characteristic of the detector's operating mode.
Second, a setup where the detector is plugged to the source (faint laser source in a blackbox) and hence also
covered best as possible in darkness, so one can compare how many photons are are measured from the environment.

The measurement setup is
\subsection*{Results and Discussion}

\section{Efficiency}
In \textbf{efficiency}, there are three types of efficiencies that describe independent loss processes in detection.
An efficiency can be equated with the probability that a process under consideration will occur.\\
These three efficiencies are the coupling efficiency ($\eta_{\text{K}}$),
the absorption efficiency ($\eta_{\text{A}}$) and the registration efficiency ($\eta_{\text{R}}$).
The graph \ref{fig: single_efficiency_terms} shows that when a photon is sent to a detector via an optical fibre,
not all photons can be coupled into the fibre due to material and symmetry properties.
The probability of coupling is called the \textit{coupling efficiency}. \\
When photons hit the detector, there is always a probability that the photon will not be absorbed by the detector.
This is described by the \textit{absorption efficiency}. \\
Finally, there is always a probability that the photon will not be registered by the measuring electronics.
This is expressed with the \textit{Registration efficiency}. \\

\begin{figure}[hhh]
\includegraphics[width=7cm]{Effizienzen}
\caption{Sketch of the components in the detector setup where photons can be lost and consequently there is a probability ($\eta_{\text{K}}$, $\eta_{\text{A}}$ or $\eta_{\text{R}}$) that the photon is not taken into account in the detection process.}
\label{fig: single_efficiency_terms}
\end{figure}

These three terms are brought together in two efficiency terms: the device efficiency
($\eta_{\text{G}} = \eta_{\text{A}} \cdot \eta_{\text{R}}$) and the system efficiency
($\eta_{\text{S}} = \eta_{\text{A}} \cdot \eta_{\text{R}} \cdot \eta_{\text{K}}$).
The device's efficiency $\eta_{\text{G}}$ is that of the device itself and the efficiency would be if photons were sent
to the detector in a free environment.
The system's efficiency $\eta_{\text{S}}$ also takes into account the coupling losses in the optical fibre.
This is the case if the detector is connected to a fibre, as the device properties or the experiment does not allow
photon detection in a free environment.\\

\subsection*{Measurement set up}
\subsection*{Results and Discussion}
\section{Recovery time}
The concept of \textbf{recovery time} is visually described in graph\ref{fig:Recovery_time}. \\
\begin{figure}[hhh]
\includegraphics[width=7cm]{recovery_time_principle}
\caption{Schematic efficiency curve for the detection of a photon\cite{shalm_single-photon_2013}. On the Y axis is the
efficiency $\eta$, where $\eta_{G}$ is the device efficiency. On the X axis is the time course of the efficiency}.
\label{fig:Recovery_time}
\end{figure}

When a photon hits the detector of an EPD and is absorbed, the efficiency of the detector drops to zero and no further
photon can be measured for a period of time.
This elapsed time is called the dead time.
The efficiency then rises again to its original device efficiency.
This period is called the reset time.
The characteristic curve between the two times forms the start of the increase to full efficiency.
The sum of both times forms the recovery time.

\subsection*{Measurement set up}

\subsection*{Results and Discussion}
\section{Discussion}
Influence of timing jitter on efficiency and recovery time.
- No Afterpulsing
- Temperature 2.9 instead of 2.5Í