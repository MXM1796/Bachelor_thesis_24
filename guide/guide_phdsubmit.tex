\subsection{Submission}

\begin{enumerate}
\item Use the file \texttt{PhD\_Submit\_Title.tex} for the title
  pages. 
  This is selected by passing the options \Option{PhD, Submit}.
  to the \Macro{documentclass} or the \Package{ubonn-thesis} package.
  Leave the \foreignquote{ngerman}{Tag der Promotion} and
  \foreignquote{ngerman}{Erscheinungsjahr} blank.
\item You are required to also submit a CV\index{CV} and a summary of your
  thesis. A skeleton CV is provided as the file
  \texttt{thesis\_cv.tex}.
  Your CV and the summary should be printed separately.
\item You have to print and bind five copies of your thesis for the
  \foreignlanguage{ngerman}{Promotionsbüro}. Nowadays these are
  usually in colour.
  One of these copies will go to the department library.
\item The first and second referees for your thesis often like to also
  have an extra copy of the thesis so that they can make comments when they read
  your thesis -- ask them if they want one. You can usually save the
  institute some money and print these copies in black \& white.
  Some referees even prefer to get the extra copy as a PDF file.
\end{enumerate}


\subsection{Printing the final version}

\begin{enumerate}
\item Use the file \texttt{PhD\_Final\_Title.tex} for the title page.
  This is selected by passing the options \Option{PhD, Final}.
  to the \Macro{documentclass} or the \Package{ubonn-thesis} package.
\item Do not include your CV\@.
\item There are probably some small corrections you or the referees
  found during the time between submission and your examination. These
  should be corrected before you submit your thesis to the university
  library (ULB).\index{ULB}\index{university library}\index{library!university}
\item Almost everyone submits their thesis electronically to the ULB\@.
  As of 2024 you no longer have to submit hardcopies.
  \footnote{In the past you had to print two copies for them.
  This used to be five, but was reduced in 2015.
  The ULB is quite strict on the quality of the binding etc. The university
  print shop is not able to fulfil the requirements, so you have to print
  these versions externally. When you do this do not forget to
  uncomment the \Macro{hypersetup} command as mentioned above, if you
  want to print them in colour.}
\item The department library\index{department library}\index{library!department}
  (in the Physikalisches Institut) needs seven printed copies with the file
  \texttt{PhD\_Cover.tex} as the cover and
  \texttt{PhD\_Final\_Title.tex} for the title pages.
  These are selected by passing the options \Option{PhD, PILibrary}
  to the \Macro{documentclass} or the \Package{ubonn-thesis} package.
  You have to get
  the \enquote{BONN-IR-YYYY-nnn} number from the librarian. You
  should also include an abstract (in English) on the cover page. You
  can also use this abstract when you submit your thesis
  electronically to the ULB\@.
  Before printing these copies, you should check with the PI librarian that these rules still apply.
\item The department library version of the thesis is the one that you usually print if
  you need extra copies for your experiment or research group.
\end{enumerate}

Note that when you want to get your degree certificate, you will get
some forms from the \foreignlanguage{ngerman}{Promotionsbüro} that have
to fill out.
These forms have to be signed by your supervisor.
One of the forms asks you if you have published significant parts of your
thesis elsewhere.
This means your actual thesis and not a paper that uses the results from your thesis.
If you submit your thesis electronically to the ULB, then you should not fill out this form.
It only applies if you actually publish your thesis elsewhere
(which is allowed by the \foreignlanguage{ngerman}{Promotionsordnung}).
