% !TEX root = thesis_guide.tex
% chktex-file 1 chktex-file 21

%==============================================================================
\chapter{Tables}%
\label{sec:table}
%==============================================================================

\LaTeX{} file: \url{./guide_tables.tex}\\[1ex]
\noindent
You almost certainly have tables with results that you want to include
in your thesis. You probably even know about the \Env{tabular}
environment, but what about fine-tuning to get the tables exactly in
the form that you want? How do you line up the decimal points in a
list of cross-sections and/or \(\pm\) for the errors? How can you
handle a table that goes over more than one page and what if it is so
wide that you would like to rotate it by \ang{90}?

Also, how can you make your tables look more professional with for
example thick and thin lines in appropriate places? This is easy to
solve --- use the \Macro{toprule}, \Macro{midrule} and \Macro{bottomrule}
commands from the \Package{booktabs} package. These commands also
produce much better spacing between these lines and the rows above and
below. The \Package{booktabs} package gives you some advice on how
good tables should look, as well as some guidelines on how to make your
tables look better.

The \Macro{rule} command is very useful for adding more space between
rows. Kopka has several examples of its usage. You can also use
\Macro{arraystretch}, e.g.\ \verb+\renewcommand{\arraystretch}{1.5}+ to
increase the spacing by \SI{50}{\percent}. This command is often useful if table
cells contain subscripts and/or superscripts. \Macro{toprule}
etc.\ mean that it is usually not needed in headers.

If you want columns of expandable width you can use the
\Package{tabularx} package. The environment \Env{tabular*} has instead
expandable intercolumn spacing.

Tables usually go inside the \Env{table} environment so their position
can \enquote{float}. In this chapter I use some tables inside \Env{table}
and some inline.

Don't forget to use the package \Package{cleveref} and macros such as \Macro{cref}
to reference the tables --- see \cref{sec:package:appearance} for a few more details.

There are good arguments for the table caption being above the table,
while the figure caption should be below the figure.
Include the \Macro{caption} command before the \Env{tabular},
in order for the caption to be placed above.
The spacing of the caption and the table is then optimal
when the option \Option{captions=tableheading}
is set in \KOMAScript.
This can be achieved by setting the option \Option{tableheading} (as of version 8.0), which is set to \enquote{true} by default.
If you use this option, you should use \Macro{centering} to centre the table
and not the environment \Env{center}.
You can get the same spacing, without the \Option{captions=tableheading} option,
by using the macro \Macro{captionabove} macro instead of \Macro{caption}.

Trying to include footnotes in tables can be tricky.
See \cref{sec:layout:footnote} for some guidance on how this can be done.


%------------------------------------------------------------------------------
\section{Use of \texorpdfstring{\Macro{phantom}}{phantom}}%
\label{sec:table:phantom}
%------------------------------------------------------------------------------

Although extra packages can help, a very useful command is
\Macro{phantom}. This inserts white space corresponding to the width
of the argument. Compare the results in the following table with two
numbers 0.76 and 83.1:

\begin{tcblisting}{textable}
\centering
\begin{tabular}{cc|rr}
  \multicolumn{2}{c|}{Centred} &
  \multicolumn{2}{c}{Right justified} \\
  & Phantom & & Phantom\\
  \midrule
  0.76 & \phantom{0}0.76 & 0.76 & 0.76\\
  83.1 & 83.1\phantom{0} & 83.1 & 83.1\phantom{0}
\end{tabular}
\qquad
{\setlength{\extrarowheight}{0.5ex}
\centering
\begin{tabular}{cc|rr}
  \multicolumn{2}{c|}{Centred} &
  \multicolumn{2}{c}{Right justified} \\
  & Phantom & & Phantom\\
  \midrule
  0.76 & \phantom{0}0.76 & 0.76 & 0.76\\
  83.1 & 83.1\phantom{0} & 83.1 & 83.1\phantom{0}
\end{tabular}
}
\end{tcblisting}
\par\noindent
The difference between the two tables is that the one on the right
has the length \Macro{extrarowheight} set to 0.5ex.
The first two columns are centred, the last two are right justified.
This is clearly a bit clumsy, but it does work!
For illustration, you can also see the effect of adjusting \Macro{extrarowheight}.


%------------------------------------------------------------------------------
\section{Using \Package{siunitx} and the \Option{S} column option}%
\label{sec:table:siunitx}
%------------------------------------------------------------------------------

The \Package{siunitx} package contains some nice tools that make the
correct alignment of numbers simpler. 

A first simple example is shown in \cref{tab:viscosity}.
In fact I show the table twice, once
with the language set to default and once with it set to German. The
\Env{tabular} contents are identical, the second tabular is inside a
\Macro{foreignlanguage}. Note the use of
\Option{table-format}
to centre the temperatures as the heading is wider than the numbers.
Umlaute etc.\ can normally be inserted directly.
Such characters can be included directly in listings that use the \Env{tcblisting} environment,
if one uses  with \LuaLaTeX\ and \XeLaTeX\ for compilation.
However, that does not work in the \Env{tcblisting} environment if one uses pdf\LaTeX\ to compile the guide.
Hence this example uses the old fashioned style \verb|\"{u}| etc.

\begin{table}[htbp]
  \captionabove{A table of viscosities in the default language and German.
    Typeset using the \Macro{captionabove} macro.}%
  \label{tab:viscosity}
\begin{tcblisting}{textable}
\centering
\begin{tabular}{lS[table-format=4.2]S}
  \toprule
  Liquid & {Temp.} & {Viscosity \(\eta\)}\\
  &  {[\si{\celsius}]} & {(\si{\metre\pascal\second})} \\
  \midrule
  Blood     & 37 & 4.0 \\
  Glycerine &  0 & 10000 \\
  & 20 & 1410 \\
  Oil (SAE 10) & 30 & 200 \\
  Water   &  0 & 1.8 \\
  & 20 & 1.00 \\
  Air     & 20 & 0.018 \\
  \bottomrule
\end{tabular}
\qquad
\foreignlanguage{ngerman}{%
\begin{tabular}{lS[table-format=4.2]S}
  \toprule
  Fl\"{u}ssigkeit & {Temp.} & {Viskosit\"{a}t \(\eta\)}\\
  &  {[\si{\celsius}]} & {(\si{\metre\pascal\second})} \\
  \midrule
  Blut     & 37 & 4.0 \\
  Glyzerin &  0 & 10000 \\
  & 20 & 1410 \\
  \"{O}l (SAE 10) & 30 & 200 \\
  Wasser   &  0 & 1.8 \\
  & 20 & 1.00 \\
  Luft     & 20 & 0.018 \\
  \bottomrule
\end{tabular}
}
\end{tcblisting}
\end{table}

\Cref{tab:xsect0} shows a more complicated table set using the tools.
You can either enclose all numbers in \Macro{tablenum} (or \Macro{num}
or use the \Option{S} column descriptor.
If you use \Option{S}, it usually centres the contents of the column.
You can use the \Option{table-number-alignment} option to change this.
\Option{S} and \Macro{tablenum} cannot generally be mixed in a single column though.
If you want to use \Macro{tablenum} in an
\Option{S} column you have to enclose it in braces.
You also need to do this with regular text,
such as column headings --- see \cref{tab:viscosity}.
Note the use of \Macro{sisetup\{table-format=4.0\}}
to define a default format for numbers in an \enquote{S} column,
or that are formatted using \Macro{numtable}.

The spacing of the captions in \cref{tab:viscosity,tab:xsect0} is the same,
as the option \Option{captions=tableheading} is set in this guide.

\begin{table}[htbp]
  \caption{A selection of cross-section measurements!
    Typeset using the \Macro{caption} macro.}%
  \label{tab:xsect0}
\begin{tcblisting}{textable,
  listing options={basicstyle=\ttfamily\footnotesize}}
\centering
\renewcommand{\arraystretch}{1.2}
\sisetup{round-mode = places}
\sisetup{table-format=4.0}
\begin{tabular}{
    S[table-format=+1.2] @{\,:\,}
    S[table-format=+1.2, table-number-alignment=center-decimal-marker]
    r@{\,}c@{\,}l@{\,}l
    S[table-format=5.0, round-precision=0]@{\,}l
    S[table-format=1.2, round-precision=2]}
  \toprule
  \multicolumn{2}{c}{\etajet} & \multicolumn{4}{c}{\diffetab} 
  & \multicolumn{2}{c}{\diffnloetab} & \Cbhad \\
  \multicolumn{2}{c}{} & \multicolumn{4}{c}{[\si{\pico\barn}]} &
  \multicolumn{2}{c}{[\si{\pico\barn\per\GeV}]} & \\
  \midrule
  -1.6 & -1.1 & \tablenum[table-format=2.1, round-precision=1]{57.4} &
  \(\pm\) & \tablenum[table-format=1.1, round-precision=1]{9.4} &
  \(^{+13}_{-3}\) &  72 & \(^{+22}_{-13}\) & 0.704 \\
  -1.1 & -0.8 & \tablenum[round-precision=0]{121.3} &
  \(\pm\) & \tablenum[table-format=2.0, round-precision=0]{21.1} &
  \(^{+16}_{-16}\) & 182 & \(^{+50}_{-30}\) & 0.781 \\
  \multicolumn{9}{l}{\ldots}\\
  -0.2 & +0.1 & \tablenum[round-precision=0]{264.1} &
  \(\pm\) & \tablenum[table-format=2.0, round-precision=0]{22.0} &
  \(^{+28}_{-23}\) & 342 & \(^{+91}_{-55}\) & 0.808 \\
  +0.1 & +0.5 & \tablenum[round-precision=0]{316.0} &
  \(\pm\) & \tablenum[table-format=2.0, round-precision=0]{21.1} &
  \(^{+23}_{-17}\) & 346 & \(^{+96}_{-57}\) & 0.847 \\
  +0.5 & +1.4 & \tablenum[round-precision=0]{288.1} &
  \(\pm\) & \tablenum[table-format=2.0, round-precision=0]{15.4} &
  \(^{+20}_{-30}\) & 265 & \(^{+82}_{-48}\) & 0.926 \\
  \bottomrule
\end{tabular}
\end{tcblisting}
\end{table}

With \Macro{tablenum} you can also specify the precision and format of each
number is shown separately.
\Macro{tablenum} should be preferred over \Macro{num} in tables,
as you can use alignment options with it.
With \Option{S} you specify the format for the whole column.
You should make sure
that you specify the rounding mode in the preamble of the table (usually
\Option{figures} or \Option{places} --- you can also choose \Option{off}).

If you know that you are going to make such a table, it is very easy
to use either of these options to write it out in this format using a program. 
Using \Macro{tabenumnum} or \Macro{num} solves the very common problem of your
program writing out the results with too many significant digits and
you have to correct them all by hand later (as well as every time they
get updated)!  A slightly different approach that one could follow is
to use \Option{S} for simple numbers in tables and \Macro{num} for
more complicated number typesetting.  For asymmetric errors you could
consider defining something similar to \Macro{numpmerr}, which
internally uses \Macro{num}.

A common and closely related problem that occurs is that your analysis
spits out a result such as \(24.36789^{+0.36423}_{-0.45236}\). You copy
and paste this into a table and then a referee (or your supervisor)
complains that you clearly don't understand statistics, as you should
never quote an error to 5 significant digits. You can go ahead and
edit all the numbers by hand, but what do you do if you rerun your
analysis and all the numbers change.
Reformatting by hand is then an error prone and lengthy process.
As discussed above, you can either use options such as 
\Option{round-precision} in the \Option{S} command or use the macro
\Macro{tablenum} or \Macro{num} 
and option \Option{dp}/\Option{round-precision} to do the
rounding\index{rounding} for you.

\Cref{tab:rounding1,tab:rounding2} shows and compares two different approaches
on how this can be done, even for asymmetric errors.
While the form
may appear to be a bit clumsy at first, it is easy enough to get your
program to write out the lines. 
You should use the options \Option{round-mode} and
\Option{round-precision} to specify how the rounding should be done.
In the first line of \cref{tab:rounding1}
I show what to do if you need to change the precision of a single number.
As you can see this is rather trivial.
However, then the alignment on the decimal point is no longer perfect.
While this is probably OK for internal notes etc.,
theses or papers (should) have tougher requirements.
Another way of achieving the same thing and
avoiding the use of \Option{round-mode} and
\Option{round-precision}\footnote{%
  The \Option{round-mode} should be set in the preamble of the table and
  not for every number.}
is shown in \cref{tab:rounding2}.
Note the use of options for the \Option{S} command
and the use of \Macro{num} enclosed in braces
to format the row that requires a different precision.

While the rounding works very nicely,
it slows down the compilation of your document.
You can turn off any rounding (and all other parsing of numbers)
by using \Macro{sisetup} and the option \Option{parse-numbers=false}: 
\Macro{sisetup\{parse-numbers=false\}}.
If you set this option inside a table, it will only be applied to that table.

It takes a while to learn what the different options mean and their
consequences. I hope that these examples cover most problems and at
least give ideas as to what is possible.
In particular the \cref{tab:rounding2} solution is very nice, as
all numbers except the one that requires an extra digit are written
without any special formatting!

\begin{table}[htbp]
  \caption{Another selection of cross-section measurements! Note the
    use of \Macro{sisetup} to keep the plus signs on the positive errors.
    This example uses \Macro{tablenum}.
  }%
  \label{tab:rounding1}
\begin{tcblisting}{textable}
\centering
\renewcommand{\arraystretch}{1.2}
\sisetup{retain-explicit-plus}
\sisetup{round-mode = places}
\begin{tabular}{%
    S@{\,:\,}S
    r@{\,}@{\(\pm\)}@{\,}l@{\,}l
      }
  \toprule
  \multicolumn{2}{c}{\etajet} & \multicolumn{3}{c}{\diffetab} \\
  \multicolumn{2}{c}{} & \multicolumn{3}{c}{[\si{\pico\barn}]} \\
  \midrule
  {\num{-1.6}} & -1.1 & \num[round-precision=3]{0.574} &
  \num[round-precision=3]{0.094} &
  \(^{\num[round-precision=3]{+0.035}}_{\num[round-precision=3]{-0.031}}\) \\
  {\num{-1.1}} & -0.8 & \num[round-precision=2]{1.213} &
  \num[round-precision=2]{0.211} &
  \(^{\num[round-precision=2]{+0.162}}_{\num[round-precision=2]{-0.162}}\) \\
  \multicolumn{5}{l}{\ldots}\\
  {\num{-0.2}} & +0.1 & \num[round-precision=2]{2.641} &
  \num[round-precision=2]{0.220} &
  \(^{\num[round-precision=2]{+0.283}}_{\num[round-precision=2]{-0.233}}\) \\
  {\num{+0.1}} & +0.5 & \num[round-precision=2]{3.160} &
  \num[round-precision=2]{0.211} &
  \(^{\num[round-precision=2]{+0.232}}_{\num[round-precision=2]{-0.172}}\) \\
  {\num{+0.5}} & +1.4 & \num[round-precision=2]{2.881} &
  \num[round-precision=2]{0.154} &
  \(^{\num[round-precision=2]{+0.201}}_{\num[round-precision=2]{-0.301}}\) \\
  \bottomrule
\end{tabular}
\end{tcblisting}
\end{table}

\begin{table}[htbp]
  \caption{Another selection of cross-section measurements! Note the
    use of \Macro{sisetup} to keep the plus signs on the positive errors.}%
    This example uses \enquote{S} columns.
  \label{tab:rounding2}
\begin{tcblisting}{textable}
\centering
\renewcommand{\arraystretch}{1.2}
\sisetup{retain-explicit-plus}
\sisetup{round-mode = places, round-precision = 2}
\begin{tabular}{%
    S[table-format=3.2, table-number-alignment=right]@{\,:\,}S
    S[round-mode = places, round-precision = 2,
    table-format = 1.3, table-number-alignment=right]
    @{\(\,\pm\,\)}
    S[round-mode = places, round-precision = 2,
    table-format = 1.3, table-number-alignment=left]
    @{\,}l
      }
  \toprule
  \multicolumn{2}{c}{\etajet} & \multicolumn{3}{c}{\diffetab} \\
  \multicolumn{2}{c}{} & \multicolumn{3}{c}{[\si{\pico\barn}]} \\
  \midrule
  -1.6 & -1.1 & {\num[round-precision=3]{0.574}} &
  {\num[round-precision=3]{0.094}} &
  \(^{\num[round-precision=3]{+0.035}}_{\num[round-precision=3]{-0.031}}\) \\
  -1.1 & -0.8 & 1.213 & 0.211 & \(^{\num{+0.162}}_{\num{-0.162}}\) \\
  \multicolumn{5}{l}{\ldots}\\
  -0.2 & +0.1 & 2.641 & 0.220 & \(^{\num{+0.283}}_{\num{-0.233}}\) \\
  +0.1 & +0.5 & 3.160 & 0.211 & \(^{\num{+0.232}}_{\num{-0.172}}\) \\
  +0.5 & +1.4 & 2.881 & 0.154 & \(^{\num{+0.201}}_{\num{-0.301}}\) \\
  \bottomrule
\end{tabular}  
\end{tcblisting}
\end{table}

Another example using \Package{siunitx} tools that contains a similar
problem is:
\begin{tcblisting}{textable}
\centering
\begin{tabular}{
  S[table-number-alignment=center-decimal-marker]@{\(\,\pm\)}
  S[table-number-alignment=center-decimal-marker]|
  c
  S[output-decimal-marker={,}]@{\(\,\pm\!\!\)}
  S[output-decimal-marker={,}]
}
\toprule
\multicolumn{2}{c|}{English} &
\multicolumn{1}{c}{German1} &
\multicolumn{2}{c}{German2} \\
{Value} & {Error} & {Wert} & \multicolumn{2}{c}{Messung}\\
\midrule
0.76 & 0.14 & \num[output-decimal-marker={,}]{0.89(16)} & 0.89 & 0.16\\
83.1 &  7.6 & \num[output-decimal-marker={,}]{94.2(83)} & 94.2 & 8.3\\
\bottomrule
\end{tabular}
\end{tcblisting}

As you can see, the \enquote{English} column formats things nicely using
the \Option{S} column descriptor. The \enquote{German1} column successfully
converts the decimal point to a comma and also the parentheses with
the error to \(\pm\). However, the alignment of the numbers is now
messed up. The \enquote{German2} column looks better. I had to do some dirty
tricks with the formatting of the intercolumn separator
\verb+@{\,\pm\!\!}+ to get the spacing nice! This confirms my
statement above that the \Option{S} format is most useful for aligning
simple numbers easily, while \Macro{tablenum} is very nice for rounding to
a given precision --- note that you can use either the \Option{dp} or
\Option{sf} options to achieve what you want.

If your header is wider than the content,
you may also have to tweak a few things to get the correct alignment.
A good way of doing this is to insert some \Macro{hspace*} into the header.
The table on the left of \cref{tab:heading} is without \Macro{hspace}, while the one on the right is
with an \Macro{hspace} of 2em.
The tables contain vertical lines that I would normally not include to illustrate the alignment.

\begin{table}[htbp]
  \caption{Table illustrating how to centre numbers
    if the heading is wider than them.}
  \label{tab:heading}
\begin{tcblisting}{textable}
\renewcommand{\arraystretch}{1.4}
\sisetup{round-mode=places, retain-explicit-plus}
\centering
\begin{tabular}{|S[table-format=5.0, round-precision=0]@{\,} l|}
  \toprule
  \multicolumn{2}{| c |}{\diffnloetab} \\
  \multicolumn{2}{| c |}{[\si{\pico\barn\per\GeV}]} \\
  \midrule
    72 & \(^{+22}_{-13}\) \\
    182 & \(^{+50}_{-30}\) \\
    255 & \(^{+69}_{-42}\) \\
  \bottomrule
\end{tabular}
\qquad
\begin{tabular}{|S[table-format=5.0, round-precision=0]@{\,}
  l@{\hspace*{2em}}|}
  \toprule
  \multicolumn{2}{| c |}{\diffnloetab} \\
  \multicolumn{2}{| c |}{[\si{\pico\barn\per\GeV}]} \\
  \midrule
    72 & \(^{+22}_{-13}\) \\
    182 & \(^{+50}_{-30}\) \\
    255 & \(^{+69}_{-42}\) \\
  \bottomrule
\end{tabular}
\end{tcblisting}
\end{table}

%------------------------------------------------------------------------------
\section{Using \Package{dcolumn}}%
\label{sec:tab:dcolumn}
%------------------------------------------------------------------------------

An alternative is the \Package{dcolumn} package. You can also use this
package to convert numbers written with \enquote{.} as the decimal point
into German-style numbers with \enquote{,}.\footnote{See
  \cref{sec:layout:german} for hints on how to get
  around problems with the \Package{ziffer} package.}
You can line up measurements and errors by putting each of them in its own column. If
your errors are symmetric you can put \(\pm\) as the intercolumn
separator:

\begin{tcblisting}{textable}
\centering
\begin{tabular}{cD{.}{.}{3} | cD{.}{,}{2} | rl |
  D{.}{.}{2}@{\(\,\pm\,\)}D{.}{.}{2}}
  \multicolumn{2}{c|}{English} &
  \multicolumn{2}{c|}{German} &
  \multicolumn{2}{c|}{Val \(\pm\) Err} &
  \multicolumn{1}{c}{Val} & \multicolumn{1}{l}{Err}\\
  \midrule
  0.76 & 0.76 & 0,76 & 0.76 & 0.76 & \(\pm\) 0.14 & 0.76 & 0.04\\
  83.1 & 83.1 & 83,1 & 83.1 & 83.1 & \(\pm\) 4.2  & 83.1 & 4.2
\end{tabular}
\end{tcblisting}

\Cref{tab:xsect1} shows quite a complicated table
set in 2 different ways. It is rotated by \SI{90}{\degree} to
illustrate how that can be done.

\begin{table}[htbp]
  \caption{Cross-section measurements!}%
  \label{tab:xsect1}
  \begin{sideways}
    \centering
    \begin{tabular}{r@{ : }l|c|c|c} % chktex 26 chktex 44
      \toprule
      \multicolumn{2}{c|}{\pTjet} & \diffptb & \diffnloptb & \Cbhad \\
      \multicolumn{2}{c|}{(GeV)} & (pb/GeV) & (pb/GeV) & \\
      \midrule
      \(\phantom{1}\)6 & 11 & \(95.6\phantom{2}\pm 4.9\phantom{4}^{+9.8\phantom{2}}_{-7.0\phantom{2}}\) &
      \(109\phantom{.22}^{+31\phantom{.22}}_{-19\phantom{.22}}\) & 0.83 \\
      11 & 16 & \(24.8\phantom{2}\pm 1.2\phantom{4}^{+1.8\phantom{2}}_{-1.4\phantom{2}}\) &
      \(\phantom{1}29.1\phantom{2}^{+\phantom{1}7.9\phantom{2}}_{-\phantom{1}4.7\phantom{2}}\) & 0.89 \\
      16 & 21 & \(\phantom{2}6.02\pm 0.49^{+0.6\phantom{2}}_{-0.6\phantom{2}}\) &
      \(\phantom{10}7.1\phantom{2}^{+\phantom{1}2.0\phantom{2}}_{-\phantom{1}1.2\phantom{2}}\) & 0.92 \\
      21 & 27 & \(\phantom{2}0.93\pm 0.22^{+0.31}_{-0.20}\) &
      \(\phantom{10}1.87^{+\phantom{1}0.54}_{-\phantom{1}0.34}\) & 0.95 \\
      27 & 35 & \(\phantom{2}0.30\pm 0.12^{+0.14}_{-0.12}\) &
      \(\phantom{10}0.46^{+\phantom{1}0.13}_{-\phantom{1}0.08}\) & 1.05 \\
      \bottomrule
      \multicolumn{5}{c}{}\\
      \toprule
      \multicolumn{2}{c|}{\etajet} & \diffetab & \diffnloetab & \Cbhad \\
      \multicolumn{2}{c|}{} & (pb) & (pb) & \\
      \midrule
       \(-1.6\) & \(-1.1\) & \(\phantom{2}57\pm 22^{+13}_{-\phantom{1}3}\) &
       \(\phantom{1}72^{+22}_{-13}\) & 0.70 \\
       \(-1.1\) & \(-0.8\) & \(121\pm 21^{+16}_{-16}\) &
       \(182^{+50}_{-30}\) & 0.78 \\
       \(-0.8\) & \(-0.5\) & \(214\pm 22^{+22}_{-12}\) &
       \(255^{+69}_{-42}\) & 0.79 \\
       \(-0.5\) & \(-0.2\) & \(233\pm 21^{+28}_{-21}\) &
       \(307^{+83}_{-50}\) & 0.79 \\
       \(-0.2\) & \(\phantom{-}0.1\) & \(264\pm 22^{+28}_{-23}\) &
       \(342^{+91}_{-55}\) & 0.81 \\
       \(\phantom{-}0.1\) & \(\phantom{-}0.5\) & \(316\pm 21^{+23}_{-17}\) &
       \(346^{+96}_{-57}\) & 0.86 \\
       \(\phantom{-}0.5\) & \(\phantom{-}1.4\) & \(288\pm 15^{+20}_{-30}\) &
       \(265^{+82}_{-48}\) & 0.93 \\
       \bottomrule
    \end{tabular}

    \qquad

    \(\begin{array}{r@{\,:\,}l|D{.}{.}{2}@{\,}r@{}D{.}{.}{2}@{\,}l|D{.}{.}{2}@{\,}l|c} % chktex 44
      \toprule
      \multicolumn{2}{c|}{\pTjet} & \multicolumn{4}{c|}{\diffptb} & \multicolumn{2}{c|}{\diffnloptb} & \Cbhad \\
      \multicolumn{2}{c|}{[\si{\GeV}]} & \multicolumn{4}{c|}{[\si{\pico\barn\per\GeV}]} & \multicolumn{2}{c|}{[\si{\pico\barn\per\GeV}]} & \\
      \midrule
       6 & 11 & 95.6 & \pm & 4.9  & ^{+9.8}_{-7.0}  &  109  & ^{+31}_{-19} & 0.83 \\
      11 & 16 & 24.8 & \pm & 1.2  & ^{+1.8}_{-1.4}  & 29.1  & ^{+7.9}_{-4.7} & 0.89 \\
      16 & 21 & 6.02 & \pm & 0.49 & ^{+0.6}_{-0.6}  &  7.1  & ^{+2.0}_{-1.2} & 0.92 \\
      21 & 27 & 0.93 & \pm & 0.22 & ^{+0.31}_{-0.20} & 1.87 & ^{+0.54}_{-0.34} & 0.95 \\
      27 & 35 & 0.30 & \pm & 0.12 & ^{+0.14}_{-0.12} & 0.46 & ^{+0.13}_{-0.08} & 1.05 \\
      \bottomrule
      \multicolumn{9}{c}{}\\
      \toprule
      \multicolumn{2}{c|}{\etajet} & \multicolumn{4}{c|}{\diffetab} & \multicolumn{2}{c|}{\diffnloetab} & \Cbhad \\
      \multicolumn{2}{c|}{} & \multicolumn{4}{c|}{[\si{\pico\barn}]} & \multicolumn{2}{c|}{[\si{\pico\barn}]} & \\
      \midrule
       -1.6 & -1.1 &  57 & \pm & 22 & ^{+13}_{-3}  &  72 & ^{+22}_{-13} & 0.70 \\
       -1.1 & -0.8 & 121 & \pm & 21 & ^{+16}_{-16} & 182 & ^{+50}_{-30} & 0.78 \\
       -0.8 & -0.5 & 214 & \pm & 22 & ^{+22}_{-12} & 255 & ^{+69}_{-42} & 0.79 \\
       -0.5 & -0.2 & 233 & \pm & 21 & ^{+28}_{-21} & 307 & ^{+83}_{-50} & 0.79 \\
       -0.2 & +0.1 & 264 & \pm & 22 & ^{+28}_{-23} & 342 & ^{+91}_{-55} & 0.81 \\
       +0.1 & +0.5 & 316 & \pm & 21 & ^{+23}_{-17} & 346 & ^{+96}_{-57} & 0.86 \\
       +0.5 & +1.4 & 288 & \pm & 15 & ^{+20}_{-30} & 265 & ^{+82}_{-48} & 0.93 \\
       \bottomrule
    \end{array}\)
  \end{sideways}
\end{table}

The 2nd version (right) is certainly simpler to typeset and does not
really use any tricks to line things up. Note the use of \Env{array}
rather than \Env{tabular} which means that the contents are typeset in
math mode rather than text mode. For tables of numbers this is often
preferred. You just have to enclose the array in
\texttt{\textbackslash(\ldots\textbackslash)} or
\BegEnv{math}\ldots\EndEnv{math}. Close inspection of the right-hand
table shows that it is, however, not perfect. It is questionable
whether one wants to to write \(+0.5\) or just \(0.5\). The fact that both
\pT as well as \(\eta\) cross-sections are in a single tabular, but the
numerical values are so different makes it difficult to line things up
perfectly. An alternative, which uses the headers to fix the width of
the columns is given in \cref{tab:xsect2a,tab:xsect2b}.
Note that this uses
\Env{sidewaystable} rather than \Env{sideways} inside \Env{table},
which also rotates the caption.

\begin{sidewaystable}
  \caption[Cross-sections using \Env{sidewaystable}, which also
  rotates the caption.]{Cross-sections using \Env{sidewaystable},
    which also rotates the caption.
    Note the dirty trick used to
    get the \Cbhad values nicely in the centre of the column.}%
  \label{tab:xsect2a}
\begin{tcblisting}{textable}
\centering
\renewcommand{\arraystretch}{1.2}
\begin{math}
  \begin{array}{D{.}{.}{4.0}@{\,:}D{.}{.}{3.1} |
    D{.}{.}{2}@{\,}r@{}D{.}{.}{2}@{\,}l |
    D{.}{.}{2}@{\,}l|c}
    \toprule
    \multicolumn{2}{p{2.0cm}|}{\rule[-1.5ex]{0pt}{4.0ex}\centering\pTjet} &
    \multicolumn{4}{p{3.0cm}|}{\centering\diffptb} &
    \multicolumn{2}{p{2.2cm}|}{\centering\diffnloptb} &
    \multicolumn{1}{p{1.5cm}}{\centering\Cbhad} \\
    \multicolumn{2}{c|}{[\si{\GeV}]} &
    \multicolumn{4}{c|}{[\si{\pico\barn\per\GeV}]} &
    \multicolumn{2}{c|}{[\si{\pico\barn\per\GeV}]} & \\
    \midrule
      6 & 11 & 95.6 & \pm & 4.9  & ^{+9.8}_{-7.0}  &  109  & ^{+31}_{-19} & 0.83 \\
    11 & 16 & 24.8 & \pm & 1.2  & ^{+1.8}_{-1.4}  & 29.1  & ^{+7.9}_{-4.7} & 0.89 \\
    16 & 21 & 6.02 & \pm & 0.49 & ^{+0.6}_{-0.6}  &  7.1  & ^{+2.0}_{-1.2} & 0.92 \\
    21 & 27 & 0.93 & \pm & 0.22 & ^{+0.31}_{-0.20} & 1.87 & ^{+0.54}_{-0.34} & 0.95 \\
    27 & 35 & 0.30 & \pm & 0.12 & ^{+0.14}_{-0.12} & 0.46 & ^{+0.13}_{-0.08} & 1.05 \\
    \bottomrule
  \end{array}
\end{math}
\end{tcblisting}
\end{sidewaystable}

\begin{sidewaystable}
  \caption[Cross-sections using \Env{sidewaystable}, which also
    rotates the caption.]{Cross-sections using \Env{sidewaystable},
    which also rotates the caption. Just for fun the numbers
    indicating the \(\eta\) range of the bins in the lower half have
    been converted to German format!
    Note also the dirty trick used to
    get the \Cbhad values nicely in the centre of the column.}%
  \label{tab:xsect2b}
\begin{tcblisting}{textable}
\centering
\renewcommand{\arraystretch}{1.2}
\begin{math}
  \begin{array}{D{.}{,}{1}@{\,:}D{.}{,}{1} |
    D{.}{,}{5.0}@{\,}r@{}D{.}{,}{0}@{\,}l |
    D{.}{,}{5.0}@{\,}l|D{.}{,}{4}}
    \toprule
    \multicolumn{2}{p{2.0cm}|}{\rule[-1.5ex]{0pt}{4.0ex}\centering\etajet} &
    \multicolumn{4}{p{3.0cm}|}{\centering\diffetab} &
    \multicolumn{2}{p{2.2cm}|}{\centering\diffnloetab} &
    \multicolumn{1}{p{1.5cm}}{\centering\Cbhad} \\
    \multicolumn{2}{c|}{} & \multicolumn{4}{c|}{[\si{\pico\barn}]} &
    \multicolumn{2}{c|}{[\si{\pico\barn}]} & \\
    \midrule
    -1.6 & -1.1 &  57 & \pm & 22 & ^{+13}_{-3}  &  72 & ^{+22}_{-13} & 0.70 \\
    -1.1 & -0.8 & 121 & \pm & 21 & ^{+16}_{-16} & 182 & ^{+50}_{-30} & 0.78 \\
    -0.8 & -0.5 & 214 & \pm & 22 & ^{+22}_{-12} & 255 & ^{+69}_{-42} & 0.79 \\
    -0.5 & -0.2 & 233 & \pm & 21 & ^{+28}_{-21} & 307 & ^{+83}_{-50} & 0.79 \\
    -0.2 &  0.1 & 264 & \pm & 22 & ^{+28}_{-23} & 342 & ^{+91}_{-55} & 0.81 \\
      0.1 &  0.5 & 316 & \pm & 21 & ^{+23}_{-17} & 346 & ^{+96}_{-57} & 0.86 \\
      0.5 &  1.4 & 288 & \pm & 15 & ^{+20}_{-30} & 265 & ^{+82}_{-48} & 0.93 \\
    \bottomrule
  \end{array}
\end{math}
\end{tcblisting}
\end{sidewaystable}  

This version of the table also adds a few extra bells and whistles. It
uses a \Macro{rule} of zero width to give a bit more space above and
below the cross-sections. \texttt{p\{\ldots\}} switches to paragraph
mode, so \Macro{centering} is needed to get centred headers. It adds a
bit more space between the rows using \Macro{arraystretch}.
You have to play around a bit with the column widths. If you set one
of them too small it gets expanded anyway, so the two parts of the
table would not line up. Just for fun the bottom half of the table
uses \enquote{,} instead of \enquote{.} for the decimal point!
Admittedly the header is a bit complicated, but the numbers are nice
and easy to write!

%------------------------------------------------------------------------------
\section{Hiding columns}
\label{sec:tab:hide}
%------------------------------------------------------------------------------

Sometimes you write out a table with so many columns that it does not fit
on a page, even if you rotate it.
In sich cases you can of course reduce the number of columns by hand,
but it would be much nicer to just \enquote{hide} some of the columns
(as one often does in a spreadsheet).
The \Package{array} package allows one to define new column types
and you can use this feature to define a column that is not shown.

Following the idea found in
\url{https://texblog.org/2014/10/24/removinghiding-a-column-in-a-latex-table/}
we define two new column types \enquote{H} and \enquote{E}:

\newcolumntype{H}{>{\setbox0=\hbox\bgroup}c<{\egroup}@{}}
\newcolumntype{E}{>{\setbox0=\hbox\bgroup\let\pm\relax}c<{\egroup}@{}}
\begin{tcblisting}{listing only}
\newcolumntype{H}{>{\setbox0=\hbox\bgroup}c<{\egroup}@{}}
\newcolumntype{E}{>{\setbox0=\hbox\bgroup\let\pm\relax}c<{\egroup}@{}}
\end{tcblisting}{listing only}

The \enquote{E} variant is for columns that include \Macro{pm}.
This idea can also be extended to other content if necessary.
\Cref{tab:column1,tab:column2,tab:column3} show the original table
with variants hiding different columns.

\begin{table}[htbp]
  \caption{Simple table with columns that we want to hide.}%
  \label{tab:column1}
\begin{tcblisting}{textable}
\centering
\begin{tabular}{SSS[table-format=1.1(1)]
  S[table-format=3.1]@{\(\,\pm\,\)}
  S[table-format=2.1, table-number-alignment=left]
  S[table-format=3.1]@{\(\,\pm\!\!\)}S[table-format=1.1]}
  \toprule
  {Col.\ 1} & {Col.\ 2} & {Col.\ 3} &
  \multicolumn{2}{c}{Col.\ 4 with err.} &
  \multicolumn{2}{c}{Col.\ 5 with err.} \\
  \midrule
  1.1 & 2.1 & 3.1 \pm 0.2 & 4.1 & 0.1 & 5.1 & 0.3 \\
  1.2 & 2.2 & 3.2 \pm 0.3 & 4.2 & 12.2 & 5.2 & 0.4 \\
  1.3 & 2.3 & 3.3 \pm 0.4 & 4.3 & 0.3 & 5.3 & 0.5\\
  \bottomrule
\end{tabular}
\end{tcblisting}
\end{table}

\begin{table}[htbp]
  \caption{Simple table with columns 2 and 3 hidden.
    Note the use of the \enquote{E} column when the number contains an uncertainty.}%
  \label{tab:column2}
\begin{tcblisting}{textable}
\centering
\begin{tabular}{SHE
  S[table-format=3.1]@{\(\,\pm\,\)}
  S[table-format=2.1, table-number-alignment=left]
  S[table-format=3.1]@{\(\,\pm\!\!\)}S[table-format=1.1]}
  \toprule
  {Col.\ 1} & {Col.\ 2} & {Col.\ 3} &
  \multicolumn{2}{c}{Col.\ 4 with err.} &
  \multicolumn{2}{c}{Col.\ 5 with err.} \\
  \midrule
  1.1 & 2.1 & 3.1 \pm 0.2 & 4.1 & 0.1 & 5.1 & 0.3 \\
  1.2 & 2.2 & 3.2 \pm 0.3 & 4.2 & 12.2 & 5.2 & 0.4 \\
  1.3 & 2.3 & 3.3 \pm 0.4 & 4.3 & 0.3 & 5.3 & 0.5\\
  \bottomrule
\end{tabular}
\end{tcblisting}
\end{table}

\begin{table}[htbp]
  \caption{Simple table with column 4 (actually 4 and 5) hidden.
  The column heading, which is defined using \Macro{multicolumn}, also needs to be typeset with column type \enquote{H}.}%
  \label{tab:column3}
\begin{tcblisting}{textable}
\centering
\begin{tabular}{SSS[table-format=1.1(1)]
  H%S[table-format=3.1]@{\(\,\pm\,\)}
  H%S[table-format=2.1, table-number-alignment=left]
  S[table-format=3.1]@{\(\,\pm\!\!\)}S[table-format=1.1]}
  \toprule
  {Col.\ 1} & {Col.\ 2} & {Col.\ 3} &
  \multicolumn{2}{H}{Col.\ 4 with err.} &
  \multicolumn{2}{c}{Col.\ 5 with err.} \\
  \midrule
  1.1 & 2.1 & 3.1 \pm 0.2 & 4.1 & 0.1 & 5.1 & 0.3 \\
  1.2 & 2.2 & 3.2 \pm 0.3 & 4.2 & 12.2 & 5.2 & 0.4 \\
  1.3 & 2.3 & 3.3 \pm 0.4 & 4.3 & 0.3 & 5.3 & 0.5\\
  \bottomrule
\end{tabular}
\end{tcblisting}
\end{table}

\Cref{tab:column1} also illustrates different ways of including numbers with errors.
Column 4 is the one that looks best to me.