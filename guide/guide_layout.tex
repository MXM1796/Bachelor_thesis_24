% !TEX root = thesis_guide.tex
% chktex-file 1 chktex-file 21

%==============================================================================
\chapter{Layout and language}%
\label{sec:layout}
%==============================================================================

\LaTeX\ file: \url{./guide_layout.tex}\\[1ex]
\noindent
If you are not happy with the layout, fonts, etc.\ that are used in
this guide or the thesis skeleton, this chapter includes some examples
on how you can change things.


%------------------------------------------------------------------------------
\section{Page layout}%
\label{sec:layout:page}\index{page}
%------------------------------------------------------------------------------


Page geometry can be changed using either the \Package{typearea} or
the \Package{geometry} package. In this guide I use \Package{typearea}
as it is closely tied into \KOMAScript. The basic command is to
specify how many pieces to divide the page into. This guide uses
\Option{DIV=12} by default with a binding correction of
\Option{BCOR=5mm}.  You can then let \KOMAScript{} and
\Package{typearea} sort out the rest. Note that \Package{typearea}
will issue a warning that you have \enquote{Bad type area settings}
when using \Option{DIV=14}. If you want to avoid this warning, you can
set \Option{DIV} to 11 or smaller, but then the amount of text on each
page is reduced by quite a bit and your thesis will cover more pages.

The alternative is to use \Package{geometry}, where by default I say
that the text should cover about 75\% of the page. In practice, I have
found \Package{geometry} easier to use if you want to specify exactly
the layout, e.g.\ making single Feynman graphs with
\Package{feynmf}. \Package{typearea} is probably to be preferred for
documents on normal paper sizes.

See the beginning of the \KOMAScript{} guide for a detailed
introduction on how a page should be laid out.

The \Package{setspace} package has a number of useful options to change
spacing in a fairly easy way, if such options are not already
available in \KOMAScript.


%------------------------------------------------------------------------------
\section{Footnotes}%
\label{sec:layout:footnote}\index{footnote}
%------------------------------------------------------------------------------

A few tweaks may well be useful in order to get footnotes looking the
way you want them rather than using the default \KOMAScript{}
settings.

The default setting of \KOMAScript{} is:
\begin{tcblisting}{listing only}
\deffootnote[1em]{1.5em}{1em}{\textsuperscript{\thefootnotemark}}
\end{tcblisting}
In this guide and in \texttt{ubonn-thesis.sty} I have changed this to:
\begin{tcblisting}{listing only}
\deffootnote{1em}{1em}{\textsuperscript{\thefootnotemark}\ }
\end{tcblisting}
The differences are:
\begin{itemize}
\item The optional argument is missing that sets the box width for the
  footnote mark (which is right-adjusted). In this case the width of
  the first required argument is used instead, which also defines how
  much all other lines are indented. Hence all lines in the footnote
  are now left-aligned, while in the default setting the first line is
  indented more than the others.
\item In the last argument an extra space has been added so that the
  footnote mark is not quite so close to the text.
\end{itemize}

\deffootnote[1em]{1.5em}{1em}{\textsuperscript{\thefootnotemark}}
You might think that a nice way to write footnotes is:
\begin{tcblisting}{}
We want to include a footnote
\footnote{
  This is the footnote text.
}
about what the footnote should look like.
\end{tcblisting}
% If you do this you will find some extra space where it should not be.
% This is illustrated in footnote~\footref{foot:short}.
% \begin{tcblisting}{}
% To try out footnotes we try first an example which is probably how you
% would first try to write a footnote:
% \footnote{
%   This is an example of a footnote with extra space at the beginning.%
%   \label{foot:short}
% }
% \end{tcblisting}
\noindent
This example uses the default footnote settings.
As you can see this is not really satisfactory. You have a spurious
space between the colon and the footnote mark \(^{a}\). If may even
be the case that the footnote mark appears on the next line! 
To get the spacing
correct and still have \enquote{nice} \LaTeX{} you have to add some
judicious \texttt{\%} signs:
\begin{tcblisting}{}
We want to include a footnote%
\footnote{\label{foot:one}
  This is the footnote text.
  it is longer than the previous one, but about nothing in particular.
  It goes over more than one line to make sure that we can also have a proper
  look at the effect of indentation.
  It also demonstrates that footnotes can be given a label,
  if they need to be referred to from elsewhere.
}
about what the footnote should look like.
The space between \enquote{\(^{a}\)} and \enquote{This} should not actually be there
with the default footnote settings.
If you look very carefully \enquote{This} is not properly aligned with \enquote{than}.
\end{tcblisting}
\deffootnote{1em}{1em}{\textsuperscript{\thefootnotemark}\ }
\begin{tcblisting}{}
The same footnote with the \File{ubonn\_thesis.sty} default settings
demonstrates the effect of the changed spacing.%
\footnote{\label{foot:two}%
  This is a longer footnote about nothing in particular that goes
  over more than one line to make sure that we can also have a proper
  look at the effect of indentation.
  It also demonstrates that footnotes can be given a label,
  if they need to be referred to from elsewhere.
}
\end{tcblisting}

In standard \LaTeX{} you can use \Macro{footnotemark}
to set the symbol for a footnote and refer to it.
\KOMAScript{} has a better solution for this.
You define a \Macro{label} inside the footnote and
then refer to it via the label and \Macro{footref}.
You can see how to do this in footnote~\footref{foot:one}.
The \Package{cleverref} package claims to also know about footnote label.
However, if I use \Macro{cref\{foot:two\}} to refer to it, \cref{foot:two},
it points to the section and not the footnote.
Using the \Macro{footref} mechanism it is even
possible to click on the reference if you use pdf\LaTeX.

When a footnote contains a complete sentence, people still often
forget to add a full stop at the end of it --- try not to!

If you try to use footnotes in tables, you will probably have some
problems. The \Package{ctable} package allows you to include footnotes
that are part of the table. It also supports \Package{booktabs}, so it
should be possible to keep the same format. If you use the package
\Package{xtab} you can use the environment \Env{mpxtabular} 
instead of \Env{xtabular} to get footnotes
--- see \cref{sec:app:tables} for an example. 
If you use the \Package{longtable} package, use \Env{longtable}
rather than \Env{table} and \Env{tabular} in order to easily include
footnotes.
Another recent package (\TeXLive 2011) is
\Package{tablefootnote} which provides the command
\Macro{tablefootnote}. This is advertised to also work in sideways tables.


%------------------------------------------------------------------------------
\section{Thesis in German}%
\label{sec:layout:german}\index{German}
%---------------------------------------------------- --------------------------

\Package{babel} is a powerful package for handling different
languages. The languages to be used in a document can either be given
as options to \Macro{documentclass} or to \Package{babel}. I choose to
use \Macro{documentclass} here so that the style file
\texttt{ubonn-thesis.sty} is independent of the thesis language. If
you give more than one language make sure that the default language
comes last.

Your thesis is in German rather than English. What do you have to
worry about? First, set the languages in the \Macro{documentclass} to
\Option{UKenglish,ngerman} rather than the other way round.

Second, if you want to use a comma rather than a full stop for the
decimal point you may notice that numbers are sometimes not written
properly. In text mode they are OK, e.g.\ 91,1234, while in math mode
there is a small space after the comma, e.g.\ \(91,1234\).

If you use the \Package{siunitx} package, you can simply specify the
locale: \Macro{sisetup}\texttt{\{locale = DE\}}. This is already done
for you in \texttt{ubonn-thesis.sty} in such a way that if you
select \Option{ngerman} as the language of your document
\Macro{num}\texttt{\{2.3\}} will be printed as \num[locale=DE]{2.3}. I find this
by far the best way to handle such things.

You can also do such things locally in a
single table for example by using constructions such as
\enquote{\texttt{S[decimalsymbol=comma]}} in the column description of a table to
change full stops into commas --- see \cref{sec:table}.

Another way to avoid this problem by using the \Package{ziffer} package. It is
commented out in \texttt{ubonn-thesis.sty}. An alternative
is to remove the comments on the \TeX{} code snippet at the top
of the file \texttt{thesis\_defs.sty}:
\begin{tcblisting}{}
\mathchardef\CommaOrdinary="013B
\mathchardef\CommaPunct   ="613B
\mathcode`,="8000   % , im Math-Mode aktiv ("8000) machen
{\catcode`\,=\active
 \gdef ,{\obeyspaces\futurelet\next\CommaCheck}}
\def\CommaCheck{\if\space\next\CommaPunct\else\CommaOrdinary\fi}
\end{tcblisting}
As far as I have able to tell this above code does just what one wants
and so is probably better than trying to use \Package{ziffer}.

I have spent some effort to try to get around the problems that occur
if you try to use \Package{ziffer} and \Package{dcolumn} together. A way
that seems to work is to use the full stop as the decimal point in
columns that are formatted using \Package{dcolumn} and then just change
full stop to a comma, i.e.\ use the form \verb+D{.}{,}{2}+. The table
below illustrates this usage:
\begin{tcblisting}{listing side text}
\begin{otherlanguage}{ngerman}
\begin{tabular}{l|
  D{.}{,}{4}@{\(\,\pm\,\)}
  D{.}{,}{4}}
  Teilchen & \multicolumn{2}{c}{Masse [\si{\GeVovercsq}]}\\
  \midrule
  \(Z\) boson & 91.0372 & 0.0002\\
  \(W\) boson & 83 & 0.035\\
  Top quark & 173.3 & 1.1\\
\end{tabular}
\end{otherlanguage}
\end{tcblisting}
If you use \Package{ziffer} and want to keep the decimal symbol,
e.g.\ \verb+D{.}{.}+ or \verb+D{,}{,}+ this does
not seem to work. Hence if you try to use \Package{ziffer} with this
guide you will get error messages.

The other question is whether to use the traditional \TeX{} way of
writing letter with umlauts: \verb+\"{u}+ to get \"{u}, or \verb+\"a+
to get \"a, or just to type in the character directly. I strongly
recommend typing the characters directly. It makes your \LaTeX{} much
easier to read and spell-check. You can either switch your keyboard to
German (this is simple under Windows, macOS, KDE or Gnome (Unity)) or set the
compose key and then type the sequence \texttt{Compose " a} to get ä for example. % chktex 18


%------------------------------------------------------------------------------
\section{Fonts}%
\label{sec:layout:font}
%------------------------------------------------------------------------------

Fonts to use for the various parts of a
document can usually be set using the \Macro{setkomafont} command.
I have set a few such fonts in \texttt{ubonn-thesis.sty}.
In particular
I set the fonts for the title page so that it conforms more closely to
the requirements for theses and also have a few more commented out
examples there.

Which font should I use?
\url{http://tug.ctan.org/tex-archive/info/Free_Math_Font_Survey/en/survey.html}
has a list of fonts that one can consider using.
I have included
commented out packages that use some of these in the style file.
Good alternatives to \Package{newtx} and \Package{txfonts} (with the \Option{varg} option) seem
to be either \Package{lmodern} or \Package{pxfonts}.
See \cref{sec:tips:units} for some of the things you have to worry
about if you use a font for which the text mode and math mode display
numbers differently.

I have tried to compile the guide using \XeLaTeX\index{XeLaTeX} and \LuaLaTeX\index{LuaLaTeX}.
After a few tweaks both work.
Older versions of \LuaLaTeX\ may have problems with some tables, as well as listings,
but they seem to be OK now.
You can use the command \Command{make guidexe} to try to compile using \XeLaTeX.
You can use the command \Command{make guidelua} to try to compile using \LuaLaTeX.
The TeX Gyre fonts Pagella or Termes are used, depending on whether you pass the
\Option{palatino} option to the document class or not.
The advantage of these tools is that you can use any font that you have installed on your computer.
However, the number of (free) fonts that also have all the math symbols is very limited.
I need to tweak settings more before the options can be considered anything other than experimental!


%------------------------------------------------------------------------------
\section{Other languages}%
\label{sec:layout:lang}
%------------------------------------------------------------------------------

If you have a few words in the foreign language just use
\Macro{foreignlanguage},\\
e.g. \texttt{\textbackslash foreignlanguage\{ngerman\}\{Das
  Physikalische Institut der Universität Bonn\}}  to produce \foreignlanguage{ngerman}{Das
  Physikalische Institut der Universität Bonn}. You might ask why
bother? Hyphenation, in particular, is not the same in different
languages, so telling \LaTeX{} which language the text is in certainly
helps. To change the language at a certain point in the document use
\Macro{selectlanguage}. To set a block of text (inside an environment)
use the \Env{otherlanguage} environment. I used this in the title
pages, as (except for the cover page) they are in German. It certainly
does no harm to specify the language of the abstract
within \Macro{thesisabstract}.

The \Package{csquotes} package advertises itself as the way to cope
with quoting\index{quotes} things in different languages. The basic
command to use is \Macro{enquote}\index{quotation marks}
for quoting things in the current default language, while you
use \Macro{foreignquote} to quote things in another language. For
example, John said \enquote{I have too much to do at the moment},
while \foreignlanguage{ngerman}{Johannes sagte}
\foreignquote{ngerman}{ich habe im Moment alle Zeit der Welt}.
Note that \Package{babel} uses \Option{ngerman} for \enquote{new} German
spelling, while \Package{csquotes} options calls the same thing
\Option{german}. However, this only affects the setting of options. If
you use commands such as \Macro{foreignquote} you can specify
\Option{ngerman} as the language.

Having used \Package{csquotes} for several years now, I find it a
really nice way of quoting things properly in the language you are
writing your text in without having to worry about using the correct
opening and closing quotes, so I warmly recommend it.

%------------------------------------------------------------------------------
\section{Coloured links}%
\label{sec:layout:link}\index{color!link}
%------------------------------------------------------------------------------

In the default version of the thesis, links in the table of contents
are coloured blue, citations are coloured dark magenta and URLs are
coloured dark green. These settings are in \texttt{ubonn-thesis.sty}.

For the printed version you probably do not want these things to be
coloured. You can change the \Package{hyperref} options using the
\Macro{hypersetup} command as indicated in the main file of your
thesis. Using these changes, such links will be surrounded by a
coloured box when viewed on the screen, while the box will not be shown when the thesis
is printed.


%------------------------------------------------------------------------------
\section{Chapter headings}%
\label{sec:layout:chapter}\index{chapter!heading}
%------------------------------------------------------------------------------

The standard appearance of the chapter headings is not very exciting!
You can make some changes with the \KOMAScript\ options, but nothing
very radical.
A number of packages exist that can make larger changes. 
I tried out \Package{fncychap}, \Package{quotchap} and \Package{titlesec}.
A variant of \Package{titlesec} was used for this guide and for theses (version 3.0).
As combining \Package{titlesec} with \KOMAScript{} is not recommended,
the settings are now done by hand by default.
The problem with this version was that the bibliography was given a chapter number,
if it was part of \Macro{mainmatter} rather than \Macro{backmatter}.
As of version 4.0, I use a very similar style,
but this time the changes are made using \KOMAScript{} adjustment as suggested by the author.\footnote{%
\url{http://www.komascript.de/chapterwithlines}}
This seems to work better.
If you want to get back to the
usual style, just comment out the appropriate lines in
\texttt{ubonn-thesis.sty}. See the documentation on
\Package{titlesec} on how to make further adjustments.

%%% Local Variables:
%%% mode: latex
%%% TeX-master: "./thesis_guide"
%%% End:
