% !TeX root = thesis_guide.tex
% chktex-file 1 chktex-file 46

%==============================================================================
\chapter{\TeX{} setup and packages}%
\label{sec:app:tex}
%==============================================================================

\LaTeX\ file: \href{run:./guide_appendix_texsetup.tex}{guide\_appendix\_texsetup.tex}\\[1ex]
\noindent
Some packages that I recommend,
and that are still subject to active development,
are \Package{biblatex} and \Package{siunitx}.
\Package{TikZ-Feynhand}, \Package{diffcoef} and \Package{derivative}
are new and very interesting packages.
If you want to fully exploit their capabilities,
you should make sure you \LaTeX{} installation is as recent as
possible --- I recommend at least \TeXLive 2020 and preferably later.

The institute is a member of Dante
and I therefore receive a copy of the \TeX\ Collection DVD every year.
You can also download the installation you need from the \TeX\
Users (TUG) group web page: \url{http://www.tug.org/texcollection}.


%------------------------------------------------------------------------------
\section{Integrated environments}%
\label{sec:app:compile}
%------------------------------------------------------------------------------

As mentioned in \cref{sec:tips:do}, I highly recommend
that you use an integrated environment for editing and compiling your
thesis. All such tools allow you to define projects, which then know
about which files should be included when compiling.

\TeXstudio\index{TeXstudio} and
\TeXmaker\index{TeXmaker} can be installed under Windows, macOS or
Linux. I have tried \TeXstudio in all three systems and it works nicely. It
can be downloaded from \url{http://texstudio.sourceforge.net/}.
Others report good experience with \TeXmaker (under Ubuntu).
It is available as an Ubuntu package or from
\url{http://www.xm1math.net/texmaker}.
\TeXmaker and \TeXstudio (\TeX makerX) split in 2009, so both packages
contain a lot of similarities.
Visual Studio Code\index{Visual Studio Code}\footnote{%
\url{https://code.visualstudio.com}}
is a fast and powerful editor with a very nice extension for \LaTeX\ documents.
It is also available for Windows, macOS and Linux.
Kile\index{kile} is quite popular, especially if you use the KDE
desktop environment. 

I used to use \texttt{emacs}\index{emacs} and AUCTeX, as I can
then use the same editor for everything. Note that the
RefTeX\index{RefTeX} mode in \texttt{emacs} also provides powerful
tools for finding and inserting cross-references and the names of
citations easily.


%------------------------------------------------------------------------------
\subsection{\TeXstudio}%
\label{sec:app:texstudio}\index{TeXStudio}

I used \TeXstudio as my main tool for creating \LaTeX\ documents starting in 2013.
One very nice feature is that it works under macOS, Windows and Linux.
It is fairly easy to adapt to your preferences.
Additional keyboard shortcuts can be added and the editor can be customised.
I usually change the \texttt{TAB} behaviour so that spaces rather than TABs are inserted.

If you use the integrated PDF viewer, it is very easy to switch between the PDF file and
the \LaTeX\ source code, which is extremely helpful.
You can also usually click on an error in the Message window and the cursor will jump to the right spot.
This makes debugging and proofreading much easier.

You can configure \TeXstudio to use \Command{latexmk}\index{latexmk} by default.
It is easy to switch between \texttt{biber} and \texttt{bibtex} using the Build preferences.
If you want to use \Command{latexmk} with \TeXstudio under Windows,
you may have to install Perl from e.g.\ \url{https://strawberryperl.com/}.

\TeXstudio should also know about your citations.
If you do not get a list of citations when you type \verb+\cite+,
you should first close the file and open 
in \TeXstudio \Option{Options \(\rightarrow\) Configure TexStudio \(\rightarrow\) Completion} and check \File{biblatex.cwl}.

You can also make a glossary from within \TeXstudio.
In order to do this you have to define a user build command.
You can find such commands under:
\textsf{Options} $\to$ \textsf{Configure TeXstudio} $\to$ \textsf{Build}.
I defined a new command which I called \enquote{BuildWithGlossary}.
This executes \texttt{pdflatex, biber, makeindex, makeglossary, pdflatex, pdflatex}.


%------------------------------------------------------------------------------
\subsection{Visual Studio Code}%
\label{sec:app:vscode}\index{Visual Studio Code}%
\index{vscode|see{Visual Studio Code}}

I currently use Visual Studio Code\index{Visual Studio Code} as my editor for both \LaTeX\ and other programs.
It is available for Windows, macOS and Linux.
The big advantage is that you have one editor for both documentation and code.
It works quite well for writing \LaTeX,
but assumes that you know the \LaTeX\ commands that you want to include.
Rapid development is ongoing,
also for the \LaTeX\ extension \textsf{\LaTeX\ Workshop}\index{LaTeX Workshop@\LaTeX\ Workshop} that you should certainly install.
If you have \Command{chktex} installed you should also enable it,
as it helps you to find errors 
and gives a lot of information on the quality of your \LaTeX.


%------------------------------------------------------------------------------
\subsection{Overleaf}%
\label{sec:app:overleaf}\index{Overleaf}

If you want to write your thesis using Overleaf,
you first have to upload it as a project there.
The best way to do this is to create a new thesis skeleton as usual, e.g.:
% specifying the version of \TeXLive being run by Overleaf, e.g.:
\begin{bashlisting}
make new [THESIS=projectname]
\end{bashlisting}
\noindent Compress the directory into a zip file:
\begin{bashlisting}
  zip -r projectname projectname
\end{bashlisting}
\noindent and upload it to Overleaf as a new project.

Overleaf should then be able to compile the skeleton thesis and you can start writing!


%------------------------------------------------------------------------------
\subsection{Kile}%
\label{sec:app:kubuntu:kile}\index{kile}

Many people use Kile as their environment for editing and compiling \LaTeX. 
I have not used it for a number of years,
so some of the instructions I give here may be out of date.

If you use \texttt{bibtex} for your references then the default
setup does not have to be changed. If you want to use the
\Option{biber}\index{biber} backend then you should integrate the
command into Kile.

To do this you need to do the following:
In Kile: \Option{Settings $\to$ Configure Kile $\to$ Tools $\to$ Build} and
then insert a new tool (Biber) on the left-hand side. It is probably
best to define the class of the tool as \BibTeX. Then on the
right-hand side in the \Option{General} tab the command should be
\texttt{biber} and the option \texttt{\%S}. In the \Option{Advanced}
tab you should set the \Option{Source extension} to \texttt{aux} and
the \Option{Target extension} to \texttt{bbl}.

I would then recommend adding a new configuration to the
\Option{QuickBuild} called \Option{PDFLaTeX+Biber+ViewPDF} that
executes the commands: \texttt{PDFLaTeX, Biber, PDFLaTeX, PDFLaTeX,
  ViewPDF}. This is also the series of commands that the
\texttt{Makefile} executes.

I have not yet investigated how to also make the glossary from within
Kile, but it should be possible to do it in the same way as you add a
\texttt{biber} command.


%------------------------------------------------------------------------------
\section{macOS}%
\label{sec:app:mac}\index{macOS}
%------------------------------------------------------------------------------

For the past several years,
I have been installing and running \LaTeX\ on a MacBook Pro.
The standard package you start with is \TeX Shop. 
Together with Basic\TeX\ this provides a basic environment. 
However, if you want to do more (like write a thesis), you should install \TeXLive.
There is a complete \TeXLive installation called Mac\TeX\ 
that you can download from \url{http://www.tug.org/mactex/}.
The Dante \TeX\ Collection DVD also includes Mac\TeX, which is what the \TeX Shop web page
recommends as the best combination to install.
You can also install Mac\TeX{} using Homebrew\index{Homebrew}.
Install and run the \TeX\ Live Utility to keep your installation up to date.

As discussed above, there are macOS versions of \TeXstudio and \TeXmaker which you can try.
I used to use \TeXstudio.
More recently I switched to using Visual Studio Code,
which is a very nice editor with integrated \LaTeX\ and Git support.

%------------------------------------------------------------------------------
\section{(Ku|Xu|U)buntu}% chktex 36
\label{sec:app:kubuntu}\index{Linux}\index{Ubuntu}\index{Kubuntu}\index{Xubuntu}
%------------------------------------------------------------------------------

For my tests with Ubuntu I use Xubuntu 20.04 and \TeXLive 2019.
I also check that things work with older versions.
The oldest version of \TeXLive that I have access to is 2013.

% I also made the guide on a plain Xubuntu 12.04
% installation, which used the default packages from the Xubuntu image. At
% installation time I enabled the usage of 3rd party non-free
% software. This is be needed for \texttt{acroread} among other
% things.\footnote{Nowadays you have to download \texttt{acroread} from
%   the Adobe web page.} 
If you want to write your thesis in British (UK) English you should make sure
that that language is fully installed. If you want to write your
thesis in German, then also install German as a language, even if
your system is in English.

I explicitly installed the following packages in addition:
\begin{bashlisting}
(k|x)ubuntu-restricted-extras
git #Used to be subversion
texlive
texlive-science
texlive-bibtex-extra
texlive-fonts-extra
texlive-latex-extra
biber
latexmk
texlive-lang-german
feynmf
\end{bashlisting}
\noindent I also had to install \texttt{texlive-metapost} and \texttt{tracklang} in order to compile the guide.
% The guide also needs \texttt{texlive-generic-extra}, at least for \TeXLive 2015.
If you want to use the \Package{IEEEtrantools} you have to install \texttt{texlive-publishers},
or the package \Package{IEEEtran}.
If you want to use \LuaLaTeX\ you also have to install \texttt{texlive-luatex}
and if you want to use \XeLaTeX\ you have to install \texttt{texlive-xetex}.
You may also have to install \texttt{texlive-xetex} even if you only plan to use \LuaLaTeX.
In older Ubuntu installations you may have to also install \texttt{texlive-math-extra}.
In newer versions this package has been removed.

If your \LaTeX\ installation is old, you may want to bring it up-to-date.
For Ubuntu variants there is a simple solution.
Just add \TeXLive backports to your list of sources
(use the \texttt{sudo apt-add-repository ppa:texlive-backports/ppa} command).
You should then be able to give the command
\texttt{sudo apt-get update; sudo apt-get upgrade}.
Note that this variant will overwrite your current
\LaTeX\ installation.
An alternative is the method discussed in the following paragraphs,
which allows you to have two (or more) different installations in parallel.


%------------------------------------------------------------------------------
\section{Fedora}% chktex 36
\label{sec:app:fedora}\index{Linux}\index{Fedora}
%------------------------------------------------------------------------------

For my tests with Fedora I used Fedora~33 which I downloaded from
\url{https://getfedora.org},
which includes \TeXLive 2019.
Most \LaTeX\ packages come as separate Fedora packages,
so the list of what you have to install is long.

I installed the following packages using the command \Command{dnf install \dots}:
\begin{bashlisting}
texlive
latexmk
texlive-newtx
texlive-mweights
texlive-siunitx
texlive-longtable
texlive-hep
texlive-mhchem
texlive-biblatex
texlive-background
texlive-cleveref
texlive-babel-german
texlive-hyphens-german
\end{bashlisting}

\noindent These are sufficient to compile the skeleton thesis.
For the thesis guide I needed in addition:
\begin{bashlisting}
texlive-standalone
texlive-physics
texlive-tikz-feynhand
texlive-tikz-feynman
texlive-tikz-3dplot
texlive-pgfplots
texlive-pgfopts
texlive-glossaries
texlive-import
texlive-bigfoot # needed for suffix
\end{bashlisting}
\noindent You will probably need at least some of these packages for your thesis.

\LuaLaTeX\ and \XeLaTeX\ were either installed by default,
or with one of the other packages.

For the tests that I conducted with a new installation on 2020-10-21,
I got an error when trying to use the \Package{newtx} fonts.
After some searching, I was able to fix it using the commands:
\Command{sudo updmap --sys --syncwithtrees} and \Command{sudo updmap --sys}.

  
%------------------------------------------------------------------------------
\section{Installing (a newer version of) \TeXLive by hand}%
\label{sec:app:texlive:update}
%------------------------------------------------------------------------------

A \TeX nische Komödie issue (3/2011) contained detailed instructions on
how to install a newer version of \TeXLive in parallel to the
default version.
For a while, I did this as standard practice on my laptop and it worked very well.
Another useful source of information on how to install \TeXLive can be found under
\url{http://tug.org/texlive/quickinstall.html}.

A brief summary of how to do the installation by hand: you must have
\texttt{perl-tk} installed.
Go to a directory where you want the
install script to be and then execute the following chain of commands:
\begin{bashlisting}
wget http://mirror.ctan.org/systems/texlive/tlnet/install-tl-unx.tar.gz
tar zxvf install-tl-unx.tar.gz
cd install-tl-...
sudo ./install-tl -gui perltk
  Turn off some of the languages
  Make sure installation is in /usr/local/texlive
  Install
\end{bashlisting}
\noindent I installed the scheme \Option{scheme-tetex} and added the
collections: \Option{BibTeX additional styles}, \Option{Additional fonts} and
\Option{LaTeX additional packages}.
If you want to use the package \Package{TikZ-Feynman} you probably need to include \Option{LuaTeX packages}.
If you want to use Xe\TeX{} include \Option{XeTeX and packages}.
When the installation is complete (can easily take \numrange{1}{2} hours):
\begin{bashlisting}
cd ..
echo 'export PATH=/opt/texbin:${PATH}' > texlive.sh
sudo cp texlive.sh /etc/profile.d/
sudo ln -s /usr/local/texlive/2020/bin/x86_64-linux /opt/texbin
\end{bashlisting}
\noindent You probably have to login again to activate the new \texttt{PATH}.
Check whether it is correct with
\begin{bashlisting}
  echo $PATH
\end{bashlisting}
  
If you have a 32-bit installation replace \texttt{x86\_64-linux} with \texttt{i386-linux}.
Note that the three commands set things up so
that \texttt{/opt/texbin} is at the beginning of your \texttt{PATH}
and that \LaTeX\ is then taken from there rather than the usual
\texttt{/usr/bin}.

On my test Ubuntu installation, 
I installed a very basic \TeXLive installation when setting up the system
and then added several different \TeXLive installations using this method.
This works very well.

For updating, additions of packages etc., it is useful to define an
alias in your \File{.bashrc} or \File{.bash\_aliases}:
\begin{bashlisting}
function sutlmgr () {
    if [[ -z "$@" ]]; then
      sudo /opt/texbin/tlmgr -gui
    else
      sudo /opt/texbin/tlmgr "$@"
    fi
}
\end{bashlisting}
\noindent You then give the command \texttt{sutlmgr} to start the \TeXLive
manager interface. You can use this interface to install new packages
as well as to update the ones you have installed.
If you want to update things, you should first click on \enquote{Load default} to
load a repository and then you can do things such as
\enquote{Update all installed}.

%------------------------------------------------------------------------------
\section{Windows}%
\label{sec:app:windows}\index{windows}
%------------------------------------------------------------------------------

I also tried to to make the guide and a skeleton thesis under Windows. 
My tests have been done with Windows 10.
I simply downloaded the MiK\TeX\ installer from \url{http://miktex.org}.
This is discussed in \cref{sec:app:miktex}.
You can also install \TeXLive for Windows. Some hints on how to do
this are given below (\cref{sec:app:texlive}).\footnote{%
I have had success with the 2010 \TeX{} Collection DVD from Dante to install
pro\TeX t, which is based on MiK\TeX.
Newer versions should also work without problems.
What the user sees is \TeXworks
which is deliberately supposed to look like the Mac's \TeX Shop.}

Once you have a \TeX\ installation, you have to get the thesis style files and the guide.
To do this I used \texttt{TortoiseGit} which is available from
\url{https://tortoisegit.org} as the interface between Windows and
\texttt{Git}.\index{tortoisegit}\index{Git}

To checkout the files, simply start Windows Explorer and go to the
top-level directory where you want to do the checkout, right-click on
the mouse and enter\\
\texttt{https://bitbucket.team.uni-bonn.de/scm/uni/ubonn-thesis.git}
as the URL\@.
If you do this you can stay at the cutting edge!
If you want a particular version then specify the branch you want, e.g.\ \enquote{8.0}.
You can of course simply download the appropriate tar file from the PI webpages.

To set up your thesis, you have to do by hand what the
\texttt{Makefile} does under Linux or install GNU Make.
If you want to do it by hand, you should make a new directory
\texttt{mythesis}, copy \texttt{thesis\_skel/thesis\_skel.tex} to
\texttt{mythesis/mythesis.tex}, and copy the rest of the files you need from
\texttt{thesis\_skel} into \texttt{mythesis}. In addition you should
copy the style files \texttt{ubonn-thesis.sty} and \texttt{ubonn-biblatex.sty} into the
\texttt{mythesis} directory.

To compile things, I ran the sequence pdf\LaTeX, Biber, pdf\LaTeX,
pdf\LaTeX\ to get the output file for the skeleton thesis.
You should make sure that the \BibTeX\ command in \TeXstudio or \TeXworks
is set to \texttt{biber},
unless for some reason you want to use traditional \BibTeX.

If you want to use \Command{latexmk} with \TeXstudio,
you probably have to install Perl.
I installed Perl from \url{https://strawberryperl.com},
rebooted and \Command{latexmk} then worked.

If you want to install GNU Make you can try the following:
You should install the \enquote{Minimal Gnu for Windows} from \url{http://www.mingw.org}.
I downloaded and ran \texttt{mingw-get-setup.exe}.
Once you have installed it you can run the \enquote{MinGW Installer} to install the software you want.
I think you should install \Package{msys-base} from \enquote{Basic Setup}.
This installs \texttt{make} and quite a lot of other typical Unix commands.
Finally you have to add \verb|C:\MinGW\msys\1.0\bin| to your \texttt{PATH}.
One way to do this is with the command \verb|setx PATH %PATH%;C:\MinGW\msys\1.0\bin| using a Windows Command Prompt, 
i.e.\ \texttt{cmd.exe}.
You can also use the GUI in the System Administration to set environment variables.
If you get \texttt{make} to work properly you start \texttt{cmd.exe},
navigate to the correct directory and give the usual commands to create and compile your thesis:
\begin{verbatim}
make new
make
\end{verbatim}

If you run pdf\LaTeX\ with the option \texttt{-enable-write18} and
include appropriate \texttt{mpost} commands in your \LaTeX\ file(s), %chktex 36
as discussed in \cref{sec:fig:feynman:feynmp}, Feynman graphs made
using \Package{feynmp} work well.
With \TeXstudio, you can simply
include this option in the pdf\LaTeX\ command: \textsf{Options} $\to$
\textsf{Configure TeXstudio} $\to$ \textsf{Commands}.

I have not invested any effort in trying to get \texttt{feynmf}
working under Windows. The output files are produced, but I have not
tried to run Metafont on them.


%------------------------------------------------------------------------------
\subsection{MiK\TeX}%
\label{sec:app:miktex}

I usually install \TeX\ under Windows via a download from Internet.
You should decide whether to install Mik\TeX{} for yourself or for all users.
I think maintenance is easier if you just install it for yourself.
Once the installation is finished, I also update all the
packages using the MiK\TeX{} Maintenance (Admin) $\to$ Update (Admin) program.
Be patient --- these updates always take a while!

With the minimal version of MiK\TeX{} you are then asked whether to
install the missing packages when first compiling a file that needs
them. Note that such missing packages are installed in your local
directory tree and not in the central directory tree.
This means that if you want to update MiK\TeX{} in the future
you should run the MiK\TeX{} Update both as Admin and as a normal user,
if you installed Mik\TeX{} for all users.
Be patient if the first compilation of a document appears to hang.
This usually means that new packages are being installed.

The last time I have installed \TeX\ under Windows using a DVD was
from the \TeX{} Collection 2010 DVD. 
To do this I logged in as an Administrator and then opened the DVD\@.
It opened a \TeX\ Collection window and I then clicked on pro\TeX t Quick Install and installed
MiK\TeX{} (minimal version),\index{MiKTeX}
Ghostscript\index{ghostscript} and Ghostview\index{ghostview} for all users.

Note that the Dante \TeX\ Collection 2011 DVD has dropped
\TeXnicCenter within pro\TeX t and is using the \TeX makerX front-end instead.
\TeX makerX has been superseded by \TeXstudio.
My comments on \TeXnicCenter have been relegated to \cref{sec:app:old}.
There is also the package \TeXmaker.


%------------------------------------------------------------------------------
\subsection{\TeXLive}%
\label{sec:app:texlive}

You can install \TeXLive instead of pro\TeX t. To do that you can
either use the \TeX\ Collection DVD or download \TeXLive. If you
download \TeXLive you should unzip the file that you get and go to the
directory \texttt{install-tl/install-tl-YYYYMMDD}. In that directory you
should run \texttt{install-tl-advanced.bat} as Administrator if you want to install
\TeXLive in a system directory. I installed the \enquote{medium}
scheme. The following extra collections were needed:
\begin{itemize}
\item \textsf{BibTeX additional styles}
\item \textsf{Generic additional packages}
\item \textsf{LaTeX additional packages} --- needed for \Package{csquotes}
\item \textsf{Natural and computer sciences}
\end{itemize}
I also added the German documentation. Do not forget to toggle the
option \Option{All users} if you want more than one user to be able to
use \LaTeX\ on your machine. If you forget one of the above
collections, you can install the missing packages later.

While MiK\TeX\ can install missing packages on the fly, this does not
seem to be the case for \TeXLive. To install extra packages you should
start the \TeXLive manager and before doing anything else you should
load the default repository.

You can either use \TeXworks as the front end or you can install
\TeXstudio or \TeXmaker.


%------------------------------------------------------------------------------
\section{Making the guide}% chktex 36
\label{sec:app:guide}\index{Guide}
%------------------------------------------------------------------------------

Using the installations described above I was able to issue to following commands to
produce the guide (assuming that you have \TeXLive 2016 or later):
\begin{bashlisting}
git clone https://bitbucket.team.uni-bonn.de/scm/uni/ubonn-thesis.git
cd ubonn-thesis
\end{bashlisting}
\noindent Set the \Macro{texlive} variable appropriately in \texttt{guide/thesis\_guide.tex}.
\noindent Then the following worked:
\begin{bashlisting}
cd guide
make guide|guidelua|guidexe
\end{bashlisting}
\noindent so all packages are there as well as at least the default font which I selected.
% If your version of \TeXLive is older than 2020, you should set \Macro{texlive} accordingly.
% If your version is older than 2013 you should not pass the option \texttt{newtx} to \Package{ubonn-thesis}.
Some figures or tables may not appear with versions of \TeXLive earlier than 2017,
as, for example, the \Package{TikZ-FeynHand} package was only introduced then.

The default version of the guide uses \Option{biber} for the
\Package{biblatex} package and includes\\
\texttt{./guide/refs/example\_refs-utf8.bib} which contains some umlauts.\footnote{%
The same references using normal ascii test (suitable for \texttt{bibtex}) can be found in
\texttt{./guide/refs/example\_refs-ascii.bib}.}
You give the command \texttt{make guide} to compile, which uses \texttt{biber}\index{biber} rather than
\texttt{bibtex}\index{bibtex} to process the \texttt{.bib} files.

The default version of the guide also compiles on ATLAS Bonn desktop
computers (Debian 11 with \TeXLive 2020). % chktex 36
If you want to use
\Package{feynmf} instead of \Package{feynmp} to compile the guide, you
have to pass the option \Option{feynmf} instead of \Option{feynmp} to \texttt{ubonn-thesis.sty} 
and give the command \texttt{make feynmf} before \texttt{make guide}.

If you want to switch from \texttt{biber}\index{biber} to
\texttt{bibtex}\index{bibtex} or \texttt{bibtex8}\index{bibtex8} or vice versa, give the commands
\texttt{make cleanguide cleanbbl cleanblx} before you try to compile
the guide (or your thesis).

In order to edit \LaTeX\ you need an editor such as
\texttt{Visual Studio Code} or \texttt{emacs}, or
a program such as \TeXstudio or \TeXmaker or
\texttt{kile} (and the associated \texttt{kile-doc}).
These were discussed above.

\TeXstudio knows about indexes and glossaries. You have to set up a
user command to execute \Command{makeindex} and \Command{makeglossaries}
as discussed in \cref{sec:app:compile}. With the addition of
these commands, you should also be able to compile this guide.
