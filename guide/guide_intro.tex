%==============================================================================
\chapter{Introduction}
\label{sec:intro}
%==============================================================================

\LaTeX{} file: \url{./guide_intro.tex}\\[1ex]
\noindent
When you want to start writing your thesis you usually ask a (more
senior) colleague if he or she has a \gls{LaTeX} framework that you can
start with. He or she in turn had asked their (more senior) colleague
for an example thesis several years earlier etc.! Maybe it is
surprising that we are actually using \LaTeX{} and not \TeX{} to write
theses!

\LaTeX\index{LaTeX@\LaTeX} (or more precisely the packages that one
can use in \LaTeX) is actually in a state of continual development
and improvement, so it certainly makes sense to review what packages
are available, how they should be used and whether there are better
ways of doing things than methods used 10 or more years ago.

The aim of this guide is to break with the tradition of just adapting
what your predecessor used and provide up-to-date guidelines on the
layout and packages that can or should be used for thesis writing. The
guide should also provide you with enough information for you to
concentrate on the content of your thesis, rather than having to spend
too much time making it look nice!

You may ask why bother? First and foremost a thesis is something that
you should be proud of! I therefore think it is actually worth
devoting some effort to not only making it look good, but also to
using correct and consistent notation when you write it. Figures and
tables should be legible and understandable (including the size of the
axis labels!). You should, however, not have to spend too much time
working out how to make the thesis look the way you want it to. It is
also good if you can avoid annoying or irritating your supervisor if he
or she also thinks that \si{\GeVovercsq} should be written like this and
not as GeV/c$^{2}$ or $GeV/c^{2}$ etc.\ or some mixture of the two.

The recommendations are based on experience I gained:
\begin{itemize}
\item  preparing the
lecture course \foreignquote{ngerman}{EDV für Physiker} in WS06/07 and WS13/14;
\item editing the book \enquote{Physics at the Terascale}, which was
  published in April 2011;
\item rewriting and maintaining the ATLAS \LaTeX\ document class and style files;
\item regular reading of the \foreignquote{ngerman}{TeXnische Komödie},
  which is published by \foreignlanguage{ngerman}{Dante
    (Deutschsprachige Anwendervereinigung \TeX)} 4 times a year;
\item general interest in preparing good quality documents;
\item reading quite a lot of theses!
\end{itemize}

This document does not attempt to explain how to write \LaTeX. I
assume a basic level of knowledge. The aim is more to give some
practical tips and solutions to solve problems that often occur when
you are writing your thesis. There are many books and online documents
to help you get started, so many in fact that it is difficult to know
where to start. My favourite is \enquote{Guide to \LaTeX} from
Kopka~\cite{kopka04}. Be sure to read the Fourth Edition though. It
was originally written in German where the title is
\foreignquote{ngerman}{\LaTeX: Eine Einführung}. When you want to know
what packages exist, what they can do and how to use them, consult
\enquote{The \LaTeX\ Companion} from M.~Goossens et al.~\cite{goossens04}.
A fairly comprehensive online guide is the
\enquote{A (Not So) Short Introduction to LaTeX2e.}~\cite{lshort},
which is available in many languages.
Help on getting started and a list of online
documents can be found on the CTAN (Comprehensive \TeX\ Archive
Network) information page
\url{http://tug.ctan.org/starter.html}. Other useful sources of
information that I and others use are:
\begin{itemize}
\item \url{http://www.tex.ac.uk/faq}:
  This contains an extensive FAQ (maybe even a bit better than the German
  one maintained by Dante). An interesting feature is the \enquote{Visual FAQ}
  that serves as a rather unorthodox, but very intuitive kind of index:
  \url{http://www.tex.ac.uk/tex-archive/info/visualFAQ/visualFAQ.pdf}.
\item \url{http://tex.stackexchange.com} often comes up in Google
  searches and contains a lot of very useful tips.
\item \url{http://detexify.kirelabs.org/classify.html} contains a
  little online tool to find \LaTeX\ names of symbols. It works quite
  well and can be a lot quicker than searching through the written
  documentation.
\end{itemize}

Not everyone knows about the \texttt{texdoc}\index{texdoc} command
which should be available for Linux and MacOSX \TeX{} installations. To get help
on a package, you can simply give the command \texttt{texdoc
  geometry}, etc. Note that to see the \KOMAScript{} manual you have
to know the name of the PDF file: \texttt{texdoc scrguide} or
\texttt{texdoc scrguien} for the German and English versions,
respectively. The AMS Math users guide also does not have a totally
obvious name -- try \texttt{texdoc amsldoc}.\index{AMS math} It
contains a whole host of useful information on typesetting
(complicated) equations.

The Physikalisches Institut is a member of Dante and so receives three
copies of each issue of \foreignquote{ngerman}{der TeXnische Komödie},
one of which is available in the department library in PI.  The
booklet often contains useful hints on typesetting. We also get a DVD
every year with \TeX\index{TeX distribution@\TeX\ distribution}
distributions for Unix, MacOSX and Windows. Details on how to install
a \LaTeX\ distribution can be found in Appendix~\ref{sec:app:tex}.

If you write your thesis in English,\index{English} questions are sure
to occur on how things should be written in English, what is the
correct punctuation and hyphenation, and what do you have to worry
about when you construct sentences. I will not attempt to answer such
questions here. \enquote{The guide to writing ZEUS
  papers}\index{ZEUS paper guide} from Brian Foster~\cite{ZEUSGuide}
contains a wealth of useful information. Brian kindly gave me
permission to package a PDF file of the note with this guide.

This document is structured as follows. Chapter~\ref{sec:tips} tells
you how to get started with the files and package. It also contains
several tips and tricks that it is probably good to include early. It
is sometimes not clear which version of the cover should be used when
submitting and/or printing your thesis. Some instructions are given in
Chapter~\ref{sec:submit}.  This is followed by
Chapter~\ref{sec:package}, which lists the packages used in this
document and says what they are good for. Chapters~\ref{sec:fig} and
\ref{sec:table} give some guidelines for figures and
tables. Chapter~\ref{sec:ref} discusses the tricky business of
references and their formatting. Some hints on how to solve common
layout problems, which fonts one can use and how to handle multiple
languages in a document are given in Chapter~\ref{sec:layout}. 
In the appendix I include some more
information on the \TeX{} setup I have used to test things. I have
seen a glossary (list of acronyms) in a few theses and think this is a
nice idea. The appendix shows how you can create such a list.
As big
tables are often moved to the appendix, an example of how to create
such tables is given there as well.

While this guide is structured pretty much like a thesis, I have
included a couple of extra features that are usually not needed in a
thesis. The first is a link to the relevant \LaTeX{} file at the
beginning of each chapter. I have also added an index, as
that is probably a useful complement to the table of contents.

Regular updates are made to the guide, so it is worth checking every
so often to see if a new version is available. Corrections and
suggestions for improvements are very welcome.


%%% Local Variables:
%%% mode: latex
%%% TeX-master: "./thesis_guide"
%%% End:
