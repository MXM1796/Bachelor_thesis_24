%==============================================================================
\chapter{References}
\label{sec:ref}
%==============================================================================

\LaTeX{} file: \url{./guide/guide_refs.tex}\\[1ex]
\noindent
Every thesis should also include a list of references, called the
bibliography in \LaTeX{} terminology. You \enquote{cite} a reference using
the \Macro{cite} command. For example, the book of
Kopka~\cite{kopka04} is my favourite \LaTeX{} book. In general you
should include a non-breaking space, i.e.\ \enquote{\textasciitilde} between
the text and the \Macro{cite} command. In British (UK) English the reference
should come before the punctuation; in American English it comes after
it.\footnote{After some research, it appears to me that the footnote
  number should come after the punctuation, unless the footnote only
  refers to the last word of the phrase or sentence.}

That is the easy part!  Where do you get the references from and how
do you format them? Sources of references are discussed in
Section~\ref{sec:ref:sources}.
There are two options for the formatting. Either you do it by
hand, formatting \Macro{bibitem} entries yourself or you use
\BibTeX. While \BibTeX\ may appear to be the more complicated option
at the beginning, I strongly recommend that you use it.

In addition, you have to make sure that authors' names are
printed consistently, you include the appropriate collaboration
name, the title is formatted correctly and journals are given
consistent abbreviations. Such topics are discussed in
Section~\ref{sec:ref:bib}.

What about citing a series of articles? Can you
include them in one reference or do you want to keep one article per
reference?
I give some hints on
useful options and settings for \Package{biblatex} below (Section~\ref{sec:ref:bbx}).
If you use \BibTeX, then you probably have to use the \Package{mcite}
package -- see Section~\ref{sec:ref:mcite}.
Just to take a silly example. The ZEUS collaboration
publications in
2010~\cite{Abramowicz:2010ih,Abramowicz:2010xc,Abramowicz:2010nj} were
not as numerous as in previous years. If you use the standard
\Macro{cite} command and the \Option{unsrt} option or its equivalent,
you get a list of numbers.  In the past one could use the
\Package{mcite} package to make the references nicer, put them all in
one, write the list as [m--n] etc.


%------------------------------------------------------------------------------
\section{Formatting by hand}
\label{sec:ref:bibitem}
%------------------------------------------------------------------------------

Don't! The number of references that you will need will probably grow
fast. It is quite likely that at some point you will decide that they
are not really formatted as you would like them to be. You will almost
certainly add references when you correct your thesis. How do you
make sure they are in the order you want? How do you make sure that
only articles that you actually refer to are in the bibliography?

Suppose you want to use some of the references in your thesis in
conference proceedings or a paper in a journal. Every place where you
publish will have it's own preferred format for the references that
almost certainly will not be the one you chose for your thesis.

If you insist on following this route, consult a book on \LaTeX!


%------------------------------------------------------------------------------
\section{Using \BibTeX\ and \Package{biblatex}}
\label{sec:ref:bibtex}\index{BibTeX@\BibTeX}
%------------------------------------------------------------------------------

I won't pretend that \BibTeX{} is the most user-friendly way of
handling references and there are several things that you have to pay
attention to when you use it for your references.

The two big advantages of \BibTeX{} are: only references that you
actually refer to appear in the bibliography; you can change the
format (consistently) of all articles in the bibliography simply by
changing the style!

The first step is to put your references in one or more \texttt{.bib}
files. For this document they can be found in:
\begin{itemize}
\setlength{\itemsep}{0pt}
\item \url{./guide/guide_refs.bib};
\item \url{./refs/standard_refs-bibtex.bib} or
  \url{./refs/standard_refs-biber.bib};
\item \url{./refs/example_refs-latin1.bib} or
  \url{./refs/example_refs-utf8.bib};
\item \url{./guide/refs/zeus_2009.bib} and
  \url{./guide/refs/zeus_2010.bib}.
\end{itemize}
For each article you specify things like its
title, author, journal etc.

You then include these files into your \LaTeX{} document where you
want the bibliography to appear and specify which style should be
used.

At this point you also have to decide which interface to the contents
of the \texttt{.bib} files you want to use. You have a choice of the
original \BibTeX\index{BibTeX@\BibTeX} or the more modern
\Package{biblatex}. If you use \Package{biblatex} you
need something like:
\begin{verbatim}
%
% Use biblatex for the bibliography
%
% \usepackage[backend=bibtex8,hyperref=true,bibencoding=latin1,
%   style=numeric-comp,sorting=none,block=ragged,firstinits=true]{biblatex}
\usepackage[backend=biber,
  style=numeric-comp,sorting=none,block=ragged,firstinits=true]{biblatex}
%
% Adjustments to output are in this style file:
% \usepackage{./biblatex/biblatex-num-v2009}
\usepackage{./biblatex/biblatex-num-v2011}
% \bibliography{./mythesis/thesis_refs.bib,%
%   ./refs/standard_refs-bibtex.bib}
\addbibresource{./mythesis/thesis_refs.bib}
\addbibresource{./refs/standard_refs-biber.bib}
\end{verbatim}
in the document preamble and \Macro{printbibliography} where they
should be printed. The commented out lines are for \TeXLive 2009 with
the \Option{bibtex8} backend.
\par\noindent
If you use \BibTeX{} you need something like:
\begin{verbatim}
%
% Use BibTeX for the bibliography
%
\bibliographystyle{unsrt}
\bibliography{./mythesis/thesis_refs,%
  ./refs/standard_refs-bibtex}
\end{verbatim}
at the point where the references should be printed.
Note that \LaTeX{} is sometimes picky about lists of directories that
have spaces between them, so it is safer to include all files on one line.

Which should you use? \BibTeX{} has been around for a long time and is
therefore better known. However, it has many problems when it comes to
sorting, handling more modern sources of information (e.g.\ the web),
etc. \Package{biblatex} is still developing rather quickly and
so some options and ways of doing things may change. It supports
things like online references and enables you to click on references
using the preprint number or DOI to look at a reference. It is also
easier though to change the way your references look. I therefore
strongly recommend that you use \Package{biblatex}.

Another serious problem with \BibTeX{} is that it cannot handle
umlauts etc.\ properly. While I have said elsewhere in this document
that you should use UTF-8 or latin1 encoding so that you can enter ä
etc.\ directly, this does not work with \BibTeX. You can use the
old syntax \verb+\"{a}+. This problem is completely solved if you use
\Package{biblatex} and the \Option{biber} backend.

If you use \Package{biblatex} and the \Option{bibtex8} backend (default
setting for \TeXLive 2009), then you have to encode your \texttt{.bib}
files with latin1 and use the line:
\Macro*{usepackage[bibencoding=latin1]\{biblatex\}}.\footnote{%
  If you do not include umlauts directly in the references you do not
  need this option.}
For this reason, as indicated above the thesis package includes two
versions of the file with some standard references and also of the
files with some example references:
\begin{itemize}
\setlength{\parskip}{0pt}\setlength{\itemsep}{0pt}
\item \texttt{./ref/examples\_refs-latin1.bib} with latin1 encoding
\item \texttt{./ref/examples\_refs-utf8.bib} with UTF-8 encoding.
\end{itemize}
If you try to compile the guide with the wrong file using \BibTeX, you
will get some errors as I have included some umlauts in the example
references.

That's it? Well almost! First, you will have to make sure that the
entry type that you use corresponds to the type of document that you
are citing. Second, you will probably get some or all of your
references from standard sources such as Spires\footnote{I
  will refer to both as Inspire in this chapter}, Inspire or CDS you will have
to change the entries a bit so that they get formatted the way you want.


%------------------------------------------------------------------------------
\section{\BibTeX{} entries}
\label{sec:ref:bib}

In this section I discuss how to format your \BibTeX\ databases,
i.e.\ the \texttt{.bib} files. In the following section I talk about
how you make your references look the way you want them to be in your
thesis.

%------------------------------------------------------------------------------
\subsection{Entry types}
\label{sec:ref:entry}

One question is what entry type you should use for what? I give here
recommendations on what to use for \Package{biblatex}. Some of the entry
types that \Package{biblatex} has are not part of \BibTeX.

\begin{description}
\item[@article] This is easy -- use it for articles\index{article} published in
  journals, e.g.~\cite{Abramowicz:2010ih}.
\item[@book] Just as easy -- use it for books,\index{book} e.g.~\cite{kopka04}.
\item[@proceedings, @inproceedings] The name says it all. Use
  \Option{@inproceedings} for a paper in the proceedings\index{proceedings} and
  \Option{@proceedings} for the whole volume.
\item[@collection] Use it for things such as the ATLAS Technical Design
  Report~\cite{lhc:vol1}\index{report!technical} where the names that you find are the
  editors. Use \Option{@incollection} for a single article in a
  collection.
\item[@report] Use it for conference\index{conference note}\index{note!conference} and
  internal\index{internal note}\index{note!internal} notes. This is
  probably also the best type to use for preprints. You can also use
  \Option{@online}. However, then the title is usually printed in
  italics rather than upright text inside quotes.
\item[@online] Use it for things that are only available online,
  e.g.~\cite{lshort}.
\item[@thesis] The name says it all.\index{thesis}
  \Option{@phdthesis}\index{PhD thesis}\index{thesis!PhD}  and
  \Option{@mastersthesis}\index{Master thesis}\index{thesis!Master}
  also exist. If you are using \Package{biblatex} you can and should specify
  the thesis type, e.g.\ \texttt{type = \{PhD\}}, see for example a
  PhD thesis~\cite{tlodd:2012}.
\end{description}

\Package{biblatex} also knows about multivolume proceedings etc. See
the manual for more details.

Note that Inspire will always give you a \BibTeX{} entry of type
\Option{@article}, so you should adjust it by hand according to what
the document you refer to really is. CDS tries a bit harder, but you
probably still have to set the entry type by hand.

As indicated above, \Package{biblatex} knows about preprint archives,
online references with a url etc.\ and can format the references so
that you can click on a DOI or arXiv number. Details of how these are
handled are well documented in the manual.  In order to make use of
these abilities you have to modify the Inspire format of the references
a bit so that it is fully compatible with what \Package{biblatex}
expects for preprints etc. More details on this are given below.

What else do you have to be careful about? The first thing to know is
that \Package{biblatex} and \BibTeX{} will try to format your author
names and titles. Thus, if you want the title to remain in exactly the
form you have typed it in include it in \enquote{"\{Title\}"}, i.e.\
both double quotes and braces. If not, collaborations and accelerators
tend to be converted to lowercase, e.g.\ \enquote{lhc} instead of
\enquote{LHC}. If you use an author such as ``ATLAS Collaboration'' it
gets printed as \enquote{A.\ Collaboration}.


%------------------------------------------------------------------------------
\subsection{Entries from Inspire and CDS}
\index{spires}\index{inspire}\index{CDS}
\label{sec:ref:cds}

Things like the LHC Design Report\index{design report} are by default
called \Option{@article} in Inspire~\cite{Bruning:2004ej-inspire} or
\Option{@book} in CDS~\cite{Bruening:782076-CDS}. They are in fact
best declared as \Option{@collection} with the \Option{author} field
replaced by \texttt{editor} and a field indicating the
\texttt{institution} instead of
\texttt{publisher}~\cite{lhc:vol1-final}. You can also add the CDS
link as a \texttt{url} field.

Conference notes,\index{conference note} e.g.\ from ATLAS, are defined
as \Option{@techreport} by CDS~\cite{ATLAS-CONF-2011-008-CDS}. It is
better to just call them \Option{@report}. You should add an author,
usually just \texttt{author = "\{ATLAS
  Collaboration\}",}~\cite{ATLAS-CONF-2011-008-final}. You may also
have to change the month format to avoid error messages. For internal
notes, also call them report and add \texttt{type = {internal report}}
to the entry. Again you could add the CDS link as a \texttt{url}
field. For preprints,\index{preprint} I also think it is best to use
the \Option{@report} entry type.

Books\index{book} need to be changed from \Option{@article} to
\Option{@book} and it is better to give the ISBN in the \texttt{isbn}
field~\cite{Halzen:1984mc-final} rather than the \texttt{reportNumber}
field as given in Inspire~\cite{Halzen:1984mc-inspire}.

Theses\index{thesis} should used the \Option{@thesis} entry type and
then add a \texttt{type} field. Alternatively you can use
\Option{@mastersthesis} or \Option{@phdthesis}.

In all cases you probably have to edit the titles a bit to get
things like $\sqrt{s} = \SI{7}{\TeV}$ printed properly. An open
question is whether you should assume the use of a units package in
the formatting of the title. If you want to make your \texttt{.bib}
files usable by others, it is probably best to do the formatting by
hand.

An example of a typical ATLAS paper as it comes from
Inspire~\cite{Aad:2010ey-inspire} needs a bit of work.  With \TeXLive
2011 the link to DOI and arXiv both work well~\cite{Aad:2010ey-final}.
With \TeXLive 2009, it worked on my laptop, but did not work properly
on ATLAS machines (Ubuntu 10.04) that had version 0.8e of \Package{biblatex}.
It may be that one can get it to work with a bit of help, but I have
not tried all possible options.

% Depending on which\TeXLive 2009 version you actually have, you might
% have to give a bit of help for the arXiv link to
% work%~\cite{Aad:2010ey-tex2009}.


%------------------------------------------------------------------------------
\subsection{More on names}
\label{sec:ref:names}

The best way to format author names so that they appear correctly
whatever \BibTeX{} style you use is \texttt{Surname, Name}. Any other
syntax is likely to get mangled.

What about collaboration names? If you use Inspire as the source of
your \BibTeX{} entries, you will see that it has a field for the
collaboration. This is often, but not always, formatted
correctly. However, very few \BibTeX{} styles pay any attention to
this field. The ones from Spires listed below will work properly. The
only other reliable alternative I have found is to use the following
syntax:
\begin{verbatim}
@Article{Chekanov:2009qja,
     author    = "{ZEUS Collab.} and Chekanov, S. and others",
\end{verbatim}
which then usually gets formatted as \enquote{ZEUS Collab., Chekanov S. et
al.,}.
I went through and changed the references in \texttt{zeus\_2009.bib}
accordingly.


%------------------------------------------------------------------------------
\section{Formatting references}
\label{sec:ref:format}

While \BibTeX{} or \Package{biblatex} format the references for you
from the \texttt{.bib} files, you have to tell them what format you
want!  For a start, you have to choose between an alphabetic and a
numeric scheme for the references. Most journals use a numeric
style. This corresponds to style \Option{unsrt} or a variant thereof
using standard \BibTeX.  If you use \Package{biblatex} you include the
package with option \Option{numeric-comp} or
\Option{numeric}. \Option{numeric-comp} produces more compact
citations (e.g.\ [1-4,7,9]) than \Option{numeric} (e.g.\
[1,2,3,4,7,9].

For this guide (for a change) I use an alphabetic style:
\Option{alpha} with \BibTeX{} or option \Option{alphabetic} with
\Package{biblatex}. In the thesis skeleton I use a more usual unsorted
numeric style.


%------------------------------------------------------------------------------
\subsection{\Package{biblatex} styles}
\label{sec:ref:bbx}

My experience with \Package{biblatex} has only recently been acquired and I
am still learning, but I have tried out a few things that I will
document here. The first official stable release was 19 Oct
2010. Active and rapid development is ongoing -- there were many
updates in 2011.  \TeXLive 2009 includes either Version 0.8e (ATLAS
cluster Ubuntu 10.04) or 1.4c (Kubuntu 11.10), so some things I
recommend below may work a bit differently depending on which version
you have. See Appendix~\ref{sec:app:tex} on how to install a newer
version of \TeXLive if you want a more up-do-date version of \LaTeX.

Looking for numeric styles you can either use the built-in
\Option{numeric} or \Option{numeric-comp}.  The \Option{numeric-comp}
style is used by default in the thesis skeleton. I made a few
adjustments that are included in the file
\texttt{./biblatex/biblatex-num-v2009.sty} or
\texttt{./biblatex/biblatex-num-v2011.sty}. Again which file is used
by default is steered by the \Macro{texlive} macro which is set in the
main file.

You can fine tune things even more by using hooks that are
available. For example, if you do not want to print the URL field you
can include the command:
\begin{verbatim}
\AtEveryBibitem{\clearfield{url}}
\end{verbatim}
in the preamble or in the relevant style file given in the previous
paragraph. It is not clear to me if you also need
\verb+\AtEveryCitekey{\clearfield{url}}+.

A fairly nice-looking style is \Option{ieee}. This is only be
available in recent releases (2011) of \TeXLive. After playing
around a bit with the \Option{ieee} style, I decided it is too new and
has too many settings that depend on have a new version of \Package{biblatex}
for now.

There are slowly more and more \Package{biblatex} styles around, but
not as many as \BibTeX. It is, however, much easier to change things
(usually you can just change an option) with \Package{biblatex} than it
was for \BibTeX, so you can probably start with a standard file and
just make adjustments in your preamble. I have found a number of very
useful hints on how to make changes in
\url{http://tex.stackexchange.com/} -- just search for \Package{biblatex}.

If it is available, \Option{biber} is probably the preferred backend to
\Option{bibtex8}. However, the backend is mostly relevant for
sorting, so it probably does not matter which you use if you use an
option that gives the references in the order that they were
cited. \Option{biber} seems to work well with \TeXLive 2011; it is
often not available with earlier versions.

If you get an error such as:
{\scriptsize
\begin{verbatim}
biber     thesis_guide
data source /tmp/par-62726f636b/cache-ab06f20732bfab23dfa35f56998ad4edca61bee1//inc/lib/Biber/LaTeX/recode_data.xml not found in .
Compilation failed in require at Biber/Utils.pm line 21.
\end{verbatim}
}
\noindent
then you should delete the directory \texttt{/tmp/par-...} and try to
run again.

%------------------------------------------------------------------------------
\subsection{\BibTeX{} styles}
\label{sec:ref:bst}

If you use references directly from Spires or Inspire, then it is
probably best to use one of the style files that is compatible with
their format. A list can be found on
\url{http://www.slac.stanford.edu/spires/hep/refs/bibstyles.shtml}. I
have used \Option{utphys} a few times and it works OK. I see that
there are also style files available there for common HEP journals,
which could save quite a bit of work. The big advantage of
\Option{utphys} is that it also knows about the arXiv and preprints.

The equivalent of \Package{biblatex}'s \Option{ieee} style in \BibTeX\ is
\Option{ieeetr}. It also knows about arXiv and preprints. However, it
does not know about collaborations.


%------------------------------------------------------------------------------
\section{Sources for references}
\label{sec:ref:sources}
%------------------------------------------------------------------------------

The ZEUS collaboration kept a reasonably up-to-date list of ZEUS and
H1 publications (as well as some others) in \BibTeX{} format. ATLAS
also keeps such a list and I assume that other collaborations keep
similar lists.

Within experimental high energy physics the standard way to get a
reference is to use Inspire
(\url{http://inspirehep.net}).\footnote{This has now replaced Spires
(\url{http://www.slac.stanford.edu/spires/}).
One problem with Spires was that it was very slow and regularly
timed out when you perform searches.} You can get the appropriate
Inspire entry by using the Inspire search engine. Alternatively if you
know the arXiv preprint number you can go from its entry to Inspire
directly.

To get the ZEUS references I used above I first tried the following
command in Inspire:
\begin{verbatim}
find exp zeus and date 2009
\end{verbatim}
This does not really give you the references you expect though! It
seems much more reliable to use an author name so I used:
\begin{verbatim}
find a chekanov and date 2009
\end{verbatim}
and then selecting \BibTeX{} format, saving the resulting page in a file
and removing the \texttt{<pre>} and \texttt{</pre>} entries between
references. This worked better, even though I got a whole load of
ATLAS papers as well.

If you then try to use the references, you get complaints that
something is not in math mode. You have to go through by hand and
change things such as \verb+Q^2+ to \verb+$Q^{2}$+.


%------------------------------------------------------------------------------
\section{Common wishes}
\label{sec:ref:tips}
%------------------------------------------------------------------------------

It is possible that you would like to combine several articles into a
single reference. The \Package{mcite} package was designed to do this,
but is not compatible with \Package{biblatex} and
\Package{hyperref}. \Package{biblatex} has another solution that it
calls sets.

In \texttt{standard\_refs-biber.bib} and
\texttt{standard\_refs-bibtex.bib} I have put in the three standard
references for the Standard Model~\cite{gsw}. They are combined by
using \Option{@Set}\index{biblatex!Set option@"@Set option} and the relevant
keys. If you use a recent version of \texttt{biber} (\TeXLive 2011)
this is all you have to do.  If, however, you are using \TeXLive
2009, and therefore the \texttt{bibtex8} backend, the
\texttt{crossref} field must contain the same key as the first one in
\texttt{entryset}.

One things you should always do is include all references in a single
\Macro{cite}. e.g.\ there were quite a few ZEUS publications in
2009~\cite{Chekanov:2009qja,Chekanov:2009zz,Chekanov:2009tu} is better
than~\cite{Chekanov:2009qja}\cite{Chekanov:2009zz}\cite{Chekanov:2009tu}.
If you want to get a list of references printed in the form \enquote{[m--n]},
then with \Package{biblatex} you should use the style
\Option{numeric-comp}. In 2009 there were many papers published by the
ZEUS
collaboration~\cite{Chekanov:2009qja,Chekanov:2009zz,Chekanov:2009tu}
as well as several articles from both the H1 and ZEUS
collaborations\cite{Chekanov:2009wt,Aaron:2009wg}. See
Section~\ref{sec:ref:mcite} on how to do this with \BibTeX.

You are nearing the end of your thesis and have to properly format all
the references that you have. However, they are spread over several
files and these files also contain many references that you do not use
or want to correct. How best to proceed?
\begin{verbatim}
bibtool -x mythesis.aux -o refs.bib
\end{verbatim}\index{bibtool}
will extract the entries that you use and in future you can use and
correct \texttt{refs.bib}, which only contains the references that you
actually cite.\footnote{%
I got this tip from
\url{http://tex.stackexchange.com/questions/417/how-to-split-all-bibtex-referenced-entries-from-a-big-bibtex-database-to-a-copy}. Do
not forget to change \texttt{mythesis.tex} to use
\texttt{refs.bib} instead of the previous sources.}


%------------------------------------------------------------------------------
\section{Using \Package{mcite}}
\label{sec:ref:mcite}
%------------------------------------------------------------------------------

As mentioned above, the \Package{mcite} package used to be a good way
of combining several articles into a single references and also
getting them to be printed out in the form \enquote{[m--n]}, rather
than \enquote{[l,m,n]} or \enquote{[l],[m],[n]}. How do you achieve
this?  Just put all the articles in a single \Macro{cite} and prefix
those that should be lumped together with a \enquote{*}, e.g.\
\verb+\cite{Chekanov:2009wt,*Aaron:2009wg,*Aaron:2009sma}+.  The
problem is that this package does not appear to be compatible with the
\Package{hyperref} package, so you have to choose between the
two. Given the ability that the \Package{hyperref} package offers to
jump directly to sections, equations, references referred to in a
document, I guess most of you will go with \Package{hyperref} rather
than \Package{mcite}.

As mentioned above the \Package{biblatex} package offers a
more modern alternative and different ways of achieving the same
results! It is also not compatible with \Package{mcite}.

A modified version \Package{mcite} is used by ZEUS in its LaTeX4ZEUS
environment, which is why I include a short description here.

%%% Local Variables:
%%% mode: latex
%%% TeX-master: "../thesis_guide"
%%% End:
