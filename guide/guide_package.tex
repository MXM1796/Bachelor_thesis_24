% !TeX root = thesis_guide.tex
% chktex-file 1 chktex-file 21

%==============================================================================
\chapter{Useful packages}%
\label{sec:package}
%==============================================================================

\LaTeX{} file: \url{./guide_package.tex}\\[1ex]
\noindent
\LaTeX{} has so many packages that it is often hard to find the
correct or most useful ones. It is also not a good idea to just take
one of your friend's theses and use his/her packages and conventions,
as there is a steady and regular improvement in the packages
available.

This chapter lists some useful packages --- maybe also some that
are not so commonly known. Here I only say what the package is used
for. More detailed instructions on the usage can be found in the
relevant chapters. I first list the packages used in this guide
and then give a bit of information on other packages that may be useful.

From all that I have read, \KOMAScript\index{KOMAScript@\KOMAScript}
seems to be the way to go for the overall classes. I have therefore
based the \Package{ubonn-thesis} style on this. You replace
\Package{article}, \Package{report} and \Package{book} by
\Package{scrartcl}, \Package{scrreprt} and \Package{scrbook}. For
theses I think it is best to use \Package{scrbook}, as this class also
includes the commands \Macro{frontmatter}, \Macro{mainmatter} and
\Macro{backmatter} that set up page numbering etc.\ appropriately.

Please try to use \KOMAScript\ version 3.0 or higher. The
\Macro{KOMAoptions} command is not available in earlier versions, so
you would have to modify the style file.


%------------------------------------------------------------------------------
\section{Layout and language}%
\label{sec:package:layout}
%------------------------------------------------------------------------------

There are quite a few packages related to layout and also to handling
of text input and languages. As far as layout goes, \KOMAScript{} has
many options with which you can already do a lot. You can either use
the built-in \Package{typearea} package to do the page layout, which
also includes nice options to allow for the binding, or use the
\Package{geometry} package which also contains more than enough
options. In the past I have used \Package{geometry}, but I also see no
reason not to just use \Package{typearea}. Note that you should not
include the \Package{typearea} package, you should simply set the
options using \Macro{KOMAoptions}.
Generally, all you need to do is specify \Option{DIV} (set by default to 12),
which divides the page into a number of divisions and
\Option{BCOR} (set by default to \SI{5}{\mm}),
which leaves some space for the binding.
The packages are listed in \cref{tab:package:layout}.

\begin{table}[htbp]
  \caption{Useful packages for layout.}%
  \label{tab:package:layout}
  \centering
  \begin{tabular}{lp{0.8\textwidth}}
    \toprule
    \Package{geometry} & Provides simple options for page layout such
    as \Option{scale=0.75} to cover \SI{75}{\%} of the page.\\
    \Package{typearea} & Does much the same, but here you specify how
    many elements to split the page into, e.g.\ \Option{DIV=12}.
    You do not have to include
    this package explicitly if you use \KOMAScript.\\
    \Package{setspace} & Useful options to change spacing.\\
    \Package{fontenc} & The encoding used for fonts. Recommended is
    \Option{T1}, which is given as an option.\\
    \Package{inputenc} & Use either \Option{utf8} or \Option{latin1} so
    that you can input German letters such as ä, ü and ß directly.\\
    \Package{babel} & Language specific typesetting.\\
    \Package{csquotes} & Package for quoting things using the correct
    language-dependent quotation marks.\\
    \Package{scrlayer-scrpage} & Set headers and footer.\\
    \Package{xspace} & Avoid having to put
    \enquote{\texttt{\textbackslash\ }} or
    \enquote{\texttt{\{\}}} after a macro.\\
    \bottomrule
  \end{tabular}
\end{table}

The package \Package{scrlayer-scrpage} has superseded \Package{scrpage2}.
If your version of \LaTeX{} is so old that it does not know about
\Package{scrlayer-scrpage} adjust the \Package{ubonn-thesis} style file accordingly.

%------------------------------------------------------------------------------
\section{Appearance}%
\label{sec:package:appearance}
%------------------------------------------------------------------------------

It used to be the case that nearly all \LaTeX{} documents used the
Computer Modern Fonts. That is no longer necessary. There are rather
complete font sets that are also free that you can use instead.
%In this guide I use a combination of \Package{Palatino} and other fonts.
The default font for theses is \Package{newtx} that is available as of \TeXLive 2013.
If you have this package, then I would recommend using it.
\Package{txfonts} is an older version of this package that can be used instead, if necessary.
Some of the spacings in equations have been improved in \Package{newtx}
and there is a better balance of the sizes of serif, \textsf{sans serif} and \texttt{typewriter} fonts.
Other fonts that look quite nice (e.g.\ Palatino) can also be used.
The option \Option{palatino} in \Package{ubonn-thesis} can be used to select this.
The option actually selects the font packages \Package{mathpazo}, \Package{courier} and \Package{helvet}.
Another alternative is a package such as \Package{pxfonts} 
to get both text and math fonts in the same style. 
Some examples of other possible font packages are given in the style file.
As mentioned above, certain fonts can be selected directly via options:
\Option{txfonts}, \Option{newtx} or \Option{palatino}.

Commonly used packages associated with fonts, tables and
figures are listed in \cref{tab:package:appearance}.

\begin{table}[htbp]
  \caption{Useful packages for appearance.}%
  \label{tab:package:appearance}
  \centering
  \begin{tabular}{lp{0.75\textwidth}}
    \toprule
    \Package{siunitx} & Typeset units properly with correct spacing.\\
    \Package{graphicx} & The package to use for including graphics.\\
    \Package{biblatex} & The package to use for formatting the bibliography.\\
    \Package{rotating} & Package to use for rotating tables etc. The
      \Macro{includegraphics} command can rotate figures directly.\\
    \Package{array} & Adds extra column formatting capabilities.\\
    \Package{longtable} & Can produce tables that extend over more than one page.\\
    \Package{booktabs} & Help for producing nicer tables.\\
    \Package{subcaption} & A newer alternative to \Package{subfig}.\\
    \Package{cleveref} & Provides \enquote{Fig.}, \enquote{Table} etc.\ when you use
      the \Macro{cref\{fig:plot\}} syntax.\\
    \Package{mathtools} & An extension package for \Package{amsmath} that fixes some bugs and deficiencies there.\\
    \Package{amsmath}, \Package{amssymb} & Included by \Package{mathtools}. 
      Extra math commands and symbols from AMS.\\
    \Package{hepnicenames} & A somewhat restricted list of predefined elementary particles.\\
    \Package{heppennames} & A more complete list of predefined elementary particles.
      Note that \Package{hepnicenames} also loads \Package{heppennames}.
      The packages are based on \Package{hepparticles}, which you can use to define further particles.\\
    \Package{mhchem} & Nice package for typesetting chemical elements correctly.\\
    \Package{bookmark} & Fixes appendices displaying in PDF viewer index/outline under first appendix
      instead of under Appendix heading.\\
    \Package{xfrac} & Some more options for typesetting fractions.\\
    \Package{xcolor} & Add colour commands.\\
    \Package{microtype} & Small adjustments to word spacings.\\
    \Package{ifthen} & Provides the \Macro{ifthenelse} command.\\
    \bottomrule
  \end{tabular}
\end{table}

The \Package{cleveref} package is a great package for specifying references to
figures, tables, sections etc.
You just give the command \Macro{cref\{fig:plot\}} and it will add \enquote{Fig.} or whatever
you set to the figure number and an unbreakable space.
You can correctly format a list of figures by including several comma-separated labels in the \Macro{cref} command.
Note that you should not include any whitespace between the labels.
If you want capitalised \enquote{Fig.} etc. pass the option \Option{clevercaps}
to \Package{ubonn-thesis} (this is the default setting).
Otherwise pass the option \Option{clevercaps=false}.
Note that you should include \Package{cleveref} by passing \Option{cleveref} 
as an option to \Package{ubonn-thesis}.
If you just try the usual \Macro{usepackage\{cleveref\}} you may get things in the wrong language.
This guide has been converted to use the \Package{cleveref} package everywhere.

As alternatives to \Package{longtable}, one can also use
\Package{supertabular} or \Package{xtab}. 
The syntax of \Package{longtable} is closer to that used in normal tables
and does not need definitions outside the table.
It is also maintained by the \LaTeX{} Project team.
All these packages also have the advantage that you
can specify header and footer text.
If you use the \Package{longtable} package,
you have to terminate the \Macro{caption} with \verb|\\|.
If you use the \Env{mpxtabuar} environment from \Package{xtab} 
you can include footnotes in a table.
See the \Package{xtab} documentation for more details.
It is probably best to only use one of these three packages to avoid conflicts.

The package \Package{microtype} declares itself to be responsible for
\enquote{Subliminal refinements towards typographical perfection}.
Need I say more?

%------------------------------------------------------------------------------
\section{Other packages}%
\label{sec:package:other}
%------------------------------------------------------------------------------

Some other useful packages, some of which are included in
\texttt{ubonn-thesis.sty} are listed in
\cref{tab:package:other1}.

\begin{table}[htbp]
  \caption{Other useful packages.}%
  \label{tab:package:other1}
  \centering
  \begin{tabular}{lp{0.8\textwidth}}
    \toprule
    \Package{IEEEtrantools} & Contains useful environments for multiline equations.\\
    \Package{feynmp} & Draw Feynman graphs with Metapost.\\
    \Package{axodraw} & Draw Feynman graphs.\\
    \Package{tikz} & General drawing package that can also be used for Feynman graphs.\\
    \Package{standalone} & Allows you to have a document that you can
    directly compile for each figure and also input to another document.\\
    \Package{hyperref} & Adds \Macro{url} command as well as ability
    to click on entries on table of contents etc.\\
    \Package{todonotes} & Nice package to add notes to your text.
      You can also use it to indicate missing figures.\\
    \Package{background} & Allows you to add things like DRAFT across the whole document page.\\
    \Package{subfiles} & Provides a nice alternative to
    \Macro{include}.\\
    \Package{tabularx} & Allows fixed table width with flexible column widths.\\
    \Package{floatrow} & Add ability to define own floats.\\
    \Package{diffcoeff} & A lot of useful commands for derivatives.\\
    \Package{derivative} & A new package (released in 2021) with lot of useful commands for derivatives.\\
    \Package{physics} & Some useful extra maths commands, especially for differentials.\\
    \Package{commath} & Some useful extra maths commands --- has not been updated for a while.\\
    \Package{skmath} & More maths commands that could be useful.\\
    \Package{adjustbox} & Add much more sophisticated clipping
    capabilities than offered by \Package{graphicx}.\\
    \Package{wrapfig} & Allow text to flow around figures.\\
    \Package{floatflt} & Similar capabilities to \Package{wrapfig} --- allow text to flow around figures.\\
    \Package{glossaries} & Provide commands for creating a glossary.
      This is intended to replace the \Package{glossary} package.\\
    \Package{dcolumn} & Very helpful for lining up columns on character strings such as a decimal point.
      \Package{siunitx} offers similar and better functionality.\\
    \Package{refcheck} & Check whether labels are used, i.e.\ if figures and tables are actually referenced.\\
    \Package{tcolorbox} & Fancy coloured boxes. Also works very well for code listings.\\
    \Package{pgfplotstable} & Read in tables from CSV files and format them (even as graphs).\\
    \Package{titlesec} & Change the appearance of chapter and section headings. 
      See below for more information.\\
    \bottomrule\\
  \end{tabular}
\end{table}

While the \Package{mathtools} and \Package{amsmath} packages solve many problems that occur if
you just use the normal \LaTeX{} math mode commands,
there are some things that are not so nice with long and complicated multiline equations.
\Package{mathtools} provides some extensions to \Package{amsmath} that solve some of the problems.
An alternative is the package \Package{IEEEtrantools}, in particular the \Env{IEEEeqnarray} environment.
See Ref.~\cite{lshort} or the package documentation for more details.

The packages \Package{physics}, \Package{commath}, \Package{skmath}, \Package{diffcoeff} and \Package{derivative} provide some additional maths commands.
Differentials etc.\ are particularly useful.
Both the \Package{diffcoeff} and \Package{derivative} packages contain a lot of nice and flexible definitions associated with derivatives.
They handle the spacing around things like \(\mathrm{d}x\) well,
if you use the \verb|\odif{x}| (\Package{derivative}) or \verb|\dl x| (\Package{diffcoeff}) construction.
An alternative (but not updated for a long time) package is \Package{physics} package contains a lot of nice and flexible definitions.
However, an interesting StackExchange article\footnote{%
\url{https://tex.stackexchange.com/questions/471532/alternatives-to-the-physics-package}}
suggests that the package has its drawbacks and it has not been updated since 2012.
The \Package{commath} package has not been updated for a while.
\Package{skmath} does quite a lot more then \Package{commath} and even modifies/enhances some standard commands.
The \Package{diffcoeff} has been around since 2016,
while \Package{derivative} package was released in 2021.
Both look to be nice and well maintained.
This guide uses \Package{diffcoeff}, as \Package{derivative} is too new.
By default, the option \Option{ISO} is passed to the \Package{diffcoeff} package,
so that an upright \enquote{d} is used in derivatives, e.g.\ \(\diff f/x\).

The \Package{todonotes} is a very nice package that enables you to add notes to your text.
By default they will be put in the margin of your document.
More details can be found in \cref{sec:package:todo}.

\Package{standalone} is both a package and a document class.
% It is available from \TeXLive 2012 onwards.
It allows you to have a standalone document for a \Package{tikz} or \Package{feynmf} figure
and also input this file into another document. If you run pdf\LaTeX\
on the file it also automatically crops the resulting picture. This is
one of those packages where you think \enquote{Why didn't someone
  create this years ago?}. The \Package{tikz} figures included in this
guide make use of it.

If you want to include code listings,
you can use the packages \Package{tcolorbox} or \Package{listings}.
I used \Package{listings} for quite a while,
but \Package{tcolorbox} can be used for anything associated with boxes and
has much more flexibility.
This guide can has now been adjusted to use \Package{tcolorbox}.

Use of the \Package{titlesec} package gives a warning when using \KOMAScript{}; hence
as of version 3.0 the chapter title formatting is done by hand in \Package{ubonn-thesis}.
You can switch back to using \Package{titlesec} by giving the option \Option{titlesec}.

Although I have not actually tried it,
the \Package{pgfplotstable} package has the ability to read in CSV files and
then format them as tables, or even as graphs.
There is a detailed description in the \TeX nische Komödie issue (2/2020)
on how this can be done.
This could be a really nice way of formatting data
and avoids the need to write out such tables including \enquote{\&} and \enquote{\textbackslash\textbackslash}.
Various sorting options are also possible.

I do not recommend the \Package{subfiles} package by default as it is not
included in \TeXLive by default. Have a look at
\url{http://en.wikibooks.org/wiki/LaTeX/General_Guidelines} for
example for more information. If you want to use the package you have
to download it and install for yourself. You can do much the same
thing using AUCTeX inside \texttt{emacs}.

\begin{sloppypar}
Note that there is a conflict if you use \Package{refcheck}, \Package{subcaption} and \Package{hyperref} together.
See \url{http://tex.stackexchange.com/questions/273970/conflict-refcheck-subcaption-packages-for-label-with-underscores} for a workaround.
\end{sloppypar}

It is often useful to indicate whether your thesis is in the draft stage.\index{draft}
The thesis skeleton uses the package \Package{background} for this.

A list of other packages that are commonly used is given in
\cref{tab:package:other2}. They are not
included in the list above, because they are either not really needed
or have been superseded by other packages.

\begin{table}[htbp]
  \caption{Other packages that are often used, but I have already
    given alternatives.}%
  \label{tab:package:other2}
  \centering
  \begin{tabular}{lp{0.8\textwidth}}
    \toprule
    \Package{hepunits} & Typeset units properly with correct spacing.\\
    \Package{units} & Typeset units properly with correct spacing.\\
    \Package{SIunits} & Typeset units properly with correct spacing.
      \Package{hepunits} uses this package, so it does not need to be added explicitly\\
    \Package{fancyhdr} & As the name suggests, do your own header and footer configuration.
      Within \KOMAScript{} it is recommended to use \Package{scrlayer-scrpage} instead.\\
    \Package{feynmf} & Draw Feynman graphs with Metafont.\\
    \Package{draftwatermark} & Another package that allows you to add DRAFT to the background each page.\\
    \Package{subfig} & As the name suggests make sub-figures and add
    separate captions for them. \emph{This package has apparently been
      deprecated.}\\
    \Package{subfigure} & As the name suggests make sub-figures and add
    separate captions for them. \emph{This package is deprecated.}\\
    \Package{color} & Add colour commands --- \Package{xcolor} is
    needed to colour boxes around links.\\
    \Package{float} & As far as I can tell \Package{floatflt} offers more options.\\
    \Package{caption} & Much more control on captions --- as
    \KOMAScript{} also has many options, not sure this is necessary.\\
    \Package{ziffer} & Spacing with a comma as decimal separator is
    correct.\\
    \Package{nomencl} & Another package for creating a glossary.\\
    \Package{listings} & Code listings. I switched to \Package{tcolorbox}.\\
    \Package{fncychap} & Another package for changing the style of the
    chapter heading.\\
    \Package{quotchap} & Another package for changing the style of the
    chapter heading.\\
    \bottomrule
  \end{tabular}
\end{table}

As indicated, the \Package{ziffer} package is advertised as providing
the correct spacing after a comma in math mode if you use the comma as
the decimal separator. Compare \(2,5\) with 2,5 and \(2{,}5\). The first
spacing is wrong. If you use the \Package{ziffer} package it will be
correct. However, it does seem to conflict with the use of the
\Package{dcolumn} package, so I cannot compile this guide
if \Package{ziffer} is included. Some  workarounds are discussed in
\cref{sec:layout:german}. In addition, the \Package{siunitx}
package contains the same functionality, which can simply be steered
by changing the document language, as discussed in
\cref{sec:table:siunitx}.
Hence, \Package{ziffer} is not really needed anymore.

%------------------------------------------------------------------------------
\section{ToDo Notes}%
\label{sec:package:todo}
%------------------------------------------------------------------------------

If you include the option \Option{todonotes} with \Package{ubonn-thesis},
the margin of the thesis will also be adjusted a bit to leave more space for such notes.
As an example, the \Macro{mynote} command is defined.

If you turn on the option \Option{todonotes}, the package will be loaded.
With the option \Option{shownotes} they will also be 
\mynote{displayed.}{The \Option{shownotes} also turns on \Option{todonotes}.}

Another nice feature of the \Package{todonotes} package is that you can use
\Macro*{missingfigure\{text\}}\index{\textbackslash missingfigure}
to indicate figures that you still want to include.
See for example \cref{fig:missing}. You can get a list of your notes
by adding the \Macro{listoftodos} command somewhere.

The macro \Macro{todo} has some problems inside floats.
You should pass the option \Option{inline} to \Macro{mynote} if you
want a comment inside a caption. If you also include a \enquote{List of Tables}
or \enquote{List of Figures}, then you have to provide a short caption
as an option or move the note outside (after) the caption.

\begin{figure}[htbp]
  \missingfigure{Still need to create the plot to be included here.}
  \caption[Optional caption without \Macro{mynote} so that \enquote{List of Figures} does not break.]{%
    This figure still has to be added.
    \mynote[inline]{}{Use the option \Option{inline} to include a note in a caption,
    as this is floating. A note that does not point to a block of text needs a
    blank first argument.}
  }%
  \label{fig:missing}
\end{figure}


%------------------------------------------------------------------------------
\section{pdf\LaTeX, Lua\LaTeX\ or Xe\LaTeX}%
\label{sec:package:xelatex}\index{pdfLaTeX@pdf\LaTeX}\index{LuaLaTeX@\LuaLaTeX}\index{XeLaTeX@\XeLaTeX}
%------------------------------------------------------------------------------

You may have heard that successors to the \TeX\ engine are being planned
and that \LaTeX3 may at some point replace the current pdf\TeX\ engine.
This has been the case for many years,
but so far it does not seem to be clear
which development, if any, will replace pdf\LaTeX.
The two main directions are \LuaLaTeX and \XeLaTeX.
The big advantage of these new developments is that they can use the installed fonts on your machine directly.
\LuaLaTeX\ also integrates a scripting language, Lua.
As a result, it also does not have some of the restrictions of \LaTeX\ or \XeLaTeX,
e.g.\ the number of open files for writing.

As of version 7.0 I have adapted the thesis style file able to be able to use both \LuaLaTeX\ and \XeLaTeX.
The font specifications had to be changed.
The main package that should be used is \Package{fontspec}.
However, there are several different packages that can be used to set the fonts for mathematics.
I tried out \Package{unicode-math}, \Package{mathfont} (which only works with \LuaLaTeX)
and \Package{mathspec} (which only works with \XeLaTeX).
While I got test files to compile using all three,
\Package{unicode-math} seems to be the furthest developed,
and works with both \LuaLaTeX\ and \XeLaTeX, so it is the package I have used.
You can enter Greek letters etc., such as \(\pi\) directly when using \Package{unicode-math},
but you then have to find out how to enter them on your keyboard.
\Package{unicode-math} introduces the macros
\Macro{symup}, \Macro{symit}, \Macro{symbf}, \Macro{symsf}, \Macro{symbtt}
for steering of boldface etc. in math mode.
There are also the macros \Macro{symbfit}, \Macro{symbfcal} etc.
The package recommends these macros as being preferable to the usual \Macro{mathbf} etc.
See the package documentation for more information.
 
Both the skeleton thesis and the thesis guide can be compiled with either pdf\LaTeX, \LuaLaTeX\ or \XeLaTeX.
You can compile the thesis using the commands \Command{make} or \Command{make thesis} for pdf\LaTeX,
\Command{make thesislua} for \LuaLaTeX, add
\Command{make thesisxe} for \XeLaTeX.
You can compile the thesis guide using the commands \Command{make guide} for pdf\LaTeX,
\Command{make guidelua} for \LuaLaTeX, add
\Command{make guidexe} for \XeLaTeX.

The fonts are selected using filenames rather than font names.
This follows the recommendation of Will Robertson in a TUGBoat article.
Using filenames I am able to compile theses (and this guide) under both macOS and Linux.
By default I use the TeX Gyre fonts --- see \cref{sec:layout:font}.
