% !TeX root = thesis_guide.tex
% chktex-file 1 chktex-file 21

%==============================================================================
\chapter{Useful packages}%
\label{sec:package}
%==============================================================================

\LaTeX{} file: \url{./guide_package.tex}\\[1ex]
\noindent
\LaTeX{} has so many packages that it is often hard to find the
correct or most useful ones. It is also not a good idea to just take
one of your friend's theses and use his/her packages and conventions,
as there is a steady and regular improvement in the packages
available.

This chapter lists some useful packages --- maybe also some that
are not so commonly known. Here I only say what the package is used
for. More detailed instructions on the usage can be found in the
relevant chapters. I first list the packages used in this guide
and then give a bit of information on other packages that may be useful.

From all that I have read, \KOMAScript\index{KOMAScript@\KOMAScript}
seems to be the way to go for the overall classes. I have therefore
based the \Package{ubonn-thesis} style on this. You replace
\Package{article}, \Package{report} and \Package{book} by
\Package{scrartcl}, \Package{scrreprt} and \Package{scrbook}. For
theses I think it is best to use \Package{scrbook}, as this class also
includes the commands \Macro{frontmatter}, \Macro{mainmatter} and
\Macro{backmatter} that set up page numbering etc.\ appropriately.

Please try to use \KOMAScript\ version 3.0 or higher. The
\Macro{KOMAoptions} command is not available in earlier versions, so
you would have to modify the style file.


%------------------------------------------------------------------------------
\section{Layout and language}%
\label{sec:package:layout}
%------------------------------------------------------------------------------

There are quite a few packages related to layout and also to handling
of text input and languages. As far as layout goes, \KOMAScript{} has
many options with which you can already do a lot. You can either use
the built-in \Package{typearea} package to do the page layout, which
also includes nice options to allow for the binding, or use the
\Package{geometry} package which also contains more than enough
options. In the past I have used \Package{geometry}, but I also see no
reason not to just use \Package{typearea}. Note that you should not
include the \Package{typearea} package, you should simply set the
options using \Macro{KOMAoptions}.
Generally, all you need to do is specify \Option{DIV} (set by default to 12),
which divides the page into a number of divisions and
\Option{BCOR} (set by default to \SI{5}{\mm}),
which leaves some space for the binding.
The packages are listed in \cref{tab:package:layout}.

\begin{table}[htbp]
  \centering
  \begin{tabular}{lp{0.8\textwidth}}
    \toprule
    \Package{geometry} & Provides simple options for page layout such
    as \Option{scale=0.75} to cover \SI{75}{\%} of the page.\\
    \Package{typearea} & Does much the same, but here you specify how
    many elements to split the page into, e.g.\ \Option{DIV=12}.
    You do not have to include
    this package explicitly if you use \KOMAScript.\\
    \Package{setspace} & Useful options to change spacing.\\
    \Package{fontenc} & The encoding used for fonts. Recommended is
    \Option{T1}, which is given as an option.\\
    \Package{inputenc} & Use either \Option{utf8} or \Option{latin1} so
    that you can input German letters such as ä, ü and ß directly.\\
    \Package{babel} & Language specific typesetting.\\
    \Package{csquotes} & Package for quoting things using the correct
    language-dependent quotation marks.\\
    \Package{scrlayer-scrpage} & Set headers and footer.\\
    \Package{xspace} & Avoid having to put
    \enquote{\texttt{\textbackslash\ }} or
    \enquote{\texttt{\{\}}} after a macro.\\
    \bottomrule
  \end{tabular}
  \caption{Useful packages for layout.}%
  \label{tab:package:layout}
\end{table}

The package \Package{scrlayer-scrpage} has superseded \Package{scrpage2}.
If your version of \LaTeX{} is so old that it does not know about
\Package{scrlayer-scrpage} adjust the \Package{ubonn-thesis} style file accordingly.

%------------------------------------------------------------------------------
\section{Appearance}%
\label{sec:package:appearance}
%------------------------------------------------------------------------------

It used to be the case that nearly all \LaTeX{} documents used the
Computer Modern Fonts. That is no longer necessary. There are rather
complete font sets that are also free that you can use instead.
%In this guide I use a combination of \Package{Palatino} and other fonts.
The default font for theses is \Package{newtx} that is available as of \TeXLive 2013.
If you have this package, then I would recommend using it.
\Package{txfonts} is an older version of this package that can be used instead, if necessary.
Some of the spacings in equations have been improved in \Package{newtx}
and there is a better balance of the sizes of serif, \textsf{sans serif} and \texttt{typewriter} fonts.
Other fonts that look quite nice (e.g.\ Palatino) can also be used.
The option \Option{palatino} in \Package{ubonn-thesis} can be used to select this.
The option actually selects the font packages \Package{mathpazo}, \Package{courier} and \Package{helvet}.
Another alternative is a package such as \Package{pxfonts} 
to get both text and math fonts in the same style. 
Some examples of other possible font packages are given in the style file.
As mentioned above, certain fonts can be selected directly via options:
\Option{txfonts}, \Option{newtx} or \Option{palatino}.

Commonly used packages associated with fonts, tables and
figures are listed in \cref{tab:package:appearance}.

\begin{table}[htbp]
  \centering
  \begin{tabular}{lp{0.75\textwidth}}
    \toprule
    \Package{siunitx} & Typeset units properly with correct spacing.\\
    \Package{graphicx} & The package to use for including graphics.\\
    \Package{rotating} & Package to use for rotating tables etc. The
      \Macro{includegraphics} command can rotate figures directly.\\
    \Package{array} & Adds extra column formatting capabilities.\\
    \Package{xtab} & Can produce tables that extend over more than one page.\\
    \Package{booktabs} & Help for producing nicer tables.\\
    \Package{amsmath}, \Package{amssymb} & Extra math commands and symbols from AMS.\\
    \Package{hepnicenames} & A somewhat restricted list of predefined elementary particles.\\
    \Package{heppennames} & A more complete list of predefined elementary particles.
      Note that \Package{hepnicenames} also loads \Package{heppennames}.
      The packages are based on \Package{hepparticles}, which you can use to define further particles.\\
    \Package{mhchem} & Nice package for typesetting chemical elements correctly.\\
    \Package{xfrac} & Some more options for typesetting fractions.\\
    \Package{xcolor} & Add colour commands.\\
    \Package{titlesec} & Change the appearance of chapter and section headings. 
      See below for more information.\\
    \Package{microtype} & Small adjustments to word spacings.\\
    \bottomrule
  \end{tabular}
  \caption{Useful packages for appearance.}%
  \label{tab:package:appearance}
\end{table}

As alternatives to \Package{xtab}, one can also use
\Package{supertabular} or \Package{longtable}. 
All these packages also have the advantage that you
can specify header and footer text.
If you use the \Env{mpxtabuar} environment from \Package{xtab} 
you can include footnotes in a table.
See the \Package{xtab} documentation for more details.
It is probably best to only use one of these three packages to avoid conflicts.

Use of the \Package{titlesec} package gives a warning when using \KOMAScript{}; hence
as of version 3.0 the chapter title formatting is done by hand in \Package{ubonn-thesis}.
You can switch back to using \Package{titlesec} by giving the option \Option{titlesec}.

The package \Package{microtype} declares itself to be responsible for
\enquote{Subliminal refinements towards typographical perfection}.
Need I say more?

%------------------------------------------------------------------------------
\section{Other packages}%
\label{sec:package:other}
%------------------------------------------------------------------------------

Some other useful packages, some of which are included in
\texttt{ubonn-thesis.sty} are listed in
\cref{tab:package:other1}.

\begin{table}[htbp]
  \centering
  \begin{tabular}{lp{0.8\textwidth}}
    \toprule
    \Package{ifthen} & Provides the \Macro{ifthenelse} command.\\
    \Package{IEEEtrantools} & Contains useful environments for multiline equations.\\
    \Package{feynmp} & Draw Feynman graphs with Metapost.\\
    \Package{axodraw} & Draw Feynman graphs.\\
    \Package{tikz} & General drawing package that can also be used for Feynman graphs.\\
    \Package{standalone} & Allows you to have a document that you can
    directly compile for each figure and also input to another document.\\
    \Package{hyperref} & Adds \Macro{url} command as well as ability
    to click on entries on table of contents etc.\\
    \Package{cleveref} & Provides \enquote{Fig.}, \enquote{Table} etc.\ when you use
      the \Macro{cref\{fig:plot\}} syntax.\\
    \Package{todonotes} & Nice package to add notes to your text.
      You can also use it to indicate missing figures.\\
    \Package{background} & Allows you to add things like DRAFT across the whole document page.\\
    \Package{subfiles} & Provides a nice alternative to
    \Macro{include}.\\
    \Package{subcaption} & A newer alternative to \Package{subfig}.\\
    \Package{tabularx} & Allows fixed table width with flexible column widths.\\
    \Package{floatrow} & Add ability to define own floats.\\
    \Package{physics} & Some useful extra maths commands, especially for differentials.\\
    \Package{commath} & Some useful extra maths commands --- has not been updated for a while.\\
    \Package{skmath} & More maths commands that could be useful.\\
    \Package{adjustbox} & Add much more sophisticated clipping
    capabilities than offered by \Package{graphicx}.\\
    \Package{wrapfig} & Allow text to flow around figures.\\
    \Package{floatflt} & Similar capabilities to \Package{wrapfig} --- allow text to flow around figures.\\
    \Package{glossaries} & Provide commands for creating a glossary.
      This is intended to replace the \Package{glossary} package.\\
    \Package{dcolumn} & Very helpful for lining up columns on character strings such as a decimal point.
      \Package{siunitx} offers similar and better functionality.\\
    \Package{refcheck} & Check whether labels are used, i.e.\ if figures and tables are actually referenced.\\
    \bottomrule
  \end{tabular}
  \caption{Other useful packages.}%
  \label{tab:package:other1}
\end{table}

While the \Package{amsmath} package solves many problems that occur if
you just use the normal \LaTeX{} math mode commands, there are some
things that are not so nice with long and complicated multiline
equations. 
\Package{IEEEtrantools}, in particular the \Env{IEEEeqnarray} environment, can help here.
See Ref.~\cite{lshort} or the package documentation for more details.

The packages \Package{physics}, \Package{commath} and \Package{skmath} provide some additional maths commands.
Differentials etc.\ are particularly useful.
The \Package{physics} package contains a lot of nice and flexible definitions and also handles the spacing around things like \(\dd{x}\) well,
if you use the \verb|\dd{x}| construction.
The \Package{commath} package has not been updated for a while.
\Package{skmath} does quite a lot more then \Package{commath} and even modifies/enhances some standard commands.

The \Package{todonotes} is a very nice package that enables you to add notes to your text.
By default they will be put in the margin of your document.
More details can be found in \cref{sec:package:todo}.

The \Package{cleveref} package is a great package for specifying references to
figures, tables, sections etc.
You just give the command \Macro{cref\{fig:plot\}} and it will add \enquote{Fig.} or whatever
you set to the figure number and an unbreakable space.
This guide has been converted to use the \Package{cleveref} everywhere.

\Package{standalone} is both a package and a document class. It is
available from \TeXLive 2012 onwards. It allows you to have a
standalone document for a \Package{tikz} or \Package{feynmf} figure
and also input this file into another document. If you run PDF\LaTeX\
on the file it also automatically crops the resulting picture. This is
one of those packages where you think \enquote{Why didn't someone
  create this years ago?}. The \Package{tikz} figures included in this
guide make use of it.

I do not recommend the \Package{subfiles} package by default as it is not
included in \TeXLive by default. Have a look at
\url{http://en.wikibooks.org/wiki/LaTeX/General_Guidelines} for
example for more information. If you want to use the package you have
to download it and install for yourself. You can do much the same
thing using AUCTeX inside \texttt{emacs}.

\begin{sloppypar}
Note that there is a conflict if you use \Package{refcheck}, \Package{subcaption} and \Package{hyperref} together.
See \url{http://tex.stackexchange.com/questions/273970/conflict-refcheck-subcaption-packages-for-label-with-underscores} for a workaround.
\end{sloppypar}

It is often useful to indicate whether your thesis is in the draft stage.\index{draft}
The thesis skeleton uses the package \Package{background} for this.
It is only available as of \TeXLive 2011.
Hence \enquote{Draft} is instead put in the heading for older versions.
As an alternative, you can try the package \Package{draftwatermark}.\footnote{%
\Package{draftwatermark} needs \Macro{unitlength} to be set to \SI{1}{pt}.
This is not the default setting in \Package{ubonn-thesis}.
The 2009 thesis skeleton contains \Macro{setlength\{\Macro{unitlength\}\{1pt\}}} (commented out).
Given that this is a global setting, neither it nor \Package{draftwatermark} are included by default.}

A list of other packages that are commonly used is given in
\cref{tab:package:other2}. They are not
included in the list above, because they are either not really needed
or have been superseded by other packages.

\begin{table}[htbp]
  \centering
  \begin{tabular}{lp{0.8\textwidth}}
    \toprule
    \Package{hepunits} & Typeset units properly with correct spacing.\\
    \Package{units} & Typeset units properly with correct spacing.\\
    \Package{SIunits} & Typeset units properly with correct spacing.
      \Package{hepunits} uses this package, so it does not need to be added explicitly\\
    \Package{fancyhdr} & As the name suggests, do your own header and footer configuration.
      Within \KOMAScript{} it is recommended to use \Package{scrlayer-scrpage} instead.\\
    \Package{feynmf} & Draw Feynman graphs with Metafont.\\
    \Package{draftwatermark} & Another package that allows you to add DRAFT to the background each page.\\
    \Package{subfig} & As the name suggests make sub-figures and add
    separate captions for them. \emph{This package has apparently been
      deprecated.}\\
    \Package{subfigure} & As the name suggests make sub-figures and add
    separate captions for them. \emph{This package is deprecated.}\\
    \Package{color} & Add colour commands --- \Package{xcolor} is
    needed to colour boxes around links.\\
    \Package{float} & As far as I can tell \Package{floatflt} offers more options.\\
    \Package{caption} & Much more control on captions --- as
    \KOMAScript{} also has many options, not sure this is necessary.\\
    \Package{ziffer} & Spacing with a comma as decimal separator is
    correct.\\
    \Package{nomencl} & Another package for creating a glossary.\\
    \Package{fncychap} & Another package for changing the style of the
    chapter heading.\\
    \Package{quotchap} & Another package for changing the style of the
    chapter heading.\\
    \bottomrule
  \end{tabular}
  \caption{Other packages that are often used, but I have already
    given alternatives.}%
  \label{tab:package:other2}
\end{table}

As indicated, the \Package{ziffer} package is advertised as providing
the correct spacing after a comma in math mode if you use the comma as
the decimal separator. Compare \(2,5\) with 2,5 and \(2{,}5\). The first
spacing is wrong. If you use the \Package{ziffer} package it will be
correct. However, it does seem to conflict with the use of the
\Package{dcolumn} package, so I cannot compile this guide
if \Package{ziffer} is included. Some  workarounds are discussed in
\cref{sec:layout:german}. In addition, the \Package{siunitx}
package contains the same functionality, which can simply be steered
by changing the document language, as discussed in
\cref{sec:table:siunitx}.
Hence, \Package{ziffer} is not really needed anymore.

%------------------------------------------------------------------------------
\section{ToDo Notes}%
\label{sec:package:todo}
%------------------------------------------------------------------------------

If you include the option \Option{todonotes} with \Package{ubonn-thesis},
the margin of the thesis will also be adjusted a bit to leave more space for such notes.
As an example, the \Macro{mynote} command is defined.

If you turn on the option \Option{todonotes}, the package will be loaded.
With the option \Option{shownotes} they will also be 
\mynote{displayed.}{The \Option{shownotes} also turns on \Option{todonotes}.}

Another nice feature of the \Package{todonotes} package is that you can use
\Macro*{missingfigure\{text\}}\index{\textbackslash missingfigure}
to indicate figures that you still want to include.
See for example \cref{fig:missing}. You can get a list of your notes
by adding the \Macro{listoftodos} command somewhere.

The macro \Macro{todo} has some problems inside floats.
You should pass the option \Option{inline} to \Macro{mynote} if you
want a comment inside a caption. If you also include a \enquote{List of Tables}
or \enquote{List of Figures}, then you have to provide a short caption
as an option or move the note outside (after) the caption.

\begin{figure}[htbp]
  \missingfigure{Still need to create the plot to be included here.}
  \caption[Optional caption without \Macro{mynote} so that \enquote{List of Figures} does not break.]{%
    This figure still has to be added.
    \mynote[inline]{}{Use the option \Option{inline} to include a note in a caption,
    as this is floating. A note that does not point to a block of text needs a
    blank first argument.}
  }%
  \label{fig:missing}
\end{figure}