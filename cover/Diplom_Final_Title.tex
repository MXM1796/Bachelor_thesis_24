%
% Title page layout for printed version
%
\extratitle{\rmfamily\setlength{\parindent}{0pt}
  \begin{center}
    {\fontsize{44}{50}\selectfont
      Universität Bonn}

    \vspace*{20pt}

    \begin{singlespace}
      \fontsize{30}{40}\selectfont
      \InstituteName
    \end{singlespace}

    \vspace*{40pt}

    \begin{onehalfspace}
      \bfseries\huge
      \thesistitle
    \end{onehalfspace}

    \vspace*{20pt}

    {\LARGE
      \thesisauthor
    }
  \end{center}

  \vspace*{\fill}

  \thesisabstract

  \vspace*{\fill}

  {\normalfont\normalsize
    \parbox{0.3\textwidth}{\InstituteAddress}
    \parbox{0.4\textwidth}{%
      \centering
      \includegraphics[width=5cm]{instituts_siegel}
    }
    \parbox{0.3\textwidth}{%
      \thesisnumber\\
      \thesismonth{} \thesisyear
    }
  }
}
\titlehead{%
  \centering
  {\fontsize{44}{50}\selectfont
    Universität Bonn}

  \vspace*{20pt}

  \begin{singlespace}
    \fontsize{30}{40}\selectfont
    \InstituteName
  \end{singlespace}
}
\title{\thesistitle}
\author{\LARGE
  \thesisauthor
}
\date{}
\publishers{\vspace*{\fill}
  \raggedright
  {\normalsize
    \begin{otherlanguage}{ngerman}
      Dieser Forschungsbericht wurde als Diplomarbeit von der
      Mathematisch-Naturwissenschaftlichen Fakultät der Universität Bonn
      angenommen.
    \end{otherlanguage}

    \vspace*{10ex minus 4ex}

    \noindent
    \begin{tabular}{@{}ll}
      Angenommen am:         & \thesissubmit\\
      \thesisrefereeonetext: & \thesisrefereeone\\
      \thesisrefereetwotext: & \thesisrefereetwo
    \end{tabular}
  }
}

\maketitle
