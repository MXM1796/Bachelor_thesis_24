\documentclass[a4paper,twoside,ngerman,UKenglish]{scrbook}
%------------------------------------------------------------------------------
% Wrapper for testing cover pages for theses in the
% Physikalisches Institut, Uni Bonn
%
% Set to 2009 for biblatex and bibtex8 (TeX Live 2009)
% Set to 2011 for biblatex and biber   (TeX Live 2011 or later)
% This must be set before \usepackage{pibonn-thesis}

%------------------------------------------------------------------------------
% Set to 2009 for biblatex and bibtex8 (TeX Live 2009)
% Set to 2011 for biblatex and biber   (TeX Live 2011 or later)
% This must be set before \usepackage{pibonn-thesis}
\newcommand*{\texlive}{2011}
%
\usepackage{pibonn-thesis}
\usepackage{./thesis_skel/thesis_defs}

%------------------------------------------------------------------------------
% Use bibtex8 for TeX Live 2009
%   If you do not have umlauts etc. in your refs, you can drop bibencoding=latin1
% Use biber   for TeX Live 2011 or later
%
\ifthenelse {\texlive = 2009} {%
  \usepackage[backend=bibtex8,hyperref=true,bibencoding=latin1,
    style=numeric-comp,sorting=none,block=ragged,firstinits=true]{biblatex}
}{%
  \usepackage[backend=biber,
    style=numeric-comp,sorting=none,block=ragged,firstinits=true]{biblatex}
}
% Specify the bibliography files here and not at the end!
% Use standard_refs-bibtex if you use bibtex8
% and standard_refs-biber  if you use biber
\ifthenelse {\texlive = 2009} {%
  % Adjustments to output are in this style file:
  \usepackage{./biblatex/biblatex-num-v2009}
  \bibliography{./mythesis/thesis_refs.bib,%
    ./refs/standard_refs-bibtex.bib}
}{%
  % Adjustments to output are in this style file:
  \usepackage{./biblatex/biblatex-num-v2011}
  \addbibresource{./mythesis/thesis_refs.bib}
  \addbibresource{./refs/standard_refs-biber.bib}
}

%------------------------------------------------------------------------------
% The following definitions are used to produce the title pages
% needed at various stages
\newcommand{\thesistitle}{Titel der Arbeit}
\newcommand{\thesisauthor}{Vorname Name}
\newcommand{\thesistown}{Geburtsort}
\renewcommand*{\InstituteName}{\PI}
\renewcommand*{\inInstitute}{\inPI}
\renewcommand*{\InstituteAddress}{\PIaddress}
% This is the month and year when the printed version of the thesis
% was released
\newcommand{\thesismonth}{October}
\newcommand{\thesisyear}{2012}
\newcommand{\thesisnumber}{BONN-IR-2012-XX}
% Adjust ...refereetext depending on male/female referee
\newcommand*{\thesisrefereeonetext}{1.\ Gutachter}
\newcommand*{\thesisrefereeone}{Prof.\ Dr.\ John Smith}
\newcommand*{\thesisrefereetwotext}{2.\ Gutachterin}
\newcommand*{\thesisrefereetwo}{Prof.\ Dr.\ Anne Jones}
% Date when thesis was accepted
\newcommand{\thesisaccept}{XX.YY.2012}
% Date of thesis examination
\newcommand{\thesispromotion}{XX.YY.2012}

%------------------------------------------------------------------------------
% The abstract is only needed for the printed version and should be in
% English regardless of the language of the thesis
\newcommand{\thesisabstract}{%
  This thesis is the result of many years of research into how to format
  a document using \LaTeX. It contains examples and recommendations on
  what packages to use and how to use them. It makes no attempt to be
  an introduction to \LaTeX. Indeed some basic \LaTeX{} knowledge is
  assumed. It is very much geared to (particle) physics theses, but
  may be useful for others.
  This thesis is the result of many years of research into how to format
  a document using \LaTeX. It contains examples and recommendations on
  what packages to use and how to use them. It makes no attempt to be
  an introduction to \LaTeX. Indeed some basic \LaTeX{} knowledge is
  assumed. It is very much geared to (particle) physics theses, but
  may be useful for others.
}

%------------------------------------------------------------------------------
% Give a list of directories where figures can be found. Do not leave
% any spaces in the list and end the directory name with a /
\graphicspath{{./figs/cover/}}

%------------------------------------------------------------------------------
% Set all headers and footers so I can see if they are used properly
\ihead[Scrplain ihead]{Scrheadings ihead}
\chead[Scrplain chead]{Scrheadings chead}
\ohead[Scrplain ohead]{Scrheadings ohead}
\ifoot[Scrplain ifoot]{Scrheadings ifoot}
\cfoot[Scrplain cfoot]{Scrheadings cfoot}
\ofoot[Scrplain ofoot]{Scrheadings ofoot}

%-------------------------------------------------------------------------------
\begin{document}
%
% Cover page of thesis - this is only needed for the printed final version
% % Cover page layout for the department library version

% Adjust the left margin to compensate for twoside option.
% The size of the adjustment may have to be adjusted 
% if you change the fraction of the page area that is used for the text.

{\thispagestyle{empty}
\begin{addmargin}[\UBNcoveroffset]{-\UBNcoveroffset}
  % \addtolength{\oddsidemargin}{1.0cm}\addtolength{\topmargin}{1.0cm}
  \rmfamily\setlength{\parindent}{0pt}
  \begin{center}
    {\fontsize{44}{50}\selectfont
      Universität Bonn}

    \vspace*{20pt}

    \begin{singlespace}
      \fontsize{30}{40}\selectfont
      \InstituteName
    \end{singlespace}

    \vspace*{40pt}

    \begin{onehalfspace}
      \bfseries\huge
      \thesistitle
    \end{onehalfspace}

    \vspace*{20pt}

    {\LARGE
      \thesisauthor
    }
  \end{center}

  \vspace*{\fill}

  \thesisabstract

  \vspace*{\fill}

  {\normalfont\normalsize
    \parbox{0.3\textwidth}{\InstituteAddress}
    \hspace{\fill}
    \parbox{0.35\textwidth}{%
      \centering
      \includegraphics[width=5cm]{instituts_siegel}
    }
    \hspace{\fill}
    \parbox{0.3\textwidth}{%
      \thesisnumber\\
      \thesismonth{} \thesisyear\\
      ISSN-0172-8741
    }
  }
\end{addmargin}
}

% % Cover page layout for the department library version

% Adjust the left margin to compensate for twoside option.
% The size of the adjustment may have to be adjusted 
% if you change the fraction of the page area that is used for the text.

{\thispagestyle{empty}
\begin{addmargin}[\UBNcoveroffset]{-\UBNcoveroffset}
  % \addtolength{\oddsidemargin}{1.0cm}\addtolength{\topmargin}{1.0cm}
  \rmfamily\setlength{\parindent}{0pt}
  \begin{center}
    {\fontsize{44}{50}\selectfont
      Universität Bonn}

    \vspace*{20pt}

    \begin{singlespace}
      \fontsize{30}{40}\selectfont
      \InstituteName
    \end{singlespace}

    \vspace*{40pt}

    \begin{onehalfspace}
      \bfseries\huge
      \thesistitle
    \end{onehalfspace}

    \vspace*{20pt}

    {\huge
      \thesisauthor
    }
  \end{center}

  \vspace*{\fill}

  \thesisabstract

  \vspace*{\fill}

  {\normalfont\normalsize
    \parbox{0.3\textwidth}{\InstituteAddress}
    \parbox{0.4\textwidth}{%
      \centering
      \includegraphics[width=5cm]{instituts_siegel}
    }
    \parbox{0.29\textwidth}{%
      \thesisnumber\\
      \thesismonth{} \thesisyear
    }
  }
\end{addmargin}
}

% % Cover page layout for the department library version

% Adjust the left margin to compensate for twoside option.
% The size of the adjustment may have to be adjusted 
% if you change the fraction of the page area that is used for the text.

{\thispagestyle{empty}
\begin{addmargin}[\UBNcoveroffset]{-\UBNcoveroffset}
  % \addtolength{\oddsidemargin}{1.0cm}\addtolength{\topmargin}{1.0cm}
  \rmfamily\setlength{\parindent}{0pt}
  \begin{center}
    {\fontsize{44}{50}\selectfont
      Universität Bonn}

    \vspace*{20pt}

    \begin{singlespace}
      \fontsize{30}{40}\selectfont
      \InstituteName
    \end{singlespace}

    \vspace*{40pt}

    \begin{onehalfspace}
      \bfseries\huge
      \thesistitle
    \end{onehalfspace}

    \vspace*{20pt}

    {\huge
      \thesisauthor
    }
  \end{center}

  \vspace*{\fill}

  \thesisabstract

  \vspace*{\fill}

  {\normalfont\normalsize
    \parbox{0.3\textwidth}{\InstituteAddress}
    \parbox{0.4\textwidth}{%
      \centering
      \includegraphics[width=5cm]{instituts_siegel}
    }
    \parbox{0.29\textwidth}{%
      \thesisnumber\\
      \thesismonth{} \thesisyear
    }
  }
\end{addmargin}
}

%
% Start counting pages from the title page
%
\frontmatter
% Dedication has to come before \maketitle
\dedication{For my friends and supporters}
% Use this version when you submit your thesis
% %
% Title page layout for submitted version.
%
\title{\thesistitle}
\subtitle{\vspace*{4ex}
  \begin{otherlanguage}{ngerman}
    Dissertation\\
    zur\\
    Erlangung des Doktorgrades (Dr.\ rer.\ nat.)\\
    der\\
    Mathematisch-Naturwissenschaftlichen Fakultät\\
    der\\
    Rheinischen Friedrich-Wilhelms-Universität Bonn
  \end{otherlanguage}
}
\author{%
  vorgelegt von\\
  \thesisauthor\\
  aus\\
  \thesistown
}
\date{}
\publishers{%
  Bonn \thesisyear
}
\lowertitleback{\normalsize
  \begin{otherlanguage}{ngerman}
    Angefertigt mit Genehmigung der 
    Mathematisch-Naturwissenschaftlichen Fakultät der
    Rheinischen Friedrich-Wilhelms-Universität Bonn
  \end{otherlanguage}
  
  \vspace*{10ex minus 4ex}
  
  \noindent
  \begin{otherlanguage}{ngerman}
    \begin{tabular}{@{}ll}
      \thesisrefereeonetext: & \thesisrefereeone\\
      \thesisrefereetwotext: & \thesisrefereetwo\\[2ex]
      Tag der Promotion:      & \\
      Erscheinungsjahr:       & 
    \end{tabular}
  \end{otherlanguage}
}

\maketitle

% %
% Title page layout for submitted version
%
\title{\thesistitle}
\author{\LARGE\thesisauthor}
\date{}
\publishers{\LARGE
  \begin{otherlanguage}{ngerman}
    Masterarbeit in Physik\\
    angefertigt \inInstitute\\[3ex]
    vorgelegt der\\
    Mathematisch-Naturwissenschaftlichen Fakultät\\
    der\\
    Rheinischen Friedrich-Wilhelms-Universität\\
    Bonn\\[3ex]
    \thesismonth{} \thesisyear
  \end{otherlanguage}
  \vspace*{\fill}
}
\lowertitleback{%
  % \begin{otherlanguage}{ngerman}
  %   Ich versichere, dass ich diese Arbeit selbstständig
  %   verfasst und keine anderen als die angegebenen Quellen und
  %   Hilfsmittel benutzt sowie die Zitate kenntlich gemacht habe.

  %   \vspace*{4ex minus 0.5ex}

  %   \begin{tabular}{@{}lc@{\hspace*{4.0cm}}c}
  %     Bonn, &  \makebox[3cm]{\dotfill} & \makebox[6cm]{\dotfill}\\
  %     & Datum & Unterschrift
  %   \end{tabular}
  % \end{otherlanguage}

  \begin{otherlanguage}{UKenglish}
    I hereby declare that this thesis was formulated by
    myself and that no sources or tools other than those cited were used.

    \vspace*{4ex minus 0.5ex}

    \begin{tabular}{@{}lc@{\hspace*{4.0cm}}c}
      Bonn, &  \makebox[3cm]{\dotfill} & \makebox[6cm]{\dotfill}\\
      & Date & Signature
    \end{tabular}
  \end{otherlanguage}

  \vspace*{10ex minus 4ex}

  \noindent
  \begin{otherlanguage}{ngerman}
    \begin{tabular}{@{}ll}
      \thesisrefereeonetext: & \thesisrefereeone\\
      \thesisrefereetwotext: & \thesisrefereetwo
    \end{tabular}
  \end{otherlanguage}
}

\maketitle

% %
% Title page layout for submitted version
%
\title{\thesistitle}
\author{\LARGE\thesisauthor}
\date{}
\publishers{\LARGE
  \begin{otherlanguage}{ngerman}
    Diplomarbeit in Physik\\
    angefertigt \inInstitute\\[3ex]
    vorgelegt der\\
    Mathematisch-Naturwissenschaftlichen Fakultät\\
    der\\
    Rheinischen Friedrich-Wilhelms-Universität\\
    Bonn\\[3ex]
    \thesismonth{} \thesisyear
  \end{otherlanguage}
  \vspace*{\fill}
}
\lowertitleback{%
  \begin{otherlanguage}{ngerman}
    Ich versichere, dass ich diese Arbeit selbstständig
    verfasst und keine anderen als die angegebenen Quellen und
    Hilfsmittel benutzt sowie die Zitate kenntlich gemacht habe.

    \vspace*{4ex minus 0.5ex}

    \begin{tabular}{@{}lc@{\hspace*{4.0cm}}c}
      Bonn, &  \makebox[3cm]{\dotfill} & \makebox[6cm]{\dotfill}\\
      & Datum & Unterschrift
    \end{tabular}
  \end{otherlanguage}

  % \begin{otherlanguage}{UKenglish}
  %   I hereby declare that this thesis was formulated by
  %   myself and that no sources or tools other than those cited were used.

  %   \vspace*{4ex minus 0.5ex}

  %   \begin{tabular}{@{}lc@{\hspace*{4.0cm}}c}
  %     Bonn, &  \makebox[3cm]{\dotfill} & \makebox[6cm]{\dotfill}\\
  %     & Date & Signature
  %   \end{tabular}
  % \end{otherlanguage}

  \vspace*{10ex minus 4ex}

  \noindent
  \begin{otherlanguage}{ngerman}
    \begin{tabular}{@{}ll}
      \thesisrefereeonetext: & \thesisrefereeone\\
      \thesisrefereetwotext: & \thesisrefereetwo
    \end{tabular}
  \end{otherlanguage}
}

\maketitle

%
% Title page layout for submitted version
%
\title{\thesistitle}
\author{\LARGE\thesisauthor}
\date{}
\publishers{\LARGE
  \begin{otherlanguage}{ngerman}
    Bachelorarbeit in Physik\\
    angefertigt \inInstitute\\[3ex]
    vorgelegt der\\
    Mathematisch-Naturwissenschaftlichen Fakultät\\
    der\\
    Rheinischen Friedrich-Wilhelms-Universität\\
    Bonn\\[3ex]
    \thesismonth{} \thesisyear
  \end{otherlanguage}
  \vspace*{\fill}
}
\lowertitleback{%
  \begin{otherlanguage}{ngerman}
    Ich versichere, dass ich diese Arbeit selbstständig
    verfasst und keine anderen als die angegebenen Quellen und
    Hilfsmittel benutzt sowie die Zitate kenntlich gemacht habe.

    \vspace*{4ex minus 0.5ex}

    \begin{tabular}{@{}lc@{\hspace*{4.0cm}}c}
      Bonn, &  \makebox[3cm]{\dotfill} & \makebox[6cm]{\dotfill}\\
      & Datum & Unterschrift
    \end{tabular}
  \end{otherlanguage}

  % \begin{otherlanguage}{UKenglish}
  %   I hereby declare that this thesis was formulated by
  %   myself and that no sources or tools other than those cited were used.

  %   \vspace*{4ex minus 0.5ex}

  %   \begin{tabular}{@{}lc@{\hspace*{4.0cm}}c}
  %     Bonn, &  \makebox[3cm]{\dotfill} & \makebox[6cm]{\dotfill}\\
  %     & Date & Signature
  %   \end{tabular}
  % \end{otherlanguage}

  \vspace*{10ex minus 4ex}

  \noindent
  \begin{otherlanguage}{ngerman}
    \begin{tabular}{@{}ll}
      \thesisrefereeonetext: & \thesisrefereeone\\
      \thesisrefereetwotext: & \thesisrefereetwo
    \end{tabular}
  \end{otherlanguage}
}

\forcetwosidetitle

% Use this command for the print version
% %
% Title page layout for published version
%
% \titlehead{%
%   \centering
%   {\fontsize{44}{50}\selectfont
%     Universität Bonn}

%   \vspace*{20pt}

%   \begin{singlespace}
%     \fontsize{30}{40}\selectfont
%     \InstituteName
%   \end{singlespace}
% }
\title{\thesistitle}
\subtitle{\vspace*{4ex}
  \begin{otherlanguage}{ngerman}
    Dissertation\\
    zur\\
    Erlangung des Doktorgrades (Dr.\ rer.\ nat.)\\
    der\\
    Mathematisch-Naturwissenschaftlichen Fakultät\\
    der\\
    Rheinischen Friedrich-Wilhelms-Universität Bonn
  \end{otherlanguage}
}
\author{%
  von\\
  \LARGE \thesisauthor\\
  aus\\
  \thesistown
}
\date{}
\publishers{%
  Bonn, \thesissubmit
}
\lowertitleback{\normalsize
  \raggedright
  \begin{otherlanguage}{ngerman}
    Angefertigt mit Genehmigung der Mathematisch-Naturwissenschaftlichen Fakultät der Rheinischen Friedrich-Wilhelms-Universität Bonn
    % Dieser Forschungsbericht wurde als Dissertation von der
    % Mathematisch-Naturwissenschaftlichen Fakultät der Universität
    % Bonn angenommen und ist auf dem
    % Hochschulschriftenserver der ULB Bonn
    % \url{http://hss.ulb.uni-bonn.de/diss_online} elektronisch
    % publiziert.
  \end{otherlanguage}

  \vspace*{10ex minus 4ex}

  \noindent
  \begin{otherlanguage}{ngerman}
    \begin{tabular}{ll}
      \thesisrefereeonetext: & \thesisrefereeone\\
      \thesisrefereetwotext: & \thesisrefereetwo\\[2ex]
      Tag der Promotion: & \thesispromotion\\
      Erscheinungsjahr: & \thesisyear
    \end{tabular}
  \end{otherlanguage}
}

\maketitle

% %
% Title page layout for printed version
%
\titlehead{%
  \centering
  {\fontsize{44}{50}\selectfont
    Universität Bonn}

  \vspace*{20pt}

  \begin{singlespace}
    \fontsize{30}{40}\selectfont
    \InstituteName
  \end{singlespace}
}
\title{\thesistitle}
\author{\LARGE\thesisauthor}
\date{}
\publishers{\vspace*{\fill}
  \raggedright
  {\normalsize
    \begin{otherlanguage}{ngerman}
      Dieser Forschungsbericht wurde als Masterarbeit von der
      Mathematisch-Naturwissenschaftlichen Fakultät der Universität Bonn
      angenommen.
    \end{otherlanguage}

    \vspace*{10ex minus 4ex}

    \noindent
    \begin{tabular}{@{}ll}
      Angenommen am:         & \thesissubmit\\
      \thesisrefereeonetext: & \thesisrefereeone\\
      \thesisrefereetwotext: & \thesisrefereetwo
    \end{tabular}
  }
}

\forcetwosidetitle

% %
% Title page layout for printed version
%
\extratitle{\rmfamily\setlength{\parindent}{0pt}
  \begin{center}
    {\fontsize{44}{50}\selectfont
      Universität Bonn}

    \vspace*{20pt}

    \begin{singlespace}
      \fontsize{30}{40}\selectfont
      \InstituteName
    \end{singlespace}

    \vspace*{40pt}

    \begin{onehalfspace}
      \bfseries\huge
      \thesistitle
    \end{onehalfspace}

    \vspace*{20pt}

    {\LARGE
      \thesisauthor
    }
  \end{center}

  \vspace*{\fill}

  \thesisabstract

  \vspace*{\fill}

  {\normalfont\normalsize
    \parbox{0.3\textwidth}{\InstituteAddress}
    \parbox{0.4\textwidth}{%
      \centering
      \includegraphics[width=5cm]{instituts_siegel}
    }
    \parbox{0.3\textwidth}{%
      \thesisnumber\\
      \thesismonth{} \thesisyear
    }
  }
}
\titlehead{%
  \centering
  {\fontsize{44}{50}\selectfont
    Universität Bonn}

  \vspace*{20pt}

  \begin{singlespace}
    \fontsize{30}{40}\selectfont
    \InstituteName
  \end{singlespace}
}
\title{\thesistitle}
\author{\LARGE
  \thesisauthor
}
\date{}
\publishers{\vspace*{\fill}
  \raggedright
  {\normalsize
    \begin{otherlanguage}{ngerman}
      Dieser Forschungsbericht wurde als Diplomarbeit von der
      Mathematisch-Naturwissenschaftlichen Fakultät der Universität Bonn
      angenommen.
    \end{otherlanguage}

    \vspace*{10ex minus 4ex}

    \noindent
    \begin{tabular}{@{}ll}
      Angenommen am:         & \thesissubmit\\
      \thesisrefereeonetext: & \thesisrefereeone\\
      \thesisrefereetwotext: & \thesisrefereetwo
    \end{tabular}
  }
}

\maketitle


\pagestyle{scrplain}

\tableofcontents

\mainmatter
\pagestyle{scrheadings}

% !TEX root = mythesis.tex

%==============================================================================
\chapter{Introduction}
\label{sec:intro}
%==============================================================================

The introduction usually gives a few pages of introduction to the
whole subject, maybe even starting with the Greeks.

For more information on \LaTeX{} and the packages that are available
see for example the books of Kopka~\cite{kopka04} and Goossens et
al~\cite{goossens04}.

A lot of useful information on particle physics can be found in the
\enquote{Particle Data Book}~\cite{pdg2010}.

I have resisted the temptation to put a lot of definitions into the
file \texttt{thesis\_defs.sty}, as everyone has their own taste as
to what scheme they want to use for names.
However, a few examples are included to help you get started:
\begin{itemize}
\setlength{\itemsep}{0pt}\setlength{\parskip}{0pt}
\item cross-sections are measured in \si{\pb} and integrated
  luminosity in \si{\invpb};
\item the \KoS is an interesting particle;
\item the missing transverse momentum, \pTmiss, is often called
  missing transverse energy, even though it is calculated using a vector sum.
\end{itemize}
Note that the examples of units assume that you are using the
\textsf{siunitx} package.

It also is probably a good idea to include a few well formatted
references in the thesis skeleton. More detailed suggestions on what
citation types to use can be found in the \enquote{Thesis Guide}~\cite{thesis-guide}:
\begin{itemize}
\item articles in refereed journals~\cite{pdg2010,Aad:2010ey};
\item a book~\cite{Halzen:1984mc};
\item a PhD thesis~\cite{tlodd:2012} and a Diplom thesis~\cite{mergelmeyer:2011};
\item a collection of articles~\cite{lhc:vol1};
\item a conference note~\cite{ATLAS-CONF-2011-008};
\item a preprint~\cite{atlas:perf:2009} (you can also use
  \texttt{@online} or \texttt{@booklet} for such things);
\item something that is only available online~\cite{thesis-guide}.
\end{itemize}

At the end of the introduction it is normal to say briefly what comes
in the following chapters.

The line at the beginning of this file is used by TeXstudio etc.\ to
specify which is the master \LaTeX{} file, so that you can compile your thesis
directly from this file.
The lines at the end of this file are used by AUCTeX
directly within \texttt{emacs} to do the same thing.
If your thesis is called something other than \texttt{mythesis}, adjust them as appropriate.

%%% Local Variables: 
%%% mode: latex
%%% TeX-master: "mythesis"
%%% End: 


\chapter{2nd Chapter}

\section{Blabla}
\label{sec:blabla}

Here I just add a bit more text on another page to check that heading
etc. work as expected!
\clearpage

It is also good to have a 3rd page!

\backmatter
\printbibliography[heading=bibintoc]

\end{document}
